\documentclass{report}
\usepackage[utf8]{inputenc}
\usepackage[T1]{fontenc}
\usepackage[spanish]{babel}
\usepackage{geometry}
\usepackage{hyperref}
\usepackage{microtype} % Mejora la justificación y el espaciado
\usepackage{cite}      % Para gestionar las citas bibliográficas
\usepackage{verbatim}

% --- Document Setup ---
\geometry{a4paper, margin=1in}
\hypersetup{
    colorlinks=true,
    linkcolor=blue,
    filecolor=magenta,      
    urlcolor=cyan,
}

\title{INTERNET DE LAS COSAS (IoT) CON PYTHON Y Raspberry PI}
\author{Prof. César Rodríguez}
\date{}

\begin{document}

\maketitle
\tableofcontents

\chapter{INTRODUCCIóN}
% 1.INTRODUCCIóN.
En las últimas décadas, el desarrollo tecnológico ha impulsado transformaciones profundas en la forma en que las personas 
interactúan con su entorno. Uno de los avances más significativos en este proceso es el Internet de las Cosas 
(IoT, por sus siglas en inglés), concepto que hace referencia a la interconexión de dispositivos físicos a través de redes 
digitales con el fin de recopilar, procesar y compartir información en tiempo real. Esta tendencia ha abierto un amplio 
abanico de oportunidades en diversos sectores como la salud, la industria, la educación, el transporte y la gestión de recursos, 
generando cambios en la eficiencia de los procesos y en la calidad de vida de la sociedad.

El IoT se fundamenta en la integración de sensores, actuadores, sistemas de comunicación y plataformas de análisis de datos, 
permitiendo que objetos cotidianos se conviertan en elementos inteligentes capaces de interactuar entre sí y con los usuarios. 
Su implementación ofrece beneficios como la automatización de tareas, optimización en la toma de decisiones y reducción de 
costos operativos, además de posibilitar soluciones innovadoras para problemáticas actuales como la sostenibilidad ambiental, 
la movilidad urbana o la atención médica remota.

No obstante, el crecimiento del IoT también plantea retos importantes relacionados con la seguridad de la información, la 
interoperabilidad de los dispositivos y la gestión de grandes volúmenes de datos. Estos desafíos hacen necesario el desarrollo 
de propuestas que garanticen no solo la viabilidad técnica de los sistemas, sino también su confiabilidad y aceptación social.

En este contexto, el presente proyecto de tesis (\textbf{IoT con Python y Raspberry Pi}) busca orientar en el estudio y aplicación 
del Internet de las Cosas, analizando sus fundamentos teóricos, sus potenciales aplicaciones y las limitaciones que enfrenta 
en su implementación. De esta manera, se pretende aportar conocimiento que contribuya al diseño de soluciones tecnológicas 
innovadoras y sostenibles, alineadas con las demandas actuales de la sociedad y con las perspectivas de un futuro cada vez 
más digitalizado.

En un contexto más amplio de este proyecto, en esta introducción se destacan los siguientes aspectos claves:
\begin{itemize}
    \item Naturaleza y Enfoque del Proyecto:
    \begin{itemize}
        \item El proyecto puede tomarse como una guía práctica y completa dirigida tanto a usuarios \textbf{principiantes} o 
        \textbf{intermedios} interesados en el Internet de las Cosas (IoT).
        \item Busca un \textbf{equilibrio completo} entre el \textbf{desarrollo de dispositivos en Raspberry Pi} y el 
        \textbf{desarrollo de servidores locales y remotos utilizando Python}.
    \end{itemize}

    \item Objetivo Principal y Producto Final:
    \begin{itemize}                                                                                                                        
        \item El objetivo central es construir una \textbf{plataforma en la nube en la que múltiples usuarios pueden iniciar 
        sesión de forma segura y controlar y monitorear sus dispositivos autorizados en tiempo real}.
        \item El producto final será un \textbf{\texttt{'dashboard'}} que muestre gráficos visuales con datos de sensores y 
        actuadores en tiempo real, con una \textbf{infraestructura a nivel de la nube}, que permitirá una solución IoT completa 
        desde el dispositivo hasta la nube.
    \end{itemize}

    \item Tecnologías Claves:
    \begin{itemize}
        \item Las \textbf{tecnologías principales} en las que se trabajará son \textbf{Python, Raspberry Pi, Flask, AWS y PubNub}.
    \end{itemize}
\end{itemize}

\section{Estructura del Proyecto por Etapas}
    \begin{itemize}
        \item \textbf{Etapa 1: Introducción a IoT}: Se enfoca en \textbf{entender qué es el Internet de las Cosas y sus principales componentes}, como 
              dispositivos inteligentes, sensores y actuadores. También se tendrá como objetivo entender \textbf{cuatro importantes modelos de 
              comunicación del Internet de las Cosas}.
        \item \textbf{Etapa 2: Primer Proyecto IoT}: Se \textbf{ejecutarán labores prácticas} desarrollando un proyecto IoT simple con sensores
              mostrando datos en una aplicación web, utilizando la técnica \textbf{AJAX} para la comunicación. Aquí se construirá un 
              \textbf{dispositivo IoT de defensa contra robos} que detecta movimiento, activa una alarma y envía alertas, permitiendo también la desactivación
              de la alarma.
        \item \textbf{Etapa 3: Seguridad y Protocolos de Comunicación}: Después de analizar ventajas y desventajas del proyecto de la Sección 2, se 
              estudiará en profundidad los \textbf{protocolos de comunicación en tiempo real y ligeros} para IoT, como \textbf{MQTT, WebSockets}, y se hará
              una \textbf{demostración práctica con PubNub}. También se cubrirá \textbf{seguridad en Internet y criptografía}, incluyendo SSL/TLS y 
              protocolos HTTP.
        \item \textbf{Etapa 4: Reconstrucción del Proyecto con PubNub y AWS}: Se \textbf{reconstruirá el proyecto anterior utilizando PubNub como el 
              principal protocolo de comunicación} en lugar de \textbf{AJAX}. Además, se aprenderá a \textbf{desplegar el servidor IoT en la nube de AWS}.
        \item \textbf{Etapa 5: Implementación de Seguridad}: Se centrará en el desarrollo de las terminologías de seguridad estudiadas en la Sección 3. 
              Esto incluye la adquisición de un \textbf{nombre de dominio personalizado y su aseguramiento con certificados SSL/TLS}, 
              así como la implementación de una \textbf{funcionalidad de inicio de sesión de usuario segura} y el almacenamiento de detalles de usuario 
              en una base de datos integrada. Se gestionarán las \textbf{reglas de seguridad de entrada para HTTPS en el servidor remoto de AWS}.
        \item \textbf{Etapa 6: Conexión Segura de Usuarios y Dispositivos}: Se implementará una forma segura para que usuarios y dispositivos IoT se 
              conecten al servidor. Se utilizará la \textbf{funcionalidad de administrador de acceso de PubNub} para que los usuarios administradores 
              puedan \textbf{otorgar acceso de lectura y escritura en tiempo real} a usuarios no administradores y dispositivos.
        \item \textbf{Etapa 7: Proyecto Final: Sistema de Monitoreo Remoto}: Se añadirán más sensores y actuadores para construir un 
              \textbf{sistema de monitoreo salud/batería}, familiarizándose con convertidores digitales, interfaz periférica serial y más.
    \end{itemize}

\section{Resultados del Proyecto}
    \begin{itemize}
        \item Al finalizar el proyecto, se \textbf{entenderá lo que se necesita para construir una solución IoT propia e integral}, desde la 
        simplicidad a nivel de dispositivo hasta la complejidad de la infraestructura a nivel de la nube. Podrán expandir el proyecto añadiendo más 
        dispositivos y funcionalidades según surjan necesidades.
    \end{itemize}

En esencia, en este proyecto se estudian desde los fundamentos teóricos y el desarrollo local de dispositivos, hasta 
la implementación de una plataforma IoT segura, escalable y en la nube (servidor remoto), capacitandonos para abordar problemas del mundo real con un
conjunto robusto de \textbf{tecnologías} y \textbf{conocimientos}.

\newpage
\section{Guía Prática}
% 1.3.GUIA PRACTICA.
Se pretende con este proyecto (\textbf{IoT con Python y Raspberry Pi}) construir una \textbf{guía práctica y completa} 
para usuarios principiantes e intermedios en el campo del Internet de las Cosas (IoT), con 
el objetivo de que se \textbf{entiendan y resuelvan problemas del mundo real de IoT}.

En un contexto más amplio de esta \textbf{introducción}, esta guía práctica se manifiesta a través de varias secciones 
y proyectos claves:

\begin{itemize}
    \item \textbf{Desarrollo de proyectos IoT}:
    \begin{itemize}
        \item La etapa 2 se centra en la \textbf{puesta en práctica} con el \textbf{desarrollo de un proyecto simple de IoT 
        utilizando sensores que muestran datos en una aplicación web}, empleando la técnica AJAX para la comunicación cliente-servidor.
        \item Un ejemplo detallado de este enfoque es el \textbf{proyecto de detector de movimiento antirrobo}, donde se conecta 
        un sensor PIR y un zumbador a una Raspberry Pi, y se escribe código Python para detectar movimiento y controlar el zumbador, 
        además de configurar un servidor web HTTP básico.
        \item La etapa 4 reconstruye el proyecto anterior para utilizar PubNub como protocolo de comunicación principal y enseña 
        \textbf{cómo desplegar un servidor IoT en la nube de AWS}.
        \item La etapa 7 culmina con un \textbf{último proyecto llamado sistema de monitoreo salud/batería}. En este proyecto, se 
        añadiran los sensores y actuadores necesarios para monitoria tales variable, y los participantes se familiarizan con 
        convertidores digitales, interfaz periférica serial y más. El producto final es un \textbf{tablero de control 
        que muestra gráficos visuales con datos de sensores y actuadores en tiempo real}.
    \end{itemize}

    \item \textbf{Desarrollo de servidor y seguridad}:
    \begin{itemize}
        \item El proyecto proporciona un equilibrio entre el \textbf{desarrollo de dispositivos en Raspberry Pi y el desarrollo de 
        servidores locales y remotos usando Python}.
        \item La etapa 5 se enfoca en el \textbf{desarrollo de conceptos de seguridad}, incluyendo la \textbf{obtención y protección 
        de un dominio personalizado con certificados SSL/TLS de Let's Encrypt}. Esto implica instalar software de terceros como 
        \textbf{Certbot} y configurar reglas de seguridad de entrada para HTTPS.
        \item También se implementará una \textbf{funcionalidad segura de inicio de sesión de usuario} y el almacenamiento de detalles de 
        usuario en una base de datos integrada.
        \item La etapa 6 implementa una \textbf{forma segura para que los usuarios y dispositivos IoT se conecten al servidor IoT}. 
        También utilizara la funcionalidad de \textbf{administrador de acceso de PubNub} para que los usuarios administradores puedan 
        \textbf{otorgar acceso de lectura y escritura en tiempo real a usuarios no administradores y dispositivos}. Esto se visualizara en 
        un \textbf{tablero de control para administradores} que listara los usuarios en línea y ofrecera botones para conceder permisos.
    \end{itemize}
\end{itemize}

\section{Plataforma IoT en la Nube}
% 1.4.PLATAFORMA IoT EN LA NUBE.
Se describe la \textbf{Plataforma IoT en la Nube} como un componente esencial para construir una \textbf{solución integral de IoT}, 
que abarca desde la simplicidad a nivel de dispositivo hasta la complejidad de la infraestructura a nivel de nube. En el contexto del 
\textbf{intrucción del proyecto}, se aprenderá a desarrollar una \textbf{plataforma en la nube segura (secure cloud platform)} donde 
múltiples usuarios pueden iniciar sesión, controlar y monitorear sus dispositivos autorizados en tiempo real. 
El proyecto se equilibra entre el \textbf{desarrollo de dispositivos en Raspberry Pi y el desarrollo de servidores locales y remotos 
utilizando Python}. 
La Plataforma IoT en la Nube se construira y se utilizará a lo largo de varias etapas:
\begin{itemize}
    \item \textbf{Despliegue en la nube de AWS}:
    \begin{itemize}
        \item La etapa 4 se centra en la \textbf{reconstrucción del proyecto} inicial para utilizar PubNub como protocolo de 
        comunicación principal y se enseñara \textbf{cómo desplegar un servidor IoT en la nube de AWS (deploy IoT server into AWS cloud)}.
        \item La configuración de seguridad para este servidor en la nube es crucial. Por ejemplo, al asegurar un dominio 
        personalizado con certificados SSL/TLS, se destaca la necesidad de \textbf{asignar reglas de seguridad de entrada para HTTPS 
        en el servidor remoto de AWS} para permitir la conexión en el puerto 443.
    \end{itemize}

    \item \textbf{Protocolos de comunicación y seguridad en la nube}:
    \begin{itemize}
        \item Las etapas 3 y 4 abordan en profundidad los \textbf{protocolos de comunicación en tiempo real y ligeros} como MQTT 
        y WebSockets, y su implementación con \textbf{PubNub}. PubNub es descrito como un servicio de entrega sobre WebSockets, lo 
        que sugiere su rol como una capa de comunicación gestionada en la nube para IoT.
        \item La etapa 5 se dedica al \textbf{desarrollo de conceptos de seguridad} en la plataforma en la nube, incluyendo:
        \begin{itemize}
            \item La obtención y \textbf{protección de un dominio personalizado con certificados SSL/TLS de Let's Encrypt}. Esto 
            implica la instalación de software de terceros como Certbot y la configuración de las reglas de seguridad de entrada 
            para HTTPS en el servidor remoto.
            \item La implementación de una \textbf{funcionalidad segura de inicio de sesión de usuario} y el almacenamiento de los 
            detalles del usuario en una base de datos integrada.
        \end{itemize}
        \item La etapa 6 implementa una \textbf{forma segura para que los usuarios y dispositivos IoT se conecten al servidor IoT}. 
        También utiliza la funcionalidad de \textbf{administrador de acceso de PubNub (PubNub access manager functionality)} para 
        que los usuarios administradores puedan otorgar acceso de lectura y escritura en tiempo real a usuarios no administradores 
        y dispositivos. Esto se visualizara en un \textbf{tablero de control para administradores} que lista a los usuarios en línea 
        y permitira conceder permisos.
    \end{itemize}
\end{itemize}
El objetivo final es que, al finalizar el proyecto, los participantes sepan \textbf{lo que se necesita para construir su propia 
solución IoT integral}, abarcando desde la simplicidad a nivel de dispositivo hasta la complejidad de la infraestructura 
a nivel de nube.

\section{Enfoque Basado en la Resolución de Problemas}
% 1.5.ENFOQUE BASADO EN LA RESOLUCIÓN DE PROBLEMAS.
Se indica que en el proyecto  'IoT con Python y Raspberry Pi ' se adopta un \textbf{enfoque basado en la resolución de problemas}, 
particularmente centrándose en \textbf{problemas de IoT del mundo real}. Este enfoque se enmarca en una \textbf{guía práctica 
completa} destinada tanto a principiantes como a usuarios intermedios. En el contexto más amplio del \textbf{Resumen del Curso}, 
este enfoque se desarrolla de la siguiente manera:

\begin{itemize}
    \item \textbf{Comprensión y aplicación}: El proyecto sigue un \textbf{enfoque basado en la reconstrucción completa}. Esto significa 
    que ayuda a los usuarios del sistema a \textbf{entender el por qué antes del cómo y qué}. Esta metodología sugiere que se busca 
    una comprensión profunda de los desafíos antes de abordar su implementación técnica.

    \item \textbf{Proyectos prácticos para problemas del mundo real}:
    \begin{itemize}
        \item La etapa 2 del proyecto se enfoca en \textbf{ensuciarse las manos} con el desarrollo de un \textbf{proyecto simple de IoT 
        utilizando sensores que muestran datos en una aplicación web}.
        \item Un ejemplo concreto de esto es el \textbf{proyecto de detector de movimiento antirrobo}, que busca crear un 
        dispositivo de defensa contra robos que detecta movimiento, activa una alarma y envía alertas, además de permitir la desactivación 
        de la alarma por parte del usuario.
        \item La etapa 7 presenta un \textbf{último proyecto llamado sistema de monitoreo}. En este proyecto, se añaden más sensores y 
        actuadores acordes con lo que se quiera monitonear para \textbf{construir algo significativo para un caso de uso del mundo real}. 
        El producto final es un tablero de control que muestra gráficos visuales con datos de sensores y actuadores en tiempo real.
    \end{itemize}

    \item \textbf{Desarrollo de soluciones integrales}: El proyecto busca un equilibrio entre el desarrollo de dispositivos en Raspberry Pi 
    y el desarrollo de servidores locales y remotos usando Python. El objetivo final de este enfoque basado en la resolución de problemas es 
    que, al finalizar el proyecto, los participantes sepan \textbf{lo que se necesita para construir su propia solución IoT integral}, 
    abarcando desde la simplicidad a nivel de dispositivo hasta la complejidad de la infraestructura a nivel de nube. Esto incluye el 
    desarrollo de una plataforma en la nube segura donde múltiples usuarios pueden iniciar sesión, controlar y monitorear sus dispositivos 
    autorizados en tiempo real.
\end{itemize}
En resumen, el 'Enfoque Basado en Resolución de Problemas' es central en el proyecto, guiando a losdesarrolladores del sistema a través 
de la comprensión de los desafíos de IoT y proporcionándoles las herramientas y la experiencia práctica para construir soluciones funcionales 
y seguras para escenarios del mundo real.

\section{Tecnologías Claves}
% 1.6.Tecnologías Claves.
\subsection{Python}
\textbf{Python} \cite{python} es una de las \textbf{tecnologías principales} en el proyecto  'Internet de las 
Cosas con Python y Raspberry Pi'. Su rol es fundamental y se equilibra entre el desarrollo a nivel de dispositivo y el desarrollo de 
servidores. En el contexto más amplio de las \textbf{Tecnologías Clave} del proyecto, Python se utiliza para:

\begin{itemize}
    \item \textbf{Programación de Dispositivos IoT con Raspberry Pi}:
    \begin{itemize}
        \item Los desarrolladores aprenderán a escribir \textbf{código Python básico} para detectar señales de sensores (como el sensor 
        PIR para movimiento) y controlar actuadores (como el zumbador). Este código permite la detección de movimiento y la activación 
        de alarmas.
        \item También se utilizara Python para implementar la lógica de comunicación bidireccional, permitiendo a los usuarios interactuar 
        con el dispositivo, como desactivar una alarma.
    \end{itemize}

    \item \textbf{Desarrollo de Servidores Locales y Remotos}:
    \begin{itemize}
        \item Python se empleara para crear un \textbf{servidor web HTTP básico con Flask} en la Raspberry Pi, que se ejecutara en la 
        red Wi-Fi local.
        \item El proyecto mantiene un \textbf{equilibrio completo entre el desarrollo de dispositivos en Raspberry Pi y el desarrollo 
        de servidores locales y remotos usando Python}. Esto incluye el despliegue de un servidor IoT en la nube de AWS.
    \end{itemize}

    \item \textbf{Funcionalidades del Servidor IoT (con Flask)}:
    \begin{itemize}
        \item Dentro de la aplicación Flask, Python se utilizara para añadir funcionalidades esenciales como la provisión de detalles 
        adicionales como el ID de usuario y la lista de usuarios en línea a la página web.
        \item Se implementara lógica en Python para \textbf{poblar variables con registros de usuarios en línea}, incluyendo nombres, 
        IDs de usuario y estados de acceso de lectura y escritura.
        \item Python es fundamental para \textbf{crear \textit{endpoints}} en la aplicación Flask para recibir solicitudes, como las 
        de concesión de permisos.
        \item También se usara para manejar la \textbf{lógica de permisos de usuario}, almacenando permisos de lectura y escritura en la 
        base de datos y llamando al servidor de PubNub para conceder acceso específico a los usuarios.
        \item Se asegura que los paneles de control de acceso solo sean visibles para los usuarios administradores, utilizando la ID 
        de usuario para añadir sentencias condicionales en el código HTML a través de las plantillas Jinja, gestionadas por Python en 
        el servidor.
    \end{itemize}
\end{itemize}
En resumen, Python es una tecnología central que une tanto la programación del hardware (Raspberry Pi) como la creación de la 
infraestructura de software (servidores, lógica de negocio, seguridad y gestión de usuarios) que forman la solución integral 
de IoT del proyecto.

\subsection{Raspberry PI}
Se establece claramente que \textbf{Raspberry Pi} \cite{raspberrypi} es una de las \textbf{tecnologías principales} 
abordadas en el proyecto  'Internet de las Cosas con Python y Raspberry Pi '. Su papel es fundamental para el \textbf{desarrollo 
de dispositivos IoT} y para albergar servidores locales. En el contexto más amplio de las \textbf{Tecnologías Clave}, la Raspberry Pi 
se utilizara para:
\begin{itemize}
    \item \textbf{Plataforma de Hardware para Dispositivos IoT}:
    \begin{itemize}
        \item El proyecto mantiene un \textbf{equilibrio completo entre el desarrollo de dispositivos en Raspberry Pi y el desarrollo 
        de servidores locales y remotos usando Python}. Esto subraya su importancia como la plataforma física sobre la que se construyen 
        las soluciones IoT.
        \item En la etapa 2, los desarrolladores del proyecto 'se ensuciarán las manos' con un \textbf{proyecto simple de IoT} utilizando 
        sensores y mostrando datos en una aplicación web, con la Raspberry Pi como el cerebro del dispositivo.
    \end{itemize}

    \item \textbf{Implementación de Proyectos Prácticos}:
    \begin{itemize}
        \item Para el \textbf{proyecto de detector de movimiento antirrobo}, la Raspberry Pi es el dispositivo central. Se conecta con el 
        \textbf{sensor PIR (Passive Infrared)} para detectar movimiento y un \textbf{zumbador (buzzer)} como actuador para activar una alarma.
        \item Los desarrolladores escribirán código Python para detectar señales y controlar el zumbador, así como para añadir funcionalidad 
        de comunicación bidireccional, permitiendo a los usuarios desactivar la alarma.
        \item En la etapa 7, la Raspberry Pi continuará siendo la plataforma para el \textbf{sistema de monitoreo atmosférico}, 
        un proyecto que implica la adición de más sensores y actuadores para construir algo significativo para un caso de uso del mundo real.
    \end{itemize}

    \item \textbf{Alojamiento de Servidores Locales}:
    \begin{itemize}
        \item La Raspberry Pi se utilizara para ejecutar un \textbf{servidor web HTTP básico con Flask} en la red Wi-Fi local. Esto 
        permitira que los usuarios, conectados a la misma red, accedan a una página web desde sus navegadores y reciban actualizaciones 
        en vivo del sensor.
        \item También se añadira un botón en la página web para que los usuarios puedan controlar los actuadores, como desactivar el zumbador.
    \end{itemize}

    \item \textbf{Configuración y Preparación del Entorno}:
    \begin{itemize}
        \item Para empezar, se sugiere a los desarrolladores configurar sus Raspberry Pi siguiendo las instrucciones oficiales, lo que 
        incluye el uso de una tarjeta SD de al menos 8GB para el sistema operativo.
        \item Se menciona la opción de utilizar un monitor HDMI o realizar una \textbf{conexión de escritorio remoto} a la Raspberry Pi.
        \item Para la Raspberry Pi 3, se enfatiza la necesidad de \textbf{habilitar SSH} a través de la terminal usando \verb|sudo raspi_config|.
    \end{itemize}
\end{itemize}
En resumen, la Raspberry Pi es la \textbf{piedra angular del hardware} en este proyecto, permitiendo a los estudiantes interactuar 
directamente con el mundo físico a través de sensores y actuadores, y también sirve como una plataforma de bajo costo para desarrollar 
y probar servidores IoT a nivel local, antes de pasar a despliegues en la nube.

\subsection{FLASK}
Se indica que \textbf{Flask} \cite{flask} es una de las \textbf{tecnologías principales} utilizadas en el proyecto 
'Internet de las Cosas con Python y Raspberry Pi'. Su rol es crucial para el \textbf{desarrollo de servidores IoT}, tanto a nivel local 
como remoto. 
En un contexto más amplio de las \textbf{Tecnologías Clave} del proyecto, Flask se empleara para:

\begin{itemize}
    \item \textbf{Desarrollo de Servidores Web HTTP Locales}:
    \begin{itemize}
        \item Los desarrolladores aprenderán a escribir un \textbf{servidor web HTTP básico en Python Flask en Raspberry Pi}. Este servidor 
        se ejecutará en la red Wi-Fi local, permitiendo que los usuarios de la misma red accedan a una página web y reciban actualizaciones 
        en vivo de los sensores.
        \item También se añadira un botón en la página web para que los usuarios puedan controlar los actuadores, como desactivar un zumbador, 
        comunicándose con el servidor Flask.
    \end{itemize}
    
    \item \textbf{Desarrollo de Servidores Remotos y en la Nube}:
    \begin{itemize}
        \item El proyecto mantiene un \textbf{equilibrio completo entre el desarrollo de dispositivos en Raspberry Pi y el desarrollo de 
        servidores locales y remotos usando Python}, donde Flask juega un papel clave en la parte del servidor. Esto incluye el despliegue 
        de servidores IoT en la nube de AWS.
    \end{itemize}

    \item \textbf{Implementación de la Lógica del Servidor IoT}:
    \begin{itemize}
        \item \textbf{Provisión de Detalles al Cliente}: Flask se utilizara para pasar detalles adicionales como el ID de usuario y la 
        lista de usuarios en línea a la página web del cliente.
        \item \textbf{Población de Datos de Usuarios en Línea}: Se implementa lógica en Flask (usando Python) para crear un mapa 
        \verb|online_user_records| que contenga el nombre del usuario, el ID de usuario y los estados de permisos de lectura y escritura 
        (representados como 'checked' o 'unchecked' para el HTML). Este mapa es luego enviado a las plantillas Jinja en \verb|index.html| 
        para poblar dinámicamente la lista de usuarios en línea.
        \item \textbf{Gestión de Permisos de Usuario}:
        \begin{itemize}
            \item Flask recibira solicitudes de 'concesión' de permisos (grant) desde el código JavaScript del cliente a través de 
            \textit{endpoints} específicos.
            \item La aplicación Flask verifica si la solicitud proviene de un usuario administrador antes de procesar la solicitud.
            \item Se utilizara Flask para \textbf{almacenar los permisos de lectura y escritura del usuario en la base de datos y luego 
            llamar al servidor de PubNub para conceder acceso de lectura y escritura específico a este usuario}.
        \end{itemize}
        \item \textbf{Control de Visibilidad del Panel de Acceso}: La aplicación Flask, al enviar el ID de usuario al cliente, permite 
        que las plantillas Jinja utilicen sentencias condicionales (\verb|if|) para que el panel de control de acceso solo sea visible 
        para los usuarios administradores.
    \end{itemize}
\end{itemize}
En síntesis, Flask es una \textbf{pieza fundamental en la arquitectura del servidor IoT} presentada en el proyecto, facilitando la 
comunicación entre los dispositivos, los usuarios y la infraestructura de la nube. Permite la creación de la interfaz web, la gestión de 
datos de usuarios, la implementación de la lógica de permisos y la interacción bidireccional, contribuyendo a la construcción de una 
\textbf{solución IoT segura y escalable}.

\subsection{AWS}
\textbf{AWS (Amazon Web Services)} \cite{AWS} es reconocida como una de las \textbf{tecnologías principales} que se abordarán en el 
proyecto 'Internet de las Cosas con Python y Raspberry Pi'. Su rol es fundamental en el contexto más amplio de las Tecnologías Clave 
del proyecto, particularmente en el \textbf{desarrollo y despliegue de la infraestructura IoT en la nube}. En este contexto, AWS 
se utilizara para:
\begin{itemize}
    \item \textbf{Plataforma de Despliegue en la Nube}:
    \begin{itemize}
        \item El proyecto enseña a \textbf{desplegar el servidor IoT en la nube de AWS}. Esto complementa el 
        desarrollo de dispositivos en Raspberry Pi y servidores locales, logrando un \textbf{ 'equilibrio completo '} entre ambos.
        \item El objetivo es construir una \textbf{plataforma en la nube \textit{serverless}} en la que múltiples usuarios 
        puedan iniciar sesión de forma segura y controlar y monitorear sus dispositivos autorizados en tiempo real.
        \item El producto final del proyecto incluirá una \textbf{infraestructura a nivel de la nube}.
    \end{itemize}

    \item \textbf{Alojamiento del Servidor Remoto}:
    \begin{itemize}
        \item AWS es la plataforma donde reside el \textbf{servidor remoto} para la aplicación IoT. Se menciona específicamente 
        una \textbf{instancia EC2}, lo que implica el uso de servicios de cómputo virtual de AWS para alojar el servidor.
    \end{itemize}

    \item \textbf{Configuración de Seguridad en la Nube}:
    \begin{itemize}
        \item Para asegurar el servidor alojado, se gestionanran las \textbf{reglas de seguridad de entrada (inbound security rules)} 
        en AWS. Específicamente, se deben \textbf{asignar reglas de seguridad de entrada para HTTPS} en la configuración de la instancia 
        EC2 para permitir el tráfico seguro a través del puerto 443. Esto es crucial para la implementación de SSL/TLS con certificados 
        Let's Encrypt.
    \end{itemize}
\end{itemize}
En resumen, AWS es la \textbf{columna vertebral de la infraestructura en la nube} para la solución IoT del proyecto, permitiendo el 
despliegue de servidores remotos escalables y seguros, y facilitando la gestión de la conectividad y la seguridad a nivel de la nube 
para los dispositivos y usuarios.

\subsection{PubNub}
\textbf{PubNub} \cite{pubnub} es explícitamente una de las \textbf{tecnologías principales} que se abordarán en el 
proyecto  'Internet de las Cosas con Python y Raspberry Pi '. Su papel es fundamental en el contexto más amplio de las Tecnologías 
Clave, especialmente en la \textbf{gestión de la comunicación en tiempo real y la seguridad de acceso} en entornos IoT. En el contexto 
de las Tecnologías Clave, PubNub se utiliza para:
\begin{itemize}
    \item \textbf{Protocolo de Comunicación en Tiempo Real y Ligero}:
    \begin{itemize}
        \item En el proyecto se estudiará en profundidad los protocolos de comunicación en tiempo real y ligeros para Internet de las Cosas, 
        incluyendo MQTT, WebSockets y, finalmente, realizará una \textbf{demostración práctica con PubNub}.
        \item Es seleccionado como el \textbf{principal protocolo de comunicación} para un proyecto reconstruido en la etapa 4, reemplazando 
        a la técnica de \textit{long polling} con AJAX. Esto subraya su importancia para una comunicación eficiente y reactiva en la 
        aplicación IoT.
    \end{itemize}

    \item \textbf{Gestión de Acceso y Permisos de Usuario}:
    \begin{itemize}
        \item PubNub se utilizara para implementar una funcionalidad clave: el \textbf{administrador de acceso de PubNub}. Esta característica 
        permitira a los usuarios administradores \textbf{otorgar acceso de lectura y escritura en tiempo real} a todos los usuarios no 
        administradores y a los dispositivos.
        \item La lógica de la aplicación Flask interactúa directamente con PubNub. Después de almacenar los permisos de lectura y escritura en 
        la base de datos, la aplicación \textbf{llama al servidor de PubNub para conceder acceso de lectura y escritura específico a este 
        usuario}.
        \item La concesión de permisos de lectura y escritura a través de PubNub es el segundo paso de un proceso, siendo el primero la 
        generación de una clave de autorización para el usuario específico y su almacenamiento en la base de datos.
    \end{itemize}

    \item \textbf{Seguridad y Control en el Ecosistema IoT}:
    \begin{itemize}
        \item Al permitir el otorgamiento de permisos de acceso en tiempo real, PubNub contribuye significativamente a un 
        \textbf{ecosistema IoT fuerte, seguro, en tiempo real y escalable}. Esto es vital para asegurar que solo los usuarios y 
        dispositivos autorizados puedan controlar y monitorear otros dispositivos.
    \end{itemize}
\end{itemize}
En síntesis, PubNub es una \textbf{tecnología crucial} que no solo facilita la comunicación en tiempo real de baja latencia en el 
sistema IoT, sino que también es instrumental en la implementación de un \textbf{modelo de seguridad robusto} mediante la gestión dinámica 
de permisos de acceso, lo que es esencial para construir una \textbf{plataforma IoT multiusuario y segura}.

\chapter{ETAPAS DEL PROYECTO}
% 2.SECCIONES DEL CURSO
Se describe el proyecto \textbf{ 'Internet de las Cosas con Python y Raspberry Pi '} como una \textbf{ 'guía práctica completa ' (complete 
hands-on guide)} dirigida tanto a \textbf{principiantes como a usuarios intermedios}. El proyecto adopta un \textbf{ 'enfoque completamente basado en 
proyectos ' (complete project-based approach)} para ayudar a los estudiantes a \textbf{ 'entender el porqué antes del cómo y el qué ' (understand the why 
before how and what)}. Su objetivo principal es construir una \textbf{ 'plataforma en la nube \textit{serverless} en la que múltiples usuarios pueden 
iniciar sesión de forma segura y controlar y monitorear sus dispositivos autorizados en tiempo real ' (serverless cloud platform in which multiple users 
can securely log in and control and monitor their authorized devices in real time)}. Las \textbf{Secciones del Curso} están estructuradas de manera 
progresiva, llevando a los estudiantes desde los fundamentos teóricos hasta la implementación de una solución IoT completa, segura y escalable:

\begin{itemize}
    \item \textbf{Sección 1: Introducción a IoT}:
    \begin{itemize}
        \item Esta sección inicial se enfoca en establecer una \textbf{comprensión fundamental del Internet de las Cosas}.
        \item Los estudiantes aprenderán \textbf{ 'qué es el Internet de las Cosas y sus principales componentes ' (understanding what is Internet of 
        Things and its major components)}, como \textbf{dispositivos inteligentes, sensores y actuadores}.
        \item También se familiarizarán con \textbf{ 'cuatro importantes modelos de comunicación del Internet de las Cosas ' (four important Internet 
        of Things communication models)}.
    \end{itemize}

    \item \textbf{Sección 2: Primer Proyecto IoT (Detector de Movimiento Antirrobo)}:
    \begin{itemize}
        \item Aquí es donde los estudiantes \textbf{ 'se ensuciarán las manos ' (get our hands dirty)} con el desarrollo de un proyecto IoT simple.
        \item El proyecto consiste en un \textbf{ 'dispositivo IoT de defensa contra robos ' (theft defensive IOT device)} que detecta movimiento, 
        activa una alarma y envía alertas. También permite la desactivación de la alarma, logrando una \textbf{comunicación bidireccional}.
        \item Se utilizarán un \textbf{sensor PIR (detector de movimiento) y un \textit{buzzer}}.
        \item La comunicación entre el cliente y el servidor se realizará inicialmente con la \textbf{técnica AJAX}. Se montará un \textbf{servidor web 
        HTTP Flask básico en Raspberry Pi} para operar dentro de la red local.
    \end{itemize}

    \item \textbf{Sección 3: Seguridad y Protocolos de Comunicación}:
    \begin{itemize}
        \item Después de analizar las ventajas y desventajas del proyecto anterior, esta sección profundiza en \textbf{protocolos de comunicación en 
        tiempo real y ligeros} para IoT, como \textbf{WebSockets y MQTT}. Se realizará una \textbf{ 'demostración práctica con PubNub ' (hands-on demo 
        with PubNub)}.
        \item Se abordará la \textbf{seguridad en Internet y criptografía}, incluyendo \textbf{SSL/TLS y protocolos HTTP}. WebSockets permiten una 
        sesión de comunicación interactiva y bidireccional con baja latencia, mientras que MQTT es un protocolo de mensajería ligero para M2M e IoT, 
        basado en un modelo de publicación/suscripción y que puede correr sobre WebSockets.
    \end{itemize}

    \item \textbf{Sección 4: Reconstrucción del Proyecto con PubNub y AWS}:
    \begin{itemize}
        \item El proyecto inicial de la Sección 2 se \textbf{ 'reconstruirá ' (reconstruct our mas project)} para usar \textbf{PubNub como el 
        'principal protocolo de comunicación ' (major communication protocol)} en lugar de AJAX \textit{long polling}.
        \item También se aprenderá a \textbf{ 'desplegar el servidor IoT en la nube de AWS ' (deploy IOT server into a Tobias cloud)}.
    \end{itemize}

    \item \textbf{Sección 5: Implementación de Seguridad (Servidor IoT y Login de Usuario)}:
    \begin{itemize}
        \item Esta sección se dedica a la \textbf{implementación de las terminologías de seguridad} estudiadas previamente.
        \item Se obtiene un \textbf{nombre de dominio personalizado y se asegura con certificados SSL/TLS Let's Encrypt}. Este proceso implica instalar 
        \textit{Certbot} y configurar Apache para redirigir el tráfico HTTP a HTTPS.
        \item Se implementa una \textbf{funcionalidad de inicio de sesión de usuario segura} y se almacenan los detalles del usuario en una base de datos 
        integrada.
        \item Es crucial \textbf{asignar reglas de seguridad de entrada para HTTPS (puerto 443) en el servidor remoto de AWS} para permitir la conexión 
        segura. Una vez configurado, el servidor redirige de HTTP a HTTPS, mostrando un icono de candado verde y garantizando una \textbf{comunicación 
        cifrada de extremo a extremo}.
    \end{itemize}

    \item \textbf{Sección 6: Conexión Segura de Usuarios y Dispositivos}:
    \begin{itemize}
        \item Se implementa una forma segura para que \textbf{usuarios y dispositivos IoT se conecten al servidor}.
        \item Se utiliza la \textbf{ 'funcionalidad de administrador de acceso de PubNub ' (PubNub access manager functionality)} para que los usuarios 
        administradores puedan \textbf{ 'otorgar acceso de lectura y escritura en tiempo real ' (grant real time with read and write access)} a usuarios 
        no administradores y dispositivos.
        \item Esto incluye desarrollar un panel de administración para listar usuarios en línea y botones para conceder o revocar permisos, asegurando 
        que solo los usuarios administradores puedan ver y utilizar este panel. La funcionalidad se gestiona enviando solicitudes desde el \textit{frontend} 
        (JavaScript) a la aplicación Flask del servidor.
    \end{itemize}

    \item \textbf{Sección 7: Proyecto Final: Sistema de Monitoreo Atmosférico}:
    \begin{itemize}
        \item Esta sección final se centra en un \textbf{ 'sistema de monitoreo atmosférico ' (atmospheric monitoring system)} más complejo, añadiendo \
        textbf{más sensores y actuadores}.
        \item Los estudiantes se familiarizarán con \textbf{convertidores digitales, interfaz periférica serial (SPI)} y otros conceptos avanzados.
        \item El producto final es un \textbf{ 'dashboard mostrando gráficos visuales ' (dashboard displaying visual charts)} con datos de sensores y 
        actuadores en tiempo real, lo que demuestra la infraestructura completa de IoT a nivel de la nube.
    \end{itemize}
\end{itemize}
Al finalizar el proyecto, los estudiantes \textbf{ 'entenderán lo que se necesita para construir su propia solución IoT integral ' (get to know what it 
takes to build your own one-stop IOT solution)}, abarcando desde la simplicidad a nivel de dispositivo hasta la complejidad de la infraestructura a nivel 
de la nube. Podrán expandir el proyecto añadiendo más dispositivos y funcionalidades según sus necesidades.

\section{Introducción a IoT}
% 2.1.INTRODUCCIÓN a IoT
La \textbf{Sección 1: Introducción a IoT} es el punto de partida fundamental del proyecto  'Internet de las Cosas con Python y Raspberry Pi ', diseñada 
para establecer una base sólida antes de que los estudiantes se sumerjan en proyectos prácticos. En el contexto más amplio de las Secciones del Curso, 
esta primera sección es crucial porque introduce los conceptos teóricos esenciales que sustentan todo el desarrollo posterior. 
Los objetivos y contenidos clave de la Sección 1 incluyen:

\begin{itemize}
    \item \textbf{Comprender qué es el Internet de las Cosas (IoT)}.
    \item Familiarizarse con los \textbf{principales componentes del IoT}, como:
    \begin{itemize}
        \item \textbf{Dispositivos inteligentes}.
        \item \textbf{Sensores}.
        \item \textbf{Actuadores}.
    \end{itemize}
    \item Conocer \textbf{cuatro importantes modelos de comunicación del Internet de las Cosas}.
\end{itemize}
Esta sección es vital para cumplir el enfoque del proyecto de ayudar a los estudiantes a \textbf{ 'entender el porqué antes del cómo y el qué ' 
(understand the why before how and what)}. Al establecer una comprensión clara de los fundamentos y los modelos de comunicación en la Sección 1, 
los estudiantes estarán mejor equipados para abordar los proyectos y las tecnologías más avanzadas que se presentarán en las secciones posteriores, como 
el desarrollo de un detector de movimiento antirrobo (Sección 2), protocolos de comunicación en tiempo real (Sección 3), seguridad (Sección 5) y la 
construcción de una plataforma en la nube \textit{serverless} (objetivo principal del proyecto).

\subsection{Que es IoT}
En el contexto más amplio de la \textbf{Sección 1: Introducción a IoT}, se  indica que el objetivo primordial de esta sección es que los 
estudiantes logren una \textbf{comprensión fundamental de  '¿Qué es el Internet de las Cosas (IoT)? '}. Esta es la piedra angular sobre la que se construirá 
todo el conocimiento y las habilidades prácticas a lo largo del proyecto. La importancia de abordar \textbf{ '¿Qué es IoT? '} al inicio radica en el enfoque 
pedagógico del proyecto, que busca ayudar a los estudiantes a \textbf{ 'entender el porqué antes del cómo y el qué ' (understand the why before how and what)}. 
Esto significa que, antes de sumergirse en la implementación de proyectos o el uso de tecnologías específicas, es esencial comprender la naturaleza y el 
propósito general del Internet de las Cosas. Para lograr esta comprensión, la Sección 1 no solo define el concepto, sino que también introduce los elementos 
que lo constituyen y cómo interactúan:

\begin{itemize}
    \item \textbf{Componentes Mayores}: Los estudiantes se familiarizarán con los \textbf{ 'principales componentes ' (major components)} del IoT, que 
    incluyen \textbf{dispositivos inteligentes, sensores y actuadores}. Comprender estos elementos es fundamental para entender cómo los sistemas IoT 
    recogen datos del entorno (sensores), toman decisiones y actúan sobre él (actuadores), y cómo se integran en una red más amplia (dispositivos inteligentes).
    \item \textbf{Modelos de Comunicación}: La sección también aborda \textbf{ 'cuatro importantes modelos de comunicación del Internet de las Cosas ' 
    (four important Internet of Things communication models)}. Esto es crucial para entender cómo los diversos componentes del IoT se conectan e intercambian 
    información, sentando las bases para discusiones más profundas sobre protocolos y seguridad en secciones posteriores.
\end{itemize}
En resumen, la pregunta \textbf{ '¿Qué es IoT? '} es el eje central de la Sección 1, que proporciona la base teórica indispensable para que los estudiantes 
puedan abordar con éxito los desafíos de desarrollo práctico y la construcción de una plataforma IoT completa, segura y escalable que se presenta en el 
resto del proyecto.

\subsection{Componentes Mayores (Dispositivos, Sensores, Actuadores)}
En el contexto más amplio de la \textbf{Sección 1: Introducción a IoT}, se  establece que uno de los objetivos principales es que los estudiantes 
comprendan los \textbf{Componentes Mayores} del Internet de las Cosas. Esta sección es fundamental porque sienta las bases teóricas necesarias para entender 
el \textbf{ 'porqué antes del cómo y el qué ' (understand the why before how and what)}. Específicamente, la Sección 1 se encarga de:

\begin{itemize}
    \item \textbf{Entender qué es el Internet de las Cosas} y sus elementos clave.
    \item Familiarizar a los estudiantes con los \textbf{ 'principales componentes ' (major components)} del IoT. Estos componentes se desglosan en:
    \begin{itemize}
        \item \textbf{Dispositivos inteligentes (smart devices)}.
        \item \textbf{Sensores}.
        \item \textbf{Actuadores}.
    \end{itemize}
\end{itemize}
La introducción de estos componentes en la Sección 1 es crucial para el desarrollo del proyecto. Permite a los estudiantes construir un conocimiento 
fundamental que luego aplicarán en las secciones subsiguientes. Por ejemplo, en la Sección 2, se  'ensuciarán las manos ' al desarrollar un detector de 
movimiento antirrobo, utilizando específicamente un \textbf{sensor PIR (detector de movimiento)} y un \textbf{\textit{buzzer}} (actuador). La comprensión 
de qué son los sensores y actuadores, proporcionada en la Sección 1, es un prerrequisito para comprender su función y manipulación en proyectos prácticos. 
De manera similar, el proyecto final en la Sección 7, un sistema de monitoreo atmosférico, involucrará la adición de  'más \textbf{sensores y actuadores} ', 
y la comprensión de estos componentes desde la introducción es clave para entender la complejidad y funcionalidad de dicho sistema. Además de los componentes, 
la Sección 1 también cubre \textbf{ 'cuatro importantes modelos de comunicación del Internet de las Cosas ' (four important Internet of Things communication 
models)}. Esto, junto con la comprensión de dispositivos, sensores y actuadores, proporciona una visión holística inicial de cómo interactúan los 
elementos en una solución IoT completa, preparando a los estudiantes para los desafíos de comunicación y seguridad que se abordarán en secciones 
posteriores como la Sección 3 (protocolos de comunicación) y la Sección 5 (seguridad).

\subsection{Modelos de Comunicación IoT}
En el contexto más amplio de la \textbf{Sección 1: Introducción a IoT}, se  indican claramente que uno de los objetivos clave es que los estudiantes 
se familiaricen con los \textbf{ 'cuatro importantes modelos de comunicación del Internet de las Cosas ' (four important Internet of Things communication 
models)}. Esta inclusión en la primera sección del proyecto es fundamental y se alinea con el enfoque pedagógico de ayudar a los estudiantes a 
\textbf{'entender el porqué antes del cómo y el qué ' (understand the why before how and what)}. Antes de sumergirse en la implementación de protocolos 
específicos o en la construcción de sistemas, la Sección 1 establece una comprensión conceptual de cómo interactúan los diferentes componentes en el 
ecosistema IoT. Al presentar estos cuatro modelos de comunicación al principio, el proyecto busca:

\begin{itemize}
    \item \textbf{Establecer una Base Teórica}: Proporcionar a los estudiantes el conocimiento fundamental sobre las diversas formas en que los dispositivos, 
    sensores y actuadores de IoT pueden intercambiar información.
    \item \textbf{Preparar para Temas Avanzados}: Sentar las bases para las discusiones más detalladas sobre protocolos de comunicación específicos que se 
    abordarán en secciones posteriores del proyecto. Por ejemplo, en la Sección 3, se estudiarán en profundidad protocolos de comunicación en tiempo real y 
    ligeros, como \textbf{WebSockets y MQTT}, que son fundamentales para el IoT. WebSockets permite sesiones de comunicación interactivas y bidireccionales, 
    mientras que MQTT es un protocolo de mensajería ligero para la comunicación máquina a máquina, a menudo funcionando sobre WebSockets.
\end{itemize}
Por lo tanto, la discusión sobre los  'cuatro modelos de comunicación IoT ' en la Sección 1 es crucial para construir una comprensión integral del 
funcionamiento del Internet de las Cosas, permitiendo a los estudiantes contextualizar y aplicar el conocimiento de protocolos más específicos y avanzados 
en proyectos prácticos a lo largo del proyecto.

\section{Primer Proyecto IoT}
% 2.2.PRIMER PROYECTO IoT
La \textbf{Sección 2: Primer Proyecto IoT} es una etapa fundamental en el proyecto  'Internet de las Cosas con Python y Raspberry Pi ', ya que marca el 
momento en que los estudiantes pasan de la teoría a la práctica, construyendo su primer sistema IoT funcional. En el contexto más amplio de las Secciones 
del Curso, esta sección es crucial por varias razones:

\begin{itemize}
    \item \textbf{Aplicación de Conceptos Fundamentales}: La Sección 2 capitaliza los conocimientos adquiridos en la Sección 1 ( 'Introducción a IoT '), 
    donde se establecieron los fundamentos del IoT, sus componentes principales (dispositivos, \textbf{sensores y actuadores}) y los modelos de comunicación. 
    Aquí, los estudiantes  'se ensucian las manos ' aplicando directamente estos conceptos al trabajar con un \textbf{sensor PIR} (un tipo de sensor de 
    movimiento) y un \textbf{\textit{buzzer}} (un actuador).
    
    \item \textbf{Primer Proyecto Práctico}: El proyecto principal de la Sección 2 es un \textbf{detector de movimiento antirrobo}. Este proyecto abarca:
    \begin{itemize}
        \item \textbf{Diagrama de circuito de hardware}.
        \item \textbf{Código Python} para interactuar con los sensores y el servidor HTTP.
        \item La detección de movimiento que activa una alarma y envía alertas al usuario, demostrando una \textbf{comunicación unidireccional} del servidor 
        al usuario.
        \item La adición de funcionalidad para que los usuarios desactiven la alarma, estableciendo una \textbf{comunicación bidireccional} del usuario al 
        servidor.
    \end{itemize}
    
    \item \textbf{Detalles de Hardware y Programación Básica}:
    \begin{itemize}
        \item Se introduce el \textbf{sensor PIR}, que detecta el movimiento al percibir la energía térmica (radiaciones infrarrojas) emitida por humanos 
        o animales. Se explican sus pines (Tierra, VCC para 5V, Salida de nivel lógico alto si se detecta objeto) y los potenciómetros para ajustar la 
        sensibilidad y el tiempo de retardo. También se detallan los modos de disparo repetible y no repetible.
        \item Se presenta el \textbf{\textit{buzzer}}, que produce sonido al vibrar un disco metálico cuando se le aplica corriente, y se explica cómo 
        controlarlo mediante la generación de una onda cuadrada con Python.
        \item Se enseña a conectar el sensor PIR y el \textbf{\textit{buzzer}} a la Raspberry Pi y a escribir código Python para detectar señales y 
        controlar el actuador.
    \end{itemize}
    
    \item \textbf{Servidor Web Local y Comunicación AJAX}:
    \begin{itemize}
        \item Se desarrolla un servidor web HTTP básico con \textbf{Python Flask} en la Raspberry Pi, que se ejecuta en la red Wi-Fi local.
        \item Los usuarios acceden a una página web desde sus navegadores, donde se muestran el estado de detección de movimiento y el estado de conexión.
        \item Se implementa una  'solicitud \textit{keepalive} ' (latido) que el navegador envía periódicamente (cada cinco segundos) al servidor 
        de la Raspberry Pi para mantener la conexión activa y recibir actualizaciones en vivo del sensor.
        \item La comunicación entre el cliente y el servidor se realiza utilizando la \textbf{técnica AJAX}.
        \item Es importante destacar que este servidor solo es accesible localmente, dentro de la misma red Wi-Fi.
    \end{itemize}

    \item \textbf{Preparación para Secciones Futuras}: La Sección 2 sirve como un trampolín. Aunque es un proyecto funcional, se  indican que se 
    discutirán las \textbf{ventajas y desventajas} de la forma en que se desarrolló, especialmente en lo que respecta a la comunicación. Esto prepara el 
    terreno para la \textbf{Sección 3}, donde se estudiarán a fondo \textbf{protocolos de comunicación en tiempo real y ligeros}, como \textbf{MQTT y 
    WebSockets}, y se abordará la seguridad en Internet. De esta manera, el proyecto progresa desde una implementación básica y local hacia soluciones 
    más robustas, escalables y seguras en la nube.
    
    \item \textbf{Requisitos Previos}: Se aconseja a los estudiantes configurar sus Raspberry Pis, incluyendo la instalación del sistema operativo en una 
    tarjeta SD de al menos 8GB, y habilitar SSH si se desea acceso remoto.
\end{itemize}
En resumen, la Sección 2 es el primer contacto práctico y guiado con la construcción de un sistema IoT, permitiendo a los estudiantes consolidar los 
conocimientos teóricos de la Sección 1 y prepararse para las complejidades de la comunicación, seguridad y despliegue en la nube que se abordarán en 
las secciones posteriores del proyecto.

\subsection{Detector de Movimiento Antirobo}
En el contexto más amplio de la \textbf{Sección 2: Primer Proyecto IoT}, se  detallan el \textbf{Detector de Movimiento Antirrobo} como el proyecto 
central y fundamental de esta etapa del proyecto. Este proyecto sirve como la primera incursión práctica de los estudiantes en la construcción de un sistema 
IoT funcional, aplicando los conceptos teóricos aprendidos en la Sección 1. Aquí se desglosa lo que se  dice sobre este proyecto:

\begin{itemize}
    \item \textbf{Objetivo y Funcionalidad General}:
    \begin{itemize}
        \item El propósito principal es construir un \textbf{dispositivo IoT defensivo contra robos} que detecte movimiento, active una alarma y envíe alertas 
        al usuario.
        \item Inicialmente, establece una \textbf{comunicación unidireccional} del servidor al usuario para enviar alertas.
        \item Para lograr una \textbf{comunicación bidireccional}, se añade una funcionalidad que permite a los usuarios desactivar la alarma desde el 
        servidor.
    \end{itemize}
    
    \item \textbf{Componentes de Hardware}:
    \begin{itemize}
        \item \textbf{Sensor PIR (Detector de Movimiento)}: Utiliza un elemento sensor piezoeléctrico (RE200BL) que genera energía al exponerse al calor. 
        Detecta el movimiento de humanos o animales por las radiaciones infrarrojas que emiten. Es un sensor pasivo, lo que significa que no emite energía 
        para la detección, sino que detecta la energía emitida por otros objetos.
        \item Incluye una tapa de plástico para ampliar la cobertura del área de detección.
        \item Tiene tres pines: \textbf{Tierra}, \textbf{VCC (para 5V)} y un \textbf{pin de salida} que proporciona un nivel lógico alto si se detecta un 
        objeto.
        \item Posee dos potenciómetros: uno para ajustar la \textbf{sensibilidad} del sensor y otro para ajustar el \textbf{tiempo} durante el cual la señal 
        de entrada permanece en alto tras la detección (de 0.3 segundos a 5 minutos).
        \item Dispone de pines para seleccionar modos de disparo: \textbf{disparo no repetible} (la salida vuelve a bajo después del tiempo de retardo) y 
        \textbf{disparo repetible} (la salida permanece en alto mientras el objeto detectado esté presente).
        \item \textbf{\textit{Buzzer}}: Compuesto por una carcasa con tres pines (\textbf{VCC 5V, Tierra y Señal}) y un elemento piezoeléctrico con un 
        disco metálico que vibra al aplicar corriente, produciendo sonido. Se controla generando una \textbf{onda cuadrada} con Python, alternando la señal 
        entre alto y bajo para generar el sonido.
    \end{itemize}
    
    \item \textbf{Implementación Técnica y Software}:
    \begin{itemize}
        \item Los estudiantes conectarán el \textbf{sensor PIR y el \textit{buzzer}} a la Raspberry Pi.
        \item Se escribirá \textbf{código Python} para detectar las señales del sensor y controlar el \textit{buzzer}.
        \item Se desarrollará un \textbf{servidor web HTTP básico} utilizando \textbf{Python Flask} en la Raspberry Pi. Este servidor se ejecutará en la 
        \textbf{red Wi-Fi local}.
        \item Los usuarios accederán a una página web desde sus navegadores (dentro de la misma red Wi-Fi local).
        \item Una vez cargada la página, el navegador del usuario enviará una \textbf{ 'solicitud \textit{keepalive} '} (latido) al servidor de la 
        Raspberry Pi periódicamente (cada cinco segundos). Esto actúa como un latido para asegurar que la conexión con el servidor sigue activa.
        \item En cada respuesta del \textit{keepalive}, el servidor enviará el estado del sensor y los datos al usuario, proporcionando 
        \textbf{actualizaciones en vivo}.
        \item La comunicación entre el cliente y el servidor se realiza utilizando la \textbf{técnica AJAX}.
        \item Se agregará un botón en la página web para que los usuarios puedan \textbf{controlar los actuadores}, específicamente para desactivar 
        la alarma (el \textit{buzzer}).
        \item La página web mostrará el estado de detección de movimiento y el estado de la conexión.
    \end{itemize}
    
    \item \textbf{Preparación y Requisitos Previos}:
    \begin{itemize}
        \item Se aconseja a los estudiantes configurar sus Raspberry Pis, incluyendo la instalación del sistema operativo en una tarjeta SD de al menos 
        8 GB y la habilitación de SSH para acceso remoto.
    \end{itemize}

    \item \textbf{Importancia en el Contexto del Curso}:
    \begin{itemize}
        \item La Sección 2, con el proyecto del Detector de Movimiento Antirrobo, es el \textbf{primer paso práctico} del proyecto, donde los estudiantes  
        'se ensucian las manos'.
        \item Posteriormente, en la Sección 3, se discutirán las \textbf{ventajas y desventajas} de la forma en que se desarrolló la comunicación en este 
        proyecto. Esto servirá como base para estudiar en profundidad \textbf{protocolos de comunicación en tiempo real y ligeros} como MQTT y WebSockets, 
        así como temas de seguridad en Internet. De esta manera, el proyecto sienta las bases para soluciones más avanzadas y seguras.
    \end{itemize}
\end{itemize}

\subsection{Sensores (PIR)}
En el contexto más amplio de la \textbf{Sección 2: Primer Proyecto IoT}, el \textbf{Sensor PIR (Passive Infrared Sensor)} es un componente fundamental y 
el dispositivo principal de entrada para el \textbf{Detector de Movimiento Antirrobo}. Esta sección se centra en que los estudiantes se familiaricen con 
el hardware y la programación básica, y el sensor PIR juega un papel crucial en este aprendizaje práctico. A continuación, se dan los detalles de el sensor 
PIR:
\begin{itemize}
    \item \textbf{Nombre y Principio de Funcionamiento}
    \begin{itemize}
        \item El PIR es un \textbf{módulo detector de movimiento}.
        \item Su elemento sensor principal se llama \textbf{RE200BL}, que es un \textbf{sensor piezoeléctrico}.
        \item Funciona generando energía cuando se expone al calor.
        \item Detecta el movimiento porque \textbf{humanos o animales emiten energía térmica en forma de radiaciones infrarrojas}. Cuando un cuerpo entra 
        en el rango del sensor, detecta este movimiento.
        \item El término  'pasivo ' significa que el sensor \textbf{no utiliza ninguna energía para el propósito de detección}, sino que funciona detectando 
        la energía emitida por otros objetos.
    \end{itemize}

    \item \textbf{Componentes Físicos y Pines}
    \begin{itemize}
        \item El módulo incluye una \textbf{tapa de plástico especialmente diseñada} que se utiliza para expandir la cobertura del área de detección.
        \item Posee tres pines principales:
        \begin{itemize}
            \item \textbf{Ground (Tierra)}.
            \item \textbf{VCC (para 5 voltios)}, utilizado para alimentar el sensor.
            \item \textbf{Output (Salida)}, que proporciona un \textbf{nivel lógico alto} si se detecta un objeto y un nivel lógico bajo en caso contrario.
        \end{itemize}
    \end{itemize}

    \item \textbf{Potenciómetros para Ajuste}
    \begin{itemize}
        \item El módulo dispone de \textbf{dos potenciómetros} para ajustar su comportamiento:
        \begin{itemize}
            \item Uno para ajustar la \textbf{sensibilidad} del sensor.
            \item Otro para ajustar el \textbf{tiempo} durante el cual la señal de entrada permanece en un nivel alto cuando se detecta un objeto. Este 
            tiempo puede ajustarse desde \textbf{0.3 segundos hasta 5 minutos}.
        \end{itemize}
    \end{itemize}

    \item \textbf{Modos de Disparo (Trigger Modes)}
    \begin{itemize}
        \item El módulo también tiene \textbf{tres pines adicionales con un \textit{jumper}} entre dos de ellos, que sirven para seleccionar los modos de 
        disparo:
        \begin{itemize}
            \item \textbf{Disparo no repetible (Non-repeatable trigger)}: Cuando el sensor emite una señal alta y el tiempo de retardo ha terminado, la 
            salida cambia automáticamente de alto a bajo.
            \item \textbf{Disparo repetible (Repeatable trigger)}: La salida permanece en un nivel alto todo el tiempo que el objeto detectado esté presente 
            en el rango del sensor.
        \end{itemize}
    \end{itemize}

    \item \textbf{Rol en el Detector de Movimiento Antirrobo}
    \begin{itemize}
        \item En este proyecto, el sensor PIR se conecta a una Raspberry Pi.
        \item El código Python se escribirá para \textbf{detectar las señales altas y bajas} que el sensor PIR produce al detectar movimiento.
        \item Estas detecciones son la base para \textbf{activar la alarma} (controlando un \textit{buzzer}) y \textbf{enviar alertas} al usuario, 
        estableciendo así una comunicación unidireccional inicial del servidor al usuario.
    \end{itemize}
\end{itemize}
En resumen, el sensor PIR es un componente didáctico clave en la Sección 2, permitiendo a los estudiantes comprender y experimentar cómo un sensor 
real interactúa con una Raspberry Pi para recopilar datos del entorno y utilizarlos en un sistema IoT funcional de detección de movimiento.

\subsection{Actuadores (Zunbador)}
En el contexto más amplio de la \textbf{Sección 2: Primer Proyecto IoT}, el \textbf{Actuador (Zumbador)} es un componente clave del proyecto 
\textbf{Detector de Movimiento Antirrobo}, funcionando como el dispositivo de salida principal para la alarma del sistema. Su implementación es 
fundamental para establecer la funcionalidad de alerta y, posteriormente, la comunicación bidireccional en el sistema IoT. A continuación, se dan 
los detalles de el zumbador:

\begin{itemize}
    \item \textbf{Propósito en el Proyecto}:
    \begin{itemize}
        \item El objetivo principal del proyecto es crear un dispositivo IoT defensivo contra robos que \textbf{active una alarma} al detectar movimiento. 
        El zumbador es el componente encargado de producir esta alarma sonora.
        \item Inicialmente, el proyecto se enfoca en la comunicación unidireccional del servidor al usuario. Para lograr una \textbf{comunicación 
        bidireccional}, se añade una funcionalidad que permite a los usuarios \textbf{desactivar el zumbador (la alarma)} desde el servidor a través de 
        una interfaz web.
    \end{itemize}

    \item \textbf{Descripción del Hardware del Zumbador}:
    \begin{itemize}
        \item Consta de una \textbf{carcasa exterior con tres pines}: VCC (para 5 voltios), Tierra y Señal.
        \item En su interior, contiene un \textbf{elemento piezoeléctrico} rodeado por un \textbf{disco metálico que vibra}.
        \item El sonido se produce cuando se \textbf{aplica corriente al zumbador}, lo que hace que el disco se contraiga y expanda, generando vibraciones y, 
        por ende, sonido.
        \item Es posible cambiar el \textbf{tono o la  'melodía '} del zumbador al variar la frecuencia de la corriente aplicada, lo que altera la 
        velocidad de vibración del disco.
    \end{itemize}

    \item \textbf{Control y Programación}:
    \begin{itemize}
        \item Para controlar el zumbador, se generará una \textbf{onda cuadrada}.
        \item Esto se logra alternando el pin de señal entre un estado lógico alto y bajo, con pequeños intervalos de espera de milisegundos entre cada 
        cambio, repitiendo este proceso.
        \item Los estudiantes escribirán \textbf{código Python básico} para controlar el zumbador en respuesta a la detección de movimiento del sensor PIR.
    \end{itemize}

    \item \textbf{Interacción con la Interfaz de Usuario Web}:
    \begin{itemize}
        \item En la página web que sirve como interfaz del proyecto, se añadirá un \textbf{botón interruptor (switch button)}. Este botón permitirá a 
        los usuarios \textbf{controlar los actuadores}, específicamente para \textbf{desactivar el zumbador}. Esta funcionalidad es clave para la 
        comunicación bidireccional del sistema.
    \end{itemize}
\end{itemize}

\subsection{Servidor WEB con FLASK (Comunicación Unidireccional y Bidireccional)}
En el contexto más amplio de la \textbf{Sección 2: Primer Proyecto IoT}, que se centra en el desarrollo del \textbf{Detector de Movimiento Antirrobo}, el 
\textbf{Servidor Web con Flask} es el pilar central para la comunicación y la interacción con el dispositivo IoT. Se describe su implementación y 
evolución desde una comunicación unidireccional básica a una bidireccional, sentando las bases para conceptos más avanzados en el proyecto. Aquí se dan los 
detalles de el Servidor Web con Flask y su rol en la comunicación:

\begin{itemize}
    \item \textbf{Propósito del Servidor Web con Flask en la Sección 2}:
    \begin{itemize}
        \item Los estudiantes desarrollarán un \textbf{servidor web HTTP básico utilizando Python Flask} en la Raspberry Pi.
        \item Este servidor es fundamental para que el dispositivo IoT defensivo contra robos pueda \textbf{enviar alertas al usuario} y, posteriormente, 
        permitir que los usuarios \textbf{interactúen con el servidor} para controlar el dispositivo.
        \item El objetivo es mostrar datos de sensores a través de una aplicación web.
    \end{itemize}

    \item \textbf{Funcionamiento Técnico del Servidor y Comunicación Unidireccional Inicial}:
    \begin{itemize}
        \item El servidor se ejecutará en la \textbf{red Wi-Fi local} de los usuarios.
        \item Una vez que la Raspberry Pi tenga una dirección IP (por ejemplo, 192.168.1.250), los usuarios podrán acceder a una página web desde sus 
        navegadores dentro de la misma red Wi-Fi.
        \item La comunicación entre el cliente y el servidor se realizará utilizando la \textbf{técnica AJAX}.
        \item Para mantener la conexión activa y recibir actualizaciones, el navegador del usuario enviará una \textbf{ 'solicitud \textit{keepalive} 
        '(latido)} al servidor de la Raspberry Pi periódicamente (cada cinco segundos). Esto actúa como un  'latido ' para confirmar que la conexión con 
        el servidor sigue activa.
        \item En cada respuesta del \textit{keepalive}, el servidor \textbf{enviará el estado del sensor y los datos al usuario}, proporcionando así 
        \textbf{actualizaciones en vivo} de, por ejemplo, el estado de detección de movimiento y el estado de la conexión.
        \item Esta fase inicial cumple con la \textbf{comunicación unidireccional del servidor al usuario}, donde el servidor informa al usuario sobre 
        el estado del sensor y las alertas.
    \end{itemize}

    \item \textbf{Evolución hacia la Comunicación Bidireccional}:
    \begin{itemize}
        \item Para lograr una \textbf{comunicación bidireccional del usuario al servidor}, se añade una funcionalidad que permite a los usuarios 
        \textbf{comunicarse con el servidor}.
        \item Específicamente, se implementa un \textbf{botón interruptor (switch button) en la página web} que permite a los usuarios 
        \textbf{controlar los actuadores}, como \textbf{desactivar el zumbador (alarma)}.
        \item Este botón envía una solicitud desde el cliente (navegador) al servidor Flask para cambiar el estado del dispositivo. Se 
        menciona la adición de un \textit{endpoint} en la aplicación Flask para recibir estas solicitudes.
        \item Posteriormente en el proyecto (aunque ya se conceptualiza aquí), se detallará cómo el servidor también puede devolver información 
        como \verb|user_ID| y una lista de \verb|online_users| para poblar la interfaz web, lo que refuerza la naturaleza bidireccional y la capacidad 
        de gestión de usuarios.
    \end{itemize}

    \item \textbf{Limitaciones y Futuras Mejoras}:
    \begin{itemize}
        \item Es importante destacar que, en esta etapa, el servidor Flask se ejecuta \textbf{localmente en la Raspberry Pi}, lo que significa que los 
        usuarios solo pueden acceder a él desde \textbf{dentro de la misma red Wi-Fi local}.
        \item Se indica también que, después de discutir las ventajas y desventajas de este método de comunicación desarrollado en la Sección 2, 
        el proyecto procederá a estudiar en profundidad \textbf{protocolos de comunicación en tiempo real y ligeros} como \textbf{MQTT y WebSockets} en 
        la Sección 3, así como temas de seguridad en Internet. Esto implica que el enfoque inicial con Flask y AJAX, aunque funcional, se considera 
        una base para soluciones más robustas y escalables.
    \end{itemize}
\end{itemize}
En resumen, el servidor web basado en Flask en la Sección 2 es el elemento clave que convierte la Raspberry Pi en un dispositivo IoT interactivo, 
permitiendo inicialmente el monitoreo de movimiento y la activación de alarmas (unidireccional), para luego evolucionar a un control remoto de la 
alarma por parte del usuario (bidireccional), todo ello como preparación para arquitecturas IoT más complejas y seguras.

\subsection{Técnica AJAX para Comunicación}
En el contexto más amplio de la \textbf{Sección 2: Primer Proyecto IoT}, que se centra en el desarrollo de un \textbf{Detector de Movimiento Antirrobo}, 
la \textbf{técnica AJAX para comunicación} es el método principal utilizado para facilitar la interacción entre el cliente (navegador web del usuario) 
y el servidor Flask que se ejecuta en la Raspberry Pi. Es fundamental para establecer la comunicación inicial en el proyecto y proporcionar 
actualizaciones en tiempo real. Se detalla lo siguiente sobre AJAX en esta sección:

\begin{itemize}
    \item \textbf{Propósito y Rol Inicial}:
    \begin{itemize}
        \item La técnica AJAX se utiliza para la \textbf{comunicación entre el cliente y el servidor}.
        \item Su objetivo principal es permitir que el servidor web de la Raspberry Pi \textbf{envíe alertas al usuario} y muestre datos de los sensores 
        a través de una aplicación web. Esto cumple con la primera fase de \textbf{comunicación unidireccional del servidor al usuario}.
        \item Se utiliza para obtener \textbf{actualizaciones en vivo} del estado del sensor y los datos.
    \end{itemize}

    \item \textbf{Mecanismo de  'Keepalive Request ' (Latido)}:
    \begin{itemize}
        \item Una vez que el usuario recibe la página web del servidor Flask, el navegador del usuario enviará una \textbf{ 'solicitud \textit{keepalive} 
        '(latido)} al servidor web de la Raspberry Pi de forma \textbf{periódica, cada cinco segundos}.
        \item Esta solicitud actúa como un  'latido ' para \textbf{confirmar que la conexión con el servidor sigue activa}, es decir, que el servidor 
        aún está en funcionamiento.
        \item En la respuesta a cada solicitud \textit{keepalive}, el servidor tiene la capacidad de \textbf{enviar el estado del sensor y los datos 
        al usuario}. Por ejemplo, esto incluye el estado de la detección de movimiento y el estado de la conexión.
    \end{itemize}

    \item \textbf{Limitaciones y Evolución Futura}:
    \begin{itemize}
        \item Aunque eficaz para la comunicación inicial y las actualizaciones en vivo, se  sugieren que la técnica AJAX (junto con el 
        'long polling ' en general) tiene \textbf{ventajas y desventajas}.
        \item Después de esta Sección 2, el proyecto abordará en profundidad \textbf{protocolos de comunicación en tiempo real y ligeros} como 
        \textbf{MQTT y WebSockets} en la Sección 3. Esto implica que AJAX, si bien es una base de aprendizaje útil, se considera una solución 
        inicial que será reemplazada por tecnologías más avanzadas para una comunicación IoT más robusta y escalable.
    \end{itemize}
\end{itemize}
En resumen, la técnica AJAX, implementada a través de solicitudes periódicas de \textit{keepalive}, es la piedra angular de la comunicación en la 
Sección 2, permitiendo que el Detector de Movimiento Antirrobo basado en Raspberry Pi proporcione información y alertas en tiempo real al usuario a 
través de una interfaz web local.

\section{Protocolos de Comunicación y Seguridad}
% 2.3.PROTOCOLOS DE COMUNICACIÓN Y SEGURIDAD
En el contexto más amplio de las \textbf{Secciones del Curso}, la \textbf{Sección 3: Protocolos de Comunicación y Seguridad} es una etapa crucial que 
sigue a la implementación inicial del proyecto IoT básico en la Sección 2. Su objetivo principal es profundizar en las \textbf{tecnologías esenciales 
para una comunicación IoT robusta, segura y en tiempo real}, abordando las limitaciones de los métodos anteriores y sentando las bases para desarrollos 
futuros más complejos y seguros del servidor IoT. A continuación, se dan los detalles de la Sección 3:

\begin{itemize}
    \item \textbf{Transición y Propósito}:
    \begin{itemize}
        \item Después de discutir las ventajas y desventajas del método de comunicación desarrollado en la Sección 2 (que utilizaba AJAX y 
        \textit{long polling}), el proyecto avanza para estudiar a fondo los \textbf{protocolos de comunicación en tiempo real y ligeros} para el 
        Internet de las Cosas.
        \item También se estudiará la \textbf{seguridad en Internet y la criptografía}, preparando el terreno para la implementación de un servidor 
        seguro y un inicio de sesión de usuario en secciones posteriores.
    \end{itemize}

    \item \textbf{Protocolos de Comunicación en Tiempo Real}:
    \begin{itemize}
        \item La Sección 3 examinará dos de las tecnologías más famosas y adoptadas en IoT para protocolos de comunicación: \textbf{WebSockets y MQTT}.
        \item Se ofrecerán \textbf{demos prácticas con PubNub}, un servicio que probablemente se usará para aplicar estos protocolos.
    \end{itemize}

    \item \textbf{WebSockets}:
    \begin{itemize}
        \item Permiten a los clientes recibir actualizaciones solo cuando ocurren, sin que el cliente tenga que  'preguntar ' al servidor.
        \item Es una tecnología avanzada que posibilita una sesión de comunicación interactiva al abrir una única conexión TCP persistente.
        \item Es \textbf{bidireccional y \textit{full duplex}}, lo que significa que ambas partes (cliente y servidor) pueden enviar mensajes de forma 
        independiente.
        \item Se estandarizó en 2011 y es altamente adoptado en aplicaciones IoT debido a su baja latencia y naturaleza bidireccional.
        \item Funciona mediante una solicitud HTTP inicial donde el cliente pide al servidor \textbf{actualizar a un protocolo WebSockets}. Una vez 
        aceptada la conexión (handshake), la sesión se mantiene abierta y persistente hasta que una de las partes la cierra.
        \item Los paquetes de datos enviados son muy pequeños, con una longitud de trama de 2 a 14 bytes.
    \end{itemize}

    \item \textbf{MQTT (Message Queuing Telemetry Transport)}:
    \begin{itemize}
        \item Es un \textbf{protocolo de transferencia de mensajes ligero} para la comunicación máquina a máquina (M2M) y el Internet de las Cosas.
        \item Es muy \textbf{eficiente en el uso del ancho de banda}, con solo 2 bytes de sobrecarga.
        \item Permite escenarios de transmisión de datos \textbf{uno a uno, uno a muchos y muchos a muchos} para dispositivos y aplicaciones.
        \item Esto se logra mediante un \textbf{modelo de publicación y suscripción}, donde un flujo de datos específico se envía sobre un 'tema' 
        (topic), y los dispositivos suscritos a ese tema pueden recibir los datos.
        \item Se ejecuta sobre la red TCP/IP, lo que significa que \textbf{puede usarse sobre una capa WebSocket}.
        \item Es importante no confundir MQTT con WebSockets; MQTT es un servicio de entrega que puede funcionar \textit{encima} de WebSockets, como 
        si MQTT fuera un servicio de paquetería (DHL) y WebSockets proporcionaran las carreteras y el transporte.
        \item El punto central de comunicación en MQTT es el \textbf{broker}, que se encarga de despachar todos los mensajes entre el remitente y los 
        receptores.
        \item Los clientes publican mensajes al broker que incluyen un  'tema ', que es información de enrutamiento para que el broker reenvíe el 
        mensaje a los receptores suscritos a ese tema.
        \item Esto lo hace \textbf{altamente escalable}, ya que los clientes no necesitan conocerse entre sí, solo comunicarse a través del tema. 
        Sin embargo, su principal inconveniente es que si el broker falla, toda la comunicación se interrumpe.
    \end{itemize}

    \item \textbf{Seguridad en Internet y Criptografía}:
    \begin{itemize}
        \item Se abordarán \textbf{temas candentes} como:
        \begin{itemize}
            \item \textbf{Algoritmos de clave simétrica y asimétrica}.
            \item \textbf{Cifrado}.
            \item \textbf{Firma digital}.
            \item \textbf{Protocolos SSL/TLS}, también conocidos como \textbf{HTTPS}.
        \end{itemize}
    \end{itemize}
\end{itemize}
En resumen, la Sección 3 es la fase en la que los estudiantes adquieren conocimientos fundamentales sobre \textbf{cómo hacer que los sistemas IoT se 
comuniquen de manera eficiente y segura}, sentando las bases para la implementación de soluciones más complejas y empresariales en las secciones 
subsiguientes, incluyendo la protección de dominios personalizados y la gestión de inicios de sesión de usuarios.

\subsection{Protocolos en Tiempo Real}
En el contexto más amplio de la \textbf{Sección 3: Protocolos de Comunicación y Seguridad}, la discusión sobre los \textbf{Protocolos en Tiempo Real} 
es fundamental, ya que esta sección se dedica a estudiar a fondo las tecnologías avanzadas de comunicación para el Internet de las Cosas (IoT) que 
superan las limitaciones de los métodos anteriores (como AJAX y \textit{long polling}). El objetivo es permitir que los sistemas IoT se comuniquen 
de manera eficiente y robusta. Se detallan dos de los protocolos en tiempo real más importantes y adoptados en el IoT: \textbf{WebSockets} 
y \textbf{MQTT}.

\subsubsection{WebSockets}
\begin{itemize}
    \item \textbf{Naturaleza y Propósito}:
    \begin{itemize}
        \item Permiten a los clientes \textbf{recibir actualizaciones solo cuando ocurren}, sin necesidad de que el cliente esté  'preguntando' 
        al servidor constantemente.
        \item Es una tecnología avanzada que posibilita una \textbf{sesión de comunicación interactiva} al abrir una \textbf{única conexión TCP 
        persistente}.
        \item Es \textbf{bidireccional y \textit{full duplex}}, lo que significa que tanto el cliente como el servidor pueden enviar mensajes de 
        forma independiente entre sí.
        \item Fue estandarizado en 2011 y es muy utilizado en aplicaciones IoT debido a su \textbf{baja latencia y naturaleza bidireccional}.
    \end{itemize}

    \item \textbf{Funcionamiento}:
    \begin{itemize}
        \item Inicialmente, un cliente realiza una \textbf{solicitud HTTP} pidiendo al servidor que actualice el protocolo a WebSockets.
        \item El servidor responde con un \textbf{apretón de manos (handshake)}, y una vez aceptada, la sesión se mantiene \textbf{abierta y persistente} 
        hasta que una de las partes la cierra.
        \item Durante la sesión, los paquetes de datos enviados entre el cliente y el servidor son \textbf{muy pequeños}, con una longitud de trama de 2 
        a 14 bytes.
    \end{itemize}
\end{itemize}

\subsubsection{MQTT (Message Queuing Telemetry Transport)}
\begin{itemize}
    \item \textbf{Naturaleza y Propósito}:
    \begin{itemize}
        \item Es un \textbf{protocolo de transferencia de mensajes ligero} (lightweight) diseñado específicamente para la comunicación máquina a 
        máquina (M2M) y el Internet de las Cosas.
        \item Es \textbf{muy eficiente en el uso del ancho de banda}, utilizando solo 2 bytes de sobrecarga.
        \item Permite escenarios de transmisión de datos \textbf{uno a uno, uno a muchos y muchos a muchos} para dispositivos y aplicaciones.
    \end{itemize}

    \item \textbf{Modelo de Publicación y Suscripción}:
    \begin{itemize}
        \item Esto se logra mediante un \textbf{modelo de publicación y suscripción}. Un flujo de datos específico se envía sobre un 
        \textbf{'tema' (topic)}, y los dispositivos suscritos a ese tema pueden recibir los datos. Por ejemplo, un sensor de humedad puede publicar 
        lecturas en el tema  'jardín ', y una bomba de agua suscrita a ese tema puede encenderse si el nivel de humedad es bajo.
    \end{itemize}

    \item \textbf{Funcionamiento con un Broker}:
    \begin{itemize}
        \item El punto central de comunicación en MQTT es el \textbf{broker}. El broker es el encargado de despachar todos los mensajes entre el 
        remitente y los receptores.
        \item Cada cliente que publica un mensaje al broker incluye un  'tema ' dentro del mensaje. Esta información de tema es crucial para que el 
        broker reenvíe el mensaje a los receptores que están suscritos a ese tema. Por ejemplo, un usuario puede publicar la temperatura en un tema 
        llamado 'temperatura', y el broker reenviará estos datos a un aire acondicionado que esté suscrito al mismo tema para establecer la temperatura 
        deseada.
    \end{itemize}

    \item \textbf{Escalabilidad y Limitaciones}:
    \begin{itemize}
        \item Este diseño lo hace \textbf{altamente escalable}, ya que los clientes no necesitan conocerse entre sí; solo necesitan comunicarse a 
        través del tema.
        \item El principal inconveniente de MQTT es su \textbf{dependencia de la entidad central (el broker)}: si el broker falla, toda la comunicación 
        se interrumpe.
    \end{itemize}

    \item \textbf{Relación con WebSockets}:
    \begin{itemize}
        \item MQTT se ejecuta sobre la red TCP/IP, lo que significa que \textbf{puede utilizarse sobre una capa WebSocket}.
        \item Es importante no confundir MQTT con WebSockets; son cosas diferentes. MQTT puede considerarse un servicio de entrega (como DHL) que funciona 
        \textit{sobre} WebSockets, que a su vez proporciona las  'carreteras y el transporte '. Un paquete de datos MQTT se  'envuelve ' dentro de un sobre 
        WebSocket, que a su vez se envuelve en un sobre TCP/IP y se envía por Internet, desempaquetándose en orden inverso al llegar.
    \end{itemize}
\end{itemize}
En resumen, la Sección 3 establece una base sólida para la comunicación IoT al introducir estos protocolos en tiempo real, permitiendo a los desarrolladores 
elegir la solución más adecuada para la interactividad y la eficiencia en sus proyectos. Además, se  mencionan que esta sección también aborda 
la seguridad en Internet y la criptografía, preparando el terreno para servidores seguros y gestión de usuarios.

\subsubsection{MQTT}
En el contexto más amplio de los \textbf{Protocolos en Tiempo Real} que se estudian en la Sección 3: Protocolos de Comunicación y Seguridad, 
\textbf{MQTT (Message Queuing Telemetry Transport)} se presenta como una de las tecnologías más adoptadas y eficientes para la comunicación en el 
Internet de las Cosas (IoT). Su inclusión subraya la necesidad de superar las limitaciones de métodos de comunicación anteriores, como AJAX y 
\textit{long polling}, para lograr una comunicación robusta, segura y en tiempo real. Aquí se dan los detalles de MQTT:

\begin{itemize}
    \item \textbf{Definición y Propósito}:
    \begin{itemize}
        \item MQTT es un \textbf{protocolo de transferencia de mensajes ligero} (lightweight).
        \item Está diseñado específicamente para la \textbf{comunicación máquina a máquina (M2M)} y el \textbf{Internet de las Cosas}.
        \item Es notablemente \textbf{eficiente en el uso del ancho de banda}, con solo 2 bytes de sobrecarga.
    \end{itemize}

    \item \textbf{Modelo de Publicación y Suscripción}:
    \begin{itemize}
        \item Permite escenarios de \textbf{transmisión de datos uno a uno, uno a muchos y muchos a muchos} para dispositivos y aplicaciones.
        \item Esto se logra mediante un \textbf{modelo de publicación y suscripción}.
        \item Un flujo de datos específico se envía sobre un \textbf{ 'tema ' (topic)}, y los dispositivos suscritos a ese tema pueden recibir los 
        datos. Por ejemplo, un sensor de humedad puede publicar el nivel de humedad en un tema llamado  'jardín ', y una bomba de agua suscrita a 
        ese tema puede activarse si el nivel es bajo.
    \end{itemize}

    \item \textbf{El Broker Central}:
    \begin{itemize}
        \item El punto central de comunicación en MQTT es el \textbf{broker}.
        \item El broker es el encargado de \textbf{despachar todos los mensajes} entre el remitente y los receptores.
        \item Cada cliente que publica un mensaje al broker incluye un  'tema ' dentro del mensaje. Esta información de tema sirve como 
        \textbf{información de enrutamiento} para que el broker reenvíe el mensaje a los receptores que están suscritos a ese tema. Por ejemplo, 
        un usuario puede publicar la temperatura en un tema llamado  'temperatura ', y el broker reenvía estos datos a un aire acondicionado 
        suscrito al mismo tema para establecer la temperatura deseada.
    \end{itemize}

    \item \textbf{Escalabilidad y Limitaciones}:
    \begin{itemize}
        \item Este modelo hace que MQTT sea \textbf{altamente escalable}, ya que los clientes no necesitan conocerse entre sí; solo tienen que 
        comunicarse a través del tema.
        \item Sin embargo, el \textbf{principal inconveniente} de este protocolo es su \textbf{dependencia de la entidad central (el broker)}: 
        si el broker falla, toda la comunicación se interrumpe.
    \end{itemize}

    \item \textbf{Relación con WebSockets}:
    \begin{itemize}
        \item MQTT se ejecuta sobre la red \textbf{TCP/IP}, lo que significa que \textbf{puede utilizarse sobre una capa WebSocket}.
        \item Es importante \textbf{no confundir MQTT con WebSockets}, ya que son cosas diferentes.
        \item Se ofrece una analogía: se puede considerar MQTT como un \textbf{servicio de entrega} (como DHL) que funciona \textit{sobre} 
        WebSockets, que a su vez proporciona las  'carreteras y el transporte '. En esta metáfora, un paquete de datos MQTT se  'empaqueta ' dentro de un 
        sobre WebSocket, que luego se envuelve en un sobre TCP/IP y se envía por Internet, desempaquetándose en orden inverso al llegar a su destino.
    \end{itemize}
\end{itemize}
En el contexto de los protocolos en tiempo real, MQTT se presenta como una solución robusta y eficiente para la comunicación en IoT, especialmente cuando 
se requiere un manejo escalable de mensajes y un bajo consumo de ancho de banda. Su estudio, junto con WebSockets, en la Sección 3 es fundamental para 
construir un ecosistema IoT seguro, en tiempo real y escalable, sentando las bases para funcionalidades más avanzadas como la gestión de permisos 
de acceso en tiempo real para usuarios y dispositivos.

\subsubsection{WebSockets}
En el contexto más amplio de los \textbf{Protocolos en Tiempo Real} dentro de la Sección 3: Protocolos de Comunicación y Seguridad, \textbf{WebSockets} 
se presenta como una tecnología fundamental y ampliamente adoptada en el Internet de las Cosas (IoT). Esta sección se enfoca en el estudio de 
protocolos avanzados para la comunicación en IoT, buscando superar las limitaciones de técnicas anteriores como AJAX y \textit{long polling}, 
para permitir una comunicación eficiente, bidireccional y robusta. A continuación, se dan los detalles de WebSockets:

\begin{itemize}
    \item \textbf{Naturaleza y Propósito}:
    \begin{itemize}
        \item WebSockets es una \textbf{tecnología avanzada} que permite a los clientes \textbf{recibir actualizaciones solo cuando ocurren}, 
        eliminando la necesidad de que el cliente esté consultando constantemente al servidor. Esto significa que no hay necesidad de  'preguntar' 
        o hacer \textit{polling} al servidor para obtener actualizaciones.
        \item Habilita una \textbf{sesión de comunicación interactiva} al abrir una \textbf{única conexión TCP persistente}.
        \item Es \textbf{bidireccional y \textit{full duplex}}, lo que permite que tanto el cliente como el servidor envíen mensajes de forma independiente 
        entre sí.
    \end{itemize}

    \item \textbf{Características Clave}:
    \begin{itemize}
        \item Fue \textbf{estandarizado en 2011}.
        \item Aunque inicialmente fue adoptado por aplicaciones como juegos, sistemas de chat y aplicaciones web, su \textbf{baja latencia y naturaleza 
        bidireccional} lo han hecho altamente adoptado en aplicaciones IoT.
        \item Los paquetes de datos enviados durante una sesión abierta son \textbf{muy pequeños}, con una longitud de trama de 2 a 14 bytes.
    \end{itemize}

    \item \textbf{Funcionamiento}:
    \begin{itemize}
        \item Inicialmente, un cliente realiza una \textbf{solicitud HTTP} solicitando al servidor que actualice el protocolo a WebSockets.
        \item El servidor responde con un \textbf{apretón de manos (handshake)}, y una vez aceptado, la sesión se mantiene \textbf{abierta y persistente} 
        hasta que una de las partes la cierra.
    \end{itemize}

    \item \textbf{Relación con MQTT}:
    \begin{itemize}
        \item Es crucial \textbf{no confundir WebSockets con MQTT}; son dos cosas diferentes.
        \item MQTT, que se ejecuta sobre la red TCP/IP, \textbf{puede utilizarse sobre una capa WebSocket}.
        \item Se utiliza una analogía para clarificar su relación: \textbf{MQTT puede considerarse un servicio de entrega} (como DHL) que funciona 
        \textit{sobre} WebSockets, siendo estos últimos los que proporcionan las  'carreteras y el transporte '. Un paquete de datos MQTT se  'empaqueta' 
        dentro de un sobre WebSocket, que a su vez se envuelve en un sobre TCP/IP y se envía por Internet, desempaquetándose en orden inverso al llegar 
        a su destino.
    \end{itemize}
\end{itemize}
En resumen, WebSockets es un pilar fundamental para la comunicación en tiempo real en IoT, ofreciendo una solución eficiente, interactiva y de baja 
latencia que supera las limitaciones de los métodos tradicionales, facilitando así el desarrollo de sistemas IoT avanzados y seguros.

\subsubsection{DEMOSTRACIÓN CON PUBNUB}
En el contexto más amplio de los \textbf{Protocolos en Tiempo Real}, se  mencionan a \textbf{PubNub} como una tecnología relevante, particularmente 
en la \textbf{Sección 3: Protocolos de Comunicación y Seguridad}, donde se estudian en profundidad los protocolos de comunicación en tiempo real y ligeros 
para el Internet de las Cosas (IoT), como MQTT y WebSockets. Se indica lo siguiente sobre PubNub en relación con los protocolos en tiempo real 
y las demostraciones:

\begin{itemize}
    \item \textbf{Integración con Protocolos en Tiempo Real}: Después de discutir las ventajas y desventajas del método de comunicación utilizado en un 
    proyecto anterior (probablemente AJAX y \textit{long polling}), la Sección 3 se dedica a estudiar los protocolos de comunicación en tiempo real y ligeros, 
    mencionando específicamente a \textbf{MQTT, WebSockets y, finalmente, algunas demostraciones prácticas con PubNub}. Esto sugiere que PubNub se presenta 
    como una plataforma o servicio que utiliza o facilita la comunicación en tiempo real.

    \item \textbf{Reconstrucción de Proyectos y Despliegue en la Nube}: La \textbf{Sección 4} del proyecto se centra en \textbf{reconstruir un proyecto 
    anterior para usar PubNub como el principal protocolo de comunicación}, en lugar de las técnicas menos eficientes como AJAX y \textit{long polling}. 
    En esta sección, también se aprende a desplegar un servidor IoT en la nube de AWS (Amazon Web Services). Esto indica que PubNub no es solo un 
    protocolo, sino una solución que se integra en la arquitectura de un servidor IoT en la nube para manejar la comunicación.

    \item \textbf{Gestión de Accesos en Tiempo Real}: En la \textbf{Sección 6}, se implementa una forma segura en la que los usuarios y dispositivos IoT 
    pueden conectarse de manera segura con el servidor IoT. Aquí, se utiliza la \textbf{funcionalidad de \textit{Access Manager} de PubNub} para que los 
    usuarios administradores puedan otorgar \textbf{permisos de lectura y escritura en tiempo real} a usuarios no administradores y dispositivos. 
    Un ejemplo de esto se muestra en un panel de control donde los administradores pueden ver una lista de usuarios en línea y, mediante botones de 
    interruptor, conceder o revocar permisos de lectura y escritura, que se aplican en tiempo real al seleccionar un botón de  'aplicar'.

    \item \textbf{Implementación de la Lógica de Permisos}:
    \begin{itemize}
        \item Para implementar estos permisos, se describe un proceso de \textbf{generación de una clave de autorización} para un usuario específico y 
        su almacenamiento en la base de datos como un primer paso.
        \item Luego, un administrador envía una solicitud desde el código JavaScript a la aplicación Flask, que escucha en cualquier botón de interruptor 
        con un ID que comienza con  'XS ', extrae el ID del usuario, el estado de lectura y el estado de escritura.
        \item Finalmente, esta solicitud se envía al servidor para \textbf{otorgar acceso de lectura y escritura a un usuario específico a través del 
        servidor PubNub}. La respuesta se maneja en el cliente, y si el acceso es concedido, el canal se vuelve a suscribir.
    \end{itemize}
\end{itemize}
En resumen, PubNub se presenta como una \textbf{plataforma clave para implementar la comunicación en tiempo real en sistemas IoT}, 
ofreciendo funcionalidades que van más allá de un simple protocolo. Permite la \textbf{reconstrucción de proyectos para usarlo como el principal 
mecanismo de comunicación}, se integra con el despliegue en la nube, y es crucial para la \textbf{gestión de permisos de acceso de usuarios y 
dispositivos en tiempo real}, lo que lo convierte en una parte integral del ecosistema IoT seguro, en tiempo real y escalable que se busca construir.

\subsection{Seguridad en Internet y Criptografía}
En el contexto más amplio de la \textbf{Sección 3: Protocolos de Comunicación y Seguridad}, se  indican que se realiza un estudio en profundidad 
sobre la \textbf{seguridad en Internet y la criptografía}. Este estudio es fundamental para la construcción de un ecosistema IoT robusto, seguro y en 
tiempo real. Aquí se dan los detalles de la seguridad en Internet y la criptografía:

\begin{itemize}
    \item \textbf{Temas Candentes de la Criptografía y Seguridad en Internet}: La Sección 3 aborda  'temas candentes ' dentro de la seguridad en Internet 
    y la criptografía, que incluyen:
    \begin{itemize}
        \item \textbf{Algoritmos de clave simétrica y asimétrica}.
        \item \textbf{Cifrado (Encryption)}.
        \item \textbf{Firma digital (Digital Signature)}.
        \item \textbf{SSL/TLS}, que también son conocidos como \textbf{protocolos HTTP}.
    \end{itemize}

    \item \textbf{Importancia en el Contexto de Protocolos en Tiempo Real (IoT)}: El estudio de estos conceptos de seguridad, junto con protocolos de 
    comunicación en tiempo real y ligeros como MQTT y WebSockets, es crucial para lograr la meta de desarrollar un \textbf{ecosistema IoT que sea fuerte, 
    seguro, en tiempo real y escalable}. La seguridad de la comunicación es un pilar esencial cuando se trabaja con dispositivos y datos en el Internet 
    de las Cosas.

    \item \textbf{Aplicación Práctica de los Conceptos de Seguridad (en secciones posteriores)}: Aunque los principios se estudian en la Sección 3, las 
    secciones subsiguientes demuestran su aplicación práctica para asegurar la plataforma IoT:
    \begin{itemize}
        \item \textbf{Asegurar un Dominio Personalizado con SSL/TLS}:
        \begin{itemize}
            \item La \textbf{Sección 5} se centra en la aplicación de las  'terminologías de seguridad ' estudiadas en la Sección 3, específicamente para 
            \textbf{asegurar un dominio personalizado recién creado con certificados SSL/TLS}.
            \item Se utiliza \textbf{Let's Encrypt}, una autoridad de certificación gratuita, automatizada y de código abierto, respaldada por importantes 
            patrocinadores.
            \item El proceso implica:
            \begin{itemize}
                \item Instalar software de terceros como \texttt{Certbot} en el servidor.
                \item Configurar el certificado SSL para Apache utilizando el cliente \texttt{Certbot}.
                \item Se pregunta al usuario si desea \textbf{redirigir el tráfico HTTP a HTTPS}, lo cual es una práctica recomendada y deseada.
                \item Es fundamental \textbf{asignar reglas de seguridad de entrada para HTTPS (puerto 443)} en el servidor remoto de AWS, ya que, de lo 
                contrario, la conexión podría fallar.
                \item Los certificados de Let's Encrypt son \textbf{válidos por aproximadamente tres meses}, requiriendo una renovación periódica.
                \item Una vez implementado, se verifica que el sitio web está \textbf{completamente seguro}, y que cada cliente conectado al servidor 
                tendrá una \textbf{comunicación encriptada de extremo a extremo}. El servidor mostrará un candado verde en el navegador, indicando una 
                conexión segura.
            \end{itemize}
        \end{itemize}
        \item \textbf{Implementación de Inicio de Sesión Seguro y Gestión de Accesos}:
        \begin{itemize}
            \item La \textbf{Sección 5} también incluye la implementación de una \textbf{funcionalidad de inicio de sesión de usuario seguro} y el 
            almacenamiento de los detalles del usuario en una base de datos integrada.
            \item En la \textbf{Sección 6}, se implementa una forma segura para que los usuarios y dispositivos IoT puedan conectarse de manera segura 
            con el servidor IoT.
            \item Esto se logra utilizando la funcionalidad de \textbf{\texttt{Access Manager}} de PubNub para permitir que los usuarios administradores 
            otorguen \textbf{permisos de lectura y escritura en tiempo real} a usuarios no administradores y dispositivos.
            \item La implementación implica generar una \textbf{clave de autorización para un usuario específico} y almacenarla en la base de datos. 
            Las solicitudes para conceder acceso se envían al servidor de Flask, que luego llama al servidor PubNub para otorgar los permisos, asegurando 
            que la solicitud provenga de un usuario administrador. Si el acceso es concedido, el canal se vuelve a suscribir para aplicar los cambios.
        \end{itemize}
    \end{itemize}
\end{itemize}
En síntesis, la Sección 3 proporciona la base teórica sobre la seguridad en Internet y la criptografía, presentando conceptos clave como el cifrado, 
las firmas digitales y SSL/TLS. Estas bases son luego aplicadas y demostradas en las secciones posteriores del proyecto para asegurar los servidores IoT, 
los dominios web y la gestión de acceso de usuarios y dispositivos, lo cual es esencial para construir una solución IoT fiable y protegida.

\subsubsection{ALGORITMOS (SIMÉTRICO, ASIMÉTRICO)}
En el contexto más amplio de la \textbf{Seguridad en Internet y la Criptografía}, se  indican que la \textbf{Sección 3: Protocolos de Comunicación 
y Seguridad} se dedica a un estudio en profundidad de estos temas, incluyendo específicamente los \textbf{Algoritmos de clave simétrica y asimétrica}. 
Estos se presentan como  'temas candentes ' dentro del ámbito de la seguridad en Internet y la criptografía. Aquí se dan los detalles de estos algoritmos:

\begin{itemize}
    \item \textbf{Estudio Teórico Fundamental}:
    \begin{itemize}
        \item La Sección 3 del proyecto aborda un estudio en profundidad de la \textbf{seguridad en Internet y la criptografía}, con el objetivo de 
        comprender cómo construir un ecosistema IoT \textbf{fuerte, seguro, en tiempo real y escalable}.
        \item Dentro de este estudio, se mencionan explícitamente los \textbf{algoritmos de clave simétrica y asimétrica} como componentes clave.
        \item Aunque no se  proporcionan una definición detallada o un análisis técnico de cada tipo de algoritmo, su inclusión subraya su 
        importancia teórica para entender la base de la seguridad digital.
    \end{itemize}

    \item \textbf{Conceptos Relacionados}: Los algoritmos simétricos y asimétricos se estudian junto con otros pilares de la criptografía y la 
    seguridad en Internet, como:
    \begin{itemize}
        \item \textbf{Cifrado (Encryption)}.
        \item \textbf{Firma digital (Digital Signature)}.
        \item \textbf{SSL/TLS}, también conocidos como protocolos HTTP seguros.
    \end{itemize}

    \item \textbf{Aplicación Implícita en la Seguridad Práctica}: Aunque no se  describen la implementación directa de estos algoritmos, los 
    conceptos de seguridad que ellos sustentan se aplican en las secciones prácticas subsiguientes para asegurar la plataforma IoT:
    \begin{itemize}
        \item \textbf{Asegurar dominios con SSL/TLS}: En la Sección 5, se aplican las  'terminologías de seguridad ' estudiadas en la Sección 3 para 
        \textbf{asegurar un dominio personalizado con certificados SSL/TLS} utilizando Let's Encrypt. La implementación de SSL/TLS inherentemente depende 
        de la criptografía asimétrica para el intercambio seguro de claves y de la criptografía simétrica para la encriptación eficiente de los datos de 
        la comunicación una vez establecida la conexión. Esto asegura una \textbf{comunicación encriptada de extremo a extremo} para cada cliente conectado 
        al servidor.
        \item \textbf{Funcionalidad de inicio de sesión seguro}: La Sección 5 también involucra la implementación de una \textbf{funcionalidad de inicio 
        de sesión de usuario seguro} y el almacenamiento de detalles de usuario en una base de datos integrada, lo que a menudo implica el uso de hashing 
        (una forma de criptografía) para almacenar contraseñas de forma segura.
        \item \textbf{Gestión de accesos en tiempo real}: En la Sección 6, se implementa una forma segura para que usuarios y dispositivos IoT se conecten 
        con el servidor IoT. Esto se realiza utilizando la funcionalidad de \textbf{\texttt{Access Manager}} de PubNub para otorgar permisos de lectura y 
        escritura en tiempo real, lo que implica la \textbf{generación de una clave de autorización para usuarios específicos} y su gestión segura. 
        Este proceso se basa en principios criptográficos para garantizar que solo los usuarios autorizados puedan realizar ciertas acciones.
    \end{itemize}
\end{itemize}
En síntesis, los algoritmos simétricos y asimétricos son presentados como componentes conceptuales vitales en la Sección 3, proporcionando las bases 
teóricas necesarias para comprender cómo se construyen sistemas IoT seguros. Aunque no se detallan a nivel de código, su presencia es fundamental para 
las implementaciones prácticas de seguridad como SSL/TLS, el inicio de sesión de usuarios y la gestión de accesos, que se abordan en las secciones 
posteriores del proyecto.

\subsubsection{CIFRADO}
En el contexto más amplio de la \textbf{Seguridad en Internet y la Criptografía}, se  destacan que el \textbf{cifrado (encryption)} es un tema 
fundamental y  'candente ' que se estudia en profundidad. Aquí se dan los detalles de el cifrado:

\begin{itemize}
    \item \textbf{El Cifrado como Componente Clave de la Criptografía}:
    \begin{itemize}
        \item La \textbf{Sección 3: Protocolos de Comunicación y Seguridad} se enfoca en un estudio en profundidad de la seguridad en Internet y la 
        criptografía.
        \item Dentro de este estudio, el \textbf{cifrado (encryption)} se menciona explícitamente como uno de los  'temas candentes ', junto con los 
        algoritmos de clave simétrica y asimétrica, la firma digital y los protocolos SSL/TLS.
        \item La comprensión del cifrado es esencial para construir un \textbf{ecosistema IoT fuerte, seguro, en tiempo real y escalable}.
    \end{itemize}

    \item \textbf{El Cifrado y SSL/TLS}:
    \begin{itemize}
        \item Se vincula directamente el cifrado con los \textbf{protocolos SSL/TLS}, mencionando que estos son  'también conocidos como 
        protocolos HTTP '.
        \item La \textbf{Sección 5} del proyecto está dedicada al desarrollo de las terminologías de seguridad estudiadas en la Sección 3. Un 
        objetivo principal es \textbf{asegurar un dominio personalizado recién creado con certificados SSL/TLS}.
        \item La implementación de SSL/TLS se lleva a cabo utilizando \textbf{Let's Encrypt}, una autoridad de certificación gratuita, automatizada y 
        de código abierto.
    \end{itemize}

    \item \textbf{Implementación Práctica del Cifrado a través de SSL/TLS}:
    \begin{itemize}
        \item Para asegurar un sitio web, se instala software de terceros como \texttt{Certbot} en el servidor.
        \item Luego, se configura el certificado SSL para Apache utilizando el cliente \texttt{Certbot}, proporcionando el nombre de dominio y una 
        dirección de correo electrónico para recuperación.
        \item Una acción crucial es redirigir el tráfico HTTP a HTTPS, lo cual es altamente deseado para garantizar la seguridad.
        \item Es imperativo \textbf{asignar reglas de seguridad de entrada para HTTPS (puerto 443)} en el servidor remoto (por ejemplo, AWS), ya que 
        de lo contrario la conexión fallaría.
        \item Una vez implementado, se puede verificar el estado del certificado SSL a través de herramientas como \verb|ssllabs.com|. El certificado 
        es emitido por Let's Encrypt y tiene una validez de aproximadamente tres meses, lo que requiere renovaciones periódicas.
        \item El resultado final es que el sitio web se considera \textbf{ 'completamente seguro '}, y cada cliente conectado al servidor tendrá 
        una \textbf{ 'comunicación encriptada de extremo a extremo '}. El navegador mostrará un  'candado verde ' indicando una conexión segura.
    \end{itemize}

    \item \textbf{Contexto en la Plataforma IoT}:
    \begin{itemize}
        \item La implementación del cifrado, especialmente a través de SSL/TLS, es un paso fundamental para asegurar los servidores IoT.
        \item Este enfoque garantiza que la comunicación entre los clientes (usuarios o dispositivos IoT) y el servidor sea privada y segura, 
        protegiendo los datos que se transmiten.
    \end{itemize}
\end{itemize}
En resumen, se  enfatizan que el \textbf{cifrado} es un pilar fundamental de la seguridad en Internet y la criptografía, enseñado 
teóricamente en la Sección 3 y aplicado directamente a través de \textbf{SSL/TLS} en la Sección 5 para garantizar una \textbf{comunicación 
encriptada de extremo a extremo} en la plataforma IoT.

\subsubsection{FIRMA DIGITAL}
En el contexto más amplio de la \textbf{Seguridad en Internet y la Criptografía}, se  indican que la \textbf{firma digital (digital 
signature)} es un tema fundamental y  'candente ' que forma parte de un estudio en profundidad. Aquí se dan los detalles de la firma digital:

\begin{itemize}
    \item \textbf{Tema Candente de la Criptografía y Seguridad en Internet}:
    \begin{itemize}
        \item La \textbf{Sección 3: Protocolos de Comunicación y Seguridad} (aunque se  menciona inicialmente Sección 4, otra clarifica que es 
        Sección 3 para el estudio teórico) se dedica a un estudio en profundidad de la seguridad en Internet y la criptografía.
        \item Dentro de este estudio, la \textbf{firma digital} se menciona explícitamente como uno de los  'temas candentes '.
        \item Este estudio es crucial para construir un \textbf{ecosistema IoT fuerte, seguro, en tiempo real y escalable}.
    \end{itemize}

    \item \textbf{Contexto con Otros Conceptos de Seguridad}: La firma digital se presenta junto con otros pilares de la criptografía y la seguridad 
    en Internet, tales como:
    \begin{itemize}
        \item Algoritmos de clave simétrica y asimétrica.
        \item Cifrado (Encryption).
        \item SSL/TLS (también conocidos como protocolos HTTP).
    \end{itemize}

    \item \textbf{Ausencia de Detalles Específicos sobre su Implementación}: Es importante señalar que, si bien se  identifican la firma 
    digital como un tema clave a estudiar, \textbf{no proporcionan detalles adicionales} sobre qué es una firma digital, cómo funciona, o cómo se 
    implementa en la práctica dentro del proyecto IoT descrito. La mención se limita a su inclusión como un concepto teórico relevante dentro del 
    ámbito de la seguridad en Internet y la criptografía.
\end{itemize}
En resumen, se  establece que la \textbf{firma digital} es un concepto teórico importante en la seguridad en Internet y la criptografía, 
abordado en la Sección 3 del proyecto. Se la presenta como parte del conocimiento fundamental necesario para desarrollar soluciones IoT seguras, 
aunque no se ofrecen detalles específicos sobre su funcionamiento o aplicación práctica en los extractos proporcionados.

\subsubsection{SSL/TLS (HTTPS)}
En el contexto más amplio de la \textbf{Seguridad en Internet y la Criptografía}, se  enfatizan que \textbf{SSL/TLS (HTTPS)} es un protocolo 
fundamental, un  'tema candente ', y una implementación práctica crucial para asegurar la comunicación en línea. Aquí se dan los detalles de SSL/TLS:

\begin{itemize}
    \item \textbf{Identificación como  'Protocolos HTTP Seguros ' y Tema Candente}:
    \begin{itemize}
        \item La \textbf{Sección 3: Protocolos de Comunicación y Seguridad} del proyecto se dedica a un estudio en profundidad de la seguridad en 
        Internet y la criptografía.
        \item Dentro de esta sección, \textbf{SSL/TLS} se menciona explícitamente como uno de los  'temas candentes ', también conocido como 
        \textbf{protocolos HTTP seguros}.
        \item Su estudio es esencial para comprender cómo construir un \textbf{ecosistema IoT fuerte, seguro, en tiempo real y escalable}.
    \end{itemize}

    \item \textbf{Implementación Práctica en la Seguridad del Servidor IoT}:
    \begin{itemize}
        \item La \textbf{Sección 5} del proyecto se centra específicamente en el desarrollo de las terminologías de seguridad estudiadas en la 
        Sección 3, y el primer video de esta sección trata sobre \textbf{ 'secure HTTP IOT server and user login '}.
        \item El objetivo principal es \textbf{asegurar un dominio personalizado} recién creado con \textbf{certificados SSL/TLS}.
    \end{itemize}

    \item \textbf{Uso de Let's Encrypt como Autoridad de Certificación}:
    \begin{itemize}
        \item Para obtener el certificado SSL, se utiliza \textbf{Let's Encrypt}, una autoridad de certificación \textbf{gratuita, 
        automatizada y de código abierto}.
        \item Es respaldada por importantes patrocinadores y utilizada por muchos desarrolladores y empresas.
    \end{itemize}

    \item \textbf{Proceso de Configuración y Aseguramiento}:
    \begin{enumerate}
        \item \textbf{Instalación de \texttt{Certbot}}: Se instala software de terceros como \texttt{Certbot} en el servidor. Esto se hace agregando 
        el repositorio de Certbot (\verb|sudo add_apt_repository ppa:certbot/certbot|), actualizando la lista de paquetes (\verb|sudo apt-get update|) 
        e instalando \verb|python_certbot_apache| (\verb|sudo apt-get install python_certbot_apache|).
        \item \textbf{Configuración del Certificado SSL para Apache}: Se ejecuta el cliente Certbot 
        \verb|sudo certbot --apache -d your-domain-name -d www.your-domain-name|, proporcionando el nombre de dominio (con y sin \texttt{www}) y una dirección 
        de correo electrónico para recuperación de clave.
        \item \textbf{Redirección de HTTP a HTTPS}: Se pregunta si se desea \textbf{redirigir el tráfico HTTP a HTTPS}, lo cual es enfáticamente 
        deseado y se selecciona esta opción.
        \item \textbf{Configuración de Reglas de Seguridad de Entrada (Inbound Security Rules)}: Un paso crítico es \textbf{asignar reglas de seguridad 
        de entrada para HTTPS (puerto 443)} en el servidor remoto (ej. AWS EC2), ya que la conexión fallaría sin esto. Inicialmente, solo se tienen 
        reglas para HTTP y SSH, por lo que se debe añadir HTTPS.
        \item \textbf{Verificación del Certificado}: Una vez completada la instalación, los archivos del certificado se encuentran en 
        \verb|/etc/letsencrypt/live|. El estado del certificado SSL se puede verificar utilizando herramientas como SSL Labs (\verb|ssllabs.com/ssltest|).
        \item \textbf{Validación y Renovación}: El certificado emitido por Let's Encrypt es válido por \textbf{aproximadamente tres meses}, 
        requiriendo \textbf{renovaciones periódicas}.
    \end{enumerate}

    \item \textbf{Resultados y Beneficios del SSL/TLS (HTTPS)}:
    \begin{itemize}
        \item El sitio web se considera \textbf{ 'completamente seguro '}.
        \item \textbf{Cada cliente conectado al servidor tendrá una comunicación encriptada de extremo a extremo}.
        \item El navegador mostrará un \textbf{ 'candado verde '} (green lock) indicando una conexión segura.
        \item El certificado se emite a nombre del servidor (ej.  'pact IOT server ') y es firmado por Let's Encrypt.
    \end{itemize}
\end{itemize}
En síntesis, se  posicionan a \textbf{SSL/TLS} como un concepto teórico crucial en la criptografía y la seguridad en Internet (Sección 3), 
que se traduce directamente en una \textbf{implementación práctica esencial} para garantizar la seguridad de la plataforma IoT (Sección 5). Su uso 
asegura la \textbf{confidencialidad e integridad de la comunicación} entre el servidor y sus clientes, transformando el tráfico HTTP no seguro en 
HTTPS encriptado y verificable.

\section{Reconstrucción del Proyecto y Despliegue en la NUBE}
% 2.4.RECONSTRUCCIÓN DEL PROYECTO Y DESPLIEGUE EN LA NUBE.
En el contexto más amplio de las \textbf{Secciones del Curso}, se  describen la \textbf{Sección 4: Reconstrucción del Proyecto y Despliegue 
en la Nube} como una fase fundamental para mejorar y escalar el proyecto IoT existente. Aquí se dan los detalles de esta sección:

\begin{itemize}
    \item \textbf{Reconstrucción del Proyecto Principal}: En la Sección 4, el proyecto previamente desarrollado se \textbf{reconstruirá}. Esto implica 
    una revisión y modificación significativa de la arquitectura de comunicación.
    \item \textbf{Transición a PubNub para Comunicación}: El cambio principal en esta reconstrucción es la adopción de \textbf{PubNub como el 
    principal protocolo de comunicación}. Esto reemplaza la técnica de  'AJAX long polling' que se había utilizado anteriormente. PubNub es una 
    de las tecnologías que se estudian en profundidad en el proyecto.
    \item \textbf{Despliegue del Servidor IoT en la Nube}: Además de la reconstrucción del protocolo de comunicación, la Sección 4 también cubre 
    cómo \textbf{desplegar el servidor IoT en la nube de AWS (Amazon Web Services)}. Esto es crucial para la escalabilidad y accesibilidad del ecosistema IoT.
\end{itemize}
En resumen, la \textbf{Sección 4} se centra en la \textbf{evolución del proyecto IoT} desde una implementación inicial a una solución más robusta y 
escalable. Introduce \textbf{PubNub} como una tecnología clave para la comunicación en tiempo real y enseña los pasos para el \textbf{despliegue en 
la infraestructura de la nube de AWS}, sentando las bases para un  'ecosistema IoT fuerte, seguro, en tiempo real y escalable ' como se menciona en el 
contexto general del proyecto.

\subsection{Uso de PubNub Como Protocolo Principal (en lugar de AJAX)}
En el contexto más amplio de la \textbf{Sección 4: Reconstrucción del Proyecto y Despliegue en la Nube}, se  destacan el \textbf{uso de PubNub como 
protocolo de comunicación principal} como una mejora fundamental, que \textbf{reemplaza la técnica de AJAX long polling} utilizada anteriormente. 
Aquí se dan los detalles de este cambio:

\begin{itemize}
    \item \textbf{Problemas con AJAX en la Comunicación IoT (Implícito)}:
    \begin{itemize}
        \item En la Sección 2, el proyecto inicial utiliza la técnica de \textbf{AJAX para la comunicación} entre el cliente y el servidor.
        \item Se menciona que, después de discutir las  'ventajas y desventajas ' de la forma en que se desarrolló el proyecto anterior en 
        la Sección 2, se pasa a estudiar protocolos de comunicación en tiempo real y ligeros en la Sección 3, lo que sugiere que AJAX tiene limitaciones 
        para un ecosistema IoT avanzado. Específicamente, AJAX \textit{long polling} se menciona como la técnica a reemplazar.
    \end{itemize}

    \item \textbf{Introducción y Estudio de PubNub}:
    \begin{itemize}
        \item PubNub se identifica como una de las \textbf{principales tecnologías} con las que se trabajará en el proyecto, junto con Python, Raspberry 
        Pi, Flask y AWS.
        \item En la Sección 3, se realiza un estudio en profundidad de protocolos de comunicación en tiempo real y ligeros para IoT, que incluyen MQTT, 
        WebSockets, y finalmente, se presenta una \textbf{demostración práctica con PubNub}. Esto prepara el terreno para su adopción posterior.
    \end{itemize}

    \item \textbf{Reconstrucción del Proyecto en la Sección 4}:
    \begin{itemize}
        \item La \textbf{Sección 4} se centra específicamente en \textbf{reconstruir el proyecto principal} para \textbf{utilizar PubNub como el principal 
        protocolo de comunicación}, en lugar de AJAX \textit{long polling}.
        \item Este cambio es una parte integral de la evolución del proyecto, buscando mejorar su arquitectura de comunicación.
    \end{itemize}

    \item \textbf{Beneficios del Cambio a PubNub}:
    \begin{itemize}
        \item La adopción de PubNub como protocolo principal, junto con el despliegue en la nube de AWS (también cubierto en la Sección 4), contribuye a 
        la creación de un \textbf{ 'ecosistema IoT fuerte, seguro, en tiempo real y escalable '}.
        \item Aunque no se  detallan directamente las \textit{ventajas específicas} de PubNub sobre AJAX en este contexto, la implicación es 
        que PubNub ofrece una solución más adecuada para la comunicación en tiempo real y bidireccional que requiere un sistema IoT moderno, superando 
        las limitaciones de las técnicas de polling como AJAX. Otros protocolos estudiados como WebSockets y MQTT se describen como bidireccionales y 
        eficientes en ancho de banda, lo que sugiere las características buscadas con la adopción de PubNub.
    \end{itemize}
\end{itemize}
En resumen, la \textbf{Sección 4} marca un punto de inflexión en el desarrollo del proyecto IoT, donde el \textbf{cambio de AJAX \textit{long polling} a 
PubNub como protocolo de comunicación principal} es una decisión clave para modernizar y mejorar el sistema, preparándolo para ser desplegado en la nube 
y logrando un ecosistema IoT más robusto y escalable.

\subsection{Despliegue del Servidor IoT en AWS}
En el contexto más amplio de la \textbf{Sección 4: Reconstrucción del Proyecto y Despliegue en la Nube}, se  indican que el 
\textbf{despliegue del servidor IoT en AWS (Amazon Web Services)} es una etapa crucial para escalar y mejorar la infraestructura del proyecto. 
Aquí se dan los detalles de este aspecto:

\begin{itemize}
    \item \textbf{Ubicación en el Curso}:
    \begin{itemize}
        \item La \textbf{Sección 4} del proyecto se dedica explícitamente a la \textbf{reconstrucción del proyecto principal} y a enseñar cómo 
        \textbf{desplegar el servidor IoT en la nube de AWS}.
    \end{itemize}

    \item \textbf{Propósito del Despliegue en la Nube}:
    \begin{itemize}
        \item Este despliegue es un componente clave para construir un \textbf{ 'ecosistema IoT fuerte, seguro, en tiempo real y escalable '}. 
        Al mover el servidor a la nube, se superan las limitaciones de ejecutarlo localmente (como se hacía inicialmente en la Sección 2), 
        permitiendo un acceso y una gestión más robustos y globales.
    \end{itemize}

    \item \textbf{Tecnologías Involucradas}:
    \begin{itemize}
        \item \textbf{AWS} se menciona como una de las principales tecnologías con las que se trabajará en el proyecto, junto con Python, Raspberry Pi, 
        Flask y PubNub. Esto subraya su importancia como pilar tecnológico en la plataforma IoT desarrollada.
    \end{itemize}

    \item \textbf{Detalles de Implementación (implícitos y futuros)}:
    \begin{itemize}
        \item Mientras que la Sección 4 introduce el concepto de desplegar en AWS, los detalles más específicos relacionados con la 
        \textbf{seguridad del servidor en la nube}, como la asignación de reglas de seguridad de entrada para HTTPS (puerto 443) en una instancia EC2 
        de AWS, se abordan en la Sección 5. Esto sugiere que la Sección 4 sienta las bases para el despliegue, y las secciones posteriores construyen 
        sobre esa infraestructura para añadir seguridad.
    \end{itemize}
\end{itemize}
En resumen, la \textbf{Sección 4} es el punto en el que el proyecto IoT existente evoluciona de una configuración local a una \textbf{infraestructura 
basada en la nube de AWS}. Este paso es esencial para la \textbf{escalabilidad, accesibilidad y robustez} del  'ecosistema IoT ' propuesto, aunque los 
aspectos de seguridad más detallados se profundizan en secciones posteriores del proyecto.

\section{Desarrollo de Seguridad}
% 2.5.DESARROLLO DE SEGURIDAD.
En el contexto más amplio de las \textbf{Secciones del Curso}, la \textbf{Sección 5: Desarrollo de Seguridad} se presenta como una fase crucial y 
directamente práctica donde se implementan los conceptos de seguridad e criptografía estudiados previamente. Es el punto donde el proyecto IoT 
previamente reconstruido y desplegado en la nube se fortalece con medidas de seguridad esenciales. Aquí se dan los detalles de esta sección:

\begin{itemize}
    \item \textbf{Implementación de Terminologías de Seguridad}:
    \begin{itemize}
        \item La Sección 5 está diseñada para involucrar principalmente el \textbf{desarrollo de todas las terminologías de seguridad} que se 
        discutieron y estudiaron en la Sección 3. Esto significa que los conceptos teóricos de seguridad en Internet y criptografía (como SSL/TLS, 
        algoritmos de clave simétrica y asimétrica, cifrado, firma digital) encuentran su aplicación práctica aquí.
    \end{itemize}
    \item \textbf{Aseguramiento del Dominio Personalizado con SSL/TLS}:
    \begin{itemize}
        \item El \textbf{primer paso} en la Sección 5 es \textbf{obtener nuestro propio nombre de dominio personalizado} y luego \textbf{asegurar 
        ese nombre de dominio} utilizando la autoridad de certificación Let's Encrypt.
        \item Esto implica un proceso detallado que incluye:
        \begin{itemize}
            \item \textbf{Instalar el software de terceros \texttt{Certbot}} en el servidor.
            \item \textbf{Actualizar la lista de paquetes} e instalar \texttt{python-certbot-apache}.
            \item \textbf{Configurar el certificado SSL para Apache} utilizando \texttt{Certbot}, proporcionando el nombre de dominio y una dirección 
            de correo
        \end{itemize}
    \end{itemize}
    \item \textbf{Funcionalidad de Inicio de Sesión Seguro y Gestión de Usuarios}:
    \begin{itemize}
        \item Posteriormente, la Sección 5 se enfoca en \textbf{implementar una funcionalidad segura de inicio de sesión de usuario} y el 
        \textbf{almacenamiento de los detalles del usuario} en una base de datos integrada.
        \item Se implementa una manera segura en la que los \textbf{usuarios y los dispositivos IoT pueden conectarse de forma segura} a nuestro 
        servidor IoT.
        \item Un aspecto clave es la \textbf{creación de reglas para usuarios administradores y no administradores}.
        \item El panel de control (donde se gestionan los permisos) solo es visible para los usuarios administradores.
        \item Se muestra una lista de usuarios en línea en el panel de administración, donde se pueden otorgar o revocar permisos de lectura y escritura.
        \item La lógica del servidor maneja la población de esta lista de usuarios, incluyendo sus IDs y estados de permisos (leído/no leído, 
        escrito/no escrito).
    \end{itemize}
    \item \textbf{Gestión de Accesos en Tiempo Real con PubNub Access Manager}:
    \begin{itemize}
        \item Se utiliza la funcionalidad \textbf{PubNub Access Manager} para permitir a los usuarios administradores \textbf{conceder en tiempo real acceso 
        de lectura y escritura} a todos los usuarios no administradores y a los dispositivos.
        \item Cuando un administrador utiliza el botón  'aplicar ', se envía una solicitud al servidor Flask para actualizar los permisos de un usuario 
        específico.
        \item El servidor verifica que la solicitud provenga de un usuario administrador y luego \textbf{almacena los permisos de lectura y escritura del 
        usuario en la base de datos}.
        \item Finalmente, se realiza una llamada al \textbf{servidor PubNub para otorgar o denegar el acceso de lectura y escritura} al usuario específico.
        \item Antes de esto, se requiere \textbf{generar una clave de autorización} para el usuario específico y almacenarla en la base de datos.
    \end{itemize}
\end{itemize}
En resumen, la \textbf{Sección 5} es la culminación práctica de los principios de seguridad. Se pasa de un servidor local a uno desplegado en la nube 
con un \textbf{dominio personalizado asegurado por SSL/TLS}, y se construye una \textbf{sólida capa de gestión de usuarios y autenticación/autorización}. 
La integración de \textbf{PubNub Access Manager} subraya el compromiso del proyecto con la \textbf{seguridad en tiempo real y la escalabilidad} dentro del 
'ecosistema IoT fuerte, seguro, en tiempo real y escalable' que se busca construir.

\subsection{Dominio Personalizado Seguro con SSL/TLS (Encriptación)}
En el contexto más amplio de la \textbf{Sección 5: Desarrollo de Seguridad}, el aseguramiento de un \textbf{Dominio Personalizado con SSL/TLS utilizando 
Let's Encrypt} es el paso fundamental inicial para establecer una base segura para el servidor IoT. Esta fase traduce los conceptos teóricos de seguridad 
estudiados previamente en una implementación práctica, garantizando que el  'ecosistema IoT ' sea robusto y confiable. Se detalla este proceso 
de la siguiente manera:

\begin{itemize}
    \item \textbf{Propósito y Ubicación en la Sección 5}:
    \begin{itemize}
        \item La Sección 5 comienza con la tarea de \textbf{asegurar un dominio personalizado recién creado con certificados SSL/TLS de Let's Encrypt}.
        \item Este es el \textbf{primer paso} en el desarrollo de la seguridad, involucrando la aplicación práctica de  'todas las terminologías de 
        seguridad' discutidas en la Sección 3.
        \item El objetivo es que \textbf{cada cliente conectado al servidor tenga una comunicación cifrada de extremo a extremo}.
    \end{itemize}
    \item \textbf{Let's Encrypt como Autoridad de Certificación}:
    \begin{itemize}
        \item Let's Encrypt es la autoridad de certificación elegida, descrita como \textbf{ 'gratuita, automatizada y de código abierto '}. Es 
        ampliamente utilizada por desarrolladores y empresas y cuenta con el respaldo de importantes patrocinadores.
        \item Los certificados generados por Let's Encrypt son \textbf{válidos por aproximadamente tres meses} y requieren renovación después de su vencimiento.
    \end{itemize}
    \item \textbf{Pasos para la Implementación de SSL/TLS}:
    \begin{enumerate}
    \item \textbf{Instalación de \texttt{Certbot}}: El primer paso técnico es \textbf{instalar software de terceros, \texttt{Certbot}}, en el servidor. 
    Esto implica agregar el repositorio de Certbot (\verb|sudo add-apt-repository ppa:certbot/certbot|).
    \item \textbf{Actualización e Instalación del Paquete}: Posteriormente, se debe \textbf{actualizar la lista de paquetes} (\verb|sudo apt-get update|) 
    y \textbf{finalmente instalar \texttt{python-certbot-apache}} desde el nuevo repositorio.
    \item \textbf{Configuración del Certificado SSL para Apache}: Una vez instalado \texttt{Certbot}, se ejecuta el cliente para \textbf{configurar el 
    certificado SSL para Apache}, usando el comando \verb|sudo certbot --apache -d [nombre-de-dominio]| y añadiendo también el subdominio \texttt{www}. 
    Se requiere una dirección de correo electrónico para la recuperación de claves y se pregunta si se desea \textbf{redirigir el tráfico HTTP a HTTPS}, 
    lo cual se debe aceptar. Los archivos de certificado se guardan en \verb|/etc/letsencrypt/|.
        \item \textbf{Configuración de Reglas de Seguridad en AWS}: Un paso crítico, a menudo pasado por alto, es \textbf{asignar reglas de seguridad de 
        entrada para HTTPS (puerto 443) en el servidor remoto de AWS (instancia EC2)}. Sin esta regla, el servidor no sería accesible a través de HTTPS, 
        lo que resultaría en un fallo en la verificación del certificado. Se debe editar la configuración de entrada para agregar esta regla.
    \end{enumerate}
    \item \textbf{Verificación y Resultado}:
    \begin{itemize}
    \item El estado del certificado SSL puede \textbf{verificarse utilizando SSL Labs} (\verb|ssllabs.com/ssltest|).
        \item Una vez que todos los pasos se han completado correctamente, el resultado es que al visitar el dominio personalizado del servidor IoT, se 
        produce una \textbf{redirección de HTTP a HTTPS, y el navegador muestra un  'candado verde ' (green lock)}, confirmando que el sitio es seguro y 
        que la comunicación está cifrada. Los detalles del certificado mostrarán que fue emitido a  'pact IOT server ' por la autoridad de Let's Encrypt.
    \end{itemize}
\end{itemize}
En resumen, la Sección 5 establece un \textbf{dominio web seguro con SSL/TLS} como el pilar fundamental del  'Desarrollo de Seguridad', garantizando una 
\textbf{comunicación cifrada de extremo a extremo} y sentando las bases para funcionalidades de seguridad más avanzadas, como el inicio de sesión de 
usuarios y la gestión de accesos.

\subsection{Funcionalidad Segura de Inicio de Sesión de Usuario}
En el contexto más amplio de la \textbf{Sección 5: Desarrollo de Seguridad}, la \textbf{Funcionalidad Segura de Inicio de Sesión de Usuario} es un 
componente esencial que sigue a la securización del dominio con SSL/TLS. Esta funcionalidad transforma el servidor IoT en una plataforma multiusuario 
donde los dispositivos y usuarios pueden interactuar de forma segura y controlada. Se detallan los siguientes aspectos clave de esta funcionalidad:

\begin{itemize}
    \item \textbf{Propósito General}:
    \begin{itemize}
        \item La Sección 5 se enfoca en la implementación de una \textbf{funcionalidad segura de inicio de sesión de usuario} y el \textbf{almacenamiento 
        de los detalles del usuario en una base de datos integrada}.
        \item El objetivo es permitir que \textbf{múltiples usuarios puedan iniciar sesión de forma segura y controlar y monitorear sus dispositivos 
        autorizados en tiempo real}.
    \end{itemize}
    \item \textbf{Implementación de Roles de Usuario (Administrador y No Administrador)}:
    \begin{itemize}
        \item Se crean \textbf{reglas para usuarios administradores y no administradores}.
        \item El \textbf{panel de control del servidor, que muestra la lista de usuarios en línea y permite gestionar permisos, es visible exclusivamente para 
        los usuarios administradores}.
        \item Para implementar esto, se utiliza una sentencia \texttt{if} en el código HTML (\verb|index.html|) que verifica si el \texttt{user ID} de la 
        sesión coincide con el \texttt{user ID} del administrador antes de mostrar el panel de control. Esto asegura que los usuarios no administradores no 
        tengan acceso a esta interfaz.
        \item Un ejemplo práctico muestra que un usuario no administrador, al iniciar sesión, no obtiene acceso al panel de control de acceso, mientras 
        que el administrador sí puede ver la lista de usuarios en línea.
    \end{itemize}
    \item \textbf{Visualización y Gestión de Usuarios en el Panel de Administración}:
    \begin{itemize}
        \item El panel de administración muestra una \textbf{lista de usuarios en línea}.
        \item Junto al nombre de cada usuario en línea, hay \textbf{botones de conmutación (switch buttons) para otorgar permisos de lectura y escritura}.
        \item Un botón de  'aplicar ' permite guardar los cambios de permisos.
        \item La interfaz (\verb|index.html|) está diseñada con secciones \verb|div| y un panel Bootstrap que contiene un encabezado para  'Online Users' 
        y una tabla (\texttt{li} dentro de \texttt{ul}) para la lista de usuarios. Cada fila muestra el nombre del usuario y los botones de lectura/escritura.
    \end{itemize}
    \item \textbf{Población Dinámica de la Lista de Usuarios}:
    \begin{itemize}
        \item La lógica en el lado del servidor (en \texttt{my DB}) recupera los usuarios conectados y sus detalles, incluyendo el ID de usuario, el nombre, 
        y los estados de permisos de lectura y escritura.
        \item Estos datos se almacenan en un mapa (\verb|online_user_records|) y se envían a la página web principal (\verb|index.html|) utilizando 
        plantillas Jinja. Se utiliza un bucle \texttt{for} en HTML para generar dinámicamente las filas de la tabla para cada usuario en línea, mostrando 
        su nombre, ID de usuario y el estado 'checked' o  'unchecked ' de los botones de permiso según los datos del servidor.
    \end{itemize}
    \item \textbf{Manejo Seguro de Permisos (Lectura/Escritura)}:
    \begin{itemize}
        \item Cuando un administrador utiliza el botón  'aplicar ', se envía una \textbf{solicitud POST desde el código JavaScript 
        (\texttt{main.js}) al servidor Flask}. Esta solicitud incluye el ID de usuario, y los estados de lectura y escritura.
        \item El servidor Flask, al recibir la solicitud, realiza una \textbf{verificación crucial para asegurar que la solicitud proviene de un 
        usuario administrador}. Si no es así, se deniega el acceso.
        \item Si la solicitud es válida (proviene de un administrador), se realizan dos acciones fundamentales:
        \begin{enumerate}
            \item \textbf{Almacenar los permisos de lectura y escritura del usuario en la base de datos}.
            \item \textbf{Realizar una llamada al servidor PubNub para otorgar o denegar el acceso de lectura y escritura a este usuario específico}.
        \end{enumerate}
        \item \textbf{Generación de Clave de Autorización}: Antes de poder conceder permisos de lectura y escritura, es necesario \textbf{generar una 
        clave de autorización para cada usuario específico y almacenarla en la base de datos}. Esta es una  'primera etapa ' esencial antes de conceder 
        permisos de lectura y escritura.
    \end{itemize}
    \item \textbf{Integración con PubNub Access Manager}:
    \begin{itemize}
        \item Se utiliza la funcionalidad de \textbf{PubNub Access Manager} para permitir a los usuarios administradores \textbf{conceder en tiempo real 
        acceso de lectura y escritura} a todos los usuarios no administradores y a los dispositivos.
    \end{itemize}
\end{itemize}
En síntesis, la \textbf{Funcionalidad Segura de Inicio de Sesión de Usuario} en la Sección 5 abarca no solo la autenticación inicial sino también un 
robusto sistema de gestión de acceso y roles. Esto incluye la visualización de usuarios en línea, la definición clara de roles 
(administrador/no administrador), la gestión de permisos en tiempo real a través de una interfaz segura y la integración con una base de datos y 
PubNub para un control de acceso granular y escalable en el ecosistema IoT.

\subsection{Almacenamiento de Detalles de Usuario en DB Integrada}
En el contexto más amplio de la \textbf{Sección 5: Desarrollo de Seguridad}, el \textbf{Almacenamiento de Detalles de Usuario en DB Integrada} 
es una parte fundamental, que se implementa después de asegurar el dominio con SSL/TLS. Este proceso es clave para habilitar una 
\textbf{funcionalidad segura de inicio de sesión de usuario} y para gestionar el acceso de múltiples usuarios a los dispositivos IoT de forma 
controlada y segura. Se describe este aspecto de la siguiente manera:

\begin{itemize}
    \item \textbf{Propósito Fundamental en la Sección 5}:
    \begin{itemize}
        \item La Sección 5 incluye la implementación de una funcionalidad segura de inicio de sesión de usuario y el \textbf{almacenamiento de los 
        detalles del usuario en una base de datos integrada}. Esto permite que múltiples usuarios puedan iniciar sesión de forma segura, controlar 
        y monitorear sus dispositivos autorizados en tiempo real.
        \item Este almacenamiento es crucial para el desarrollo de todas las terminologías de seguridad discutidas en la Sección 3, aplicándolas 
        a un ecosistema IoT.
    \end{itemize}
    \item \textbf{Detalles Específicos del Usuario Almacenados}:
    \begin{itemize}
    \item La base de datos (referida como \texttt{my DB}) es utilizada para almacenar y gestionar diversos detalles de los usuarios.
        \item Se utiliza un mapa (\verb|online_user_records|) que se \textbf{rellena con el nombre del usuario (índice 0), el ID del usuario (índice 1), 
        el estado de permiso de lectura (índice 2) y el estado de permiso de escritura (índice 3)}.
        \item Estos estados de lectura y escritura (\texttt{read} y \texttt{write}) se transforman en 'checked' o  'unchecked ' para la interfaz HTML, 
        dependiendo de si el acceso es \texttt{1} (concedido) o \texttt{0} (denegado).
        \item Antes de conceder permisos de lectura y escritura, un \textbf{ 'primer paso ' esencial es generar una clave de autorización 
        (\texttt{authorization key}) para cada usuario específico y almacenarla en la base de datos}.
    \end{itemize}
    \item \textbf{Mecanismo de Almacenamiento y Recuperación}:
    \begin{itemize}
        \item Cuando se envían datos a la página principal (\verb|index.html|), se incluyen detalles adicionales como el \verb|user_ID| de la sesión y 
        la \textbf{lista de usuarios en línea obtenida de la función \texttt{get all logged in users} de \texttt{my DB}}.
        \item La lista de usuarios en línea se construye en el lado del servidor, recuperando información de la base de datos y organizándola en un 
        formato que puede ser utilizado por las plantillas Jinja en el HTML para poblar dinámicamente la interfaz.
        \item Cuando un administrador utiliza el botón  'aplicar ' para cambiar permisos, se envía una solicitud POST al servidor Flask. Si la solicitud 
        es válida (proviene de un administrador), el servidor realiza dos acciones clave:
        \begin{enumerate}
            \item \textbf{Almacenar los permisos de lectura y escritura del usuario en la base de datos}.
            \item Realizar una llamada al servidor PubNub para otorgar o denegar el acceso de lectura y escritura a ese usuario.
        \end{enumerate}
    \end{itemize}
    \item \textbf{Rol en la Gestión de Acceso y Roles (Admin/No-Admin)}:
    \begin{itemize}
        \item La información almacenada en la base de datos es fundamental para diferenciar entre usuarios administradores y no administradores.
        \item El \textbf{panel de control, que muestra la lista de usuarios en línea y permite gestionar sus permisos, solo es visible para los usuarios 
        administradores}, utilizando una sentencia \texttt{if} que verifica el \texttt{user ID} de la sesión contra el \texttt{user ID} del administrador.
        \item Los botones de conmutación para los permisos de lectura y escritura, así como el botón  'aplicar ', interactúan directamente con la base de 
        datos para persistir los cambios.
    \end{itemize}
\end{itemize}
En síntesis, el \textbf{Almacenamiento de Detalles de Usuario en la DB Integrada} es la columna vertebral que soporta la gestión de usuarios, roles y 
permisos dentro de la Sección 5. Permite un control granular sobre quién puede acceder y qué acciones puede realizar en el ecosistema IoT, asegurando 
que la comunicación y la interacción con los dispositivos sean tanto seguras como eficientes.

\section{Conexión Segura de Usuarios Y Dispositivos}
% 2.6.CONEXIÓN SEGURA DE USUARIOS Y DISPOSITIVOS.
En el contexto más amplio de las \textbf{Secciones del Curso}, la \textbf{Sección 6: Conexión Segura de Usuarios y Dispositivos} se posiciona como una 
etapa crucial para establecer un ecosistema IoT robusto y controlado, construyendo sobre los fundamentos de seguridad y gestión de usuarios establecidos 
en secciones anteriores. Se detalla lo siguiente sobre esta sección:

\begin{itemize}
    \item \textbf{Propósito Principal}: La Sección 6 se centrará en la implementación de una \textbf{manera segura para que los usuarios y los dispositivos 
    IoT puedan conectarse de forma segura con el servidor IoT}. Esto es fundamental para garantizar que solo las entidades autorizadas puedan interactuar 
    con la plataforma y sus dispositivos.
    \item \textbf{Funcionalidad Clave}: Para lograr esta conexión segura y la gestión de permisos, la sección utilizará la \textbf{funcionalidad PubNub 
    Access Manager}. Esta herramienta permitirá a los usuarios administradores (admins) \textbf{otorgar acceso de lectura y escritura en tiempo real} a 
    todos los usuarios no administradores y a los propios dispositivos.
    \item \textbf{Relación con Secciones Anteriores}:
    \begin{itemize}
        \item \textbf{Fundamentos de Seguridad (Sección 3)}: La Sección 6 implica el desarrollo de todas las terminologías de seguridad que se discutieron 
        y estudiaron en la Sección 3. Esto sugiere que los principios de seguridad y criptografía (mencionados en la Sección 3 como temas como cifrado, 
        firmas digitales y protocolos SSL/TLS) son aplicados en la práctica en esta etapa.
        \item \textbf{Desarrollo de Seguridad (Sección 5)}: La Sección 6 sigue directamente a la Sección 5. En la Sección 5, se establecen las bases con 
        la obtención de un nombre de dominio personalizado, su aseguramiento con certificados SSL/TLS (Let's Encrypt) y la implementación de una 
        funcionalidad segura de inicio de sesión de usuario con el almacenamiento de los detalles del usuario en una base de datos integrada. 
        La interfaz de administración que permite a los usuarios administradores ver la lista de usuarios en línea y conceder permisos de lectura y 
        escritura a través de botones de conmutación también se desarrolla en la Sección 5. Por lo tanto, la Sección 6 parece ser la implementación 
        de la lógica detrás de estos permisos, utilizando PubNub.
    \end{itemize}
    \item \textbf{Contexto en el Ecosistema IoT}: Al implementar una conexión segura y una gestión de acceso granular, la Sección 6 contribuye a la 
    creación de un \textbf{ecosistema IoT fuerte, seguro, en tiempo real y escalable}. Esto es esencial para la plataforma en la que múltiples usuarios 
    pueden iniciar sesión de forma segura y controlar y monitorear sus dispositivos autorizados en tiempo real.
    \item \textbf{Transición a Secciones Posteriores}: Una vez que se ha desarrollado este  'fuerte, seguro, en tiempo real y escalable ecosistema IoT', 
    la Sección 7 se encarga de \textbf{agregar más sensores y actuadores para construir un caso de uso significativo para el mundo real}, como un 
    sistema de monitoreo atmosférico. Esto significa que la funcionalidad de conexión segura y gestión de permisos de la Sección 6 es una base 
    habilitadora para la expansión y aplicación práctica del sistema IoT.
\end{itemize}
En resumen, la Sección 6 es el punto donde la teoría de la seguridad y la gestión de usuarios converge para crear un mecanismo práctico y en tiempo 
real para controlar el acceso a los dispositivos y a la información en el ecosistema IoT, utilizando herramientas como PubNub Access Manager y 
construyendo sobre la infraestructura segura establecida en las secciones previas del proyecto.

\subsection{Conexión Segura con Servidor IoT}
En el contexto más amplio de las \textbf{Secciones del Curso}, la \textbf{Sección 6: Conexión Segura de Usuarios y Dispositivos} se dedica específicamente 
a establecer la \textbf{Conexión Segura con el Servidor IoT} para usuarios y dispositivos, construyendo sobre la infraestructura de seguridad establecida 
en las secciones previas. Se detalla los siguientes aspectos clave sobre la Conexión Segura con el Servidor IoT en la Sección 6:

\begin{itemize}
    \item \textbf{Propósito Fundamental de la Sección 6}: El objetivo principal de esta sección es implementar una \textbf{manera segura para que los 
    usuarios y los dispositivos IoT puedan conectarse de forma segura con el servidor IoT}. Esto asegura que solo las entidades autorizadas puedan 
    interactuar con la plataforma.
    \item \textbf{Tecnología Clave para la Conexión Segura}: Para lograr esta seguridad en la conexión y la gestión de permisos, la Sección 6 
    utiliza la \textbf{funcionalidad PubNub Access Manager}. Esta herramienta permite a los usuarios administradores (admins) \textbf{otorgar acceso de lectura y escritura en tiempo real} tanto a los usuarios no administradores como a los propios dispositivos.
    \item \textbf{Proceso para Conceder Permisos de Acceso}:
    \begin{itemize}
        \item \textbf{Paso Inicial: Generación de Claves de Autorización}: Antes de conceder permisos de lectura y escritura, un \textbf{ 'primer paso' 
        esencial es generar una clave de autorización (\texttt{authorization key}) para cada usuario específico y almacenarla en la base de datos}. 
        Esta clave es fundamental para el proceso de autorización.
        \item \textbf{Mecanismo de Otorgamiento de Permisos}: Cuando un administrador utiliza un botón  'aplicar' en el panel de control (desarrollado 
        en la Sección 5), se envía una solicitud al servidor Flask. Si la solicitud es válida (confirmando que proviene de un administrador), el servidor realiza dos acciones clave para la conexión segura:
        \begin{enumerate}
            \item \textbf{Almacenar los permisos de lectura y escritura del usuario en la base de datos}.
            \item \textbf{Llamar al servidor PubNub para otorgar o denegar el acceso de lectura y escritura a ese usuario específico}.
        \end{enumerate}
    \end{itemize}
    \item \textbf{Fundamentos de Seguridad Previos}: La Conexión Segura con el Servidor IoT en la Sección 6 se apoya en los desarrollos de secciones anteriores:
    \begin{itemize}
        \item \textbf{Seguridad a Nivel de Servidor (Sección 5)}: Previamente, en la Sección 5, se asegura el servidor web mismo. Se obtiene un nombre 
        de dominio personalizado y se protege con \textbf{certificados SSL/TLS (Let's Encrypt)}. Esto garantiza que \textbf{cualquier cliente conectado 
        con el servidor tendrá una comunicación cifrada de extremo a extremo}. La redirección de HTTP a HTTPS, indicada por un  'candado verde' en el 
        navegador, confirma que el sitio es seguro.
        \item \textbf{Login de Usuario Seguro y DB Integrada (Sección 5)}: También en la Sección 5, se implementa una funcionalidad de inicio de sesión 
        de usuario segura y el almacenamiento de los detalles del usuario en una base de datos integrada. Esta base de datos es utilizada para gestionar 
        los permisos que luego serán aplicados por PubNub Access Manager. El panel de control para administrar estos permisos es visible solo para usuarios 
        administradores.
        \item \textbf{Terminologías de Seguridad (Sección 3)}: La Sección 6 implica el \textbf{desarrollo de todas las terminologías de seguridad} 
        discutidas y estudiadas en la Sección 3, lo que sugiere que principios de criptografía, firmas digitales, SSL/TLS, etc., son aplicados en esta etapa.
    \end{itemize}
    \item \textbf{Impacto en el Ecosistema IoT}: Al implementar esta conexión segura y la gestión de acceso granular, la Sección 6 contribuye a la 
    creación de un \textbf{ecosistema IoT robusto, seguro, en tiempo real y escalable}. Esto es vital para una plataforma donde múltiples usuarios 
    pueden iniciar sesión de forma segura, controlar y monitorear sus dispositivos autorizados en tiempo real.
\end{itemize}
En resumen, la Conexión Segura con el Servidor IoT en la Sección 6 es el resultado de asegurar la comunicación del servidor con SSL/TLS (Sección 5), 
implementar un sistema de gestión de usuarios y permisos (Sección 5), y luego utilizar el \textbf{PubNub Access Manager} para aplicar esos permisos 
en tiempo real, controlando el acceso de usuarios y dispositivos al servidor mediante la generación de claves de autorización y la llamada al servicio PubNub.

\subsection{PubNub Access Manager (Usuarios Admin/No Admin)}
En el contexto de la \textbf{Sección 6: Conexión Segura de Usuarios y Dispositivos}, el \textbf{PubNub Access Manager} es una funcionalidad clave y 
central para la implementación de un sistema de seguridad que permite la distinción y gestión de permisos entre usuarios administradores (admins) y 
no administradores al conectarse al servidor IoT. Aquí se dan los detalles de PubNub Access Manager, especialmente en relación con los roles de 
usuarios admin y no admin:

\begin{itemize}
    \item \textbf{Propósito Fundamental en la Sección 6}: La Sección 6 tiene como objetivo implementar una \textbf{forma segura para que los usuarios y 
    dispositivos IoT puedan conectarse de forma segura con el servidor IoT}. Para lograr esto, se utiliza la \textbf{funcionalidad PubNub Access Manager}, 
    que permite a los usuarios administradores \textbf{otorgar acceso de lectura y escritura en tiempo real} tanto a los usuarios no administradores como 
    a los propios dispositivos.
    \item \textbf{Capacidades de los Usuarios Administradores (Admins)}:
    \begin{itemize}
        \item \textbf{Panel de Control Exclusivo}: Los usuarios administradores tienen acceso a un \textbf{panel de control (dashboard)} dedicado. 
        Este panel es \textbf{visible exclusivamente para ellos}.
        \item \textbf{Visualización de Usuarios en Línea}: En este panel, los administradores pueden ver una \textbf{lista de todos los usuarios en línea}. 
        Cada fila de la tabla representa un usuario, mostrando su nombre de usuario.
        \item \textbf{Gestión de Permisos en Tiempo Real}: Junto al nombre de cada usuario en línea, el panel de control muestra \textbf{botones de conmutación 
        (switch buttons)} para \textbf{conceder permisos de lectura (\texttt{read permissions}) y escritura (\texttt{write permissions})}.
        \item \textbf{Aplicación de Cambios}: Un \textbf{botón  'aplicar ' (\texttt{apply button})} permite a los administradores registrar y aplicar los 
        cambios en los permisos en tiempo real.
        \item \textbf{Verificación del Rol}: Al procesar las solicitudes de cambio de permisos, el servidor Flask \textbf{verifica que la solicitud provenga 
        de un usuario administrador}; de lo contrario, la solicitud es denegada con un mensaje de  'acceso denegado'.
    \end{itemize}
    \item \textbf{Limitaciones de los Usuarios No Administradores}:
    \begin{itemize}
        \item \textbf{Sin Acceso al Panel de Control}: Los usuarios que no son administradores \textbf{no tienen acceso al panel de control de acceso}. 
        Un ejemplo claro muestra cómo un usuario no administrador no verá este panel.
        \item \textbf{Permisos Otorgados por Admins}: Sus permisos de lectura y escritura son \textbf{gestionados y otorgados por los usuarios 
        administradores} a través del PubNub Access Manager.
    \end{itemize}
    \item \textbf{Mecanismo para Conceder Permisos con PubNub Access Manager}:
    \begin{enumerate}
        \item \textbf{Generación de Clave de Autorización}: El \textbf{primer paso} antes de conceder permisos de lectura y escritura es \textbf{generar una 
        clave de autorización (\texttt{authorization key}) para cada usuario específico y almacenarla en la base de datos}.
        \item \textbf{Interacción en el Dashboard}: Cuando un administrador manipula los botones de conmutación de lectura/escritura y presiona el botón  
        'aplicar ' en el dashboard, se envía una solicitud POST desde el código JavaScript (en \texttt{main.js}) a la aplicación Flask del servidor. Esta 
        solicitud contiene el ID del usuario objetivo, así como el estado de los permisos de lectura y escritura.
        \item \textbf{Procesamiento en el Servidor Flask}:
        \begin{itemize}
            \item El servidor Flask recibe la solicitud y, como se mencionó, \textbf{verifica que el remitente sea un usuario administrador}.
            \item Si la solicitud es válida, el servidor realiza dos acciones cruciales:
            \begin{itemize}
                \item \textbf{Almacena los permisos de lectura y escritura del usuario en la base de datos}.
                \item \textbf{Realiza una llamada al servidor PubNub para otorgar o denegar el acceso de lectura y escritura a ese usuario específico}.
            \end{itemize}
            \item Después de que se concede el acceso, el cliente puede necesitar \textbf{reiniciar su suscripción al canal PubNub}.
        \end{itemize}
    \end{enumerate}
    \item \textbf{Integración con Secciones Previas}: La implementación de esta gestión de usuarios y permisos mediante PubNub Access Manager se basa 
    en las funcionalidades desarrolladas en secciones anteriores, como el \textbf{inicio de sesión de usuario seguro y el almacenamiento de los detalles 
    del usuario en una base de datos integrada} (Sección 5). La Sección 6 implica la aplicación de todas las \textbf{terminologías de seguridad} 
    estudiadas en la Sección 3.
\end{itemize}
En definitiva, el \textbf{PubNub Access Manager} es la herramienta clave que permite a los usuarios administradores tener un \textbf{control granular 
y en tiempo real} sobre el acceso de otros usuarios y dispositivos al servidor IoT. Esto asegura que solo las entidades autorizadas puedan interactuar 
con la plataforma, contribuyendo a la creación de un \textbf{ecosistema IoT robusto, seguro, en tiempo real y escalable}, donde múltiples usuarios 
pueden iniciar sesión de forma segura y controlar o monitorear solo sus dispositivos autorizados.

\subsection{Panel de Control (Lista de Usuarios online, Permisos de Lectura/Escritura)}
En el contexto más amplio de la \textbf{Sección 6: Conexión Segura de Usuarios y Dispositivos}, el \textbf{Panel de Control} es una interfaz esencial 
que implementa la funcionalidad del PubNub Access Manager, permitiendo a los usuarios administradores gestionar de forma segura y en tiempo real el 
acceso de otros usuarios y dispositivos al servidor IoT. Se proporcionan los siguientes detalles sobre el Panel de Control, las listas de 
usuarios online y los permisos de lectura/escritura:

\begin{itemize}
    \item \textbf{Propósito y Visibilidad en la Sección 6}:
    \begin{itemize}
        \item La Sección 6 se enfoca en implementar una \textbf{forma segura para que los usuarios y dispositivos IoT se conecten con el servidor IoT}, 
        utilizando la funcionalidad del PubNub Access Manager.
        \item El Panel de Control es una \textbf{sección adicional en el dashboard} del servidor IoT, diseñada para gestionar estos permisos.
        \item Es crucial destacar que \textbf{este panel de control es visible exclusivamente para los usuarios administradores}. Si un usuario no 
        administrador intenta acceder, no lo verá.
    \end{itemize}
    \item \textbf{Contenido y Componentes del Panel de Control}:
    \begin{itemize}
        \item El panel se presenta como una nueva sección debajo de las existentes en el dashboard.
        \item Incluye una \textbf{lista de todos los usuarios online}.
        \item Cada fila de la lista muestra el \textbf{nombre de un usuario en línea}.
        \item Junto al nombre de cada usuario, hay \textbf{botones de conmutación (switch buttons)}. Uno para \textbf{otorgar permisos de lectura 
        (\texttt{read permissions})} y otro para \textbf{otorgar permisos de escritura (\texttt{write permissions})}.
        \item Los estados iniciales de estos botones (activado/desactivado) se determinan si el acceso de lectura o escritura es uno (activado) o cero 
        (desactivado) en la base de datos.
        \item También cuenta con un \textbf{botón  'aplicar ' (\texttt{apply button})} para registrar y aplicar los cambios realizados en los permisos.
        \item La estructura del panel utiliza un panel primario de Bootstrap con un encabezado para  'usuarios en línea ' y una clase de tabla como  
        'list group ', con cada fila definida por un elemento (\texttt{<li>}).
    \end{itemize}
    \item \textbf{Funcionamiento de las Listas de Usuarios Online y Permisos}:
    \begin{enumerate}
        \item \textbf{Población Dinámica de la Lista}:
        \begin{itemize}
            \item El servidor Flask es el encargado de \textbf{proporcionar el ID de usuario y la lista de usuarios online} a la página HTML principal.
            \item Una función \textbf{\texttt{get all logged in users}} en la base de datos retorna un mapa con la clave \textbf{\texttt{online users}} 
            y un valor que es una lista de registros de usuarios online.
            \item Cada registro incluye el nombre del usuario (índice 0), el ID del usuario (índice 1), y los estados de los permisos de lectura 
            (índice 2) y escritura (índice 3).
            \item La página \texttt{index.html} utiliza \textbf{plantillas Jinja} y un bucle \texttt{for} para iterar sobre la lista de usuarios 
            online y crear dinámicamente una fila para cada uno, mostrando su nombre y configurando los botones de lectura/escritura según sus 
            permisos actuales.
        \end{itemize}
        \item \textbf{Modificación de Permisos por Administradores}:
        \begin{itemize}
            \item Cuando un administrador interactúa con los botones de conmutación de lectura/escritura y presiona el botón  'aplicar', el código 
            JavaScript (\texttt{main.js}) envía una \textbf{solicitud POST} a la aplicación Flask del servidor.
            \item Esta solicitud (\texttt{send event}) contiene el ID del usuario objetivo, el estado de lectura y el estado de escritura.
            \item En la aplicación Flask, se añade un endpoint (\verb|/grant/user_id/read/write|) para recibir esta solicitud.
        \end{itemize}
        \item \textbf{Validación y Aplicación en el Servidor}:
        \begin{itemize}
            \item El servidor Flask primero \textbf{verifica que la solicitud provenga de un usuario administrador}. Si no es así, la respuesta es  
            'acceso denegado'.
            \item Si la solicitud es válida (es decir, proviene de un administrador), el servidor realiza dos acciones cruciales:
            \begin{itemize}
                \item \textbf{Almacena los nuevos permisos de lectura y escritura del usuario en la base de datos}.
                \item \textbf{Realiza una llamada al servidor PubNub para otorgar o denegar el acceso de lectura y escritura a ese usuario específico}.
            \end{itemize}
            \item Después de conceder el acceso, el cliente puede necesitar \textbf{reiniciar su suscripción al canal PubNub}.
        \end{itemize}
    \item \textbf{Paso Previo Requerido}: Antes de conceder estos permisos de lectura y escritura, el \textbf{primer paso es generar una clave de 
    autorización (\texttt{authorization key}) para ese usuario específico y almacenarla en la base de datos}.
    \end{enumerate}
\end{itemize}
En resumen, el Panel de Control con sus listas de usuarios online y controles de permisos de lectura/escritura es la \textbf{interfaz clave del PubNub 
Access Manager}. Permite a los \textbf{administradores un control en tiempo real y granular} sobre quién puede leer o escribir datos en la plataforma 
IoT, lo que es fundamental para la \textbf{conexión segura de usuarios y dispositivos} descrita en la Sección 6, asegurando que solo los usuarios 
autorizados puedan interactuar con sus dispositivos.

\section{Proyecto Final: Sistema de Monitoreo (Paciente Diabético/Nivel de Carga de Batería)}
% 2.7.PROYECTO FINAL: SISTEMA DE MONITOREO (PACIENTE DIABETICO/NIVEL DE CARGA DE BATERÍA).
La \textbf{Sección 7: Proyecto Final: Sistema de Monitoreo Atmosférico} se presenta como la culminación del proyecto  'Internet de las Cosas con Python 
y Raspberry Pi ', integrando y expandiendo los conocimientos adquiridos en las secciones anteriores para construir una solución IoT completa y 
significativa. En el contexto más amplio de las \textbf{Secciones del Curso}, la Sección 7 se posiciona como el \textbf{proyecto final}, donde se 
aplica todo lo aprendido para desarrollar un sistema funcional. Aquí se dan los detalles de la Sección 7:

\begin{itemize}
    \item \textbf{Propósito y Ubicación en el Curso}:
    \begin{itemize}
        \item Es el \textbf{ 'último proyecto '} del proyecto, denominado  'Sistema de Monitoreo Atmosférico '.
        \item Tiene como objetivo \textbf{añadir más sensores y actuadores} para construir algo significativo para un caso de uso del mundo real.
        \item Representa la fase donde se desarrolla un \textbf{ecosistema IoT robusto, seguro, en tiempo real y escalable}, utilizando las bases 
        sentadas en secciones previas, como la comunicación segura (Sección 6).
    \end{itemize}
    \item \textbf{Nuevos Conceptos y Habilidades a Aprender}:
    \begin{itemize}
        \item Durante esta sección, los estudiantes se familiarizarán con \textbf{convertidores digitales}, la \textbf{interfaz periférica serie 
        (SPI)}, y  'mucho más '. Esto sugiere la exploración de nuevas formas de interacción con hardware y adquisición de datos.
    \end{itemize}
    \item \textbf{Producto Final y Características}:
    \begin{itemize}
        \item El \textbf{producto final será un dashboard (panel de control) que mostrará gráficos visuales}.
        \item Estos gráficos presentarán \textbf{datos en tiempo real de los sensores y actuadores}.
        \item El objetivo es que este dashboard sea  'bastante agradable a la vista ' ( 'quite possible to the eyes ').
        \item El sistema buscará ser una \textbf{solución IoT integral ( 'one-stop IOT solution ')}, abarcando desde la simplicidad a nivel de 
        dispositivo hasta la complejidad de la infraestructura a nivel de la nube.
    \end{itemize}
    \item \textbf{Escalabilidad y Aplicabilidad Futura}:
    \begin{itemize}
        \item Al finalizar, los participantes serán capaces de \textbf{expandir fácilmente este proyecto}.
        \item Podrán \textbf{añadir más dispositivos y funcionalidades} según sus necesidades empresariales.
    \end{itemize}
\end{itemize}
En resumen, la Sección 7 actúa como el \textbf{gran proyecto integrador} del proyecto. Tras haber cubierto desde los fundamentos de IoT (Sección 1), 
pasando por la comunicación (Sección 2, 3 y 4) y, crucialmente, la seguridad y la gestión de acceso de usuarios (Sección 5 y 6), el  'Sistema de 
Monitoreo Atmosférico ' permite a los estudiantes aplicar todas estas habilidades para construir una solución IoT avanzada y con un \textbf{impacto 
en el mundo real}, culminando con un dashboard visual y en tiempo real.

\subsection{Mas Sensores y Actuadores}
En el contexto más amplio de la \textbf{Sección 7: Proyecto Final: Sistema de Monitoreo Atmosférico}, la adición de \textbf{ 'Más Sensores y Actuadores'} 
es un componente fundamental que permite transformar los conocimientos adquiridos en el proyecto en un sistema IoT completo y aplicable a un caso de uso real. 
Se detalla lo siguiente sobre la incorporación de más sensores y actuadores en esta sección:

\begin{itemize}
    \item \textbf{Propósito de Añadir Más Sensores y Actuadores}:
    \begin{itemize}
        \item La Sección 7 se enfoca en el \textbf{ 'último proyecto '} del proyecto, denominado  'Sistema de Monitoreo Atmosférico '.
        \item El objetivo principal es \textbf{ 'añadir más sensores y actuadores '} para construir algo \textbf{ 'significativo para un caso de uso del 
        mundo real '}. Esto indica una progresión desde proyectos más básicos (como el detector de movimiento antirrobo de la Sección 2, que utilizaba un 
        sensor PIR y un zumbador) hacia una solución más compleja y funcional.
        \item La meta es desarrollar un \textbf{ 'ecosistema IoT robusto, seguro, en tiempo real y escalable '}.
    \end{itemize}
    \item \textbf{Nuevas Tecnologías Asociadas a los Sensores y Actuadores}:
    \begin{itemize}
        \item Para la integración de estos nuevos dispositivos, los estudiantes se familiarizarán con \textbf{ 'convertidores digitales '} y la 
        \textbf{ 'interfaz periférica serie (SPI) '}. Esto sugiere que los nuevos sensores y actuadores podrían requerir métodos de comunicación y 
        procesamiento de datos más avanzados que los GPIO básicos usados anteriormente.
    \end{itemize}
    \item \textbf{Resultados y Visualización de Datos de los Sensores/Actuadores}:
    \begin{itemize}
        \item El producto final de la Sección 7 será un \textbf{ 'dashboard (panel de control) que mostrará gráficos visuales '}.
        \item Estos gráficos presentarán \textbf{ 'datos en tiempo real de los sensores y actuadores '}.
        \item El objetivo es que este dashboard sea  'bastante agradable a la vista '.
    \end{itemize}
    \item \textbf{Escalabilidad del Proyecto Final}:
    \begin{itemize}
        \item Al completar el proyecto, los participantes serán capaces de \textbf{ 'expandir fácilmente este proyecto '}, añadiendo 
        \textbf{ 'más dispositivos y funcionalidades '} según sus necesidades empresariales. Esto subraya la flexibilidad de la arquitectura 
        construida en la Sección 7 para integrar futuros sensores y actuadores.
    \end{itemize}
\end{itemize}
En resumen, la adición de \textbf{ 'Más Sensores y Actuadores '} en la Sección 7 es un paso crucial para construir un \textbf{'Sistema de Monitoreo 
Atmosférico'} que simula una aplicación IoT del mundo real. Este proyecto final no solo implica la integración de hardware más complejo (requiriendo 
el aprendizaje de convertidores digitales y SPI), sino que también culmina en la visualización en tiempo real de los datos de estos sensores y el 
ontrol de los actuadores a través de un dashboard amigable, validando la capacidad de los estudiantes para desarrollar soluciones IoT escalables y completas.

\subsection{Convertidores Digitales}
En el contexto más amplio de la \textbf{Sección 7: Proyecto Final: Sistema de Monitoreo Atmosférico}, se  indican que los 
\textbf{Convertidores Digitales} son una de las nuevas tecnologías y habilidades con las que los estudiantes se familiarizarán. Específicamente, 
se  mencionan lo siguiente:

\begin{itemize}
    \item \textbf{Introducción en la Sección 7}: La Sección 7, descrita como el  'último proyecto ' del proyecto, tiene como objetivo construir un  
    'Sistema de Monitoreo Atmosférico '. En esta sección, los participantes se familiarizarán con  'los convertidores digitales, la interfaz periférica 
    serie y mucho más '. Esto sugiere que la comprensión y el uso de convertidores digitales son esenciales para el desarrollo de este proyecto final.
    \item \textbf{Contexto de Nuevos Sensores y Actuadores}: La Sección 7 se centra en  'añadir más sensores y actuadores y construir algo significativo 
    para un caso de uso del mundo real '. La mención de  'convertidores digitales ' junto con la  'interfaz periférica serie (SPI) ' implica que los 
    nuevos sensores y actuadores que se incorporarán en este proyecto podrían ser de naturaleza más compleja o requerir una forma específica de 
    interconexión con la Raspberry Pi, como la conversión de señales analógicas a digitales.
    \item \textbf{Desarrollo de un Ecosistema IoT Robusto}: El proyecto busca desarrollar un  'ecosistema IoT robusto, seguro, en tiempo real y escalable'. 
    La introducción de convertidores digitales es fundamental para lograr esta robustez, ya que permite la integración de una gama más amplia de sensores 
    que pueden generar datos analógicos, transformándolos en un formato digital procesable por los dispositivos IoT.
\end{itemize}
En resumen, los \textbf{Convertidores Digitales} son un concepto técnico clave introducido en la \textbf{Sección 7} que permitirá a los estudiantes 
interactuar con \textbf{sensores y actuadores más avanzados}. Su estudio y aplicación son fundamentales para la recolección de datos de manera efectiva 
en el  'Sistema de Monitoreo Atmosférico ', facilitando la creación de un dashboard que muestre  'datos en tiempo real de los sensores y actuadores' 
y expandiendo las capacidades del proyecto para incluir una diversidad de componentes del mundo real.

\subsection{Iinterfaz Periférica Serial}
En el contexto más amplio de la \textbf{Sección 7: Proyecto Final: Sistema de Monitoreo Atmosférico}, se  indican que la 
\textbf{Interfaz Periférica Serial (SPI)} es una de las nuevas tecnologías y habilidades con las que los estudiantes se familiarizarán. 
Específicamente, se  mencionan lo siguiente:

\begin{itemize}
    \item \textbf{Introducción en la Sección 7}: La Sección 7, descrita como el  'último proyecto ' del proyecto, tiene como objetivo construir un 
     'Sistema de Monitoreo Atmosférico '. En esta sección, los participantes se familiarizarán con  'los convertidores digitales, la interfaz periférica serie 
    y mucho más '. Esto sugiere que la comprensión y el uso de la Interfaz Periférica Serial son esenciales para el desarrollo de este proyecto final.
    \item \textbf{Contexto de Nuevos Sensores y Actuadores}: La Sección 7 se centra en  'añadir más sensores y actuadores y construir algo significativo para un 
    caso de uso del mundo real '. La mención de  'interfaz periférica serie (SPI) ' junto con los  'convertidores digitales ' implica que los nuevos sensores y 
    actuadores que se incorporarán en este proyecto podrían ser de naturaleza más compleja o requerir una forma específica de interconexión con la Raspberry Pi, 
    como la comunicación en serie.
    \item \textbf{Desarrollo de un Ecosistema IoT Robusto}: El proyecto busca desarrollar un  'ecosistema IoT robusto, seguro, en tiempo real y escalable '. 
    La introducción de la Interfaz Periférica Serial es fundamental para lograr esta robustez, ya que permite la comunicación eficiente con múltiples 
    dispositivos periféricos, como sensores y convertidores, utilizando menos pines que las interfaces paralelas.
\end{itemize}
En resumen, la \textbf{Interfaz Periférica Serial (SPI)} es un concepto técnico clave introducido en la \textbf{Sección 7} que permitirá a los estudiantes 
interactuar con \textbf{sensores y actuadores más avanzados}. Su estudio y aplicación son fundamentales para la comunicación efectiva entre los componentes 
en el  'Sistema de Monitoreo Atmosférico ', facilitando la creación de un dashboard que muestre  'datos en tiempo real de los sensores y actuadores' y 
expandiendo las capacidades del proyecto para incluir una diversidad de componentes del mundo real.

\subsection{Panel con Gráficos Visuales en Tiempo Real}
Se indica que el \textbf{Panel con Gráficos Visuales en Tiempo Real} es una característica central y el producto final del proyecto, desarrollado 
en el contexto más amplio de la sección 7: \textbf{Proyecto Final: Sistema de Monitoreo Atmosférico}.
A continuación, se detalla este aspecto:

\begin{itemize}
    \item \textbf{Producto final del Curso}: El proyecto concluye con un  'dashboard ' que muestra gráficos visuales con los datos de los sensores y actuadores 
    en tiempo real. Se describe como una interfaz  'bastante agradable a la vista '.
    \item \textbf{Objetivo de la Sección 7}: La Sección 7 se dedica a añadir más sensores y actuadores para construir un caso de uso significativo en el 
    mundo real. El Sistema de Monitoreo Atmosférico es este proyecto final, y el panel de control con gráficos en tiempo real es la interfaz que presentará 
    los datos recogidos por estos dispositivos.
    \item \textbf{Contexto de Desarrollo}: Este panel se construye después de haber desarrollado un ecosistema IoT robusto, seguro, en tiempo real y escalable. 
    La creación de este  'dashboard ' representa la culminación de los conocimientos adquiridos en el proyecto, abarcando desde la simplicidad a nivel de 
    dispositivo hasta la complejidad de la infraestructura en la nube.
    \item \textbf{Capacidad de Ampliación}: Al finalizar el proyecto, los participantes estarán capacitados para expandir este proyecto añadiendo más 
    dispositivos y funcionalidades según sus propias necesidades.
\end{itemize}
En resumen, el Panel con Gráficos Visuales en Tiempo Real en la Sección 7 no es solo una parte, sino la representación visual y operativa del Sistema de 
Monitoreo Atmosférico, consolidando todas las habilidades y tecnologías aprendidas para construir una solución IoT completa y funcional.

\chapter{PRIMER PROYECTO IoT: DETECTOR DE MOVIMIENTO.}
% 3.PRIMER PROYECTO IoT: DETECTOR DE MOVIMIENTO
Se describe el Primer Proyecto IoT: Detector de Movimiento (Anti-theft motion detector) como el primer proyecto práctico del proyecto 
'Internet de las Cosas con Python y Raspberry Pi ', ubicado específicamente en la Sección 2.
En el contexto más amplio del proyecto, este proyecto sirve como una introducción práctica al desarrollo de IoT, donde los participantes  
'se ensuciarán las manos ' creando un sistema simple con sensores que muestran datos a través de una aplicación web.

A continuación, se detallan los aspectos claves del proyecto:
\begin{itemize}
    \item \textbf{Propósito y Funcionalidad}:
    \begin{itemize}
        \item El objetivo es construir un  dispositivo IoT de defensa contra robos.
        \item Detecta movimiento, activa una alarma y envía alertas al usuario, estableciendo una comunicación unidireccional del servidor al usuario.
        \item También incluye una funcionalidad para que los usuarios desactiven la alarma, logrando así una comunicación bidireccional. 
    \end{itemize}
    \item \textbf{Componentes de Hardware}:
    \begin{itemize}
        \item Sensor PIR (Passive Infrared): Este es el módulo detector de movimiento.
            \begin{itemize}
                \item Utiliza un elemento sensor llamado RE200BL, un sensor piroeléctrico que genera energía al exponerse al calor.
                \item Detecta el movimiento de cuerpos humanos o animales al captar la energía de calor (radiación infrarroja) que emiten.
                \item Tiene tres pines: tierra (ground), VCC (para 5 voltios de alimentación) y un pin de salida que da un nivel lógico alto si se detecta un 
                objeto.
                \item Incluye dos potenciómetros: uno para ajustar la sensibilidad y otro para ajustar el tiempo que la señal de entrada permanece en alto 
                tras la detección (de 0.3 segundos a 5 minutos).
                \item También tiene pines para seleccionar modos de disparo (trigger modes):  'non repeatable trigger ' (la salida vuelve a bajo 
                automáticamente tras el tiempo de retardo) y  'repeatable trigger ' (la salida se mantiene en alto mientras el objeto detectado esté presente).
            \end{itemize}
        \item Buzzer (Zumbador): Se activa como una alarma.
            \begin{itemize}
                \item Consiste en una carcasa exterior con tres pines (VCC, tierra y señal) y un elemento piezoeléctrico.
                \item Produce sonido al aplicar corriente, lo que causa que el disco metálico vibre.
                \item Se controlará generando una  onda cuadrada , alternando el pin de señal entre alto y bajo con pequeñas pausas.
            \end{itemize}
    \end{itemize}
    \item \textbf{Funcionamiento Técnico y Software}:
    \begin{itemize}
        \item Se conectarán el sensor PIR y el buzzer a una Raspberry Pi.
        \item Se escribirá código Python básico para detectar señales altas/bajas del movimiento y controlar el buzzer.
        \item Se desarrollará un servidor web HTTP básico en Python Flask que correrá en la Raspberry Pi dentro de la red Wi-Fi local.
        \item Para la comunicación, se utilizará la técnica AJAX.
        \item El usuario accede a una página web a través de la dirección IP local de la Raspberry Pi.
        \item Una vez cargada la página, el navegador del usuario enviará solicitudes  'keep-alive ' periódicamente (cada cinco segundos) al servidor web de la 
        Raspberry Pi. La respuesta a estas solicitudes incluye el estado y los datos del sensor, proporcionando actualizaciones en vivo al usuario.
        \item Un botón en la página web permitirá a los usuarios controlar actuadores, como desactivar el buzzer.
        \item Inicialmente, el servidor funciona localmente, lo que significa que los usuarios deben estar conectados a la misma red Wi-Fi para acceder a él.
    \end{itemize}
    \item \textbf{Preparación y Configuración del Entorno}:
    \begin{itemize}
        \item Se requiere una tarjeta SD de al menos 8GB para instalar el sistema operativo en la Raspberry Pi.
        \item Se puede usar un monitor HDMI o acceso remoto (Remote Desktop).
        \item Para Raspberry Pi 3, el SSH está deshabilitado por defecto y debe activarse mediante el comando \verb|sudo raspi-config| en la terminal.
    \end{itemize}
\end{itemize}
Este Primer Proyecto IoT establece las bases del proyecto, sentando las primeras experiencias con hardware, programación de sensores y creación de servidores 
web básicos, para luego evolucionar hacia conceptos más avanzados de comunicación y seguridad IoT en secciones posteriores.

\section{Descripción del Proyecto}
% 3.1.DESCRIPCIóN DEL PROYECTO
Se describe el Primer Proyecto IoT: Detector de Movimiento como un proyecto práctico introductorio del proyecto  'Internet de las Cosas con Python y 
Raspberry Pi ', ubicado en la Sección 2.

En el contexto más amplio de este primer proyecto, la descripción general se centra en la construcción de un dispositivo IoT funcional que aborda un 
caso de uso real:

\begin{itemize}
    \item \textbf{Propósito Principal}
    \begin{itemize}
        \item El objetivo es construir un dispositivo IoT de defensa contra robos.
    \end{itemize}
    \item \textbf{Funcionalidad Central}
    \begin{itemize}
        \item Detección de Movimiento: El dispositivo detectará movimiento utilizando un sensor PIR.
        \item Activación de Alarma: Tras la detección de movimiento, se activará una alarma, utilizando un zumbador (buzzer).
        \item Envío de Alertas: El sistema enviará alertas al usuario, estableciendo una comunicación unidireccional del servidor al usuario.
        \item Desactivación Remota: Para lograr una comunicación bidireccional, se añadirá una funcionalidad que permitirá a los usuarios desactivar la alarma 
        desde una interfaz web.
    \end{itemize}
    \item \textbf{Visión General Técnica}
    \begin{itemize}
        \item El proyecto implica la conexión de un sensor PIR y un buzzer a una Raspberry Pi.
        \item Se utilizará código Python básico para la detección de señales y el control del buzzer.
        \item Se desarrollará un servidor web HTTP básico con Flask en Python que se ejecutará en la Raspberry Pi, accesible a través de la red Wi-Fi local.
        \item Para la comunicación entre el cliente (navegador web) y el servidor, se empleará la técnica AJAX, enviando solicitudes  'keep-alive' 
        periódicamente (cada cinco segundos) para recibir actualizaciones en vivo del estado y datos del sensor.
        \item La interfaz web incluirá un botón para que los usuarios puedan controlar los actuadores, como desactivar el buzzer.
    \end{itemize}
    \item \textbf{Experiencia del Usuario}
    \begin{itemize}
        \item La página web mostrará el estado de la detección de movimiento y el estado de la conexión, indicando si la conexión está activa.
    \end{itemize}
        \item \textbf{Naturaleza Introductoria}
    \begin{itemize}
        \item Este proyecto es la primera oportunidad para que los participantes  'se ensucien las manos ' con el desarrollo IoT, aprendiendo a usar sensores y 
        actuadores en un entorno y a mostrar datos a través de una aplicación web.
    \end{itemize}
\end{itemize}
En resumen, la descripción del proyecto destaca la creación de un sistema de seguridad básico pero funcional, sentando las bases para una comprensión práctica 
de los componentes de hardware, la programación y la comunicación web en el ámbito del IoT.

\subsection{Dispositivo IoT de Defensa Antirrobo}
El Dispositivo IoT de Defensa Antirrobo es el proyecto inicial y fundamental desarrollado en el proyecto, diseñado para detectar movimiento y activar una 
alarma, integrándose en un contexto más amplio de una plataforma IoT segura y controlable por múltiples usuarios.

A continuación, se dan los detalles de este dispositivo en el contexto de la descripción del proyecto:

\begin{itemize}
\item Propósito Principal.
    \begin{itemize}
        \item El objetivo es crear un  'dispositivo IoT de defensa contra robos que detectará movimiento '.
        \item Basado en la detección de movimiento, el dispositivo  'activará una alarma '.
        \item También  'enviará alertas al usuario ', estableciendo una comunicación unidireccional del servidor al usuario.
        \item Además, incorpora una funcionalidad para  'desactivar la alarma ' por parte del usuario, lo que permite la comunicación bidireccional del 
        usuario al servidor.
    \end{itemize}
\item Componentes Principales.
    \begin{itemize}
        \item Sensor PIR (Infrarrojo Pasivo): Es el módulo detector de movimiento.
        \begin{itemize}
            \item Utiliza un elemento sensor llamado RE200BL, un sensor piezoeléctrico que genera energía cuando se expone al calor.
            \item Detecta movimiento porque los humanos o animales emiten energía térmica en forma de radiaciones infrarrojas.
            \item Es  'pasivo ' porque no usa energía para detectar, solo detecta la energía emitida por otros objetos.
            \item Incluye una tapa de plástico para expandir el área de cobertura de detección.
            \item Tiene tres pines: tierra, VCC (5 voltios) para alimentación y un pin de salida que emite un nivel lógico alto si se detecta un objeto 
            y viceversa.
            \item Dispone de dos potenciómetros: uno para ajustar la sensibilidad y otro para ajustar el tiempo que la señal de entrada permanece alta 
            (desde 0.3 segundos hasta 5 minutos).
            \item Ofrece  dos modos de disparo :  'no repetible ' (la salida cambia de alta a baja una vez finalizado el tiempo de retardo) y  'repetible ' 
            (la salida permanece alta mientras el objeto detectado esté presente).
        \end{itemize}
        \item Zumbador (Buzzer): Actúa como el sistema de alarma.
        \begin{itemize}
            \item Consiste en una carcasa exterior con tres pines: VCC (5 voltios), tierra y señal.
            \item En su interior, un elemento piezoeléctrico rodeado por un disco de vibración metálico, que al recibir corriente, se contrae y expande, 
            generando sonido.
            \item Se controla generando una onda cuadrada; se activa la señal (HIGH), se espera unos milisegundos, se desactiva (LOW) y se repite el proceso.
        \end{itemize}
    \end{itemize}

\item Implementación y Funcionamiento.
    \begin{itemize}
        \item Los sensores y actuadores (PIR y buzzer) se conectan a una Raspberry Pi.
        \item Se escribe código Python para detectar las señales del sensor PIR y controlar el buzzer.
        \item Se implementa un servidor web HTTP básico con Flask en la Raspberry Pi que corre en la red Wi-Fi local.
        \item El usuario accede a una página web desde su navegador utilizando la dirección IP de la Raspberry Pi (ej.  '192.168.1.250 ').
        \item Una vez cargada la página, el navegador del usuario envía solicitudes  'keepalive ' (mantener vivo) al servidor cada cinco segundos; 
        estas solicitudes actúan como un  'latido ' para asegurar la conexión.
        \item En cada respuesta keepalive, el servidor envía el estado del sensor y los datos al usuario, lo que permite actualizaciones en tiempo real.
        \item Se añade un botón en la página web que permite a los usuarios  'controlar los actuadores ', específicamente para  'desactivar el zumbador '.
        \item Inicialmente, el servidor funciona localmente, por lo que los usuarios solo pueden acceder desde la misma red Wi-Fi.
    \end{itemize}

\item Contexto en la Descripción General del Proyecto.

    \begin{itemize}
        \item Este proyecto es la primera experiencia práctica en la Sección 2, donde se usa un sensor para mostrar datos en una aplicación web.
        \item Se utiliza la técnica AJAX para la comunicación entre el cliente y el servidor en esta etapa.
        \item La página web inicial muestra un  'panel negro que muestra el estado de la detección de movimiento ' y un  'estado de conexión' en la 
        parte superior.
        \item El proyecto general busca construir una  'plataforma en la nube escalable donde múltiples usuarios pueden iniciar sesión de forma segura y 
        controlar y monitorear sus dispositivos autorizados en tiempo real '. El dispositivo antirrobo es el primer ejemplo de un  'dispositivo autorizado ' 
        dentro de este ecosistema.
        \item Posteriormente, en secciones avanzadas, el proyecto se reconstruirá para usar protocolos de comunicación más ligeros como PubNub, y se desplegará 
        en la nube de AWS. Finalmente, se añadirán más sensores y actuadores para casos de uso más complejos, como un sistema de monitoreo atmosférico.
    \end{itemize}

\end{itemize}

\subsection{Detección de Movimiento}
La detección de movimiento es la funcionalidad central del primer proyecto IoT desarrollado en el proyecto, el Dispositivo IoT de Defensa Antirrobo. 
Este proyecto inicial sienta las bases para una plataforma IoT más compleja y escalable.
A continuación, se dan los detalles de la detección de movimiento en el contexto de la descripción del proyecto:
\begin{itemize}
    \item \textbf{El Sensor PIR como Base de la Detección de Movimiento}
    \begin{itemize}
        \item Identificación del Sensor: El componente clave para la detección de movimiento es el módulo detector de movimiento PIR (Infrarrojo Pasivo).
    \end{itemize}

    \item \textbf{Principios de Funcionamiento}:
    \begin{itemize}
        \item Utiliza un elemento sensor llamado RE200BL, que es un sensor piezoeléctrico que genera energía cuando se expone al calor.
        \item La detección de movimiento ocurre porque los cuerpos humanos o animales emiten energía térmica en forma de radiaciones infrarrojas. 
        Cuando una persona o animal entra en el rango del sensor, este detecta ese movimiento.
        \item El término  'pasivo ' significa que el sensor no utiliza energía propia para detectar, sino que funciona detectando la energía emitida por 
        otros objetos.
        \item Incorpora una tapa de plástico especialmente diseñada que se usa para expandir el área de cobertura de detección.
    \end{itemize}

    \item \textbf{Conectividad y Salida}
    \begin{itemize}
        \item El módulo tiene tres pines: tierra, VCC (5 voltios) para alimentación y un pin de salida que emite un nivel lógico alto si se detecta un 
        objeto y viceversa.
    \end{itemize}

    \item \textbf{Configuración y Modos de Disparo}
    \begin{itemize}
        \item Dispone de dos potenciómetros: uno para ajustar la sensibilidad del sensor y otro para ajustar el tiempo que la señal de entrada permanece alta 
        después de detectar un objeto, ajustable desde 0.3 segundos hasta 5 minutos.
        \item Ofrece dos modos de disparo mediante un jumper.
            \begin{itemize}
                \item No repetible: La salida cambia de alta a baja una vez finalizado el tiempo de retardo.
                \item Repetible: La salida permanece alta todo el tiempo mientras el objeto detectado esté presente dentro del rango del sensor.
            \end{itemize}
    \end{itemize}
\end{itemize}

\begin{itemize}
    \item Detección de Movimiento en el Contexto del Proyecto Inicial  'Antirrobo '.
    \begin{itemize}
        \item Propósito del Dispositivo: El objetivo principal es construir un  'dispositivo IoT de defensa contra robos que detectará movimiento '.
    \end{itemize}

    \item Acciones Post-Detección.
    \begin{itemize}
        \item Basado en la detección de movimiento, el dispositivo  'activará una alarma ' (a través de un zumbador).
        \item También  'enviará alertas al usuario '
    \end{itemize}

    \item Implementación Técnica Inicial.
    \begin{itemize}
        \item El sensor PIR se conecta a una Raspberry Pi, donde se escribe código Python para  'detectar las señales altas y bajas al detectar movimiento ' 
        y controlar el zumbador.
        \item Se implementa un servidor web HTTP básico con Flask en la Raspberry Pi que funciona en la red Wi-Fi local.
        \item La página web inicial, a la que el usuario accede desde su navegador (por ejemplo, usando la IP de la Raspberry Pi), muestra un  'panel negro 
        que muestra el estado de la detección de movimiento '.
        \item El servidor envía el  'estado del sensor y los datos al usuario ' en las respuestas a las solicitudes  'keepalive ' (mantener vivo) que el 
        navegador envía periódicamente (cada cinco segundos), proporcionando actualizaciones en tiempo real.
    \end{itemize}

    \item Comunicación Bidireccional: El proyecto incluye la funcionalidad para que el usuario pueda  'desactivar la alarma ' desde la página web, completando 
    la comunicación bidireccional entre el usuario y el servidor.
\end{itemize}

\begin{itemize}
    \item Detección de Movimiento en el Contexto Más Amplio de la Descripción del Proyecto.
    \begin{itemize}
        \item Fundamento del Curso: La creación de este dispositivo antirrobo es la  'primera experiencia práctica ' del proyecto, permitiendo a los estudiantes 
        interactuar con un sensor y mostrar sus datos en una aplicación web.
        \item Evolución hacia una Plataforma Escalable: Aunque el proyecto inicial se ejecuta localmente y utiliza AJAX para la comunicación, la detección de 
        movimiento es un caso de uso fundamental para la meta general del proyecto: construir una  'plataforma en la nube escalable donde múltiples usuarios 
        pueden iniciar sesión de forma segura y controlar y monitorear sus dispositivos autorizados en tiempo real '.
        \item Futuras Mejoras: Se indica que, después de este proyecto inicial, se estudiarán y adoptarán protocolos de comunicación más ligeros y en 
        tiempo real como PubNub (en lugar de HTTP/AJAX) y se implementará en la nube de AWS. Esto demuestra cómo la funcionalidad básica de detección de 
        movimiento se integra en una arquitectura IoT más avanzada y segura, capaz de manejar múltiples dispositivos y usuarios.
        \item Ampliación de Casos de Uso: El proyecto culminará en la sección 7 con un proyecto más complejo llamado  'sistema de monitoreo atmosférico ', lo que 
        subraya que la detección de movimiento es solo un ejemplo de las muchas funcionalidades basadas en sensores que se pueden integrar en la plataforma IoT 
        desarrollada.
    \end{itemize}

    \item En resumen, la detección de movimiento, facilitada por el sensor PIR, es el pilar del primer proyecto práctico del proyecto. Este proyecto de defensa 
    antirrobo, aunque simple en sus inicios (servidor local, AJAX), es crucial para entender los conceptos básicos de IoT y sienta las bases para el desarrollo 
    de una plataforma en la nube segura, escalable y en tiempo real que puede monitorear y controlar diversos dispositivos, incluyendo aquellos que detectan 
    movimiento.
\end{itemize}

\subsection{Activación de la Alarma}
La activación de alarma es una funcionalidad central del primer proyecto práctico del proyecto, el Dispositivo IoT de Defensa Antirrobo, y se integra como un 
ejemplo clave de control de actuadores dentro del marco más amplio de una plataforma IoT escalable y segura.
A continuación, se dan los detalles de la activación de alarma en el contexto de la descripción del proyecto:
\begin{itemize}
    \item \textbf{Mecanismo de Activación y Propósito Principal}
    \begin{itemize}
        \item \textbf{Detección de Movimiento como Disparador}
        \begin{itemize}
            \item El dispositivo IoT de defensa antirrobo está diseñado para detectar movimiento, y en base a ello, activará una alarma. Esta es la acción 
            directa resultante de la detección de un intruso.
        \end{itemize}

        \item \textbf{Envio de Alertas al Usuario}
        \begin{itemize}
            \item Además de la alarma sonora, el sistema también enviará alertas al usuario, lo que cumple con una comunicación unidireccional del servidor al 
            usuario.
        \end{itemize}
    \end{itemize}

    \item \textbf{El Actuador de Alarma: El Zumbador (Buzzer)}
    \begin{itemize}
        \item \textbf{Componente Físico}
        \begin{itemize}
            \item La alarma se activa mediante un zumbador (buzzer).
            \begin{itemize}
                \item El buzzer consta de una carcasa exterior con tres pines: VCC (5 voltios), tierra y señal.
                \item En su interior, tiene un elemento piezoeléctrico rodeado por un disco de vibración metálico que genera sonido al recibir corriente.
            \end{itemize}
        \end{itemize}

        \item \textbf{Control de Sonido}
        \begin{itemize}
            \item La frecuencia del zumbador puede cambiarse para modificar la velocidad de vibración del disco, lo que a su vez altera el tono del sonido 
            generado.
        \end{itemize}

        \item \textbf{Control Mediante Onda Cuadrada}
        \begin{itemize}
            \item Para controlar el zumbador, se genera una onda cuadrada. En términos simples, esto implica alternar el pin de señal a un nivel alto (HIGH), 
            esperar unos milisegundos, luego a un nivel bajo (LOW), esperar otros milisegundos y repetir el proceso.
        \end{itemize}
    \end{itemize}

    \item \textbf{Implementación Técnica en el Proyecto Inicial}
    \begin{itemize}
        \item \textbf{Conexión a Raspberry Pi}
        \begin{itemize}
            \item El zumbador se conecta a una Raspberry Pi junto con el sensor PIR.
        \end{itemize}

        \item \textbf{Código Python}
        \begin{itemize}
            \item Se escribe código Python para detectar las señales altas y bajas al detectar movimiento y controlar el zumbador.
        \end{itemize}

        \item \textbf{Servidor Web y Dashboard}
        \begin{itemize}
            \item  El dispositivo inicial utiliza un servidor web HTTP básico con Flask en la Raspberry Pi. Aunque la página web muestra el estado 
            de la detección de movimiento, la activación de la alarma ocurre en el dispositivo.
        \end{itemize}

        \item \textbf{Control del Usuario para Desactivación}
        \begin{itemize}
            \item Para permitir una comunicación bidireccional del usuario al servidor, se añade una funcionalidad en la que los usuarios pueden desactivar 
            la alarma.
            \begin{itemize}
                \item Esto se implementa con un botón de interruptor en la página web que permite a los usuarios controlar los actuadores, específicamente 
                para desactivar el zumbador.
                \item Este botón es parte de la sección de caja negra del dashboard web que muestra el estado de movimiento.
            \end{itemize}
        \end{itemize} 
    \end{itemize}

    \item \textbf{La Activación de Alarma en el Contexto Más Amplio del Proyecto}
    \begin{itemize}
        \item \textbf{Primera Experiencia Práctica}
        \begin{itemize}
            \item La implementación de la activación y desactivación de la alarma es parte de la primera experiencia práctica del proyecto en la Sección 2, 
            donde se utilizan sensores y se muestran datos en una aplicación web, demostrando la interacción cliente-servidor a través de la técnica AJAX.
        \end{itemize}

        \item \textbf{Evolución hacia una Plataforma Segura y Escalable}
        \begin{itemize}
            \item Aunque inicialmente el servidor se ejecuta localmente, el proyecto de defensa antirrobo sienta las bases para una plataforma en la nube 
            escalable donde múltiples usuarios pueden iniciar sesión de forma segura y controlar y monitorear sus dispositivos autorizados en tiempo real.
        \end{itemize}

        \item \textbf{Seguridad y Control de Acceso}
        \begin{itemize}
            \item En secciones posteriores del proyecto, se implementarán características de seguridad como la protección de dominios con certificados SSL/TLS, un 
            login de usuario seguro, y un sistema de gestión de acceso donde los usuarios administradores pueden otorgar permisos de lectura y escritura en tiempo 
            real a usuarios no administradores y dispositivos. Esto significa que la capacidad de activar o desactivar la alarma, o cualquier otro actuador, se 
            integrará en un sistema de permisos detallado y seguro, gestionado a través de un dashboard de administrador.
        \end{itemize}

        \item \textbf{Protocolos de Comunicación Avanzados}
        \begin{itemize}
            \item La funcionalidad de la alarma, que inicialmente se comunica a través de solicitudes HTTP keepalive y AJAX, eventualmente se reconstruirá para 
            usar protocolos de comunicación en tiempo real y más ligeros como PubNub. Esto mejorará la eficiencia y la capacidad de respuesta para la activación 
            y control de la alarma en un entorno IoT distribuido.
        \end{itemize}
    \end{itemize}
\end{itemize}

\subsection{Envío de Alertas (Unidireccional)}
El envío de alertas unidireccionales es una funcionalidad fundamental e inicial en el desarrollo del proyecto IoT, particularmente en el Dispositivo IoT 
de Defensa Antirrobo de la Sección 2. Este tipo de comunicación es el primer paso para permitir al usuario monitorear el estado de sus dispositivos.

Aquí se dan los detalles de el envío de alertas unidireccionales en el contexto más amplio de la descripción del proyecto:

\begin{itemize}
    \item \textbf{Definición y Propósito Inicial}
    \begin{itemize}
        \item \textbf{Comunicación Unidireccional}
        \begin{itemize}
            \item El proyecto inicial de defensa antirrobo está diseñado para enviar alertas al usuario, lo que cumple con la comunicación unidireccional del 
            servidor al usuario.
        \item
        \end{itemize}
        \item \textbf{Activación por Evento}
        \begin{itemize}
            \item Estas alertas se envían en base a la detección de movimiento realizada por el sensor PIR. Así, cuando se detecta un intruso, el dispositivo no 
            solo activa una alarma local, sino que también informa al usuario.
        \end{itemize}
    \end{itemize}

    \item \textbf{Implementación Técnica Inicial}
    \begin{itemize}
        \item \textbf{Servidor Web Local y Raspberry Pi}
        \begin{itemize}
            \item Las alertas son gestionadas por un servidor web HTTP básico con Flask en la Raspberry Pi. 
        \end{itemize}

        \item \textbf{Solicitudes Keepalive y AJAX}
            \begin{itemize}
                \item Una vez que el usuario carga la página web en su navegador, se envían solicitudes keepalive desde el navegador del usuario al servidor 
                web de la Raspberry Pi periódicamente cada cinco segundos.
            \end{itemize}

        \item \textbf{Envío de Estado y Datos}
        \begin{itemize}
            \item La respuesta a cada solicitud keepalive se considera un latido para asegurar la conexión. En estas respuestas, el servidor también envía 
            el estado del sensor y los datos al usuario, lo que permite que el usuario reciba actualizaciones en tiempo real sobre la detección de 
            movimiento.
        \end{itemize}

        \item \textbf{Interfaz de Usuario}
        \begin{itemize}
            \item La página web inicial, que es un panel negro, muestra el estado de la detección de movimiento. Aunque no se especifica si las alertas 
            son una notificación visual explícita o la simple actualización del estado del sensor, la comunicación de estado y datos constituye la base 
            de la alerta.
        \end{itemize}
    \end{itemize}

    \item \textbf{Rol en el Contexto Más Amplio del Proyecto}
    \begin{itemize}
        \item \textbf{Paso Fundamental}
        \begin{itemize}
            \item El envío unidireccional de alertas es parte de la primera experiencia práctica del proyecto, donde los estudiantes aprenden a usar sensores y 
            mostrar datos en una aplicación web utilizando la técnica AJAX para la comunicación cliente-servidor.
        \end{itemize}
        \item \textbf{Transición a la Comunicación Bidireccional}
        \begin{itemize}
            \item Si bien el envío de alertas comienza siendo unidireccional (servidor al usuario), el mismo proyecto inicial incorpora la funcionalidad para que 
            el usuario pueda desactivar la alarma desde la página web, completando así la comunicación bidireccional del usuario al servidor. 
            Esto demuestra que la unidireccionalidad es una etapa temprana en el camino hacia sistemas más interactivos.
        \end{itemize}
        \item \textbf{Evolución hacia Protocolos en Tiempo Real}
        \begin{itemize}
            \item Las limitaciones de la comunicación basada en HTTP/AJAX para alertas y actualizaciones se abordan en secciones posteriores. El proyecto explora y 
            adopta protocolos de comunicación en tiempo real y ligeros como WebSockets y MQTT, y específicamente PubNub, para mejorar la eficiencia y la 
            inmediatez de la comunicación, reemplazando las solicitudes keepalive.
        \end{itemize}
        \item \textbf{Integración en una Plataforma Escalable y Segura}
        \begin{itemize}
            \item La capacidad de enviar alertas se integra en la visión general del proyecto de construir una plataforma en la nube escalable donde múltiples 
            usuarios pueden iniciar sesión de forma segura y controlar y monitorear sus dispositivos autorizados en tiempo real.
            \item Aunque las alertas básicas pueden ser unidireccionales, la plataforma general se vuelve más robusta con la implementación de seguridad HTTP 
            IOT, login de usuario seguro y la protección de un dominio personalizado con certificados SSL/TLS.
            \item Además, se establecen reglas de acceso para usuarios administradores y no administradores, permitiendo a los administradores otorgar permisos 
            de lectura y escritura en tiempo real a usuarios y dispositivos. Esto asegura que las alertas y el acceso a los datos del sensor se gestionen de 
            forma segura y con control de permisos, incluso si la alerta en sí es un flujo de datos de un solo sentido desde el dispositivo.
        \end{itemize}
    \end{itemize}
\end{itemize}
En conclusión, el envío de alertas unidireccionales es un componente esencial del proyecto inicial de defensa antirrobo, proporcionando al usuario la 
capacidad básica de monitorear eventos (como la detección de movimiento) desde el servidor a través de una aplicación web. Este mecanismo, inicialmente 
implementado con HTTP y AJAX, sienta las bases para la comunicación en tiempo real y la gestión de datos en una plataforma IoT más avanzada, segura y 
bidireccional que se desarrolla a lo largo del proyecto.

\subsection{Funcionalidad para Desactivar Alarma (Bidireccional)}
La funcionalidad para desactivar la alarma es un componente crucial que establece la comunicación bidireccional dentro del proyecto del Dispositivo IoT 
de Defensa Antirrobo, evolucionando desde una simple interacción local hasta una operación segura y gestionada en la nube.

A continuación, se dan los detalles de esta funcionalidad en el contexto más amplio del proyecto:
\begin{itemize}
    
    \item \textbf{Establecimiento de Comunicación Bidireccional}
    \begin{itemize}
        \item \textbf{Complemento a las Alertas Unidireccionales}
        \begin{itemize}
            \item Inicialmente, el proyecto de defensa antirrobo se centra en la detección de movimiento para activar una alarma y enviar alertas al usuario, 
            lo que cumple con la comunicación unidireccional del servidor al usuario. Sin embargo, para completar la comunicación bidireccional del usuario al 
            servidor, se añade la funcionalidad para que los usuarios puedan desactivar la alarma.
        \end{itemize}
        \item \textbf{Control de Actuadores}
        \begin{itemize}
            \item Esta capacidad de desactivar la alarma es un ejemplo clave de cómo los usuarios pueden controlar los actuadores del sistema, en este 
            caso, el zumbador (buzzer).
        \end{itemize}
    \end{itemize}

    \item \textbf{Implementación Técnica Inicial}
    \begin{itemize}
        \item \textbf{Interfaz de Usuario}
        \begin{itemize}
            \item La desactivación se logra mediante un botón de interruptor añadido a la página web del proyecto. Este botón se encuentra en la sección 
            de caja negra del dashboard web, junto al estado de detección de movimiento.
        \end{itemize}

        \item \textbf{Conexión y Lógica del Zumbador}
        \begin{itemize}
            \item La alarma es generada por un zumbador (buzzer) que se controla generando una onda cuadrada. La funcionalidad de desactivación interrumpe 
            o modifica esta onda para silenciar el zumbador.
        \end{itemize}

        \item \textbf{Servidor Web y Raspberry Pi}
        \begin{itemize}
            \item El control se realiza a través de un servidor web HTTP básico con Flask en la Raspberry Pi. La comunicación inicial utiliza solicitudes
            keepalive y la técnica AJAX para enviar y recibir datos. Cuando el usuario interactúa con el botón de desactivación, se envía una solicitud 
            al servidor local.
        \end{itemize}

        \item \textbf{Acceso Local}
        \begin{itemize}
            \item En esta etapa inicial, el servidor se ejecuta localmente, lo que significa que los usuarios solo pueden acceder a él desde dentro de la 
            red (conectados al mismo router Wi-Fi).
        \end{itemize}
    \end{itemize}

    \item \textbf{Evolución hacia una Plataforma Segura y Escalable}
    \begin{itemize}

        \item \textbf{Gestión de Permisos (Sección 5 y 6)}
        \begin{itemize}
            \item La capacidad de desactivar la alarma se integra en un sistema de gestión de acceso más sofisticado.
            \item Se crean reglas para usuarios administradores y no administradores.
            \item Los usuarios administradores tienen un panel de control donde pueden ver una lista de usuarios en línea y, a través de botones de 
            interruptor, otorgar permisos de lectura y escritura en tiempo real a usuarios no administradores y dispositivos. La desactivación de la alarma 
            sería una acción que requiere permisos de escritura.
            \item El servidor valida que la solicitud de cambio de permisos proviene de un usuario administrador antes de procesarla.
            \item Los permisos (lectura/escritura) se almacenan en la base de datos y se utilizan para otorgar acceso de lectura y escritura a usuarios 
            específicos en el servidor PubNub.
        \end{itemize}

        \item \textbf{Seguridad Mejorada (Sección 5)}
        \begin{itemize}
            \item Para asegurar esta comunicación bidireccional y el control de los dispositivos.
            \item El dominio personalizado se protege con certificados SSL/TLS de Let's Encrypt, garantizando una comunicación cifrada de extremo a extremo 
            entre el cliente y el servidor para todas las interacciones. Esto incluye las solicitudes para desactivar la alarma.
            \item Se implementa una funcionalidad de inicio de sesión de usuario seguro y el almacenamiento de detalles del usuario en una base de datos 
            integrada.
        \end{itemize}

        \item \textbf{Protocolos de Comunicación Avanzados (Sección 3 y 4)}
        \begin{itemize}
            \item La comunicación inicial basada en HTTP/AJAX se mejora con protocolos de comunicación en tiempo real y ligeros como WebSockets y MQTT.
            \item PubNub se adopta como el protocolo principal, reemplazando el long pooling de AJAX. PubNub es una solución basada en el modelo 
            publicar/suscribir que permite a los clientes enviar mensajes al servidor y recibir respuestas basadas en eventos sin tener que realizar 
            un sondeo. Esto es fundamental para un control eficiente y en tiempo real de los actuadores como la alarma.
            \item El uso de PubNub Access Manager permite gestionar los permisos de manera granular para que solo los usuarios autorizados puedan enviar 
            comandos (como desactivar la alarma) a los dispositivos.
        \end{itemize}
    \end{itemize}

\end{itemize}
En resumen, la funcionalidad para desactivar la alarma no solo representa la culminación de la comunicación bidireccional en el proyecto inicial del 
detector de movimiento, sino que también sirve como un ejemplo práctico de cómo el control de actuadores se integra en una plataforma IoT segura, escalable 
y en tiempo real. Esta evolución implica el uso de certificados SSL/TLS para la seguridad, un sistema de gestión de acceso basado en roles para permisos 
de lectura/escritura, y la adopción de protocolos de comunicación avanzados como PubNub para un control eficiente y en tiempo real.

\section{Componentes de Hardware}
% 3.2.COMPONENTES DE HARDWARE.
Se detallan los Componentes de Hardware esenciales para el Primer Proyecto IoT: Detector de Movimiento, que se aborda en la Sección 2 del proyecto 
Internet de las Cosas con Python y Raspberry Pi. Este proyecto tiene como objetivo construir un dispositivo IoT de defensa contra robos que detecte movimiento, 
active una alarma y envíe alertas al usuario, permitiendo también la desactivación de la alarma.

Los componentes de hardware claves para este proyecto son:
\begin{itemize}
    \item \textbf{Sensor PIR (Passive Infrared Sensor)}
    \begin{itemize}
        \item Función principal: Es el módulo detector de movimiento.
        \item Principio de funcionamiento: Utiliza un elemento sensor llamado RE200BL, que es un sensor piroeléctrico que genera energía cuando se 
        expone al calor. Detecta el movimiento de cuerpos humanos o animales al captar la energía de calor (radiación infrarroja) que emiten. Se le 
        denomina pasivo porque no emite energía para detectar, solo detecta la energía emitida por otros objetos.
        \item Cobertura: Incluye una tapa de plástico especialmente diseñada para expandir el área de cobertura de detección.
        \item Pines: Cuenta con tres pines
        \begin{itemize}
            \item Ground (Tierra).
            \item VCC: Para alimentación de 5 voltios.
            \item Output (Salida): Proporciona un nivel lógico alto si se detecta un objeto y viceversa.
        \end{itemize}
        \item Potenciómetros: Dispone de dos para ajustar su comportamiento
        \begin{itemize}
            \item Uno para ajustar la sensibilidad del sensor.
            \item Otro para ajustar el tiempo que la señal de entrada permanece en alto después de la detección, con un rango que va desde 0.3 segundos 
            hasta 5 minutos.
        \end{itemize}
        \item Modos de disparo (Trigger Modes): Tiene tres pines con un puente (jumper) entre dos de ellos para seleccionar los modos de disparo
        \begin{itemize}
            \item Non repeatable trigger: La salida vuelve automáticamente de alto a bajo una vez que el tiempo de retardo ha terminado, incluso 
            si el objeto sigue presente.
            \item Repeatable trigger: La salida se mantiene en alto mientras el objeto detectado permanezca dentro del rango del sensor.
        \end{itemize}
    \end{itemize}

    \item \textbf{Zumbador (Buzzer)}
    \begin{itemize}
        \item Función principal: Se utiliza para activar la alarma.
        \item Componentes: Consiste en una carcasa exterior con tres pines (VCC para 5 voltios, tierra y señal) y un elemento piezoeléctrico rodeado por 
        un disco de vibración metálico.
        \item Generación de sonido: Cuando se le aplica corriente, el disco metálico se contrae y expande, causando vibraciones que producen el sonido. 
        Cambiar la frecuencia de la corriente aplicada puede alterar el tono del sonido.
        \item Control: Se controlará generando una onda cuadrada, lo que implica alternar el pin de señal entre un estado alto y bajo con pequeñas pausas 
        entre cada cambio.
    \end{itemize}

    \item \textbf{Raspberry Pi}
    \begin{itemize}
        \item Rol: Sirve como el cerebro del sistema. A ella se conectarán el sensor PIR y el buzzer.
        \item Programación: Se utilizará para ejecutar el código Python que detecta las señales del sensor y controla el buzzer.
        \item Servidor Web: Ejecutará un servidor web HTTP básico en Python Flask, el cual funcionará en la red Wi-Fi local para que los usuarios puedan 
        interactuar con el dispositivo.
        \item Configuración inicial:
        \begin{itemize}
            \item Requiere una tarjeta SD de al menos 8GB para instalar el sistema operativo.
            \item Se puede usar un monitor HDMI o, para acceso remoto, es necesario habilitar SSH (deshabilitado por defecto en Raspberry Pi 3) usando 
            el comando \verb|sudo raspi-config| en la terminal.
        \end{itemize}
    \end{itemize}
\end{itemize}
Estos componentes trabajan en conjunto para permitir la detección de movimiento, la emisión de una alarma y la interacción bidireccional con el 
usuario a través de una interfaz web básica.

\subsection{Sensor PIR (Infrarrojo Pasivo)}
El Sensor PIR (Infrarrojo Pasivo) es un componente de hardware fundamental y central en el Dispositivo IoT de Defensa Antirrobo del proyecto, 
sirviendo como el principal sensor de detección de movimiento.

Aquí se dan los detalles de el Sensor PIR en el contexto más amplio de los componentes de hardware:
\begin{itemize}

    \item \textbf{Definición y Principio de Funcionamiento}
    \begin{itemize}
        \item Módulo Detector de Movimiento: El Sensor PIR se describe como un módulo detector de movimiento.
        \item Sensor Piroeléctrico: Contiene un elemento sensor llamado RE200 BL, que es un sensor piroeléctrico.
        \item Detección por Calor: Este tipo de sensor genera energía cuando se expone al calor. Detecta el movimiento porque los cuerpos humanos 
        o animales emiten energía calorífica en forma de radiaciones infrarrojas.
        \item Naturaleza Pasiva: El término pasivo en su nombre significa que el sensor no utiliza ninguna energía para el propósito de detección, 
        sino que funciona simplemente detectando la energía emitida por otros objetos.
        \item Cobertura de Detección: El módulo incluye una tapa de plástico especialmente diseñada que se utiliza para expandir la cobertura del 
        área de detección.
    \end{itemize}

    \item \textbf{Pines y Conexión}
    \begin{itemize}
        \item Tres Pines Principales: El módulo PIR tiene tres pines esenciales para su funcionamiento:
        \begin{itemize}
            \item Tierra (Ground): Para la conexión a tierra.
            \item VCC (5V): Para el suministro de energía (cinco voltios).
            \item Salida (Output): Este pin proporciona un nivel lógico alto si se detecta un objeto y viceversa (un nivel lógico bajo si no hay detección).
        \end{itemize}

        \item Conexión a Raspberry Pi:
        \begin{itemize}
            \item El Sensor PIR se conecta directamente a la Raspberry Pi.
        \end{itemize}
    \end{itemize}
    
    \item \textbf{Configuraciones Ajustables y Modos de Disparo}
    \item El módulo PIR ofrece opciones de configuración para adaptar su comportamiento.
    \begin{itemize}
        \item Potenciómetros:
        \begin{itemize}
            \item Uno para ajustar la sensibilidad del sensor.
            \item Otro para ajustar el tiempo en que la señal de entrada permanece en alto (el tiempo que la salida permanece activada después 
            de la detección). Este tiempo se puede ajustar desde 0.3 segundos hasta 5 minutos.
        \end{itemize}

        \item Modos de Disparo (Trigger Modes):
        \item Dispone de tres pines adicionales con un puente (jumper) entre dos de ellos para seleccionar los modos de disparo:
        \begin{itemize}
            \item Disparador No Repetible (Non-Repeatable Trigger): Cuando el sensor detecta y el tiempo de retardo ha terminado, la salida cambia 
            automáticamente de alta a baja.
            \item Disparador Repetible (Repeatable Trigger): La salida se mantendrá alta todo el tiempo hasta que el objeto detectado esté presente 
            en el rango de detección del sensor.
        \end{itemize}
    \end{itemize}

    \item \textbf{Rol en el Contexto del Proyecto de Hardware}
    \begin{itemize}
        \item Base del Proyecto Antirrobo:
        \begin{itemize}
            \item El Sensor PIR es el corazón del proyecto anti-theft motion detector (detector de movimiento antirrobo) en la Sección 2. Su función es 
            detectar el movimiento para activar una alarma y enviar alertas al usuario.
        \end{itemize}
        \item Interacción con Raspberry Pi y Software:
        \begin{itemize}
            \item Una vez conectado a la Raspberry Pi, se escribe un código Python básico que detecta las señales altas y bajas al detectar movimiento 
            provenientes del sensor. Estos datos son cruciales para el funcionamiento del servidor web HTTP básico en la Raspberry Pi, que a su vez envía 
            el estado del sensor y los datos al usuario para actualizaciones en tiempo real.
        \end{itemize}
        \item Componente en una Plataforma IoT Más Amplia:
        \begin{itemize}
            \item Aunque el proyecto inicial lo usa para una función específica, la comprensión y el uso de sensores como el PIR son una parte fundamental 
            del proyecto, que busca construir una plataforma en la nube escalable donde múltiples usuarios pueden iniciar sesión de forma segura y controlar 
            y monitorear sus dispositivos autorizados en tiempo real. La capacidad de un sensor PIR para proporcionar datos de eventos (detección de 
            movimiento) es la base para la toma de decisiones y el control en un sistema IoT.
        \end{itemize}
    \end{itemize}
\end{itemize}
En síntesis, el Sensor PIR es el ojo del sistema de defensa antirrobo, detectando el movimiento a través del calor infrarrojo. Sus características técnicas, 
como los pines, las opciones de sensibilidad y tiempo de retardo, y los modos de disparo, permiten configurarlo para una detección precisa. Su integración 
con la Raspberry Pi y el software de Python es el primer paso para convertir las detecciones físicas en datos utilizables por el usuario y para el control 
de actuadores como el zumbador, sentando las bases para sistemas IoT más complejos y seguros.

\subsubsection{ELEMENTO SENSOR: RE200B}
 Aquí se detallan claramente que el Elemento Sensor: RE200B es el componente clave dentro del 
Sensor PIR (Infrarrojo Pasivo).

En el contexto más amplio del Sensor PIR, se  indican lo siguiente sobre el RE200B:
\begin{itemize}
    \item Identificación del Elemento Sensor: El módulo detector de movimiento PIR (sensor de infrarrojos pasivo) tiene un elemento sensor llamado RE200B.
    \item Naturaleza del RE200B: Se describe como un sensor piroeléctrico.
    \item Principio de Funcionamiento: Este elemento sensor genera energía cuando se expone al calor. Esto es crucial porque significa que cuando un 
    cuerpo humano o animal entra en el rango del sensor, detectará un movimiento debido a que los humanos o animales emiten energía térmica en forma 
    de radiación infrarroja.
    \item Nomenclatura Pasivo: El término pasivo en Sensor Infrarrojo Pasivo proviene de que el sensor no utiliza ninguna energía para fines de 
    detección; simplemente funciona detectando la energía emitida por otros objetos. Esto es directamente aplicable a cómo el RE200B capta las 
    emisiones de calor.
    \item Función en el Proyecto Anti-robo: El sensor PIR, con su elemento RE200B, es la parte del dispositivo IoT de defensa antirrobo que detecta el 
    movimiento, lo que a su vez activará una alarma.
\end{itemize}
En resumen, el RE200B es el corazón del sensor PIR, siendo el elemento piroeléctrico responsable de detectar la radiación infrarroja (calor) emitida 
por humanos o animales, permitiendo así la detección de movimiento sin emitir su propia energía. Esta capacidad es fundamental para el funcionamiento 
del proyecto de detección de movimiento antirrobo.

\subsubsection{TIPO: PIEZOELéCTRICO}
Aquí se establece que el tipo piezoeléctrico es una característica fundamental del Sensor PIR (Infrarrojo 
Pasivo), particularmente en relación con su elemento sensor clave, el RE200B.

En el contexto más amplio del Sensor PIR, se  detallan lo siguiente sobre su naturaleza piezoeléctrica:
\begin{itemize}
    \item Identificación del Elemento Sensor como Piezoeléctrico:
    \begin{itemize}
        \item El módulo detector de movimiento PIR (sensor de infrarrojos pasivo) contiene un elemento sensor llamado RE200B, y se describe explícitamente 
        como un sensor piroeléctrico. La termología piroeléctrico se refiere a la capacidad de ciertos materiales de generar una carga eléctrica en 
        respuesta a un cambio de temperatura, lo cual es una forma de piezoelectricidad inducida por el calor.
    \end{itemize}
    \item Principio de Funcionamiento Basado en el Efecto Piezoeléctrico:
    \begin{itemize}
        \item El funcionamiento del sensor se basa directamente en esta propiedad piezoeléctrica. El elemento sensor genera energía cuando se expone al calor. 
        Esto significa que cuando un cuerpo humano o animal entra en el rango del sensor, detectará un movimiento porque los humanos o animales emiten 
        energía térmica en forma de radiación infrarroja.
    \end{itemize}
    \item Fundamento de la Detección de Movimiento:
    \begin{itemize}
        \item La capacidad del RE200B para convertir la energía térmica (infrarroja) en una señal eléctrica (energía) es lo que permite al sensor PIR 
        detectar el movimiento. Sin esta característica piezoeléctrica, el sensor no podría percibir los cambios en la radiación infrarroja del entorno.
    \end{itemize}
    \item Relación con el Término Pasivo:
    \begin{itemize}
        \item El término pasivo en Sensor Infrarrojo Pasivo se explica porque el sensor no utiliza ninguna energía para fines de detección; simplemente 
        funciona detectando la energía emitida por otros objetos. La naturaleza piezoeléctrica del RE200B es inherentemente pasiva en este sentido, ya 
        que reacciona a la energía existente en lugar de emitirla.
    \end{itemize}
\end{itemize}

En resumen, el tipo piezoeléctrico es la base del funcionamiento del Sensor PIR. A través de su elemento RE200B, este sensor aprovecha la propiedad 
de generar una señal eléctrica al exponerse al calor, lo que le permite detectar de manera pasiva el movimiento de cuerpos que emiten radiación 
infrarroja (como humanos y animales) y es fundamental para su rol en el proyecto de detección antirrobo.

\subsubsection{DETECCIóN: CALOR (RADIACIóN INFRAROJA DE CUERPOS)}
Aquí se detalla de manera concisa el mecanismo de Detección: Calor (Radiación Infrarroja de Cuerpos) 
como el principio fundamental del Sensor PIR (Infrarrojo Pasivo).

En el contexto más amplio del Sensor PIR, se  explican lo siguiente sobre su método de detección:
\begin{itemize}
    \item Naturaleza de la Detección:
    \begin{itemize}
        \item El módulo detector de movimiento PIR, que es un sensor de infrarrojos pasivo, detecta el movimiento basándose en la detección de calor.
    \end{itemize}
    \item Elemento Sensor (RE200B):
    \begin{itemize}
        \item El corazón de esta detección es el elemento sensor llamado RE200B, que es un sensor piroeléctrico. Este elemento genera energía cuando se 
        expone al calor.
    \end{itemize}
    \item Fuente del Calor (Cuerpos Humanos o Animales):
    \begin{itemize}
        \item La detección se activa cuando un cuerpo humano o animal entra en el rango del sensor. Esto se debe a que los humanos o animales emiten 
        energía térmica en forma de radiación infrarroja.
    \end{itemize}
    \item Detección de Movimiento:
    \begin{itemize}
        \item Al detectar esta radiación infrarroja (calor) emitida por cuerpos, el sensor puede detectar un movimiento.
    \end{itemize}
    \item Significado de Pasivo:
    \begin{itemize}
        \item El término pasivo en Sensor Infrarrojo Pasivo se explica porque el sensor no utiliza ninguna energía para fines de detección; simplemente 
        funciona detectando la energía emitida por otros objetos. Esto subraya que el sensor no emite su propia energía, sino que reacciona a la radiación 
        térmica presente en su entorno.
    \end{itemize}
\end{itemize}

En resumen, el Sensor PIR detecta el movimiento al identificar los cambios en la radiación infrarroja (calor) emitida de forma natural por cuerpos humanos 
o animales. Su naturaleza pasiva significa que simplemente percibe esta energía existente sin emitir la suya propia, utilizando un elemento piroeléctrico 
para convertir el calor detectado en una señal que indica movimiento.

\subsubsection{PASIVO: NO EMITE ENERGíA}
Aquí se resalta un aspecto fundamental del Sensor PIR (Infrarrojo Pasivo): su cualidad de ser 
Pasivo: No Emite Energía.

En el contexto más amplio del Sensor PIR, se  explican lo siguiente sobre esta característica:
\begin{itemize}
    \item Origen del Nombre Pasivo: El nombre pasivo en Sensor Infrarrojo Pasivo se deriva directamente de su principio de funcionamiento.
    \item Principio de No Emisión de Energía: La razón por la que se le llama pasivo es porque el sensor no utiliza ninguna energía para fines de detección. 
    Esto significa que el sensor no irradia su propia energía (por ejemplo, infrarroja o microondas) para iluminar su entorno y luego detectar reflexiones.
    \item Detección de Energía Emitida por Otros Objetos: En cambio, el sensor PIR simplemente funciona detectando la energía emitida por otros objetos. 
    Esta energía son las radiaciones infrarrojas o el calor que los humanos o animales emiten.
    \item Rol del Elemento Sensor (RE200B): El elemento sensor RE200B, que es un sensor piroeléctrico, es el encargado de generar energía cuando se expone 
    al calor. Al ser piroeléctrico, reacciona a los cambios en la radiación térmica existente en su campo de visión, sin necesidad de emitir nada por sí mismo. 
\end{itemize}

En resumen, la naturaleza pasiva del Sensor PIR significa que no emite su propia energía para detectar. En su lugar, detecta el movimiento percibiendo 
la radiación infrarroja (calor) que de forma natural desprenden cuerpos como humanos o animales que entran en su rango de detección. 
Esta característica es clave para su funcionamiento como detector de movimiento en proyectos como el dispositivo antirrobo de IoT.

\subsubsection{TAPA PLASTICA: EXPANDE áREA DE DETECCIóN}
Aquí se resalta una característica importante del Sensor PIR (Infrarrojo Pasivo): la Tapa Plástica: 
Expande Área de Detección.

En el contexto más amplio del Sensor PIR, se  indican lo siguiente sobre este componente:
\begin{itemize}
    \item Componente del Módulo: El módulo del detector de movimiento PIR también consta de una tapa de plástico especialmente diseñada.
    \item Función Principal: La función explícita de esta tapa de plástico es expandir la cobertura del área de detección.
\end{itemize}

Esta tapa, a menudo conocida como lente Fresnel, es crucial porque el elemento sensor piroeléctrico (como el RE200B) tiene un campo de visión muy limitado. 
Al expandir el área de detección, la tapa plástica permite que el sensor perciba cambios en la radiación infrarroja de un área mucho mayor, lo que es 
esencial para su eficacia en la detección de movimiento de cuerpos como humanos o animales.

En resumen, la tapa de plástico es un componente integral del Sensor PIR cuya finalidad es ampliar significativamente el área de cobertura 
en la que el sensor puede detectar la radiación infrarroja y, por ende, el movimiento, optimizando así su rendimiento en aplicaciones como un 
sistema antirrobo.

\subsubsection{PINES}
Aquí se detalla de manera específica la configuración de los pines del Sensor PIR (Infrarrojo Pasivo).

En el contexto más amplio del Sensor PIR, se  describen lo siguiente sobre sus pines y otros elementos de conexión:
\begin{itemize}
    \item Pines Principales para Conexión a Raspberry Pi: El módulo del sensor PIR tiene tres pines principales para su funcionamiento básico:
    \begin{itemize}
        \item Ground (Tierra): Para la conexión a tierra.
        \item VCC (5 voltios): Para alimentar el sensor.
        \item Output pin (Pin de Salida): Este pin da un nivel lógico alto si se detecta un objeto y viceversa. Esto es fundamental para que la Raspberry 
        Pi pueda interpretar si ha habido movimiento.
        \item  
    \end{itemize}
    \item Potenciómetros para Ajustes: Además de los pines de conexión, el módulo cuenta con dos potenciómetros para afinar su comportamiento:
    \begin{itemize}
        \item Ajuste de Sensibilidad: Uno de los potenciómetros es para ajustar la sensibilidad del sensor.
        \item Ajuste de Tiempo de Señal Alta: El otro es para ajustar el tiempo que la señal de entrada permanece en alto cuando se detecta el objeto. 
        Este tiempo se puede ajustar desde 0.3 segundos hasta 5 minutos.
    \end{itemize}
    \item Pines Adicionales para Modos de Disparo (Trigger Modes) El módulo también incorpora tres pines más con un jumper entre dos de ellos que se 
    utilizan para seleccionar los modos de disparo:
    \begin{itemize}
        \item Disparo no repetible (Non repeatable trigger): En este modo, cuando el sensor da una salida alta y el tiempo de retardo ha terminado, 
        la salida cambia automáticamente de alta a baja.
        \item Disparo repetible (Repeatable trigger): Este modo mantendrá la salida en alto todo el tiempo hasta que el objeto detectado esté presente 
        en el rango del sensor.
    \end{itemize}
\end{itemize}

En resumen, los pines del Sensor PIR son cruciales para su integración con la Raspberry Pi, permitiendo la alimentación (VCC, Ground) y la transmisión 
de la señal de detección de movimiento (Output). Los potenciómetros ofrecen una flexibilidad importante para calibrar la sensibilidad y el tiempo de 
respuesta del sensor, mientras que los pines adicionales con el jumper permiten configurar los modos de disparo, adaptando el comportamiento del sensor 
a las necesidades específicas del proyecto, como la detección antirrobo.

\subsubsection{POTENCIOMETROS}
Aquí se especifica la presencia y función de los potenciómetros como componentes ajustables clave en 
el Sensor PIR (Infrarrojo Pasivo).

En el contexto más amplio del Sensor PIR, se  describen lo siguiente sobre sus potenciómetros:
\begin{itemize}
    \item Número y Ubicación: El módulo del sensor PIR cuenta con dos potenciómetros.
    \item Ajuste de Sensibilidad: Uno de estos potenciómetros se utiliza para ajustar la sensibilidad del sensor. Esto permite al usuario calibrar 
    qué tan fácilmente el sensor detectará cambios en la radiación infrarroja, adaptándolo al entorno específico de su aplicación (por ejemplo, 
    para evitar falsas alarmas o asegurar una detección precisa).
    \item Ajuste del Tiempo de Señal Alta: El otro potenciómetro sirve para ajustar el tiempo que la señal de entrada permanece en alto cuando se detecta 
    el objeto. Esta es una función crucial para controlar la duración de la señal de detección.
    \item Rango de Ajuste: Este tiempo se puede ajustar en un rango significativo, desde 0.3 segundos hasta 5 minutos. Esta flexibilidad es vital para 
    adaptar el comportamiento del sensor a los requisitos de un proyecto, como un sistema antirrobo, donde la duración de la alarma o el estado de alerta 
    puede necesitar configuraciones específicas.
\end{itemize}

En resumen, los potenciómetros del Sensor PIR son elementos de control fundamentales que permiten a los usuarios personalizar la sensibilidad de detección 
y la duración de la señal de salida tras la detección de movimiento. Estas capacidades de ajuste son esenciales para la integración efectiva del sensor 
en proyectos de IoT, como un dispositivo antirrobo, asegurando que su rendimiento se adapte con precisión a las necesidades y condiciones del entorno.

\subsubsection{MODOS DE DISPARO (JUMPER)}
Aquí se describen los Modos de Disparo (Jumper) como una característica configurable del Sensor 
PIR (Infrarrojo Pasivo).

En el contexto más amplio del Sensor PIR, se  explican lo siguiente sobre esta funcionalidad:
\begin{itemize}
    \item Configuración Física: El módulo del sensor PIR incluye tres pines más con un jumper entre dos de ellos. Este jumper se utiliza específicamente 
    para seleccionar los modos de disparo.
    \item Modos de Disparo Disponibles: Se detallan dos modos principales:
    \begin{itemize}
        \item Disparo no repetible (Non repeatable trigger): En este modo, cuando el sensor da una salida alta y el tiempo de retardo ha terminado, la 
        salida cambia automáticamente de alta a baja. Esto significa que, después de detectar movimiento y mantener la señal alta por el tiempo configurado, 
        el sensor se reinicia a un estado bajo, esperando una nueva detección antes de volver a activarse.
        \item Disparo repetible (Repeatable trigger): Este modo mantendrá la salida en alto todo el tiempo hasta que el objeto detectado esté presente 
        en el rango del sensor. A diferencia del modo no repetible, la señal de salida permanecerá alta mientras haya movimiento continuo dentro del 
        rango de detección, lo cual es útil para aplicaciones donde se requiere una señal constante mientras el objeto está presente.
    \end{itemize}
\end{itemize}

En resumen, los Modos de Disparo (Jumper) del Sensor PIR ofrecen una flexibilidad importante para adaptar el comportamiento de la señal de salida 
del sensor después de una detección. Mediante la colocación de un jumper en pines específicos, el usuario puede elegir entre un disparo no repetible 
(donde la señal se restablece después de un retardo fijo) o un disparo repetible (donde la señal permanece activa mientras se detecta movimiento), 
lo que permite optimizar su funcionamiento en diversas aplicaciones como, por ejemplo, un sistema de seguridad antirrobo.

\subsection{Zumbador}
El Zumbador (Buzzer) es un componente de hardware esencial en el proyecto del Dispositivo IoT de Defensa Antirrobo, funcionando como el actuador principal 
para la alarma. Su integración permite que el sistema responda a la detección de movimiento y que el usuario interactúe con el dispositivo para 
desactivar dicha alarma.

A continuación, se dan los detalles de el Zumbador en el contexto más amplio de los componentes de hardware:
\begin{itemize}

    \item Rol y Función en el Proyecto.
    \begin{itemize}
        \item Generador de Alarma: El zumbador es el componente encargado de activar una alarma cuando se detecta movimiento. Es una parte crucial 
        de la funcionalidad de defensa antirrobo.
        \item Actuador Controlable por el Usuario: El proyecto permite a los usuarios controlar los actuadores, siendo el zumbador un ejemplo principal 
        de esto. Se añade un botón de interruptor en la página web para que los usuarios puedan desactivar el zumbador (que representa la alarma). 
        Esto completa la comunicación bidireccional del usuario al servidor.
    \end{itemize}

    \item Descripción Física y Conexiones
    \begin{itemize}
        \item Componentes: Un zumbador consta de una caja exterior con tres pines y, en su interior, un elemento piezoeléctrico rodeado por un 
        disco metálico vibratorio.
        \item Pines: Los tres pines del zumbador son:
        \begin{itemize}
            \item VCC (5V): Para el suministro de energía.
            \item Tierra (Ground): Para la conexión a tierra.
            \item Señal (Signal): El pin a través del cual se controla el zumbador.
        \end{itemize}
        \item Conexión a Raspberry Pi: El zumbador se conecta directamente con la Raspberry Pi.
    \end{itemize}

    \item Principio de Funcionamiento y Control
    \begin{itemize}
        \item Generación de Sonido: Cuando se aplica corriente al zumbador, el disco metálico se contrae y expande, lo que provoca su vibración y, 
        como resultado, la producción de sonido.
        \item Control del Tono (Pitch): Si se cambia la frecuencia de la corriente aplicada al zumbador, la velocidad de vibración del disco también 
        cambia, lo que a su vez cambia el tono (pitch) del sonido resultante. Esto permite generar diferentes melodías o sonidos.
        \item Control mediante Onda Cuadrada: Para controlar el zumbador, se genera una onda cuadrada. En términos sencillos, esto implica alternar el pin 
        de señal entre un estado alto y un estado bajo durante ciertos milisegundos y repetir este proceso.
        \item Programación en Python: Se escribe un código Python básico para controlar el zumbador, manejando la lógica de la onda cuadrada.
    \end{itemize}

    \item Integración en el Sistema
    \begin{itemize}
        \item Activación por Sensor PIR: El zumbador se activa en respuesta a la detección de movimiento por parte del Sensor PIR.
        \item Servidor Web y Desactivación: El estado del zumbador (activado/desactivado) se puede ver y controlar a través de un 
        servidor web HTTP básico con Flask en la Raspberry Pi. Un botón en la interfaz web permite a los usuarios desactivar el zumbador, 
        interrumpiendo o modificando la generación de la onda cuadrada que produce la alarma.
    \end{itemize}
\end{itemize}

En síntesis, el Zumbador es el principal actuador sonoro del sistema antirrobo. Su capacidad para generar una alarma visible y controlable por el 
usuario, mediante la aplicación de una onda cuadrada y la interacción a través de una interfaz web, lo convierte en un componente crucial para la 
funcionalidad bidireccional y la respuesta del sistema IoT.

\subsubsection{COMPONENTE: ELEMENTO PIEZOELéCTRICO, DISCO DE VIBRACIóN METáLICO.}
Aquí se explica de manera directa los Componentes: Elemento Piezoeléctrico, Disco de Vibración Metálico en el contexto más amplio 
de un Zumbador (Buzzer).

Según las notas técnicas:
\begin{itemize}
    \item Estructura Interna del Zumbador: Un zumbador, más allá de su carcasa exterior y sus pines de conexión (VCC, Ground, Signal), está compuesto 
    internamente por un elemento piezoeléctrico que está rodeado por un disco de vibración metálico.
    \item Mecanismo de Generación de Sonido:
    \begin{itemize}
        \item Cuando se aplica corriente al zumbador, el disco hace que el disco se contraiga y se expanda.
        \item Esta contracción y expansión, a su vez, hace que el disco circundante vibre, lo cual produce el sonido que se escucha.
    \end{itemize}
    \item Control del Sonido:
    \begin{itemize}
        \item La frecuencia de la corriente aplicada al zumbador determina la velocidad de la vibración del disco, lo que a su vez 
        cambia el tono del sonido resultante. Esto permite generar diferentes melodías o sonidos.
        \item Para controlar el zumbador, se puede generar una onda cuadrada. En términos sencillos, esto implica alternar el pin de señal entre 
        un estado alto y un estado bajo, esperando unos milisegundos entre cada cambio y repitiendo el proceso.
    \end{itemize}
\end{itemize}
En resumen, se  indican que el elemento piezoeléctrico y el disco de vibración metálico son los componentes internos clave de un zumbador. 
Estos trabajan en conjunto: el elemento piezoeléctrico, al recibir corriente, hace que el disco metálico vibre, produciendo así el sonido, cuya 
frecuencia puede ser modulada para cambiar el tono.

\subsubsection{SONIDO: VIBRACIóN DE DISCO}
Aquí se explica claramente cómo se genera el Sonido: Vibración del Disco en el contexto más amplio de un Zumbador (Buzzer).

Según notas técnicas:
\begin{itemize}
    \item Componentes Clave: Un zumbador está compuesto internamente por un elemento piezoeléctrico que está rodeado por un disco de vibración metálico.
    \item Mecanismo de Generación de Sonido:
    \begin{itemize}
        \item Cuando se aplica corriente eléctrica al zumbador, el elemento piezoeléctrico hace que el disco de vibración metálico se contraiga y se expanda.
        \item Esta acción de contracción y expansión, a su vez, provoca que el disco circundante vibre.
        \item Es precisamente esta vibración del disco la que produce el sonido que se escucha.
    \end{itemize}
    \item Control del Tono del Sonido:
    \begin{itemize}
        \item La frecuencia de la corriente que se aplica al zumbador es crucial, ya que esta determina la velocidad de la vibración del disco.
        \item Un cambio en la velocidad de vibración resulta en un cambio en el tono del sonido resultante, lo que permite generar diferentes melodías.
        \item Para controlar el zumbador y generar sonidos, se puede utilizar una onda cuadrada, que implica alternar el pin de señal entre estados alto 
        y bajo con esperas de milisegundos.
    \end{itemize}
\end{itemize}
En síntesis, el sonido de un zumbador se origina directamente de la vibración de un disco metálico, la cual es inducida por la contracción y expansión 
generada por un elemento piezoeléctrico al recibir corriente eléctrica. La capacidad de controlar la frecuencia de esta vibración permite variar el 
tono del sonido producido.

\subsubsection{PINES}
Aquí se especifica la configuración de los pines de un Zumbador (Buzzer).

El zumbador:
\begin{itemize}
    \item Está formado por una carcasa exterior a la que se le adjuntan tres pines.
    \item Estos tres pines tienen funciones específicas para su conexión y control:
    \begin{itemize}
        \item VCC (5 voltios): Para la alimentación del zumbador.
        \item Ground (Tierra): Para la conexión a tierra.
        \item Signal (Señal): Este pin es crucial para el control del zumbador, ya que a través de él se genera una onda cuadrada, alternando entre estados 
        alto y bajo para producir el sonido.
    \end{itemize}
\end{itemize}
En resumen, los pines del zumbador son esenciales para su integración y funcionamiento, permitiendo tanto su alimentación como el control de la generación 
del sonido mediante la señal.

\subsubsection{CONTROL: ONDA CUADRADA (ALTA, ESPERAR; BAJA, ESPERAR)}
Aquí se explica de manera específica el concepto de Control: Onda Cuadrada (Alta, Esperar; Baja, Esperar) en el contexto más amplio 
de un Zumbador (Buzzer).

Según notas técnicas:
\begin{itemize}
    \item Método de Control: Para controlar el zumbador y generar sonido, se utilizará la generación de una onda cuadrada.
    \item Funcionamiento Simplificado: En términos sencillos, el proceso implica:
    \begin{itemize}
        \item Poner el pin de señal en alto (High).
        \item Esperar algunos milisegundos.
        \item Ponerlo en bajo (Low).
        \item Esperar algunos milisegundos nuevamente.
        \item Repetir el proceso otra vez.
    \end{itemize}
    \item Generación de Sonido y Tono: Esta alternancia rápida entre estados altos y bajos es lo que crea la onda cuadrada, la cual, al aplicarse 
    al elemento piezoeléctrico y al disco vibratorio del zumbador, causa que el disco se contraiga y se expanda. Esta contracción y expansión, a su 
    vez, hace que el disco circundante vibre, produciendo así el sonido. Además, la frecuencia de la corriente aplicada (es decir, la velocidad a 
    la que se alternan los estados alto y bajo en la onda cuadrada) determina la velocidad de la vibración del disco, lo que permite cambiar el 
    tono del sonido resultante y, en consecuencia, generar algunas melodías hermosas.
\end{itemize}
En resumen, el control de un zumbador mediante una onda cuadrada es el método fundamental para producir sonido. Este control se logra alternando 
el pin de señal entre un estado alto y uno bajo, con pausas en milisegundos entre cada cambio. La velocidad de esta alternancia (la frecuencia de 
la onda cuadrada) es crucial, ya que permite modular la vibración del disco interno del zumbador y, por ende, el tono del sonido generado.

\section{Funcionamiento Técnico}
% 3.3.FUNCIONAMIENTO TéCNICO.
En el contexto más amplio del Primer Proyecto IoT: Detector de Movimiento, el funcionamiento técnico se refiere a cómo los componentes de hardware y 
software interactúan para crear un sistema de seguridad anti-robo funcional. Este proyecto se presenta como una introducción práctica en la Sección 2 
del proyecto Internet de las Cosas con Python y Raspberry Pi.

El funcionamiento técnico se detalla a través de los siguientes puntos:
\begin{itemize}

    \item Conexión de Hardware:
    \begin{itemize}
        \item El sensor PIR (Passive Infrared Sensor) y el  buzzer  (zumbador) se conectan físicamente a la Raspberry Pi.
        \item El sensor PIR detecta el movimiento de cuerpos (humanos o animales) captando la energía de calor (radiación infrarroja) que emiten. 
        Tiene un pin de salida que proporciona un nivel lógico alto cuando se detecta un objeto.
        \item El buzzer, que se utiliza como alarma, produce sonido al aplicarle corriente, lo que causa la vibración de su disco metálico.
    \end{itemize}

    \item Programación con Python en Raspberry Pi:
    \begin{itemize}
        \item Se escribe código Python básico en la Raspberry Pi. Este código es responsable de:
        \begin{itemize}
            \item Detectar las señales de nivel alto y bajo provenientes del sensor PIR cuando se detecta movimiento.
            \item Controlar el buzzer generando una onda cuadrada, lo que implica alternar el pin de señal entre un estado alto y bajo con 
            pequeñas pausas, para producir el sonido de la alarma.
        \end{itemize}
    \end{itemize}

    \item Servidor Web HTTP Básico con Flask:
    \begin{itemize}
        \item Se desarrolla un servidor web HTTP básico utilizando el framework Flask de Python en la Raspberry Pi.
        \item Este servidor se ejecuta en la red Wi-Fi local.
        \item Los usuarios pueden acceder a la página web del dispositivo a través de la dirección IP local asignada a la Raspberry Pi 
        (por ejemplo, 192.168.1.250) desde sus navegadores.
    \end{itemize}

    \item Comunicación Cliente-Servidor mediante AJAX (Unidireccional):
    \begin{itemize}
        \item Una vez que el navegador del usuario carga la página web, se envían solicitudes keep-alive periódicamente (cada cinco segundos) 
        desde el navegador (cliente) al servidor web de la Raspberry Pi.
        \item La respuesta a cada solicitud keep-alive se considera un latido para confirmar que la conexión con el servidor sigue activa.
        \item La Raspberry Pi aprovecha estas respuestas para enviar el estado del sensor y los datos al usuario, lo que permite 
        actualizaciones en vivo en la interfaz web. Esto establece una  comunicación unidireccional del servidor al usuario. La 
        interfaz web muestra el estado de la detección de movimiento y el estado de la conexión.
    \end{itemize}

    \item Interacción del Usuario (Comunicación Bidireccional):
    \begin{itemize}
        \item Se añade un botón o interruptor en la página web que permite a los usuarios controlar los actuadores.
        \item Específicamente, este botón permite desactivar la alarma (el buzzer), logrando así una comunicación bidireccional 
        entre el usuario y el servidor.
    \end{itemize}

    \item Configuración del Entorno:
    \begin{itemize}
        \item Se requiere una tarjeta SD de al menos 8GB para instalar el sistema operativo de la Raspberry Pi.
        \item Para acceder remotamente a la Raspberry Pi (por ejemplo, mediante Escritorio Remoto), es necesario habilitar SSH, ya 
        que está deshabilitado por defecto en la Raspberry Pi 3. Esto se hace a través del comando \verb|sudo raspi-config| en la terminal.
    \end{itemize}

\end{itemize}
Inicialmente, el servidor funciona localmente, lo que significa que los usuarios deben estar conectados a la misma red Wi-Fi que la Raspberry Pi 
para acceder al sistema. Este proyecto establece la base para futuras secciones del proyecto que explorarán protocolos de comunicación más avanzados 
como WebSockets y MQTT, así como la seguridad y el despliegue en la nube.

\subsection{Conexión PIR y Zumbador con Raspberry PI}
Aquí se  detalla la conexión y el funcionamiento técnico del Sensor PIR y el Zumbador con la Raspberry Pi en el contexto de un dispositivo 
IoT de defensa antirrobo, destacando cómo estos componentes interactúan para formar un sistema funcional.

Conexión y Funcionamiento Técnico del Sensor PIR con Raspberry Pi
El Sensor PIR (Infrarrojo Pasivo) es el módulo principal de detección de movimiento del proyecto.
\begin{itemize}
    \item Pines y Conexión a Raspberry Pi:
    \begin{itemize}
        \item El módulo PIR tiene  tres pines principales: Tierra (Ground), VCC (5V) para alimentación, y un pin de Salida (Output).
        \item Este pin de salida proporciona un nivel lógico alto si se detecta un objeto y viceversa (nivel bajo si no hay detección).
        \item El sensor PIR se conecta directamente a la Raspberry Pi.
    \end{itemize}

    \item Principio de Detección:
    \begin{itemize}
        \item Es un sensor piroeléctrico (elemento RE200 BL) que genera energía al exponerse al calor.
        \item Detecta el movimiento porque los cuerpos humanos o animales emiten energía calorífica en forma de radiaciones infrarrojas.
        \item Es pasivo porque no emite energía, solo detecta la energía emitida por otros objetos.
    \end{itemize}

    \item Configuraciones Ajustables:
    \begin{itemize}
        \item Dispone de  dos potenciómetros: uno para ajustar la sensibilidad del sensor y otro para ajustar el tiempo en que la señal de entrada 
        permanece en alto (de 0.3 segundos a 5 minutos).
        \item Tiene tres pines adicionales con un jumper para seleccionar  modos de disparo:
        \begin{itemize}
            \item Disparador No Repetible: La salida cambia de alta a baja automáticamente una vez que el tiempo de retardo ha terminado, tras una detección.
            \item Disparador Repetible: La salida se mantendrá alta todo el tiempo hasta que el objeto detectado esté presente en el rango del sensor.
        \end{itemize}
    \end{itemize}

    \item Interacción con Raspberry Pi (Software):
    \begin{itemize}
        \item Una vez conectado, se escribe un código Python básico que detecta las señales altas y bajas al detectar movimiento provenientes 
        del sensor. Estas señales son el input clave para el sistema.
    \end{itemize}
\end{itemize}

Conexión y Funcionamiento Técnico del Zumbador con Raspberry Pi
El Zumbador (Buzzer) es el actuador principal para la alarma en el proyecto.
\begin{itemize}
    \item Pines y Conexión a Raspberry Pi:
    \begin{itemize}
        \item El zumbador consta de una caja con tres pines: VCC (5V), Tierra (Ground), y Señal (Signal).
        \item Se conecta directamente a la Raspberry Pi.
    \end{itemize}

    \item Principio de Generación de Sonido:
    \begin{itemize}
        \item Internamente tiene un elemento piezoeléctrico y un disco metálico vibratorio.
        \item Cuando se aplica corriente, el disco metálico se contrae y expande, vibrando y produciendo sonido.
    \end{itemize}

    \item Control del Tono y Generación de Sonido con Raspberry Pi:
    \begin{itemize}
        \item Si se cambia la frecuencia de la corriente aplicada, la velocidad de vibración del disco cambia, lo que a su vez cambia 
        el tono (pitch) del sonido. Esto permite generar melodías.
        \item Para controlarlo, se genera una onda cuadrada. En términos técnicos, esto implica alternar el pin de señal entre un estado 
        alto y bajo por milisegundos y repetir el proceso.
        \item Un código Python básico en la Raspberry Pi se encarga de generar esta onda cuadrada para controlar el zumbador.
    \end{itemize}

\end{itemize}

Funcionamiento Técnico Conjunto y Rol de la Raspberry Pi
La Raspberry Pi actúa como el cerebro que integra y controla ambos componentes:
\begin{itemize}
    \item Detección y Activación: El Sensor PIR detecta el movimiento y envía una señal a la Raspberry Pi.
    \item Procesamiento y Alarma: La Raspberry Pi, ejecutando código Python, interpreta esta señal del PIR. Si se detecta movimiento, 
    el código instruye a la Raspberry Pi para que active el Zumbador generando una onda cuadrada en su pin de señal. Esto cumple la 
    función de activar una alarma.

    \item Comunicación Unidireccional y Bidireccional:
    \begin{itemize}
        \item Esta interacción permite la comunicación unidireccional del servidor a los usuarios (enviando alertas y activando la alarma).
        \item Para la comunicación bidireccional, se añade una funcionalidad para que los usuarios puedan desactivar el zumbador (la alarma) 
        a través de una interfaz web. Esto implica que la Raspberry Pi también recibe comandos del usuario para controlar el actuador.
    \end{itemize}

    \item Servidor Web HTTP Básico: La Raspberry Pi ejecuta un servidor web HTTP básico con Python Flask en la red local. Este servidor es responsable de:
    \begin{itemize}
        \item Enviar el estado del sensor y los datos al usuario para actualizaciones en tiempo real.
        \item Mostrar un botón de interruptor en la página web que permite a los usuarios desactivar el zumbador, 
        demostrando el control de actuadores.
    \end{itemize}
\end{itemize}
En resumen, la Raspberry Pi sirve como el intermediario inteligente que recibe información del Sensor PIR (entrada de datos del mundo 
físico), la procesa con lógica programada en Python, y luego usa esa información para controlar un Zumbador (actuador que interactúa con 
el mundo físico), así como para comunicarse con los usuarios a través de una interfaz web, logrando un sistema IoT funcional.

\subsection{Código Python para Detección de Movimiento y Control del Zumbador}
% 3.3.2 CóDIGO PYTHON PARA DETECCIóN DE MOVIMIENTO Y CONTRO DEL ZUMBADOR.
Aquí se  proporcionan detalles sobre el código Python utilizado para la detección de movimiento del Sensor PIR y el 
control del Zumbador en el contexto del funcionamiento técnico del Dispositivo IoT de Defensa Antirrobo. La Raspberry Pi actúa como 
el centro de control, ejecutando este código para integrar ambos componentes.

Aquí se detalla este aspecto:

Código Python para la Detección de Movimiento del Sensor PIR
\begin{itemize}
    \item Conexión con Raspberry Pi:
    \begin{itemize}
        \item El Sensor PIR se conecta directamente a la Raspberry Pi.
    \end{itemize}
    
    \item Detección de Señales:
    \begin{itemize}
        \item Se escribe un código Python básico que detecta las señales altas y bajas al detectar movimiento provenientes del sensor.
        \item El pin de salida del PIR proporciona un nivel lógico alto si se detecta un objeto y un nivel bajo si no hay detección. El código 
        Python monitorea estos cambios de estado.
    \end{itemize}
    
    \item Propósito:
    \begin{itemize}
        \item Este código es fundamental para la funcionalidad principal del dispositivo, que es detectar movimiento 
        y, basándose en ello, activar una alarma.
    \end{itemize}
\end{itemize}


Código Python para el Control del Zumbador
\begin{itemize}
    \item Conexión con Raspberry Pi: 
    \begin{itemize}
        \item El Zumbador también se conecta directamente a la Raspberry Pi.
    \end{itemize}

    \item Generación de Sonido: 
    \begin{itemize}
        \item El zumbador genera sonido cuando se le aplica corriente, causando la contracción y expansión de un disco metálico vibratorio.
    \end{itemize}

    \item Control mediante Onda Cuadrada: 
    \begin{itemize}
        \item Para controlar el zumbador y generar el sonido de la alarma, se genera una onda cuadrada.
        \item El código Python básico para el zumbador implica alternar el pin de señal del zumbador entre un estado alto y un estado bajo 
        durante ciertos milisegundos y repetir el proceso.
        \item Se escribirá un código de una sola línea para controlar el zumbador, sugiriendo una implementación concisa.
    \end{itemize}

    \item Control del Tono (Pitch): 
    \begin{itemize}
        \item Al cambiar la frecuencia de la corriente aplicada (es decir, la velocidad de alternancia entre alto y bajo en 
        la onda cuadrada), el código Python puede cambiar el tono (pitch) del sonido, permitiendo la generación de diferentes melodías.
    \end{itemize}

    \item Propósito: 
    \begin{itemize}
        \item El código permite que la Raspberry Pi controle los actuadores, siendo el zumbador el principal para activar una alarma.
    \end{itemize}
\end{itemize}

Funcionamiento Técnico Conjunto e Integración en Python
La Raspberry Pi, con su capacidad de ejecutar código Python, es el cerebro que orquesta la interacción entre el sensor y el actuador:

\begin{itemize}
    \item Flujo del Sistema:
    \begin{itemize}
        \item El proceso técnico es el siguiente: se conectan el sensor PIR y el zumbador con la Raspberry Pi. 
        Luego, se escribe el código Python básico que primero detecta las señales altas y bajas al detectar movimiento del PIR y, en consecuencia, 
        controla el zumbador.
    \end{itemize}

    \item Servidor Web con Flask:
    \begin{itemize}
        \item La Raspberry Pi no solo gestiona los componentes físicos, sino que también ejecuta un servidor web HTTP 
        básico con Python Flask en la red local.
        \item Este servidor web Python es crucial para la comunicación bidireccional: no solo envía el estado del sensor y los datos al usuario 
        (permitiendo actualizaciones en vivo del estado de detección de movimiento), sino que también permite al usuario 
        desactivar el zumbador (la alarma) a través de un botón de interruptor en la página web.
        \item La funcionalidad de desactivación del zumbador a través de la interfaz web demuestra cómo el código Python en el 
        servidor maneja las interacciones del usuario para controlar el actuador.
    \end{itemize}

    \item Programación y Lógica: 
    \begin{itemize}
        \item El código Python es el que implementa la lógica del proyecto, desde la lectura de entradas del sensor 
        hasta la activación de salidas en el actuador y la gestión de la interfaz web para la interacción del usuario.
    \end{itemize}
\end{itemize}
En resumen, el código Python en la Raspberry Pi es el componente clave que permite al sistema IoT de defensa antirrobo interpretar los datos 
del Sensor PIR, activar el Zumbador como alarma sonora y gestionar la interacción con el usuario a través de un servidor web Flask para un 
control bidireccional.

\subsection{Servidor WEB FLASK en Raspberry PI (Red Local).}
% 3.3.3 SERVIDOR WEB FLASK EN Raspberry PI (RED LOCAL).
El Servidor Web Flask en la Raspberry Pi desempeña un rol crucial en el funcionamiento técnico del dispositivo IoT de defensa antirrobo, 
sirviendo como la interfaz principal para la interacción del usuario y la gestión de los componentes del sistema, todo operando dentro de una red local.

A continuación, se dan los detalles de su funcionamiento técnico:

Implementación y Operación en la Red Local
\begin{itemize}
    \item Tipo de Servidor: 
    \begin{itemize}
        \item Se trata de un servidor web HTTP básico con Python Flask.
    \end{itemize}

    \item Plataforma de Ejecución: 
    \begin{itemize}
        \item El servidor Flask se ejecuta directamente en la Raspberry Pi.
    \end{itemize}
    
    \item Conectividad de Red: 
    \begin{itemize}
        \item Opera dentro de la red Wi-Fi local. Esto implica que los usuarios que desean interactuar con el dispositivo deben estar 
        conectados al mismo router Wi-Fi para acceder al servidor.
    \end{itemize}

    \item Acceso del Usuario: 
    \begin{itemize}
        \item La Raspberry Pi es asignada una dirección IP local por el router Wi-Fi (por ejemplo, 192.168.1.250). Los usuarios 
        pueden acceder a la página web del servidor a través de sus navegadores utilizando esta dirección IP local.
    \end{itemize}
\end{itemize}

Funcionalidades para la Comunicación Bidireccional

El servidor Flask facilita tanto la comunicación unidireccional (servidor a usuario) como la bidireccional (usuario a servidor).
\begin{itemize}
    \item Envío de Datos del Sensor y Estado en Tiempo Real:
    \begin{itemize}
        \item Una vez que el usuario carga la página web, el navegador envía peticiones keepalive al servidor de la Raspberry Pi 
        periódicamente cada cinco segundos.
        \item Estas peticiones actúan como un latido para confirmar que la conexión con el servidor sigue activa.
        \item En respuesta a cada petición keepalive, el servidor envía el estado del sensor y los datos al usuario, lo que permite que el 
        usuario reciba actualizaciones en vivo del sensor en la página web.
        \item La interfaz web mostrará el estado de la detección de movimiento y el estado de la conexión.
    \end{itemize}

    \item Recepción de Comandos del Usuario para el Control de Actuadores:
    \begin{itemize}
        \item La página web generada por Flask incluye botones de interruptor que permiten a los usuarios controlar los actuadores 
        conectados a la Raspberry Pi.
        \item Un ejemplo específico es la funcionalidad para desactivar el zumbador (la alarma), lo que establece una comunicación bidireccional 
        completa desde el usuario hacia el servidor.
        \item Estas acciones del usuario se traducen en solicitudes que el código JavaScript del navegador envía a la aplicación Flask. 
        Por ejemplo, un botón para aplicar cambios de permisos envía una solicitud con el formato grant - user ID - read state - right stick.
        \item La aplicación Flask debe tener endpoints configurados para recibir y procesar estas solicitudes (por ejemplo, un endpoint para 
        /grant user ID read and write).
    \end{itemize}

\end{itemize}

Gestión de Usuarios y Permisos (Características Avanzadas)
\begin{itemize}
    \item Población Dinámica de la Interfaz: El servidor Flask es responsable de enviar detalles adicionales a la página web principal 
    (index.html), como el user ID del usuario actual y una lista de usuarios en línea.
    \item Obtención de Datos de Usuarios: Para obtener la lista de usuarios en línea, el servidor llama a una función get all logged in users 
    que devuelve un mapa (online user records) con información de cada usuario (nombre, ID, permisos de lectura y escritura).
    \item Visualización en Plantillas Jinja: Utiliza plantillas Jinja en el index.html para iterar sobre la lista de online user records 
    y mostrar dinámicamente cada usuario en una fila, con botones de interruptor para otorgar permisos de lectura y escritura. 
    Los estados de los interruptores se determinan por variables checked o unchecked enviadas desde el servidor Python.
    \item Control de Acceso (Admin/No-Admin): El servidor implementa lógica para asegurar que el panel de control completo solo sea 
    visible para  'usuarios administradores. Esto se logra añadiendo una sentencia if en el código HTML basada en el
     user ID proporcionado por el servidor.
    \item Procesamiento de Solicitudes de Permisos: Cuando el servidor Flask recibe una solicitud grant (por ejemplo, de un botón aplicar), 
    verifica si el remitente es un usuario administrador. Si la verificación es exitosa, procede a almacenar los permisos de lectura y 
    escritura del usuario en la base de datos y, si es necesario, a llamar al servidor PubNub para otorgar estos permisos específicos al usuario.
\end{itemize}

En resumen, el servidor web Flask en la Raspberry Pi es el núcleo técnico que une los sensores y actuadores con la interfaz de usuario, 
permitiendo la monitorización en tiempo real y el control de los dispositivos IoT a través de una red local, además de gestionar aspectos 
más complejos como la autenticación y los permisos de usuario.

\subsection{Solisitud de HTTP del Navegador del Usuario}
% 3.3.4 SOLISITUD DE HTTP DEL NAVEGADOR DEL USUARIO.
Las solicitudes HTTP del navegador del usuario son un componente fundamental en el funcionamiento técnico del dispositivo IoT de defensa antirrobo, 
ya que constituyen el principal medio de interacción entre el usuario y el servidor web Flask que se ejecuta en la Raspberry Pi. Estas solicitudes 
permiten la monitorización del estado del sensor y el control de los actuadores del sistema.

Aquí se dan los detalles de este aspecto:

\textbf{Acceso Inicial a la Página Web}
\begin{itemize}
    \item Petición Inicial: El funcionamiento técnico comienza cuando el usuario introduce la dirección IP local de la Raspberry Pi 
    (por ejemplo, 192.168.1.250) en su navegador. Al hacerlo, el usuario puede ir a esta dirección desde su navegador y solicitar una página web.
    \item Respuesta del Servidor: Como respuesta a esta solicitud HTTP, el servidor web Flask envía la página web principal (index.html) 
    al navegador del usuario.
\end{itemize}

\textbf{Solicitudes Keepalive para Actualizaciones en Vivo}
\begin{itemize}
    \item Comunicación Unidireccional (Servidor a Usuario): Una vez que el usuario ha cargado la página web, el navegador inicia un proceso de envío de 
    peticiones keepalive al servidor web de la Raspberry Pi periódicamente cada cinco segundos.
    \item Propósito: Estas solicitudes actúan como un latido para asegurar que la conexión con el servidor aún existe (el servidor sigue vivo).
    \item Envío de Datos: En respuesta a cada petición keepalive, el servidor permite que el servidor envíe el estado del sensor y los datos al usuario, 
    lo que facilita las actualizaciones en vivo del sensor en la página web del usuario. La página mostrará el estado de la detección de movimiento y el 
    estado de la conexión.
    \item Técnica Subyacente: Este mecanismo de keepalive es una forma de comunicación que se asemeja al long polling de AJAX, mencionada 
    como la técnica inicial del proyecto antes de la posible adopción de PubNub.
\end{itemize}

\textbf{Solicitudes de Control para Actuadores}
\begin{itemize}
    \item Comunicación Bidireccional (Usuario a Servidor): El navegador del usuario también envía solicitudes HTTP para permitir la interacción del 
    usuario con los actuadores. Se implementa un botón de interruptor en la página web que permite a los usuarios controlar los actuadores.
    \item Ejemplo Específico (Desactivar Alarma): Un caso de uso clave es la capacidad de desactivar el zumbador (la alarma). Cuando el usuario 
    interactúa con estos botones en la interfaz web, el código JavaScript del navegador genera y envía una solicitud HTTP al servidor Flask.
    \item Formato de la Solicitud: Por ejemplo, una solicitud para aplicar cambios de permisos se envía como grant - ID de usuario - estado 
    de lectura - estado de escritura. Estas son solicitudes POST enviadas al servidor.
    \item Endpoint del Servidor: La aplicación Flask en la Raspberry Pi tiene un endpoint para recibir esta solicitud, por ejemplo, /grant user 
    ID read and write.
\end{itemize}

\textbf{Gestión de Permisos (Solicitudes Grant)}
\begin{itemize}
    \item JavaScript a Flask: Específicamente, el archivo main.js en el navegador del usuario contiene una función que escucha los botones de 
    interruptor. Si un botón de interruptor comienza con XS, se extrae el ID del usuario, se lee el estado de los interruptores de lectura y 
    escritura, y se envía la solicitud como grant - ID de usuario - estado de lectura - estado de escritura al servidor Flask.
    \item Verificación de Admin: Antes de procesar los permisos, la aplicación Flask es capaz de verificar si esta solicitud grant proviene de 
    un usuario administrador.
    \item Almacenamiento y Notificación: Si el acceso es concedido, el servidor almacena los permisos de lectura y escritura en la base de datos y 
    puede llamar al servidor PubNub para otorgar estos permisos específicos al usuario.
\end{itemize}

\textbf{Consideraciones de Seguridad}
\begin{itemize}
    \item Cifrado SSL/TLS: Inicialmente, las solicitudes HTTP del navegador podrían ser inseguras. Sin embargo, se  describen un proceso para 
    asegurar nuestro dominio personalizado con certificados SSL/TLS de Let's Encrypt.
    \item Redirección a HTTPS: Una vez configurado, el servidor está diseñado para redirigir el tráfico HTTP a HTTPS.
    \item Comunicación Cifrada: Esto significa que cada cliente conectado con este servidor tendrá una comunicación cifrada de extremo a extremo, 
    garantizando la seguridad de los datos enviados entre el navegador del usuario y el servidor IoT.
    \item Reglas de Seguridad: Se requiere configurar las reglas de seguridad de entrada para HTTPS (puerto 443) en el servidor remoto 
    (AWS EC2 en el ejemplo) para permitir que el navegador se conecte de forma segura.
\end{itemize}

En resumen, las solicitudes HTTP del navegador del usuario son el mecanismo esencial para la interacción remota con el dispositivo IoT. 
Permiten desde la carga inicial de la interfaz, la recepción continua de datos del sensor, hasta el envío de comandos de control y la gestión 
de permisos, todo ello protegido mediante SSL/TLS para garantizar la seguridad de la comunicación.

\subsection{Solisitudes KEEPALIVE Periódicas (cada 5s)}
% 3.3.5 SOLISITUDES KEEPALIVE PERIóDICAS (CADA 5s).
Las solicitudes Keepalive Periódicas (Cada 5 segundos) son un elemento técnico fundamental en el funcionamiento del dispositivo IoT de defensa antirrobo, 
especialmente para la comunicación unidireccional del servidor al usuario y la monitorización en tiempo real.

A continuación, se dan los detalles de su funcionamiento técnico:

\textbf{Mecanismo y Frecuencia de las Solicitudes Keepalive}
\begin{itemize}
    \item Origen de la Solicitud:  Una vez que el usuario ha cargado la página web desde su navegador (accediendo a la dirección IP local de la Raspberry Pi), 
    el 'navegador del usuario '  comienza a enviar estas solicitudes.
    \item Destino:  Las peticiones se envían   'al servidor web de la Raspberry Pi ' .
    \item Periodicidad:  Estas solicitudes se envían   'periódicamente cada cinco segundos ' .
    \item Propósito Inicial (Latido):  La respuesta a cada solicitud keepalive se considera un   'latido para asegurarse de que la conexión con el servidor 
    aún existe ' , confirmando que   'el servidor sigue vivo ' .
\end{itemize}

\textbf{Función en la Actualización de Datos en Tiempo Real}
\begin{itemize}
    \item Envío de Datos del Sensor:  Más allá de solo verificar la conexión, la función principal de estas solicitudes en el contexto del proyecto es 
    permitir que el servidor   'envíe el estado del sensor y los datos al usuario '  en respuesta a cada keepalive.
    \item Actualizaciones en Vivo:  Este mecanismo asegura que el usuario reciba   'actualizaciones en vivo del sensor '  en la página web. La interfaz 
    web mostrará el   'estado de la detección de movimiento '  y el   'estado de la conexión ' .
    \item Comunicación Unidireccional:  Esto establece una   'comunicación unidireccional del servidor al usuario ' , un objetivo clave del proyecto.
\end{itemize}

\textbf{Contexto Técnico y Evolución}
\begin{itemize}
    \item Tecnología Subyacente:  Inicialmente, para la comunicación entre el cliente y el servidor, el proyecto utilizaba la   'técnica AJAX '. 
    Las solicitudes keepalive periódicas son una implementación que se asemeja al *long polling* de AJAX para simular la comunicación en tiempo real 
    en la interfaz web.
    \item Limitaciones y Mejoras:  Se  sugieren que, aunque este método permite las actualizaciones en vivo, el proyecto evolucionaría para usar 
    protocolos de comunicación más avanzados  y ligeros para IoT, como  WebSockets y MQTT , y específicamente la plataforma  PubNub , en lugar del 
    *long polling* de AJAX. PubNub se usaría como el principal protocolo de comunicación en fases posteriores. Esto implica que, si bien las solicitudes 
    keepalive con AJAX fueron el punto de partida, tecnologías más eficientes se adoptarían para mejorar la latencia y la bidireccionalidad.
    \item Acceso Local:  Es importante recordar que este servidor, en esta etapa del proyecto, está   'funcionando localmente ' . Por lo tanto, los 
    usuarios solo pueden acceder a él   'desde dentro de la red ' , es decir, deben estar   'conectados al mismo router Wi-Fi '  que la Raspberry Pi 
    para que estas solicitudes keepalive y la interacción general funcionen.
\end{itemize}
En resumen, las  solicitudes Keepalive Periódicas cada 5 segundos  son un pilar técnico para la  monitorización en tiempo real del estado de los sensores  
en el dispositivo IoT. Permiten al navegador del usuario mantener una  conexión activa  con el servidor Flask en la Raspberry Pi y recibir  actualizaciones 
constantes y en vivo  sobre la detección de movimiento y el estado del sistema, facilitando la comunicación unidireccional inicial del servidor al usuario.

\subsection{Botón en Página WEB para Controlar Actuadores (Desactivar Zumbador)}
% 3.3.6 BOTóN EN PáGINA WEB PARA CONTROLAR ACTUADORES (DESACTIVAR ZUMBADOR).
El  botón en la página web para controlar actuadores , específicamente para  desactivar el zumbador  (alarma), es un componente técnico crucial que 
habilita la  comunicación bidireccional del usuario al servidor  en el dispositivo IoT de defensa antirrobo. Permite que el usuario interactúe directamente 
con el sistema para controlar sus funciones.

A continuación, se detalla el funcionamiento técnico:
\textbf{Propósito y Presentación en la Interfaz de Usuario}
\begin{itemize}
    \item Comunicación Bidireccional:  Uno de los objetivos del proyecto es añadir una funcionalidad en la que los usuarios puedan comunicarse con el 
    servidor, cumpliendo con la comunicación bidireccional. El botón de control es la interfaz para esta interacción.
    \item Caso de Uso Principal:  El uso más destacado de esta funcionalidad es la capacidad de   'desactivar el zumbador ' , que actúa como la alarma 
    del sistema.
    \item Ubicación en la Web:  Una vez que el usuario accede a la página web ('index.html'), se añade un  'botón de interruptor ' o *switch button* 
    en la interfaz para permitir el control de los actuadores. Inicialmente, se  describen una sección con el  'estado de movimiento ' y un botón 
    para controlar el zumbador.
    \item Control de Permisos (Panel Admin):  En una etapa posterior del proyecto, se implementa un panel de control visible solo para usuarios 
    administradores, donde se listan los usuarios en línea y, junto a cada nombre, se encuentran botones de interruptor para   'conceder permisos de 
    lectura y escritura ' . También hay un  'botón para aplicar los cambios '. Estos botones tienen IDs dinámicos que incorporan el ID del usuario 
    (por ejemplo, `read user ID`, `write user ID`, `access user ID`).
\end{itemize}

\textbf{Mecanismo de Envío de Solicitudes (Front-end: JavaScript)}
\begin{itemize}
    \item Interacción del Usuario:  Cuando el usuario interactúa con estos botones en la página web, el código JavaScript (`main.js`) del navegador entra 
    en acción.
    \item Detección y Procesamiento:  El JavaScript tiene un método que  'escucha ' cualquier botón de interruptor en el dashboard. Si la ID de un 
    interruptor comienza con  'XS ' (para el caso de los permisos de acceso), se extrae el ID del usuario de la ID del botón.
    \item Formación de la Solicitud:  Luego, se lee el estado actual del interruptor de lectura y el interruptor de escritura. Finalmente, se  
    'envía la solicitud como `grant - ID de usuario - estado de lectura - estado de escritura` ' .
    \item Tipo de Solicitud:  Esta solicitud se envía al servidor como una   'solicitud POST '  utilizando un método como `send event`.
\end{itemize}

\textbf{Procesamiento en el Servidor (Back-end: Flask)}
\begin{itemize}
    \item Endpoint en Flask:  La aplicación Flask en la Raspberry Pi está configurada con un   'endpoint para recibir esta solicitud ' . Un ejemplo de 
    este endpoint sería `/grant user ID read and write`.
    \item Verificación de Admin:  Una vez recibida la solicitud, es recomendable  'verificar si esta solicitud grant proviene de un usuario administrador' 
    antes de procesarla. Si no es un administrador, la respuesta puede ser  'acceso denegado '.
    \item Actualización de la Base de Datos y PubNub:  Si la solicitud es válida y proviene de un administrador, el servidor realiza dos acciones principales:
    \begin{itemize}
        \item  'Almacenar los permisos de lectura y escritura del usuario en la base de datos ' .
        \item  'Llamar al servidor PubNub para conceder estos permisos específicos al usuario ' . Esto se hace a través de la funcionalidad de 
        'Access Manager de PubNub '.
    \end{itemize}
    \item Respuesta al Cliente:  El servidor envía una respuesta al navegador del usuario indicando si el acceso fue  'concedido ' o  'denegado'. 
    Si el acceso es concedido, el JavaScript en el cliente puede  'resetear la suscripción al canal '.
\end{itemize}

\textbf{Funcionamiento Técnico General y Seguridad}
\begin{itemize}
    \item Acceso Local:  Es fundamental recordar que, en esta etapa del proyecto, el servidor Flask funciona localmente, lo que significa que los 
    usuarios deben estar   'conectados al mismo router Wi-Fi '  que la Raspberry Pi para poder acceder a la página web y utilizar estos botones de control.
    \item Seguridad (SSL/TLS):  Todas las solicitudes HTTP enviadas desde el navegador del usuario, incluyendo las que provienen de los botones de control, 
    están sujetas a las medidas de seguridad implementadas. Una vez que el dominio personalizado se asegura con certificados SSL/TLS de Let's Encrypt y se configura la redirección de HTTP a HTTPS,   'cada cliente conectado con este servidor tendrá una comunicación cifrada de extremo a extremo ' . Esto garantiza la confidencialidad e integridad de los comandos de control enviados al dispositivo IoT. Se requiere configurar reglas de seguridad de entrada para HTTPS (puerto 443) en el servidor remoto para permitir esta conexión segura.
\end{itemize}
En síntesis, el  botón en la página web para controlar actuadores  representa una sofisticada interacción que comienza con una acción del usuario en la 
interfaz web, se traduce en una solicitud HTTP POST gestionada por JavaScript, es procesada por un servidor Flask que verifica permisos y actualiza 
la información en la base de datos y en PubNub, todo ello bajo una  comunicación cifrada de extremo a extremo  para garantizar la seguridad.

\subsection{Acceso solo Dentro de la Red Local}
% 3.3.7 ACCESO SOLO DENTRO DE LA RED LOCAL.
Se enfatizan que el  Acceso Solo Dentro de la Red Local  es una característica fundamental del funcionamiento 
técnico inicial del servidor IoT de defensa antirrobo, que se ejecuta en la Raspberry Pi. Esta limitación define cómo los usuarios pueden interactuar 
con el dispositivo en las primeras etapas del proyecto.

Aquí se dan los detalles de este punto en el contexto del funcionamiento técnico:

\textbf{Naturaleza del Servidor y Requisito de Acceso Local}
\begin{itemize}
    \item Servidor Local en Raspberry Pi:  El proyecto comienza con la creación de un servidor web HTTP básico de Python Flask en la Raspberry Pi. 
    Este servidor está diseñado para   'ejecutarse en nuestra red Wi-Fi local ' .
    \item Acceso Exclusivo Dentro de la Red:  Debido a que el servidor   'funciona localmente ' , los usuarios   'solo pueden acceder a él desde dentro 
    de la red ' . Esto significa que, para interactuar con el servidor, los usuarios   'deben estar conectados al mismo router Wi-Fi '  que la Raspberry Pi.
    \item Acceso por Dirección IP:  Para acceder a la página web del servidor, el usuario debe ir a la dirección IP local asignada a la Raspberry Pi por 
    el router Wi-Fi (por ejemplo, `192.168.1.250`) desde su navegador.
\end{itemize}

\textbf{Implicaciones en las Funcionalidades Técnicas}
\begin{itemize}
\item Este requisito de acceso local afecta directamente cómo funcionan varios componentes técnicos del sistema:
    \begin{itemize}
        \item Carga de la Página Web:  La interfaz web ('index.html'), que muestra el estado del sensor y los controles, solo puede ser cargada por 
        usuarios que están en la misma red local.
        \item Solicitudes Keepalive Periódicas:  Las   'solicitudes keepalive periódicas '  que el navegador del usuario envía al servidor cada cinco 
        segundos para verificar la conexión y recibir actualizaciones en tiempo real, solo son posibles si el usuario está conectado al mismo router 
        Wi-Fi [conversación previa].
        \item Control de Actuadores (Botón para Desactivar Zumbador):  De manera similar, la funcionalidad de usar un botón en la página web para 
        controlar actuadores, como   'desactivar el zumbador ' , también está restringida a usuarios dentro de la misma red local. Las solicitudes 
        POST generadas por JavaScript para controlar estos actuadores solo llegarán al servidor Flask si el cliente está en la red [conversación previa].
    \end{itemize}
\end{itemize}

\textbf{Evolución del Proyecto más Allá del Acceso Local}
\begin{itemize}
\item Aunque el acceso local es la configuración inicial y fundamental, el proyecto y el proyecto están diseñados para evolucionar y superar esta limitación:
    \begin{itemize}
        \item Despliegue en la Nube y Dominio Personalizado:  En secciones posteriores, el proyecto se centra en   'desplegar el servidor IoT en la nube 
        de AWS '  y asegurar un   'nombre de dominio personalizado '  con certificados SSL/TLS. Esto indica una transición de un entorno estrictamente 
        local a una infraestructura accesible a través de Internet.
        \item Seguridad Mejorada:  Al migrar a un servidor remoto con un dominio seguro (HTTPS),   'cada cliente conectado con este servidor tendrá una 
        comunicación cifrada de extremo a extremo ' . Esto es un contraste con la seguridad implícita (o su ausencia) en una red local inicial, 
        donde el acceso es intrínsecamente más limitado por la geografía.
    \end{itemize}
\end{itemize}
En resumen, el  acceso solo dentro de la red local  es una característica definitoria del  funcionamiento técnico inicial  del servidor IoT, donde 
la Raspberry Pi actúa como un servidor Flask básico. Requiere que todos los usuarios estén   'conectados al mismo router Wi-Fi '  para interactuar 
con la interfaz web, recibir datos en tiempo real y controlar actuadores. Si bien esta configuración simplifica el inicio del proyecto, se  
demuestran que el objetivo final es una solución IoT más robusta y accesible globalmente a través del despliegue en la nube y protocolos seguros.

\section{Configuración de Raspberry PI}
% 3.4.CONFIGURACIóN DE Raspberry PI.
En el contexto más amplio del  Primer Proyecto IoT: Detector de Movimiento  (Sección 2 del proyecto  'Internet de las Cosas con Python y Raspberry Pi'), 
la configuración de la Raspberry Pi es fundamental, ya que actúa como el  'cerebro ' central del sistema.

Aquí se detallan los siguientes aspectos clave para la  Configuración de la Raspberry Pi :
\begin{itemize}
    \item Requisitos de Almacenamiento:
    \begin{itemize}
        \item Se requiere una  tarjeta SD de al menos 8GB  para instalar el sistema operativo de la Raspberry Pi.
    \end{itemize}

    \item Instalación del Sistema Operativo :
    \begin{itemize}
        \item Se recomienda seguir las  instrucciones oficiales de configuración de Raspberry Pi  disponibles en su sitio web para instalar el sistema 
        operativo.
    \end{itemize}

    \item Acceso Remoto y SSH :
    \begin{itemize}
        \item Es posible realizar una  conexión de escritorio remoto  a la Raspberry Pi, una opción que el instructor utilizará.
        \item Para la  Raspberry Pi 3 , el protocolo  SSH está deshabilitado por defecto .
        \item Para habilitar SSH, lo cual es necesario para el acceso remoto por escritorio, los usuarios deben ir a la terminal en la Raspberry Pi, 
        escribir el comando \verb|sudo raspi-config| y seguir las instrucciones para activarlo. El uso de  Putty  para conectarse al servidor, como se 
        menciona en la sección 5, implica que se utilizará una conexión SSH.
    \end{itemize}
    
    \item Conectividad de Red :
    \begin{itemize}
        \item La Raspberry Pi alojará un  servidor web HTTP básico con Flask en Python  que se ejecutará en la  red Wi-Fi local .
        \item Los usuarios podrán acceder a la página web del dispositivo desde su navegador a través de la  dirección IP local  asignada a la 
        Raspberry Pi (por ejemplo, 192.168.1.250).
        \item Es importante destacar que el servidor funcionará localmente, lo que significa que los usuarios deben estar  conectados al mismo router 
        Wi-Fi  que la Raspberry Pi para poder acceder al sistema.
    \end{itemize}

    \item Una vez configurada, la Raspberry Pi se encargará de:
    \begin{itemize}
        \item Conectar el  sensor PIR  y el  buzzer .
        \item Ejecutar el  código Python básico  para detectar señales del sensor y controlar el buzzer.
        \item Alojar la  aplicación web  que permitirá la monitorización y el control del dispositivo.
    \end{itemize}
\end{itemize}

Esta configuración sienta las bases para el desarrollo del proyecto, permitiendo la interacción entre el hardware, el software y la interfaz de 
usuario a través de la red local.

\subsection{Tarjeta SD de 8GB Mínimo (OS)}
% 3.4.1 TARJETA SD DE 8GB MíNIMO (OS).
Aquí se detalla un requisito específico para la  Tarjeta SD de 8GB Mínimo  en el contexto de la 
Configuración de Raspberry Pi .

En concreto, para configurar la Raspberry Pi y comenzar con el proyecto, se establece que:
\begin{itemize}
    \item Requisito de Tarjeta SD:  Se requiere una   'tarjeta SD de 8GB como mínimo ' .
    \item Propósito (Instalación del Sistema Operativo):  El propósito principal de esta tarjeta SD es la   'instalación del sistema operativo'  en ella.
\end{itemize}

Esto significa que la tarjeta SD es un componente esencial para el funcionamiento básico de la Raspberry Pi, ya que alberga el sistema operativo que 
permite que el dispositivo arranque y ejecute el servidor IoT y los scripts de Python. Sin esta tarjeta SD con el sistema operativo instalado, la 
Raspberry Pi no podría iniciar ni realizar ninguna de las tareas del proyecto.

Además, la configuración general de la Raspberry Pi mencionada incluye pasos como:
\begin{itemize}
    \item Seguir las instrucciones oficiales de configuración de Raspberry Pi para instalar el sistema operativo en la tarjeta SD.
    \item Posiblemente necesitar un monitor HDMI o realizar una conexión de escritorio remoto (Remote Desktop) a la Raspberry Pi para completar la 
    configuración inicial.
    \item Habilitar SSH si está deshabilitado por defecto (como en Raspberry Pi 3) para permitir el escritorio remoto.
\end{itemize}

\subsection{Monitor HDMI O Escritorio Remoto}
% 3.4.2 MONITOR HDMI O ESCRITORIO REMOTO.
Para la  Configuración de Raspberry Pi  en el contexto del proyecto de IoT, se  indican que es esencial contar con un método para interactuar 
con la Raspberry Pi, ya sea a través de un  Monitor HDMI  o mediante  Escritorio Remoto .

Aquí se detalla este aspecto:
\begin{itemize}
    \item Necesidad de un Método de Interacción:  
    \begin{itemize}
        \item Para la configuración inicial de la Raspberry Pi, que incluye seguir las instrucciones oficiales de configuración, se requiere una forma 
        de visualizar y controlar el sistema.
        \begin{itemize}
            \item Opción 1: Monitor HDMI:  Se menciona que   'puede que necesite un monitor HDMI '  para realizar la configuración. Esto implica 
            una conexión física directa a la Raspberry Pi para acceder a su interfaz gráfica o terminal.
            \item Opción 2: Escritorio Remoto:  Si no se dispone de un monitor HDMI, el   'escritorio remoto a la Raspberry Pi '  es una alternativa 
            viable que el instructor mismo utilizará. Existen  'instrucciones sencillas en el video de YouTube proporcionado ' para llevar a cabo esta 
            conexión.
        \end{itemize}
    \end{itemize}
    \item Habilitación de SSH para Escritorio Remoto (Específico para Raspberry Pi 3):  Un punto técnico crucial es que, para la Raspberry Pi 3, el 
    protocolo SSH (Secure Shell)   'está deshabilitado por defecto ' . Para poder realizar el escritorio remoto, es 'necesario habilitar el SSH'.
    \item Procedimiento para Habilitar SSH:  Para activar SSH, el usuario debe ir a la terminal en la Raspberry Pi y 'escribir \verb|sudo raspi-config| y 
    seguir las instrucciones para activar el SSH ' .
\end{itemize}
En resumen, la capacidad de ver y controlar la Raspberry Pi durante su configuración inicial es fundamental, ofreciendo la flexibilidad de elegir entre 
una conexión física con un  Monitor HDMI  o una conexión virtual a través de  Escritorio Remoto , siendo este último un proceso que en ciertos modelos 
como la Raspberry Pi 3 requiere la  habilitación explícita de SSH .

\subsection{Habilitar SSH}
% 3.4.3 HABILITAR SSH.
En el contexto más amplio de la  Configuración de Raspberry Pi  para el proyecto de IoT, se destacan la 
importancia de  habilitar SSH (\verb|sudo raspi-config|), especialmente cuando se utiliza el 'Escritorio Remoto' como método de interacción.

Aquí se detalla este aspecto:
\begin{itemize}
    \item Necesidad del Escritorio Remoto:  Si un usuario no dispone de un monitor HDMI para configurar su Raspberry Pi, el  Escritorio Remoto  es 
    una alternativa viable para interactuar con el dispositivo. De hecho, el instructor mismo utilizará este método.
    \item SSH Deshabilitado por Defecto (\textbf{en Raspberry Pi 3}):  Un punto técnico crucial es que, para modelos específicos como la  Raspberry Pi 3 , el 
    protocolo SSH (Secure Shell)   'está deshabilitado por defecto ' .
    \item Requisito para el Escritorio Remoto:  Para poder realizar la conexión de Escritorio Remoto a la Raspberry Pi, es 'necesario habilitar el SSH '.
    \item Procedimiento para Habilitar SSH:  Se proporcionan instrucciones claras sobre cómo activar SSH:
    \begin{itemize}
        \item El usuario debe 'ir a la terminal en la Raspberry Pi'.
        \item Una vez en la terminal, debe escribir \verb|sudo raspi-config|.
        \item Finalmente, debe   'seguir las instrucciones para activar el SSH '  dentro de la interfaz de configuración de `raspi-config`.
    \end{itemize}
\end{itemize}
En síntesis, la habilitación de SSH es un paso fundamental en la  configuración de la Raspberry Pi , particularmente cuando se opta por el acceso remoto en 
lugar de una conexión física con un monitor HDMI. Este proceso asegura la capacidad de establecer una conexión segura y de controlar la Raspberry 
Pi a distancia, siendo un requisito indispensable para ciertos modelos como la Raspberry Pi 3.

\chapter{PROTOCOLOS DE COMUNICACIóN IoT}
% 4.PROTOCOLOS DE COMUNICACIóN IoT.
En el contexto más amplio del proyecto   'Internet de las Cosas con Python y Raspberry Pi ' , los  Protocolos de Comunicación IoT  son un pilar fundamental 
para el desarrollo de sistemas que permiten a los dispositivos interactuar entre sí y con los usuarios. El proyecto explora varias tecnologías, desde 
las más básicas hasta las más avanzadas y seguras, para establecer una comunicación eficaz y robusta.

A continuación, se dan los detalles de estos protocolos:
\begin{itemize}
    \item Modelos de Comunicación Iniciales (Sección 1) :
    \begin{itemize}
        \item El proyecto comienza familiarizando a los estudiantes con  cuatro modelos importantes de comunicación  en Internet de las Cosas. 
        Aunque no se  detalla, se establece una base para entender las interacciones entre dispositivos y servidores.
    \end{itemize}

    \item HTTP y AJAX para Comunicación Unidireccional y Bidireccional (Sección 2 - Primer Proyecto IoT) :
    \begin{itemize}
        \item En el  Primer Proyecto IoT: Detector de Movimiento , se utiliza un enfoque inicial basado en  HTTP  y la técnica  AJAX  para la comunicación 
        entre el cliente (navegador del usuario) y el servidor (Raspberry Pi).
        \item Servidor HTTP Básico : La Raspberry Pi ejecuta un servidor web HTTP básico con Flask en Python en la red Wi-Fi local. Los usuarios acceden 
        a la página web a través de la dirección IP local de la Raspberry Pi.
        \item Comunicación Unidireccional (Servidor a Usuario) : Para obtener actualizaciones en vivo, el navegador del usuario envía solicitudes 
        'keep-alive ' periódicamente (cada cinco segundos) al servidor de la Raspberry Pi. La respuesta a estas solicitudes permite al servidor enviar 
        el estado del sensor y los datos al usuario, estableciendo una comunicación unidireccional. La interfaz web muestra el estado de detección de 
        movimiento y el estado de la conexión.
        \item Comunicación Bidireccional (Usuario a Servidor) : Se añade una funcionalidad para que los usuarios puedan interactuar con los actuadores, 
        como un botón en la página web para desactivar la alarma (buzzer), lo que logra una comunicación bidireccional desde el usuario al servidor.
        \item Limitación Local : Es importante destacar que, en este proyecto inicial, el servidor funciona localmente, requiriendo que los usuarios 
        estén conectados a la misma red Wi-Fi que la Raspberry Pi para acceder al sistema.
    \end{itemize}

    \item Protocolos de Comunicación en Tiempo Real y Ligeros (Sección 3) :
    \begin{itemize}
        \item Tras discutir las ventajas y desventajas del enfoque inicial de la Sección 2, el proyecto profundiza en  protocolos de comunicación en tiempo 
        real y ligeros  más avanzados para IoT:  MQTT  y  WebSockets , incluyendo demostraciones prácticas con  PubNub .
    \end{itemize}

    \item WebSockets :
    \begin{itemize}
        \item Es una tecnología avanzada que permite una  sesión de comunicación interactiva  abriendo una única conexión TCP.
        \item Su característica principal es que permite a los clientes recibir actualizaciones  solo cuando ocurren , sin necesidad de  'polling' 
        (preguntar constantemente al servidor).
        \item Es  bidireccional y full-duplex , lo que significa que ambas partes pueden enviar mensajes de forma independiente.
        \item Fue estandarizado en 2011 y es muy adoptado en aplicaciones IoT debido a su  baja latencia y naturaleza bidireccional.
        \item Funcionamiento : Inicialmente, un cliente realiza una solicitud HTTP para  'actualizar ' la conexión al protocolo WebSockets. 
        El servidor reconoce el  'handshake ', y la sesión permanece abierta y persistente hasta que cualquiera de las partes la cierra. 
        Los paquetes de datos enviados durante la sesión son muy pequeños, con una longitud de trama de 2 a 14 bytes.
    \end{itemize}
    
    \item MQTT (Message Queuing Telemetry Transport):
    \begin{itemize}
        \item Es un protocolo de transferencia de mensajes  ligero  para comunicación máquina a máquina (M2M) e IoT.
        \item Es altamente  eficiente en el uso de ancho de banda , con solo 2 bytes de sobrecarga.
        \item Ofrece escenarios de transmisión de datos  uno a uno, uno a muchos y muchos a muchos  mediante un  modelo de publicación y suscripción .
        \item Funcionamiento : Se basa en un  broker  central que actúa como punto de comunicación y es el encargado de despachar todos los mensajes 
        entre remitentes y receptores. Los clientes publican mensajes en un  tópico  específico (que sirve como información de enrutamiento), y los 
        dispositivos que están suscritos a ese tópico reciben los datos.
        \item Escalabilidad : Este modelo lo hace altamente escalable, ya que los clientes no necesitan conocerse entre sí, solo comunicarse a través 
        del tópico.
        \item Inconveniente : Su principal desventaja es que, al depender de una entidad central (el broker), si este falla, toda la comunicación se interrumpe.
        \item Relación con WebSockets : MQTT funciona sobre la capa de red TCP/IP y puede utilizarse sobre la capa de WebSockets. Se aclaran 
        que no deben confundirse, ya que MQTT es un  'servicio de entrega ' que opera *encima* de WebSockets, empacando los paquetes de datos de MQTT 
        dentro de un  'sobre ' de WebSocket, que a su vez se envuelve en un  'sobre ' TCP/IP antes de ser enviado por internet.
    \end{itemize}

    \item Seguridad y Cifrado (SSL/TLS - HTTPS) (Secciones 3 y 5) :
    \begin{itemize}
        \item El proyecto también aborda la  seguridad en internet y la criptografía , lo que incluye protocolos  SSL/TLS (HTTPS) .
        \item En la Sección 5, se enfoca en asegurar un dominio personalizado con certificados  SSL/TLS de Let's Encrypt . Let's Encrypt es una 
        autoridad de certificación gratuita, automatizada y de código abierto.
        \item El proceso implica instalar software de terceros ('certbot') en el servidor para obtener los certificados.
        \item Una vez configurado, el tráfico  HTTP se redirige a HTTPS , lo que garantiza una  comunicación cifrada de extremo a extremo  para 
        cada cliente conectado al servidor.
        \item Para que funcione correctamente, se deben añadir  reglas de seguridad de entrada para HTTPS (puerto 443)  en el servidor remoto 
        (por ejemplo, en AWS EC2).
        \item Los certificados de Let's Encrypt son válidos por tres meses y deben renovarse al expirar.
    \end{itemize}
\end{itemize}
En resumen, el proyecto avanza desde una comunicación básica local con HTTP y AJAX para un proyecto inicial, para luego explorar protocolos más 
avanzados, ligeros y en tiempo real como WebSockets y MQTT, que son fundamentales para la escalabilidad y eficiencia en IoT. Además, se enfatiza 
la importancia de la seguridad en la comunicación mediante la implementación de SSL/TLS para cifrar las conexiones.

\section{Websockets}
% 4.1 WEBSOCKETS
Se proporciona una discusión detallada sobre  WebSockets  en el contexto de los  Protocolos de Comunicación IoT , destacando su funcionalidad, 
ventajas e integración con otros protocolos como MQTT y PubNub.

Aquí se desglosa lo que se  dice sobre WebSockets:
\begin{itemize}
    \item 1. Definición y Propósito Principal
    \begin{itemize}
        \item  Tecnología Avanzada para Comunicación Interactiva : WebSockets se describe como una tecnología avanzada que permite abrir sesiones de 
        comunicación interactivas mediante la apertura de una única conexión TCP.
        \item  Actualizaciones Sin Petición Explícita : Permiten que los clientes reciban actualizaciones solo cuando ocurren,  sin necesidad de que el 
        cliente esté  'preguntando ' o sondeando al servidor  (polling). Esto significa que el cliente puede enviar mensajes al servidor y recibir 
        respuestas basadas en eventos sin tener que solicitar actualizaciones constantemente.
        \item  Comunicación Bi-direccional y Full Duplex : WebSockets son fundamentalmente  bi-direccionales y full duplex , lo que significa que 
        ambas partes (cliente y servidor) pueden enviar mensajes de forma independiente entre sí.
    \end{itemize}

    \item 2. Cómo Funcionan los WebSockets
    \begin{itemize}
        \item  Proceso de Negociación (Handshake) : Inicialmente, un cliente realiza una solicitud HTTP al servidor, pidiéndole que actualice el 
        protocolo a WebSockets.
        \item  Conexión Persistente : Si el servidor reconoce y acepta esta negociación (handshake), la  sesión se mantiene abierta y persistente  
        durante todo el tiempo, a menos que alguna de las partes la cierre.
        \item  Pequeños Paquetes de Datos : Durante una sesión abierta, los paquetes de datos que se envían entre el cliente y el servidor son muy 
        pequeños, con una  longitud de trama de 2 a 14 bytes .
    \end{itemize}

    \item 3. Ventajas y Adopción en IoT
    \begin{itemize}
        \item  Baja Latencia : Debido a su naturaleza bi-direccional y persistente, WebSockets ofrecen  baja latencia  en la comunicación.
        \item  Escalabilidad en Aplicaciones en Tiempo Real : Aunque se estandarizaron en 2011 y fueron adoptados inicialmente por aplicaciones 
        como juegos y sistemas de chat, su  baja latencia y naturaleza bi-direccional  los hacen altamente adoptados en aplicaciones IoT.
        \item  Componente del ProyectoIoT : El proyecto  'Internet de las Cosas con Python y Raspberry Pi ' se enfoca en comprender y resolver 
        problemas del mundo real de IoT, y WebSockets es uno de los  protocolos de comunicación en tiempo real y ligeros  que se estudian en 
        profundidad en la Sección 3.
    \end{itemize}

    \item 4. Relación con MQTT
    \begin{itemize}
        \item  MQTT sobre WebSockets : Se aclaran que WebSockets y MQTT son cosas diferentes, pero  MQTT puede ejecutarse sobre la capa de 
        WebSockets .
        \item  Analogía de  'Entrega de Servicio ' : Se utiliza una analogía para explicar su relación: MQTT se considera un  'servicio de entrega' 
        (como DHL), mientras que  WebSockets proporcionan la  'infraestructura ' o las  'carreteras ' .
        \item  Anidación de Paquetes : Un paquete de datos MQTT se  empaqueta dentro de un  'sobre ' de WebSocket , que a su vez se envuelve en un 
        'sobre ' TCP/IP antes de ser enviado por Internet. El proceso inverso ocurre al desempaquetarse.
    \end{itemize}
\end{itemize}

\subsection{Actualizaciones solo Cuando Ocurren (sin sondeo)}
% 4.1.1 ACTUALIZACIONES SOLO CUANDO OCURREN (SIN SONDEO).
Aquí se explica de manera detallada el concepto de 'Actualizaciones Solo Cuando Ocurren (Sin Sondeo)'  
en el contexto más amplio de  WebSockets.

En el ámbito de la comunicación en el Internet de las Cosas (IoT), se  destacan que los  WebSockets  permiten a los clientes recibir 
actualizaciones   'solo cuando ocurren '  y, crucialmente,   'sin que el cliente pregunte al servidor' . Esto contrasta con métodos más tradicionales 
que requieren que el cliente  'sondee ' (o  'polee ') al servidor periódicamente para verificar si hay nuevas actualizaciones.

Las características clave que permiten este modelo de actualización son:
\begin{itemize}
    \item Comunicación Interactiva Bidireccional:  WebSockets es una tecnología avanzada que posibilita abrir una   'sesión de comunicación 
    interactiva'  mediante la apertura de una   'única conexión TCP ' . Esta sesión se mantiene   'abierta y persistente '  a menos que una 
    de las partes la cierre.
    \item Respuestas Dirigidas por Eventos:  Gracias a esta conexión persistente, el cliente puede   'enviar mensajes al servidor y recibir una 
    respuesta impulsada por eventos sin tener que sondear o preguntar al servidor '  por actualizaciones. Esto significa que el servidor envía datos 
    al cliente tan pronto como están disponibles, en lugar de esperar una solicitud del cliente.
    \item Naturaleza Full-Duplex:  La comunicación es   'bidireccional y full-duplex ' , lo que implica que   'ambas partes pueden enviar mensajes 
    independientemente una de la otra ' . Esta capacidad permite un flujo de datos más eficiente y en tiempo real.
    \item Baja Latencia y Eficiencia:  Debido a su   'baja latencia y naturaleza bidireccional ' , WebSockets es una tecnología  'altamente adoptada 
    en aplicaciones de IoT ' . El tamaño de los paquetes de datos enviados durante una sesión abierta es   'muy pequeño' , 
    de 2 a 14 bytes, lo que contribuye a su eficiencia.
    \item Proceso de Conexión:  Inicialmente, un cliente realiza una   'solicitud HTTP y pide al servidor que actualice al protocolo WebSockets '. 
    El servidor   'reconoce el handshake como respuesta ' , y luego la sesión permanece abierta.
\end{itemize}
En resumen, los  WebSockets  facilitan las   'actualizaciones solo cuando ocurren (sin sondeo) '  al establecer una conexión 
persistente y bidireccional entre el cliente y el servidor. Esto permite que el servidor envíe datos al cliente en tiempo real, 
tan pronto como estén disponibles, eliminando la necesidad de que el cliente realice solicitudes periódicas, lo cual es 
fundamental para aplicaciones de IoT que requieren una comunicación eficiente y de baja latencia.

\subsection{Sesión de Comunicación Interactiva Persistente (una conexión TCP)}
% 4.1.2 SESIóN DE COMUNICACIóN INTERACTIVA PERSISTENTE (UNA CONEXIóN TCP).
Aquí se explica que la  Sesión de Comunicación Interactiva Persistente (Una Conexión TCP)  es una característica fundamental 
de los  WebSockets  que revoluciona la forma en que los clientes y servidores se comunican, especialmente en el ámbito del Internet de las Cosas (IoT).

En el contexto más amplio de los WebSockets, se  detallan lo siguiente sobre esta sesión:
\begin{itemize}
    \item Establecimiento de una Conexión Única y Duradera:  WebSockets es una tecnología avanzada que permite establecer una   'sesión de 
    comunicación interactiva '  mediante la apertura de una   'única conexión TCP ' . Una vez establecida, esta sesión se   'mantiene abierta y persistente '  
    durante todo el tiempo, a menos que una de las partes decida cerrarla.
    \item Proceso de Conexión Inicial ( 'Handshake '):  La conexión se inicia cuando un cliente envía una   'solicitud HTTP '  
    al servidor, pidiéndole que   'actualice al protocolo WebSockets ' . El servidor, al reconocer esta solicitud, envía una respuesta que se conoce 
    como 'handshake ' , y a partir de ese momento, la sesión se considera abierta y persistente.
    \item Comunicación Bidireccional y Full-Duplex:  La naturaleza de esta conexión persistente permite una comunicación   'bidireccional y full-duplex '. 
    Esto significa que   'ambas partes pueden enviar mensajes independientemente una de la otra ' .
    \item Actualizaciones Basadas en Eventos (Sin Sondeo):  Gracias a esta sesión persistente, los clientes pueden   'enviar mensajes al servidor y recibir 
    una respuesta impulsada por eventos sin tener que sondear o preguntar al servidor '  por actualizaciones. Como se debatió previamente, esto se traduce 
    en   'actualizaciones solo cuando ocurren ' , eliminando la necesidad de que el cliente solicite información periódicamente.
    \item Beneficios en Aplicaciones IoT:  Debido a su   'baja latencia y naturaleza bidireccional ' , la tecnología WebSocket ha sido  
    'altamente adoptada en aplicaciones de IoT ' . La eficiencia también es un factor clave, ya que los paquetes de datos enviados durante 
    una sesión abierta son   'muy pequeños ' , con una longitud de trama de 2 a 14 bytes.
\end{itemize}
En resumen, la  sesión de comunicación interactiva persistente sobre una única conexión TCP  es el corazón del funcionamiento de WebSockets. 
Permite una interacción en tiempo real, bidireccional y eficiente entre clientes y servidores, facilitando que las actualizaciones se entreguen 
solo cuando ocurren y sin necesidad de sondeo, lo cual es vital para aplicaciones como el monitoreo y control en el Internet de las Cosas.

\subsection{Bidireccional Y Full Duplex}
% 4.1.3 BIDIRECCIONAL Y FULL DUPLEX.
Se describen los conceptos de comunicación bidireccional y full duplex principalmente en el contexto de WebSockets.
Los WebSockets son una tecnología avanzada que posibilita abrir una sesión de comunicación interactiva manteniendo una única conexión TCP. 
Esta característica los hace bidireccionales y full duplex.
A continuación, se detalla lo que significan estos términos:
\begin{itemize}
    \item Bidireccional y Full Duplex: Con WebSockets, ambas partes (el cliente y el servidor) pueden enviar mensajes de forma independiente 
    una de la otra. Esto significa que los clientes pueden recibir actualizaciones solo cuando estas ocurren, sin necesidad de que el cliente 
    le pregunte al servidor. Permiten enviar mensajes al servidor y recibir respuestas impulsadas por eventos sin tener que sondear o preguntar 
    al servidor por actualizaciones. Esta naturaleza de baja latencia y bidireccionalidad ha llevado a su adopción en aplicaciones de IoT.
    \item Históricamente, antes de su estandarización en 2011, aplicaciones como juegos, sistemas de chat y aplicaciones web ya aprovechaban 
    al máximo el potencial de los WebSockets. En el contexto de IoT, los WebSockets son una de las tecnologías más adoptadas para protocolos 
    de comunicación.
\end{itemize}
En resumen, los WebSockets establecen una conexión persistente que permite un flujo de comunicación constante y simultáneo en ambas direcciones 
entre el cliente y el servidor, lo que se define como bidireccional y full duplex.

\subsection{Baja Latencia}
% 4.1.4 BAJA LATENCIA.
En el contexto de WebSockets, se  destacan que una de sus características clave es la baja latencia:
\begin{itemize}
    \item Específicamente, los WebSockets son una tecnología avanzada que posibilita abrir una sesión de comunicación interactiva manteniendo 
    una única conexión TCP. Permiten que los clientes reciban actualizaciones solo cuando estas ocurren, sin necesidad de que el cliente le 
    pregunte al servidor. Esto significa que tanto el cliente como el servidor pueden enviar mensajes de forma independiente el uno del otro, 
    una cualidad que los hace bidireccionales y full duplex.
    \item Debido a su naturaleza de baja latencia y bidireccional, los WebSockets han sido ampliamente adoptados en aplicaciones de Internet 
    de las Cosas (IoT). Esta característica es fundamental porque permite una comunicación más rápida y eficiente, donde las actualizaciones 
    se entregan casi instantáneamente en el momento en que suceden, sin retrasos significativos causados por el sondeo constante al servidor.
\end{itemize}

\subsection{Funcionamiento}
% 4.1.5 FUNCIONAMIENTO.
Se proporcionan una descripción clara del funcionamiento de los WebSockets, especialmente en el contexto de la comunicación en el 
Internet de las Cosas (IoT) y aplicaciones en tiempo real:
\begin{itemize}
    \item 1. Establecimiento de la Conexión: Inicialmente, un cliente realiza una solicitud HTTP al servidor, pidiéndole que actualice el 
    protocolo a WebSockets.
    \item 2. Handshake y Persistencia: El servidor reconoce esta solicitud mediante un 'handshake' como respuesta. Una vez que este 
    handshake es exitoso, la sesión se mantiene abierta y persistente durante todo el tiempo, a menos que alguna de las partes la cierre.
    \item 3. Comunicación Bidireccional y Full Duplex: Una vez establecida la conexión, los WebSockets permiten una comunicación interactiva 
    donde tanto el cliente como el servidor pueden enviar mensajes de forma independiente el uno del otro. Esto significa que los clientes 
    pueden recibir actualizaciones solo cuando ocurren, sin necesidad de consultar o preguntar constantemente al servidor por ellas.
    \item 4. Paquetes de Datos Ligeros: Durante la sesión, los paquetes de datos que se envían entre el cliente y el servidor son muy 
    pequeños, con una longitud de trama de 2 a 14 bytes. Esta característica contribuye a su baja latencia.
\end{itemize}
Los WebSockets se estandarizaron en 2011, y su naturaleza de baja latencia y bidireccional los ha hecho muy adoptados en aplicaciones IoT, 
juegos, sistemas de chat y aplicaciones web. Es importante destacar que, aunque MQTT puede ejecutarse sobre la capa de WebSocket, son 
protocolos diferentes; los WebSockets proporcionan la 'infraestructura' de comunicación (como las carreteras), mientras que MQTT es un 
'servicio de entrega' que opera sobre ella.

\subsubsection{CLIENTE SOLICITA ACTUALIZACIóN A PROTOCOLO WEBSOCKETS (HTTP)}
% 4.1.5.1 CLIENTE SOLICITA ACTUALIZACIóN A PROTOCOLO WEBSOCKETS (HTTP).
En el contexto del funcionamiento de los WebSockets, se  explican claramente el rol del cliente en la solicitud de actualización 
al protocolo WebSocket a través de HTTP.
El proceso se describe de la siguiente manera:
\begin{itemize}
    \item Solicitud Inicial del Cliente: Inicialmente, un cliente realiza una solicitud HTTP y le pide al servidor que actualice al protocolo 
    WebSockets. Esta es la fase de inicio para establecer la comunicación WebSocket.
    \item Reconocimiento del Servidor (Handshake): El servidor reconoce esta solicitud a través de un 'handshake' como respuesta.
    \item Conexión Persistente: Una vez que el handshake es exitoso, la sesión se mantiene abierta y persistente durante todo el tiempo, a 
    menos que alguna de las partes la cierre.
\end{itemize}
Este paso es fundamental porque, a partir de esta solicitud inicial HTTP y el subsiguiente handshake, se establece la conexión TCP única que 
permite la sesión de comunicación interactiva, bidireccional y full duplex de los WebSockets. Posteriormente, durante esta sesión abierta, 
los paquetes de datos que se envían entre el cliente y el servidor son muy pequeños, con una longitud de trama de 2 a 14 bytes, lo que contribuye 
a la baja latencia de los WebSockets.

\subsubsection{SERVIDOR RECONOCE HANDSHAKE COMO RESPUESTA}
% 4.1.5.2 SERVIDOR RECONOCE HANDSHAKE COMO RESPUESTA.
En el contexto más amplio del funcionamiento de los WebSockets, se  explican que el servidor reconoce el handshake como un paso 
crucial para establecer una comunicación persistente y bidireccional.
El proceso se desarrolla de la siguiente manera:
\begin{itemize}
    \item 1. Solicitud Inicial del Cliente: Primero, un cliente realiza una solicitud HTTP al servidor, pidiéndole que 'actualice' el 
    protocolo a WebSockets.
    \item 2. Reconocimiento del Servidor (Handshake): En respuesta a esta solicitud, el servidor reconoce el 'handshake'. Este reconocimiento 
    es la señal de que la solicitud de actualización ha sido aceptada.
    \item 3. Establecimiento de la Sesión Persistente: Una vez que el servidor ha reconocido exitosamente el handshake, la sesión se mantiene 
    abierta y persistente durante todo el tiempo, a menos que alguna de las partes decida cerrarla.
\end{itemize}
Este 'handshake' inicial es fundamental porque transforma una solicitud HTTP convencional en una conexión TCP única y persistente, lo que 
permite la comunicación interactiva y eficiente que caracteriza a los WebSockets. Después de este establecimiento, la comunicación se vuelve 
bidireccional y full duplex, permitiendo que tanto el cliente como el servidor envíen mensajes de forma independiente, con paquetes de datos 
muy pequeños (de 2 a 14 bytes), lo que contribuye a su baja latencia.

\subsubsection{SESIóN ABIERTA Y PERSISTENTE}
% 4.1.5.3 SESIóN ABIERTA Y PERSISTENTE.
En el contexto más amplio del funcionamiento de los WebSockets, se  explican que la característica de una sesión abierta y 
persistente es un resultado directo del proceso de establecimiento de la conexión.
El funcionamiento se describe de la siguiente manera:
\begin{itemize}
    \item 1. Solicitud Inicial del Cliente: Un cliente inicia el proceso realizando una solicitud HTTP al servidor, pidiéndole que actualice 
    el protocolo a WebSockets.
    \item 2. Reconocimiento del Servidor (Handshake): El servidor responde a esta solicitud reconociendo el 'handshake'.
    \item 3. Establecimiento de la Sesión Abierta y Persistente: Una vez que el handshake es exitoso, la sesión se mantiene abierta y 
    persistente durante todo el tiempo, a menos que alguna de las partes la cierre.
\end{itemize}
Esta sesión persistente es fundamental para las ventajas que ofrecen los WebSockets, como la comunicación bidireccional y full duplex, 
donde tanto el cliente como el servidor pueden enviar mensajes de forma independiente. Además, durante esta sesión abierta, los paquetes 
de datos intercambiados son muy pequeños (con una longitud de trama de 2 a 14 bytes), lo que contribuye a la baja latencia de los WebSockets. 
La capacidad de mantener una conexión única TCP y una sesión interactiva es lo que permite a los clientes recibir actualizaciones solo cuando 
ocurren, sin necesidad de consultarle al servidor constantemente.

\subsubsection{PAQUETES DE DATOS PEQUEñOS}
% 4.1.5.4 PAQUETES DE DATOS PEQUENTOS.
En el contexto más amplio del funcionamiento de los WebSockets, se  destacan la característica de los paquetes de datos pequeños 
(2 a 14 bytes) como un factor importante en su eficiencia.
El funcionamiento de los WebSockets se desarrolla de la siguiente manera:
\begin{itemize}
    \item 1. Solicitud Inicial y Handshake: El proceso comienza cuando un cliente envía una solicitud HTTP al servidor, solicitando 
    una 'actualización' al protocolo WebSockets. El servidor, a su vez, reconoce esta solicitud mediante un 'handshake'.
    \item 2. Sesión Abierta y Persistente: Una vez que el handshake es exitoso, se establece una sesión que se mantiene abierta y persistente. 
    Esta conexión TCP única permite una comunicación interactiva continua entre el cliente y el servidor.
    \item 3. Intercambio de Paquetes de Datos Pequeños: Durante esta sesión abierta, los paquetes de datos que se envían entre el cliente y el 
    servidor son muy pequeños, con una longitud de trama de 2 a 14 bytes.
\end{itemize}
Esta característica de los paquetes de datos ligeros contribuye directamente a la baja latencia de los WebSockets, lo que los hace ideales para 
aplicaciones donde las actualizaciones deben ser casi instantáneas, como en juegos, sistemas de chat y aplicaciones IoT.

\section{MQTT (Message Queuing Telemetry Transport)}
% 4.2 MQTT (MESSAGE QUEUING TELEMETRY TRANSPORT)
Se proporcionan una visión detallada de  MQTT (Message Queuing Telemetry Transport)  en el contexto de los Protocolos de Comunicación IoT, 
destacando su eficiencia, modelo de funcionamiento y su relación con otras tecnologías.

Aquí se desglosa lo que se  dice sobre MQTT:
\begin{itemize}
    \item 1. Definición y Características Fundamentales
    \begin{itemize}
        \item Protocolo de Mensajería Ligero : MQTT se describe como un   'protocolo de transferencia de mensajería ligero'  
        (Message Queuing Telemetry Transport) diseñado específicamente para la comunicación máquina a máquina (M2M) e Internet de las Cosas.
        \item Eficiencia en Ancho de Banda : Es   'muy eficiente en ancho de banda ' , utilizando solo 2 bytes de sobrecarga, lo que lo hace 
        ideal para entornos con reproyectos limitados.
        \item Comunicación en Tiempo Real y Ligera : Junto con WebSockets, MQTT es considerado uno de los 'protocolos de comunicación en tiempo real 
        y ligeros '  más adoptados en aplicaciones IoT.
    \end{itemize}

    \item 2. Modelo de Funcionamiento: Publicar/Suscribir
    \begin{itemize}
        \item Diversos Escenarios de Datos : MQTT soporta escenarios de transmisión de datos  uno a uno, uno a muchos y muchos a muchos  para 
        dispositivos y aplicaciones.
        \item Modelo Publicar/Suscribir : Este modelo se logra a través de un esquema de   'publicar y suscribir' . Los datos se envían sobre un  
        'tópico ' específico, y los dispositivos suscritos a ese tópico pueden recibir los datos.
        \item El Broker (Agente) :
        \begin{itemize}
            \item El 'punto central de comunicación' en MQTT es el  broker .
            \item El broker es el encargado de 'despachar todos los mensajes'  entre el emisor y los receptores.
        \end{itemize}
        
        \item Tópicos :
        \begin{itemize}
            \item Cada cliente que publica un mensaje al broker incluye un   'tópico '  dentro del mensaje.
            \item Este tópico sirve como   'información de enrutamiento '  para el broker, permitiéndole reenviar el mensaje a los receptores 
            suscritos a ese tópico.
            \item Ejemplo : Un usuario puede publicar la temperatura en un tópico llamado  'temperatura ', y el broker reenviará estos datos a un 
            aire acondicionado que esté suscrito al mismo tópico, para que ajuste la temperatura deseada. Otro ejemplo es un sensor de humedad que p
            ublica niveles de humedad en un tópico  'jardín ', y una bomba de agua se activa si el nivel es bajo.
        \end{itemize}
    \end{itemize}

    \item 3. Ventajas y Desventajas
    \begin{itemize}
        \item Escalabilidad : El protocolo es 'altamente escalable '  porque los clientes no necesitan conocerse entre sí; solo necesitan comunicarse 
        a través del tópico.
        \item Dependencia Centralizada : El 'único inconveniente'  de este protocolo es su   'entidad central'  (el broker). Si el broker falla, 
        'toda la comunicación se perderá' .
    \end{itemize}

    \item 4. Relación con WebSockets
    \begin{itemize}
        \item Capas de Protocolo : MQTT y WebSockets son  'cosas diferentes ' .
        \item MQTT sobre WebSockets : MQTT puede ejecutarse   'sobre la capa de WebSockets ' . Esto es posible porque MQTT, a su vez, se ejecuta sobre 
        la red TCP/IP.
        \item Analogía de  'Entrega de Servicio ' : Para aclarar la relación, se usa una analogía:
        \begin{itemize}
            \item MQTT se compara con un   'servicio de entrega '  (como DHL).
            \item WebSockets proporcionan la   'infraestructura '  o las  'carreteras ' por las que viaja el servicio de entrega.
        \end{itemize}
        \item Empaquetamiento de Datos : Un paquete de datos MQTT se   'empaqueta dentro de un sobre de WebSocket' , que a su vez se envuelve en un  
        'sobre TCP/IP ' antes de ser enviado por Internet. El proceso inverso ocurre al desempaquetarse en el destino.
    \end{itemize}

    \item 5. Rol en el Proyecto 'Internet de las Cosas con Python y Raspberry Pi '
    \begin{itemize}
        \item Estudio en Profundidad : En la Sección 3 del proyecto, se estudia   'en profundidad'  sobre los protocolos de comunicación en tiempo 
        real y ligeros para Internet de las Cosas, incluyendo MQTT, WebSockets y una demostración práctica con PubNub. Esto subraya su importancia 
        para los participantes del proyecto que buscan comprender y resolver problemas reales de IoT.
    \end{itemize}
\end{itemize}
En resumen, MQTT es un protocolo esencial en el ecosistema IoT debido a su diseño ligero y eficiente, su modelo publicar/suscribir basado en 
brokers y tópicos, y su capacidad de coexistir con WebSockets para una comunicación robusta y escalable.

\subsection{Protocolo de Mensajería Ligero}
% 4.2.1 PROTOCOLO DE MENSAJERíA LIGERO.
Se describen a MQTT (Message Queuing Telemetry Transport) como un protocolo ligero de transferencia de mensajes con las siguientes 
características en el contexto más amplio de su funcionamiento:
\begin{itemize}
    \item Definición de Protocolo Ligero: MQTT es un protocolo ligero de transferencia de mensajes diseñado para la comunicación máquina 
    a máquina (M2M) y el Internet de las Cosas (IoT).
    \item Eficiencia de Ancho de Banda: Se destaca por ser muy eficiente en el uso del ancho de banda, requiriendo solo 2 bytes de sobrecarga. 
    Esta mínima sobrecarga es un factor clave que lo clasifica como un protocolo ligero.
    \item Modelo de Publicación y Suscripción: Opera mediante un modelo de publicación y suscripción. En este modelo, una transmisión de datos 
    específica se envía sobre un 'tema' (topic), y los dispositivos que se 'suscriben' a ese tema pueden recibir los datos. Esto permite 
    escenarios de transmisión de datos uno a uno, uno a muchos y muchos a muchos para dispositivos y aplicaciones.
    \item Base en TCP/IP: Este protocolo funciona sobre la red TCP/IP, lo que implica que puede ser utilizado incluso sobre la capa de 
    WebSocket. Sin embargo, se  aclaran que no se debe confundir MQTT con WebSockets, ya que son conceptos diferentes; MQTT es 
    un 'servicio de entrega' que trabaja sobre WebSockets, que proporcionan la 'infraestructura' (como las carreteras). Un paquete de datos 
    MQTT se encapsula dentro de un 'sobre' WebSocket, que a su vez se envuelve en un 'sobre' TCP/IP antes de ser enviado por internet.
    \item Broker Central: El punto central de comunicación en MQTT es el broker, que se encarga de despachar todos los mensajes entre 
    remitentes y receptores. Cada cliente que publica un mensaje al broker incluye un tema, que sirve como información de enrutamiento 
    para que el broker reenvíe el mensaje a los receptores suscritos a ese tema.
    \item Escalabilidad: Esta arquitectura lo hace altamente escalable, ya que los clientes no necesitan conocerse entre sí, solo comunicarse 
    a través del tema.
    \item Desventaja (Entidad Central): La única desventaja mencionada es la dependencia de una entidad central: si el broker falla, toda la 
    comunicación se interrumpe.
\end{itemize}
En resumen, la naturaleza de protocolo ligero de MQTT, su eficiencia en el ancho de banda con baja sobrecarga, y su modelo de 
publicación/suscripción lo hacen una solución muy adoptada en entornos de IoT donde los recursos y el ancho de banda pueden ser limitados.

\subsection{Máquina a Máquina (M2M) e IoT}
% 4.2.2 MáQUINA A MáQUINA (M2M) E IoT.
Se describen MQTT (Message Queuing Telemetry Transport) como un protocolo ligero de transferencia de mensajes específicamente 
diseñado para la comunicación máquina a máquina (M2M) y el Internet de las Cosas (IoT). Es una de las tecnologías más reconocidas y 
adoptadas en IoT para protocolos de comunicación.
En el contexto más amplio de su funcionamiento para M2M e IoT, se  destacan las siguientes características clave:
\begin{itemize}
    \item Eficiencia de Ancho de Banda y Ligereza: MQTT es 'muy eficiente en el uso del ancho de banda'. Su diseño ligero se evidencia en que 
    requiere solo 2 bytes de sobrecarga, lo cual es crucial para los dispositivos IoT que a menudo tienen recursos limitados y operan en entornos 
    con restricciones de ancho de banda.
    \item Modelo de Publicación y Suscripción: Este protocolo opera mediante un modelo de publicación y suscripción, el cual es fundamental 
    para los escenarios de comunicación M2M e IoT.
        \begin{itemize}
            \item Una transmisión de datos específica se envía sobre un 'tema' (topic).
            \item Los dispositivos que se 'suscriben' a ese tema pueden recibir los datos.
            \item Este modelo permite escenarios de transmisión de datos uno a uno, uno a muchos y muchos a muchos entre dispositivos y aplicaciones.
            \item Por ejemplo, un usuario puede publicar la temperatura en un tema llamado 'temperatura', y un aire acondicionado, 
            suscrito al mismo tema, recibirá esos datos para ajustar la temperatura deseada. Otro ejemplo es un sensor de humedad que publica 
            niveles de humedad en un tema llamado 'jardín', y una bomba de agua se activa si el nivel de humedad es bajo.
        \end{itemize}
    \item Broker Central y Escalabilidad: El broker es el punto central de comunicación en MQTT y se encarga de despachar todos los 
    mensajes entre los remitentes y los receptores. Cada cliente que publica un mensaje al broker incluye un tema, que actúa como 
    información de enrutamiento para que el broker reenvíe el mensaje a los receptores suscritos a ese tema. Esta arquitectura hace que 
    el protocolo sea altamente escalable, ya que los clientes no necesitan conocerse entre sí, solo comunicarse a través del tema. Esto 
    es esencial para el crecimiento y la gestión de un gran número de dispositivos en un ecosistema IoT.
    \item Funcionamiento sobre TCP/IP y WebSockets: MQTT funciona sobre la red TCP/IP y puede ser utilizado sobre la capa de WebSocket. 
    Se aclaran que MQTT y WebSockets son conceptos diferentes: MQTT se considera un 'servicio de entrega' que trabaja sobre la 
    'infraestructura' proporcionada por WebSockets. Un paquete de datos MQTT se encapsula dentro de un 'sobre' WebSocket, que a su vez 
    se envuelve en un 'sobre' TCP/IP antes de ser enviado por internet.
    \item Desventaja: La única desventaja mencionada es su dependencia de una entidad central (el broker); si el broker falla, toda la 
    comunicación se interrumpe.
\end{itemize}

\subsection{Muy Eficiente en Ancho de Banda (2 Bytes de Sobre Carga)}
% 4.2.3 MUY EFICIENTE EN ANCHO DE BANDA (2 BYTES DE SOBREcARGA).
Se destacan que MQTT (Message Queuing Telemetry Transport) es un protocolo altamente eficiente en el uso del ancho de banda, 
lo cual es una de sus características definitorias en el contexto más amplio de su aplicación en el Internet de las Cosas (IoT) y 
la comunicación máquina a máquina (M2M).
Específicamente, se  mencionan:
\begin{itemize}
    \item Baja Sobrecarga (2 Bytes de Overhead): MQTT es 'muy eficiente en el uso del ancho de banda' y utiliza solo 2 bytes de sobrecarga. 
    Esta mínima cantidad de datos adicionales es crucial y un factor clave que lo clasifica como un protocolo ligero de transferencia de mensajes.
    \item Idoneidad para M2M e IoT: Esta eficiencia lo hace especialmente adecuado para dispositivos IoT y escenarios M2M, donde a menudo los 
    recursos son limitados y el ancho de banda puede ser una preocupación. Permite la transmisión de datos sin consumir excesivos recursos de red, 
    lo que es vital para la operación de miles de dispositivos conectados.
    \item Funcionamiento sobre TCP/IP: A pesar de su ligereza, MQTT funciona sobre la red TCP/IP, e incluso puede ser utilizado sobre la 
    capa de WebSocket. Un paquete de datos MQTT se encapsula dentro de un 'sobre' WebSocket, que a su vez se envuelve en un 'sobre' TCP/IP 
    antes de ser enviado por internet, y se desempaqueta en el orden inverso al recibirlo. Esta arquitectura demuestra que puede mantener 
    su eficiencia incluso cuando se anida dentro de otras capas de comunicación.
\end{itemize}
En resumen, la característica de ser eficiente en ancho de banda con solo 2 bytes de sobrecarga es fundamental para la identidad de MQTT 
como un protocolo ligero, haciendo posible su amplia adopción en el ecosistema IoT donde la optimización de recursos es primordial.

\subsection{Escenario de Datos Uno a Uno, Uno a Muchos y Muchos a Muchos}
% 4.2.4 ESCENARIOS DE STREAMING DE DATOS.
Se  explican que MQTT (Message Queuing Telemetry Transport) está diseñado para facilitar diversos escenarios de streaming de datos, 
especialmente en el contexto de la comunicación máquina a máquina (M2M) y el Internet de las Cosas (IoT).
En el contexto más amplio de MQTT, los escenarios de streaming de datos se logran principalmente a través de su modelo de publicación y suscripción:
\begin{itemize}
    \item Tipos de Escenarios: MQTT proporciona escenarios de streaming de datos de uno a uno, uno a muchos y muchos a muchos para dispositivos y 
    aplicaciones.
    \item Modelo de Publicación y Suscripción: Esto se consigue mediante un modelo en el que una transmisión de datos específica se envía a través 
    de un 'tema' (topic), y los dispositivos que se 'suscriben' a ese tema pueden recibir los datos.
    \item Rol del Broker y los Temas: El broker es el punto central de comunicación y se encarga de despachar todos los mensajes. Cada cliente 
    que publica un mensaje al broker incluye un tema, que sirve como información de enrutamiento para que el broker reenvíe el mensaje a los 
    receptores suscritos a ese tema. 
    
    \item Se  proporcionan ejemplos concretos de como se manifiestan estos escenarios de streaming de datos en la práctica:
    \begin{itemize}
        \item Un usuario puede publicar la temperatura en un tema llamado 'temperatura'. El broker, al recibir este mensaje, reenvía los 
        datos a un aire acondicionado que está suscrito al mismo tema, permitiendo que el aire acondicionado establezca la temperatura deseada.
        \item Un sensor de humedad puede medir el nivel de humedad y publicarlo en un tema llamado 'jardín'. Una bomba de agua, que está 
        suscrita a ese tema, se activará si el nivel de humedad detectado es bajo.
    \end{itemize}

    \item Este diseño de publicación/suscripción, junto con el rol central del broker, hace que el protocolo sea altamente escalable para estos 
    escenarios de streaming, ya que los clientes no necesitan conocerse entre sí, solo comunicarse a través de los temas.

    \item Además, la naturaleza de MQTT como un protocolo ligero de transferencia de mensajes y su eficiencia en el uso del ancho de banda 
    (con solo 2 bytes de sobrecarga) lo hacen particularmente adecuado para estos escenarios de streaming en dispositivos IoT, donde los recursos 
    y el ancho de banda pueden ser limitados.
\end{itemize}

\subsubsection{Uno a Uno}
Se  indican que MQTT (Message Queuing Telemetry Transport) está diseñado para soportar una variedad de escenarios de streaming 
de datos, y explícitamente mencionan el uno a uno como una de estas capacidades.
En el contexto más amplio de los escenarios de streaming de datos que ofrece MQTT, se destaca lo siguiente:
\begin{itemize}
    \item Tipos de Escenarios: Se  afirman que MQTT proporciona escenarios de streaming de datos de uno a uno, uno a muchos y 
    muchos a muchos para dispositivos y aplicaciones. Esta declaración subraya la flexibilidad del protocolo para cubrir diversas necesidades 
    de comunicación en el ámbito del Internet de las Cosas (IoT) y la comunicación máquina a máquina (M2M).
    \item Modelo de Publicación y Suscripción: La capacidad de MQTT para facilitar estos diferentes tipos de streaming, incluyendo el uno a uno, 
    se logra a través de su modelo de publicación y suscripción. En este modelo, una transmisión de datos específica se envía sobre 
    un 'tema' (topic), y los dispositivos que se 'suscriben' a ese tema pueden recibir los datos.
    \item Rol del Broker: El broker actúa como el punto central de comunicación en MQTT, encargado de despachar todos los mensajes entre 
    remitentes y receptores. Un cliente que publica un mensaje al broker incluye un tema, que sirve como información de enrutamiento para 
    que el broker reenvíe el mensaje a los receptores suscritos a ese tema.
\end{itemize}
Aunque no se ofrecen un ejemplo específico que ilustre de forma exclusiva un escenario 'uno a uno' donde solo un suscriptor recibe 
el mensaje (los ejemplos provistos, como el sensor de temperatura y el aire acondicionado, o el sensor de humedad y la bomba de agua, 
podrían configurarse como uno a uno si solo hay un suscriptor único para un tema específico), la mención explícita de 'uno a uno' 
confirma que MQTT tiene la capacidad inherente de gestionar este tipo de comunicación directa entre un publicador y un suscriptor 
específico, lo cual se logra mediante el uso de temas dedicados o configuraciones particulares dentro del 
modelo de publicación/suscripción.

\subsubsection{Uno a Muchos}
Se  indican que MQTT (Message Queuing Telemetry Transport) está diseñado para soportar una variedad de escenarios 
de streaming de datos, y explícitamente menciona el uno a muchos como una de estas capacidades.
En el contexto más amplio de los escenarios de streaming de datos que ofrece MQTT, se destaca lo siguiente:
\begin{itemize}
    \item Tipos de Escenarios: Se  afirman que MQTT proporciona escenarios de streaming de datos de uno a uno, uno a muchos 
    y muchos a muchos para dispositivos y aplicaciones. Esta capacidad demuestra la flexibilidad del protocolo para manejar diversas 
    configuraciones de comunicación en el Internet de las Cosas (IoT) y la comunicación máquina a máquina (M2M).
    \item Modelo de Publicación y Suscripción: La funcionalidad de 'uno a muchos' se logra a través del modelo de publicación y 
    suscripción de MQTT. En este modelo, una transmisión de datos específica se envía sobre un 'tema' (topic). Múltiples dispositivos 
    o aplicaciones pueden 'suscribirse' a ese mismo tema y, por lo tanto, recibir los datos que se publican en él.
    \item Rol del Broker y los Temas: El broker es el punto central de comunicación en MQTT, encargado de despachar todos los 
    mensajes entre remitentes y receptores. Cuando un cliente publica un mensaje, incluye un tema dentro del mensaje. Este tema 
    actúa como información de enrutamiento para que el broker reenvíe el mensaje a todos los receptores que estén suscritos a ese tema coincidente.
    \item Escalabilidad: Esta arquitectura hace que el protocolo sea altamente escalable para escenarios de streaming de datos como 
    el 'uno a muchos', ya que los clientes no necesitan conocerse entre sí; solo necesitan comunicarse a través del tema. Esto es 
    fundamental para entornos IoT donde un solo sensor podría enviar datos a múltiples aplicaciones o dispositivos que 
    requieren esa información.
\end{itemize}
Aunque los ejemplos directos (temperatura a un aire acondicionado, sensor de humedad a una bomba de agua) se 
presentan de forma implícitamente 'uno a uno' en su descripción, el modelo subyacente de publicación/suscripción con el broker 
permite intrínsecamente que un publicador envíe datos a múltiples suscriptores simultáneamente, lo cual es la esencia del escenario 
'uno a muchos' que el protocolo declara soportar.

\subsubsection{Muchos a Muchos}
Se  indican que MQTT (Message Queuing Telemetry Transport) está diseñado para soportar una variedad de escenarios de 
streaming de datos, y explícitamente menciona el muchos a muchos como una de estas capacidades.
En el contexto más amplio de los escenarios de streaming de datos que ofrece MQTT, se destaca lo siguiente:
\begin{itemize}
    \item Tipos de Escenarios: Se  afirman que MQTT proporciona escenarios de streaming de datos de uno a uno, uno a muchos y 
    muchos a muchos para dispositivos y aplicaciones. Esta capacidad demuestra la flexibilidad del protocolo para manejar complejas 
    configuraciones de comunicación en el Internet de las Cosas (IoT) y la comunicación máquina a máquina (M2M).
    \item Modelo de Publicación y Suscripción: La funcionalidad de 'muchos a muchos' se logra a través del modelo de publicación y 
    suscripción de MQTT. En este modelo, múltiples dispositivos o aplicaciones pueden publicar transmisiones de datos sobre diferentes 
    'temas' (topics), y a su vez, múltiples dispositivos o aplicaciones pueden suscribirse a uno o varios de esos temas 
    para recibir los datos.
    \item Rol del Broker y los Temas: El broker es el punto central de comunicación en MQTT, encargado de despachar todos 
    los mensajes entre remitentes y receptores. Cuando un cliente publica un mensaje, incluye un tema dentro del mismo. 
    Este tema actúa como información de enrutamiento para que el broker reenvíe el mensaje a todos los receptores que estén 
    suscritos a ese tema coincidente.
    \item Escalabilidad: Esta arquitectura hace que el protocolo sea altamente escalable para escenarios de streaming de datos 
    como el 'muchos a muchos', ya que los clientes no necesitan conocerse entre sí; solo necesitan comunicarse a través de los 
    temas. Esto es fundamental para entornos IoT donde un gran número de dispositivos pueden estar generando datos y un gran 
    número de aplicaciones o usuarios pueden estar interesados en diferentes flujos de datos simultáneamente.
\end{itemize}
Aunque no se proporciona un ejemplo específico que ilustre un escenario 'muchos a muchos' de manera explícita, 
la capacidad de múltiples publicadores para enviar a un broker y de múltiples suscriptores para recibir de ese mismo broker, 
organizada por temas, es la base de esta funcionalidad declarada.

\subsection{Modelo Publicar/Suscribir}
% 4.2.5 MODELO PUBLICAR/SUSCRIBIR.
Se  explican que el Modelo Publicar/Suscribir (Publish/Subscribe Model) es una característica central y 
fundamental del funcionamiento de MQTT (Message Queuing Telemetry Transport), especialmente en el contexto de la comunicación 
áquina a máquina (M2M) y el Internet de las Cosas (IoT).
A continuación, se detalla lo que se  dice sobre este modelo:
\begin{itemize}
    \item Fundamento del Modelo: El modelo de publicación y suscripción es la forma en que MQTT logra la transferencia de mensajes. 
    En este modelo, una transmisión de datos específica se envía sobre un 'tema' (topic), y los dispositivos o aplicaciones que se 'suscriben' 
    a ese tema pueden recibir los datos.
    \item Roles Clave:
        \begin{itemize}
            \item Publicador (Publisher): Es el cliente que envía mensajes. Cada cliente que publica un mensaje al broker incluye un 
            tema dentro del mensaje.
            \item Suscriptor (Subscriber): Es el cliente que recibe mensajes. Los dispositivos se suscriben a temas específicos 
            para recibir los datos asociados a ellos.
            \item Broker: Es el punto central de comunicación en MQTT. El broker es el encargado de despachar todos los 
            mensajes entre los remitentes (publicadores) y los receptores (suscriptores). El tema incluido por el publicador 
            sirve como información de enrutamiento para que el broker reenvíe el mensaje a todos los receptores que estén 
            suscritos a ese tema.
        \end{itemize}
    \item Escenarios de Streaming de Datos: Gracias a este modelo, MQTT proporciona diversos escenarios de streaming de datos, incluyendo:
        \begin{itemize}
            \item Uno a uno.
            \item Uno a muchos.
            \item Muchos a muchos para dispositivos y aplicaciones.        
        \end{itemize}
    \item Ejemplos de Aplicación:
        \begin{itemize}
            \item Un usuario puede publicar la temperatura en un tema llamado 'temperatura'. El broker, al recibir este mensaje, 
            reenvía los datos a un aire acondicionado que está suscrito al mismo tema, lo que le permite establecer la temperatura deseada.
            \item Un sensor de humedad puede medir el nivel de humedad y publicarlo en un tema llamado 'jardín'. 
            Una bomba de agua, suscrita a ese tema, se activará si el nivel de humedad es bajo.
        \end{itemize}
    \item Escalabilidad: El modelo de publicación y suscripción, con el broker como intermediario, hace que el protocolo MQTT sea altamente 
    escalable. Esto se debe a que los clientes no necesitan conocerse directamente entre sí; solo tienen que comunicarse a través del tema y 
    el broker se encarga del enrutamiento. Esta característica es fundamental para gestionar un gran número de dispositivos en un ecosistema IoT.
    \item Integración con WebSockets: Un paquete de datos MQTT se encapsula dentro de un 'sobre' WebSocket, que a su vez se 
    envuelve en un 'sobre' TCP/IP antes de ser enviado por internet. Esto demuestra cómo MQTT, con su modelo publicar/suscribir, 
    puede operar sobre diferentes capas de red manteniendo su funcionalidad.
\end{itemize}
En resumen, el Modelo Publicar/Suscribir es la columna vertebral de MQTT, permitiendo una comunicación eficiente, flexible y 
escalable en entornos M2M e IoT al desacoplar a los publicadores de los suscriptores mediante el uso de temas y 
un broker centralizado

subsubsection{Stream de Datos Enviado sobre un 'Tópico'}
% 4.2.5.1 Stream de Datos Enviado sobre un 'Tópico'
Se  explican que el concepto de Stream de Datos Enviado sobre un 'Tópico' (Topic) es fundamental para el Modelo Publicar/Suscribir 
de MQTT (Message Queuing Telemetry Transport), especialmente en la comunicación máquina a máquina (M2M) y el Internet de las Cosas (IoT).
En el contexto más amplio del Modelo Publicar/Suscribir:
\begin{itemize}
    \item MQTT como Protocolo de Transferencia de Mensajes Ligero:
        \begin{itemize}
            \item MQTT (Message Queuing Telemetry Transport) es un protocolo de transferencia de mensajes ligero diseñado para 
            la comunicación de máquina a máquina (M2M) y el Internet de las Cosas (IoT).
            \item Es muy eficiente en cuanto al ancho de banda, utilizando solo 2 bytes de sobrecarga.
            \item Funciona sobre una red TCP/IP, lo que significa que también puede usarse sobre una capa WebSocket.        
        \end{itemize}
        
    \item El Modelo Publicar/Suscribir (Publish/Subscribe):
        \begin{itemize}
            \item MQTT se basa en un modelo Publicar/Suscribir para escenarios de transmisión de datos uno a uno, uno a muchos y 
            muchos a muchos, lo que lo hace ideal para dispositivos y aplicaciones IoT.
            \item En este modelo, una corriente de datos específica se envía sobre un 'tópico' determinado.
            \item Los dispositivos o aplicaciones que se 'suscriben' a ese tópico pueden recibir los datos.
        \end{itemize}

    \item El Rol del Broker y los Tópicos:
        \begin{itemize}
            \item El punto central de comunicación en MQTT es el broker.
            \item El broker es el responsable de despachar todos los mensajes entre el emisor (publisher) y los receptores (subscribers).
            \item Cada cliente que publica un mensaje al broker incluye un tópico dentro del mensaje.
            \item Este tópico actúa como información de enrutamiento para el broker, permitiéndole reenviar el mensaje a los receptores 
            que están suscritos a ese tópico específico.
        \end{itemize}

    \item Flujo de Comunicación y Escalabilidad:
        \begin{itemize}
            \item Los clientes (publishers) no necesitan conocerse entre sí; solo necesitan comunicarse a través del tópico, lo que 
            hace que el protocolo sea altamente escalable.
            \item Un ejemplo dado es un usuario publicando la temperatura en un tópico llamado 'temperatura', y el broker reenvía estos 
            datos a un aire acondicionado suscrito al mismo tópico para ajustar la temperatura deseada.
            \item Otro ejemplo menciona un sensor de humedad midiendo el nivel de humedad y publicando en el tópico 'jardín', lo 
            que hace que la bomba de agua se encienda si el nivel de humedad es bajo.        
        \end{itemize}

    \item Distinción con WebSockets:
        \begin{itemize}
            \item Es importante no confundir MQTT con WebSockets, ya que son cosas diferentes.
            \item Se puede considerar a MQTT como un 'servicio de entrega' que funciona sobre WebSockets, de la misma manera que DHL 
            utiliza las carreteras y vías proporcionadas por WebSockets.
            \item Un paquete de datos MQTT se 'empaqueta' dentro de un 'sobre' de WebSocket, que a su vez se 'envuelve' en un sobre 
            TCP/IP y se envía por Internet, desempaquetándose en orden inverso al recibirse.
        \end{itemize}

    \item Inconvenientes:
        \begin{itemize}
            \item La principal desventaja de este protocolo es la entidad central (el broker); si el broker falla, toda la comunicación se interrumpe.        
        \end{itemize}
\end{itemize}
En resumen, se  explican que en el modelo Publicar/Suscribir de MQTT, los flujos de datos se asocian a 'tópicos' específicos. 
Un broker central distribuye estos mensajes, enrutándolos a los suscriptores basándose en los tópicos a los que están suscritos. 
Esto permite una comunicación eficiente y escalable para aplicaciones IoT.

\subsubsection{Dispositivos se Suscriben a Tópicos para Recibir Datos}
% 4.2.5.2 Dispositivos se Suscriben a Tópicos para Recibir Datos
Aquí se ofrecen una explicación detallada sobre cómo los dispositivos se suscriben a tópicos para recibir 
datos en el contexto del modelo Publicar/Suscribir, específicamente a través del protocolo MQTT.
Aquí se desglosa este aspecto:
\begin{itemize}
    \item El Modelo Publicar/Suscribir en MQTT:
        \begin{itemize}
            \item MQTT (Message Queuing Telemetry Transport) es un protocolo de transferencia de mensajes ligero diseñado para la 
            comunicación máquina a máquina (M2M) e Internet de las Cosas (IoT).
            \item Se basa en un modelo Publicar/Suscribir que permite escenarios de transmisión de datos uno a uno, uno a muchos y 
            muchos a muchos, lo que lo hace ideal para dispositivos y aplicaciones IoT.
            \item En este modelo, una corriente de datos específica se envía sobre un 'tópico' determinado, y los dispositivos 
            que se 'suscriben' a ese tópico pueden recibir los datos.
        \end{itemize}

    \item El Rol del Broker y los Tópicos como Enrutamiento:
        \begin{itemize}
            \item El punto central de comunicación en MQTT es el broker.
            \item El broker es el encargado de despachar todos los mensajes entre el emisor (publisher) y los receptores (subscribers).
            \item Cada cliente que publica un mensaje al broker incluye un tópico dentro del mensaje.
            \item Este tópico sirve como información de enrutamiento para el broker, permitiéndole reenviar el mensaje a los 
            receptores que están suscritos a ese tópico específico.
        \end{itemize}

    \item Proceso de Suscripción y Recepción de Datos:
        \begin{itemize}
            \item Los clientes (dispositivos o aplicaciones) no necesitan conocerse entre sí; solo necesitan comunicarse a través del tópico.
            \item Cuando un dispositivo se 'suscribe' a un tópico, le está indicando al broker que desea recibir todos los mensajes publicados 
            en ese tópico.
            \item El broker, al recibir un mensaje con un tópico específico de un publisher, busca a todos los suscriptores de ese 
            mismo tópico y les entrega el mensaje.
            \item Esta arquitectura hace que el protocolo sea altamente escalable.
        \end{itemize}

    \item Ejemplos de Suscripción de Dispositivos:
        \begin{itemize}
            \item Un ejemplo proporcionado es un usuario que publica la temperatura en un tópico llamado 'temperatura'; el broker 
            reenvía estos datos a un aire acondicionado que está suscrito al mismo tópico, permitiéndole establecer la temperatura deseada.
            \item Otro ejemplo involucra un sensor de humedad que mide el nivel de humedad y publica en el tópico 'jardín'. 
            Si el nivel de humedad es bajo, la bomba de agua, que está suscrita a ese tópico, se enciende.
        \end{itemize}

    \item Funcionamiento sobre Redes y Comparación con WebSockets:
        \begin{itemize}
            \item MQTT funciona sobre una red TCP/IP, lo que significa que también puede usarse sobre una capa WebSocket.
            \item Se  aclaran que MQTT y WebSockets son distintos; se puede pensar en MQTT como un 'servicio de entrega' 
            que opera sobre WebSockets. Un paquete de datos MQTT se 'empaqueta' dentro de un 'sobre' de WebSocket, que a su vez se 
            'envuelve' en un sobre TCP/IP y se envía por Internet.
        \end{itemize}

    \item Consideraciones:
        \begin{itemize}
            \item La principal desventaja de este protocolo es la entidad central (el broker); si el broker falla, toda 
            la comunicación se interrumpe.            
        \end{itemize}
\end{itemize}
En conclusión, se  enfatizan que los dispositivos se suscriben a tópicos en el modelo Publicar/Suscribir de MQTT 
para recibir datos relevantes. El broker actúa como un intermediario inteligente, utilizando los tópicos como mecanismo de 
enrutamiento para asegurar que los mensajes lleguen a los dispositivos suscritos, lo que permite una comunicación eficiente 
y escalable en entornos IoT.

\subsection{Corre Sobre TCP/IP y Puede Correr Sobre Websockets}
% 4.2.6 CORRE SOBRE TCP/IP Y PUEDE CORRER SOBRE WEBSOCKETS.
Aquí se explica claramente que MQTT (Message Queuing Telemetry Transport) funciona sobre TCP/IP y puede 
utilizarse sobre una capa de WebSockets, estableciendo una distinción y una relación de capas entre estos protocolos.
Aquí se detalla este aspecto:
\begin{itemize}
    \item MQTT Corre sobre TCP/IP:
        \begin{itemize}
            \item MQTT es un protocolo ligero de transferencia de mensajes diseñado para la comunicación máquina a máquina (M2M) y 
            el Internet de las Cosas (IoT).
            \item Una característica fundamental es que este protocolo 'corre sobre la red TCP/IP'.
            \item Es un protocolo muy eficiente en cuanto al ancho de banda, utilizando solo 2 bytes de sobrecarga.
        \end{itemize}

    \item Capacidad de Uso sobre WebSockets:
        \begin{itemize}
            \item Debido a que MQTT opera sobre TCP/IP, se  indican que 'podemos usar este protocolo sobre la capa WebSocket también'.
            \item Esto sugiere una flexibilidad en la implementación de MQTT, permitiéndole aprovechar las capacidades de WebSockets.
        \end{itemize}

    \item Distinción y Relación entre MQTT y WebSockets:
        \begin{itemize}
            \item Es crucial no confundir MQTT con WebSockets, ya que 'ambos son cosas diferentes'.
            \item Se ofrece una analogía para clarificar su relación: se puede 'considerar a MQTT como un servicio de entrega 
            que funciona sobre WebSockets'. En esta analogía, MQTT sería como 'DHL', mientras que 'las carreteras y las vías son 
            proporcionadas por el WebSocket'.
            \item Esta analogía ilustra que WebSocket actúa como una capa de transporte subyacente para MQTT.
            \item El proceso de encapsulación se describe de la siguiente manera: un 'paquete de datos MQTT se empaqueta dentro de 
            un sobre de WebSocket, el cual es luego envuelto dentro de un sobre TCP/IP y enviado a través de Internet'. 
            El desempaquetado ocurre en orden inverso al recibirse.
        \end{itemize}

    \item Contexto de WebSockets:
        \begin{itemize}
            \item Las WebSockets, por sí mismas, son una tecnología avanzada que permite abrir una sesión de comunicación interactiva 
            persistente sobre una única conexión TCP.
            \item Habilitan la comunicación bidireccional y dúplex completo, lo que significa que ambas partes (cliente y servidor) 
            pueden enviar mensajes de forma independiente.
            \item Debido a su baja latencia y naturaleza bidireccional, WebSockets son 'altamente adoptados en aplicaciones IoT'.
        \end{itemize}
\end{itemize}
En resumen, se  establece que MQTT está inherentemente ligado a la capa de transporte TCP/IP y que esta base permite 
que se utilice eficientemente 'sobre la capa WebSocket'. Aunque son protocolos distintos, WebSockets proporcionan una 
infraestructura (como 'carreteras') sobre la cual los paquetes de datos de MQTT pueden ser transportados, aprovechando así la 
eficiencia y las capacidades de comunicación bidireccional en tiempo real para las aplicaciones IoT.

\subsection{No Confundir con Websockets (MQTT es un  'Servicio de Entrega' que Corre Encima de Websockets)}
% 4.2.7 NO CONFUNDIR CON WEBSOCKETS (MQTT ES UN  'SERVICIO DE ENTREGA ' QUE CORRE ENCIMA DE WEBSOCKETS).
Se enfatiza la importancia de no confundir MQTT con WebSockets, a pesar de que MQTT puede operar sobre 
una capa de WebSockets. Se explica esta relación y distinción de la siguiente manera:
\begin{itemize}
    \item Distinción Explícita y Analógica:
        \begin{itemize}
            \item Se establece claramente que 'no confunda MQTT con WebSockets, ambos son cosas diferentes'.
            \item Para ilustrar esta diferencia y su relación, se propone una analogía: 'considere MQTT como un servicio de entrega que funciona sobre WebSockets'.
            \item En esta analogía, MQTT sería como 'DHL', y 'las carreteras y las vías son proporcionadas por el WebSocket'. Esto significa que WebSockets proporcionan la infraestructura o el medio de transporte subyacente, mientras que MQTT es el protocolo que define cómo se entregan los mensajes específicos.
        \end{itemize}
    \item Funcionamiento y Encapsulación:
        \begin{itemize}
            \item MQTT es un protocolo ligero de transferencia de mensajes diseñado para la comunicación máquina a máquina (M2M) y el Internet de las Cosas (IoT). Es muy eficiente en cuanto al ancho de banda, utilizando solo 2 bytes de sobrecarga.
            \item Se especifica que MQTT 'corre sobre la red TCP/IP'.
            \item Debido a esta base sobre TCP/IP, se  indican que es posible 'usar este protocolo sobre la capa WebSocket también'.
            \item El proceso técnico de cómo MQTT utiliza WebSockets se describe como una encapsulación: un 'paquete de datos MQTT se empaqueta dentro de un sobre de WebSocket, el cual es luego envuelto dentro de un sobre TCP/IP y enviado a través de Internet'. Al recibirse, el paquete se desempaqueta en el orden inverso.
        \end{itemize}
    \item Características de WebSockets como Capa Subyacente:
        \begin{itemize}
            \item Las WebSockets son una tecnología avanzada que permite establecer una sesión de comunicación interactiva 
            persistente y bidireccional sobre una única conexión TCP.
            \item Permiten que los clientes reciban actualizaciones solo cuando ocurren, sin necesidad de que el cliente esté 
            'preguntando' constantemente al servidor (polling).
            \item Debido a su baja latencia y naturaleza bidireccional y dúplex completo (donde ambas partes pueden enviar mensajes 
            independientemente), las WebSockets son 'altamente adoptadas en aplicaciones IoT'.
        \end{itemize}
\end{itemize}
En resumen, se  recalcan que MQTT y WebSockets son protocolos distintos. Si bien MQTT es el protocolo de mensajería ligero 
ideal para IoT con su modelo Publicar/Suscribir, WebSockets actúan como una capa de transporte eficiente que permite a MQTT aprovechar 
sus capacidades de comunicación bidireccional y de baja latencia sobre una conexión TCP/IP, funcionando como un 
'servicio de entrega' que utiliza las 'carreteras' que WebSockets proporcionan.

\subsection{Funcionamiento}
% 4.2.8 FUNCIONAMIENTO.
Aquí se ofrece una descripción detallada del funcionamiento de MQTT (Message Queuing Telemetry Transport), 
destacando su arquitectura, modelo de comunicación y cómo interactúa con otras tecnologías de red.
Aquí se explica cómo funciona MQTT:
\begin{itemize}
    \item 1. Naturaleza y Propósito de MQTT:
        \begin{itemize}
            \item MQTT es un protocolo de transferencia de mensajes ligero diseñado específicamente para la comunicación de máquina a 
            máquina (M2M) y para el Internet de las Cosas (IoT).
            \item Es muy eficiente en cuanto al ancho de banda, utilizando solo 2 bytes de sobrecarga.
            \item Proporciona escenarios de transmisión de datos uno a uno, uno a muchos y muchos a muchos para dispositivos y aplicaciones.
        \end{itemize}
    \item 2. El Modelo Publicar/Suscribir (Publish/Subscribe):
        \begin{itemize}
            \item El funcionamiento central de MQTT se basa en un modelo de publicar y suscribir.
            \item En este modelo, una corriente de datos específica se envía sobre un 'tópico' determinado.
            \item Los dispositivos o aplicaciones que se 'suscriben' a ese tópico pueden recibir los datos.
        \end{itemize}
    \item 3. El Rol Central del Broker:
        \begin{itemize}
            \item El punto central de comunicación en MQTT es el 'broker'.
            \item El broker es el encargado de despachar todos los mensajes entre el emisor (publisher) y los receptores (subscribers).
            \item Cada cliente que publica un mensaje al broker incluye un tópico dentro del mensaje.
            \item Este tópico actúa como información de enrutamiento para el broker, permitiéndole reenviar el mensaje a los receptores 
            que están suscritos a ese tópico específico.
        \end{itemize}
    \item 4. Proceso de Publicación y Suscripción:
        \begin{itemize}
            \item Cuando un 'publisher' (dispositivo o aplicación) tiene datos para enviar, los publica en un tópico específico al broker.
            \item Simultáneamente, los 'subscribers' (otros dispositivos o aplicaciones) que desean recibir ciertos datos, se 'suscriben' 
            a tópicos específicos con el broker.
            \item Cuando el broker recibe un mensaje de un publisher para un tópico determinado, lo reenvía a todos los suscriptores que 
            se han registrado para ese mismo tópico.
            \item Un ejemplo dado es un usuario que publica la temperatura en un tópico llamado 'temperatura', y el broker 
            reenvía estos datos a un aire acondicionado suscrito al mismo tópico para ajustar la temperatura deseada. Otro ejemplo es un 
            sensor de humedad que publica en el tópico 'jardín', haciendo que una bomba de agua suscrita se encienda si el nivel es bajo.
        \end{itemize}
    \item 5. Ventajas y Desventajas:
            \item Este modelo hace que el protocolo sea altamente escalable, ya que los clientes (publishers y subscribers) no 
            necesitan conocerse entre sí; solo tienen que comunicarse a través del tópico con el broker.
            \item La principal desventaja de este protocolo es la entidad central (el broker); si el broker falla, toda la 
            comunicación se interrumpe.
    \item 6. Funcionamiento sobre Redes (TCP/IP y WebSockets):
        \begin{itemize}
            \item MQTT 'corre sobre la red TCP/IP'.
            \item Esto significa que también puede 'usarse este protocolo sobre la capa WebSocket'.
            \item Se enfatiza que MQTT y WebSockets son 'cosas diferentes'. Se utiliza la analogía de que MQTT es como un 
            'servicio de entrega' (DHL) que 'funciona sobre WebSockets', siendo WebSockets las 'carreteras y las vías'.
            \item Técnicamente, un paquete de datos MQTT se 'empaqueta dentro de un sobre de WebSocket', el cual a su vez es 
            'envuelto dentro de un sobre TCP/IP' y enviado a través de Internet. El desempaquetado ocurre en orden inverso al recibirse.
            \item WebSockets, por su parte, permiten una comunicación interactiva, bidireccional y dúplex completo, lo que las hace 
            adecuadas para aplicaciones IoT debido a su baja latencia
        \end{itemize}
\end{itemize}
En resumen, el funcionamiento de MQTT se centra en un broker que orquesta la comunicación entre publishers y subscribers a través 
de tópicos. Este modelo publicar/suscribir es altamente escalable y eficiente, operando sobre TCP/IP y pudiendo aprovechar 
las capacidades de WebSockets como una capa de transporte subyacente.

\subsection{Escalabilidad: Clientes no Necesitan Conocerse Entre SÍ, Solo Comunicarse a Través del Tópico}
% 4.2.9 ESCALABILIDAD: CLIENTES NO NECESITAN CONOCERSE ENTRE SÍ, SOLO COMUNICARSE A TRAVÉS DEL TóPICO.
Aquí se resalta que el diseño de MQTT (Message Queuing Telemetry Transport) y su modelo Publicar/Suscribir 
contribuyen significativamente a su escalabilidad, particularmente porque los clientes no necesitan conocerse entre sí.
A continuación, se detalla este aspecto:
\begin{itemize}
    \item Modelo Publicar/Suscribir como Base:
        \begin{itemize}
            \item MQTT es un protocolo ligero de transferencia de mensajes diseñado para la comunicación de máquina a máquina (M2M) y 
            el Internet de las Cosas (IoT).
            \item Este protocolo proporciona escenarios de transmisión de datos uno a uno, uno a muchos y muchos a muchos, lográndolo a 
            través de un modelo de publicar y suscribir.
            \item En este modelo, una corriente de datos específica se envía sobre un 'tópico' determinado, y los dispositivos que se 
            'suscriben' a ese tópico pueden recibir los datos.
        \end{itemize}
    \item El Broker como Intermediario Central:
        \begin{itemize}
            \item El punto central de comunicación en MQTT es el broker.
            \item El broker es el encargado de despachar todos los mensajes entre el emisor (publisher) y los receptores (subscribers).
            \item Cada cliente que publica un mensaje al broker incluye un tópico dentro del mensaje, el cual sirve como información de 
            enrutamiento para el broker. Esto permite al broker reenviar el mensaje a los receptores que están suscritos a 
            ese tópico específico.
        \end{itemize}
    \item La Clave de la Escalabilidad: Desconocimiento Mutuo de los Clientes:
        \begin{itemize}
            \item La arquitectura descrita hace que el protocolo sea 'altamente escalable'.
            \item La razón principal de esta escalabilidad es que 'los clientes no tienen que conocerse entre sí'.
            \item En cambio, los clientes 'solo tienen que comunicarse sobre el tópico' con el broker.
        \end{itemize}
    \item Ejemplos Ilustrativos:
        \begin{itemize}
            \item Un usuario puede publicar la temperatura en un tópico llamado 'temperatura', y el broker reenvía estos datos a un aire 
            acondicionado que está suscrito al mismo tópico. El aire acondicionado no necesita saber quién publicó el mensaje; solo reacciona 
            a los datos del tópico 'temperatura'.
            \item Otro ejemplo es un sensor de humedad que mide el nivel y publica en el tópico 'jardín'. Una bomba de agua, suscrita a 
            ese tópico, se enciende si el nivel es bajo, sin tener conocimiento directo del sensor.
        \end{itemize}
    \item Advertencia sobre la Entidad Central:
        \begin{itemize}
            \item A pesar de la alta escalabilidad que ofrece el modelo de desconocimiento mutuo entre clientes, la principal desventaja 
            de este protocolo es la entidad central (el broker). Si el broker falla, toda la comunicación se interrumpe.        
        \end{itemize}
\end{itemize}
En resumen, se  enfatizan que la escalabilidad de MQTT radica en su modelo Publicar/Suscribir, donde el broker central actúa 
como un intermediario que enruta los mensajes basados en tópicos, eliminando la necesidad de que los clientes (publishers y subscribers) 
tengan conocimiento directo el uno del otro. Esta desvinculación hace que el sistema sea fácil de expandir y gestionar, aunque la 
dependencia de un broker centralizado representa una vulnerabilidad potencial.

\subsection{Inconveniente: Entidad Central (BROKER), Si Flla, Toda la Comunicación se Pierde}
% 4.2.10 INCONVENIENTE: ENTIDAD CENTRAL (BROKER), SI FALLA, TODA LA COMUNICACIóN SE PIERDE.
Aquí se identifica un inconveniente clave en el funcionamiento de MQTT (Message Queuing Telemetry Transport), que 
se relaciona directamente con la entidad central, el BROKER.
Aquí se detalla este aspecto:
\begin{itemize}
    \item El Broker como Punto Central de Comunicación:
        \begin{itemize}
            \item En el modelo Publicar/Suscribir de MQTT, el broker es el punto central de comunicación.
            \item Es el encargado de despachar todos los mensajes entre el emisor (publisher) y los receptores (subscribers).
            \item El broker utiliza los 'tópicos' incluidos en los mensajes para enrutar la información a los suscriptores correspondientes.
        \end{itemize}
    \item La Desventaja de la Entidad Centralizada:
        \begin{itemize}
            \item A pesar de las ventajas de escalabilidad y eficiencia que ofrece MQTT, se  señalan una desventaja importante: 
            'el único inconveniente de este protocolo es la entidad central; si el broker muere, toda la comunicación se va a caer'.       
        \end{itemize}
\end{itemize}
En resumen, se  explican que si bien el broker es fundamental para el funcionamiento y la escalabilidad de MQTT al facilitar 
la comunicación entre publishers y subscribers, su naturaleza centralizada lo convierte en un punto único de fallo. Si este broker 
deja de funcionar, toda la red de comunicación de los dispositivos y aplicaciones que dependen de él se interrumpirá.

\subsection{Ejemplos}
% 4.2.11 EJEMPLOS.
Se ofrecen varios ejemplos concretos del funcionamiento de MQTT (Message Queuing Telemetry Transport) 
en el contexto de sus características principales, como el modelo Publicar/Suscribir y su aplicación en IoT:
\begin{itemize}
    \item 1. Control de Aire Acondicionado por Temperatura:
        \begin{itemize}
            \item Escenario: Un usuario desea controlar un aire acondicionado.
            \item Funcionamiento con MQTT: El usuario puede publicar la temperatura deseada en un tópico específico, por ejemplo, 'temperatura'. 
            El broker MQTT recibe este mensaje y lo reenvía al aire acondicionado que está suscrito a ese mismo tópico. De esta manera, el aire 
            acondicionado recibe la instrucción y ajusta la temperatura deseada.
        \end{itemize}
    \item 2. Activación de Bomba de Agua por Nivel de Humedad:
        \begin{itemize}
            \item Escenario: Mantener el nivel de humedad adecuado en un jardín.
            \item Funcionamiento con MQTT: Un sensor de humedad mide constantemente el nivel de humedad y publica estos datos en un tópico 
            como 'jardín'. Si el nivel de humedad es bajo (detectado por el sensor), la bomba de agua, que está suscrita a ese mismo tópico, 
            recibe el mensaje y se enciende automáticamente para regar el jardín.
        \end{itemize}
\end{itemize}
Estos ejemplos ilustran cómo MQTT, a través de su modelo Publicar/Suscribir y el uso de tópicos enrutados por un broker central, 
facilita la comunicación eficiente y en tiempo real entre dispositivos en un entorno IoT, permitiendo la automatización y el 
control de manera escalable, ya que los dispositivos no necesitan conocerse directamente entre sí.

\subsubsection{Usuario Publica Temperatura-> Aire Acondicionado se Suscribe}
% 4.2.11.1 Usuario Publica Temperatura-> Aire Acondicionado se Suscribe
Se ofrece el ejemplo específico de un usuario publicando la temperatura y un aire acondicionado 
suscribiéndose para ilustrar el funcionamiento de MQTT (Message Queuing Telemetry Transport) en el contexto de su modelo Publicar/Suscribir 
y sus aplicaciones en el Internet de las Cosas (IoT).
A continuación, se detalla lo que se  dice sobre este ejemplo:
\begin{itemize}
    \item El Escenario del Ejemplo:
        \begin{itemize}
            \item Se describe directamente este caso: 'un usuario puede publicar la temperatura en un tópico llamado 
            'temperatura' y el broker reenvía estos datos a un aire acondicionado que está suscrito en el mismo tópico y establece 
            la temperatura deseada'.        
        \end{itemize}
            
    \item Conceptos de MQTT Ilustrados:
        \begin{itemize}
            \item Publicador (Publisher): En este ejemplo, el usuario es el publicador. El usuario (o un dispositivo en su nombre, como un 
            termostato inteligente) genera el dato de la temperatura y lo envía al sistema.
            \item Tópico (Topic): La temperatura se publica en un tópico específico, nombrado 'temperatura'. Este tópico actúa como una categoría 
            o dirección lógica que clasifica el tipo de datos que se están enviando.
            \item Broker: El broker MQTT es la entidad central que recibe el mensaje del usuario. Su función es 'reenviar estos datos' a los 
            suscriptores apropiados.
            \item Suscriptor (Subscriber): El aire acondicionado es el suscriptor. Este dispositivo ha manifestado su interés en recibir mensajes 
            relacionados con el tópico 'temperatura', por lo que el broker le entrega los datos.
            \item Acción Basada en Datos: Una vez que el aire acondicionado recibe los datos de temperatura del tópico, puede 'establecer la 
            temperatura deseada', lo que demuestra la capacidad de los dispositivos IoT para reaccionar y actuar en función de la información 
            recibida en tiempo real.
        \end{itemize}

    \item Implicaciones para la Escalabilidad y la Comunicación:
        \begin{itemize}
            \item Este ejemplo también subraya la alta escalabilidad del protocolo, ya que el usuario (publisher) y el aire acondicionado 
            (subscriber) 'no tienen que conocerse entre sí'. Su única interacción es a través del tópico con el broker, lo que simplifica 
            la adición o eliminación de dispositivos sin afectar a otros componentes del sistema.
            \item Esto es un claro ejemplo de comunicación uno a muchos o uno a uno para el control de dispositivos en entornos IoT.
        \end{itemize}
\end{itemize}
En resumen, el ejemplo del usuario publicando la temperatura y el aire acondicionado suscribiéndose ilustra de manera efectiva cómo 
MQTT facilita la comunicación eficiente y desvinculada entre usuarios/dispositivos y actuadores en un sistema IoT, utilizando 
el modelo Publicar/Suscribir, tópicos y un broker central para el enrutamiento de mensajes.

\chapter{SERVIDOR IoT SEGURO Y LOGIN DE USUARIO}
% 5.SERVIDOR IoT SEGURO Y LOGIN DE USUARIO.
En el contexto más amplio del proyecto   'Internet de las Cosas con Python y Raspberry Pi ' , el  Servidor IoT Seguro y el Login de Usuario  
constituyen la Sección 5, donde se abordan aspectos cruciales para construir una plataforma IoT robusta, segura y multiusuario. El objetivo general 
es crear una plataforma en la nube sin satélites donde múltiples usuarios puedan iniciar sesión de forma segura, y controlar y monitorear sus 
dispositivos autorizados en tiempo real.

Se detallan los siguientes aspectos sobre la seguridad del servidor y la gestión de usuarios:
\begin{itemize}
    \item 1. Asegurar el Dominio Personalizado con Certificados SSL/TLS (HTTPS)
    \item La primera etapa para un servidor IoT seguro es asegurar la comunicación, lo cual se logra mediante certificados  SSL/TLS de Let's Encrypt .
    \begin{itemize}
        \item Autoridad de Certificación : Let's Encrypt es una autoridad de certificación gratuita, automatizada y de código abierto, respaldada por 
        importantes patrocinadores y ampliamente utilizada por desarrolladores y empresas.
        \item Proceso de Instalación y Configuración :
        \item Software de Terceros : El primer paso es instalar el software de terceros 'certbot' en el servidor. Esto implica agregar el repositorio PPA 
        para 'certbot' y `apache` (\verb|sudo add-apt-repository ppa:certbot/certbot|), actualizar la lista de paquetes (\verb|sudo apt-get update|) e 
        instalar `python-certbot-apache`.
        \item Generación del Certificado: Se ejecuta  \verb|sudo certbot --apache -d [nombre-de-dominio] -d www.[nombre-de-dominio]| para configurar el 
        certificado SSL para Apache. Durante este proceso, se solicita un correo electrónico para la recuperación de la clave.
        \item Redirección a HTTPS : Es fundamental  redirigir el tráfico HTTP a HTTPS , lo cual se selecciona durante la configuración 
        `Type 2 to enter`).
        \item Ubicación de Certificados : Los archivos de certificado generados se encuentran en `/etc/letsencrypt/live/`.
        \item Reglas de Seguridad Entrantes : Un paso crítico, y un error común, es olvidar asignar las  reglas de seguridad entrantes para HTTPS 
        (puerto 443)  en el servidor remoto (por ejemplo, en AWS EC2). Sin esto, la conexión al servidor puede fallar. Se debe añadir una regla para 
        HTTPS con el puerto 443 en la configuración de las instancias EC2.
        \item Verificación : Se puede verificar el estado del certificado SSL en `SSL labs.com/ssltest`.
        \item Validez y Renovación : Los certificados de Let's Encrypt son válidos por aproximadamente tres meses y deben renovarse al expirar.
        \item Resultado : Una vez configurado correctamente, el servidor redirigirá de HTTP a HTTPS, mostrando un candado verde en el navegador, 
        lo que indica una  comunicación cifrada de extremo a extremo  para cada cliente conectado al servidor.
    \end{itemize}

    \item 2. Funcionalidad de Login de Usuario y Gestión de Acceso
    \item El proyecto también aborda la implementación de una  funcionalidad de login de usuario segura  y el almacenamiento de los detalles del 
    usuario en una base de datos integrada. Además, se crean reglas para usuarios administradores y no administradores para gestionar el 
    acceso a los dispositivos y funcionalidades.
    \begin{itemize}
        \item Roles de Usuario (Admin y No-Admin) : Se establece un sistema donde los usuarios pueden tener roles de administrador o no administrador. 
        
        \item Panel de Control del Administrador :
        \begin{itemize}
            \item Los usuarios administradores tienen acceso a un panel de control que muestra una  lista de todos los usuarios en línea .
            \item Frente al nombre de cada usuario en línea, hay  botones de conmutación para otorgar permisos de lectura y escritura .
            \item También hay un botón  'Apply ' (aplicar) para guardar los cambios en los permisos.        
        \end{itemize}

        \item Implementación en el Servidor (Python/Flask) :
        \begin{itemize}
            \item Para poblar la lista de usuarios en el dashboard, el servidor envía detalles adicionales a la página web principal, como 
            el \verb|user_ID| de la sesión y una lista de \verb|online_user_records|.
            \item \verb|online_user_records| es un mapa que contiene el nombre del usuario, el ID del usuario, y el estado de acceso de lectura 
            y escritura (1 para 'checked'/marcado, 0 para  'unchecked '/desmarcado).
            \item La función `get-all-logged-in-users` se encarga de retornar este mapa.
        \end{itemize}

        \item Implementación en el Cliente (HTML/Jinja2) :
        \begin{itemize}
            \item La página 'index.html' utiliza plantillas Jinja para iterar sobre la lista de \verb|online_user_records| y crear filas en una 
            tabla para cada usuario. Cada fila muestra el nombre del usuario, y los ID y estados (checked/unchecked) de los botones de conmutación 
            para lectura y escritura.
        \end{itemize}
        \item Visibilidad Condicional : El panel de control completo (para gestionar permisos)  solo es visible para los usuarios administradores. 
        Esto se logra mediante una declaración `if` en el código HTML que compara el \verb|user_ID| del usuario actual con el \verb|user_ID| 
        del administrador (previamente hardcodeado o verificado). Un usuario no administrador (como  'Anam Chaudhary ' en el ejemplo) no tendrá 
        acceso a este panel de control.
        
        \item Otorgar Permisos en Tiempo Real : 
        \begin{itemize}
            \item Lado del Cliente (JavaScript) : Un método en `main.js` escucha los eventos de los botones de conmutación. Cuando se activa un 
            botón de acceso, extrae el ID del usuario, el estado de lectura y el estado de escritura. Luego, envía una solicitud POST al servidor 
            (por ejemplo, \verb|grant_user_ID_read_state_write_state|).
            \item Lado del Servidor (Aplicación Flask) : El servidor tiene un endpoint (`/grant`) para recibir estas solicitudes. Primero, verifica 
            si la solicitud proviene de un usuario administrador. Si no, deniega el acceso. Si es un administrador, almacena los permisos de lectura 
            y escritura del usuario en la base de datos y llama al servidor PubNub para otorgar estos permisos al usuario específico.
            \item PubNub Access Manager : Se utiliza la funcionalidad  PubNub Access Manager  para que los usuarios administradores puedan otorgar 
            permisos de lectura y escritura en tiempo real a usuarios no administradores y dispositivos. El primer paso es generar una clave de 
            autorización para el usuario y almacenarla en la base de datos, seguido de la concesión de permisos.
        \end{itemize}
    \end{itemize}
\end{itemize}

Este enfoque integral en la seguridad del servidor y la gestión de usuarios sienta las bases para una plataforma IoT robusta, capaz de manejar 
múltiples dispositivos y usuarios de manera controlada y protegida.

\subsection{ASEGURAR DOMINIO PERSONALIZADO CON SSL/TLS}
% 5.1 ASEGURAR DOMINIO PERSONALIZADO CON SSL/TLS.
Se detallan exhaustivamente el proceso de  asegurar un dominio personalizado con SSL/TLS  en el contexto de la creación de un  Servidor IoT 
Seguro y Login de Usuario , un componente crucial para establecer una plataforma IoT robusta y fiable. Este es el enfoque principal de la Sección 5 
del proyecto.

Aquí se desglosa la información relevante:
\begin{itemize}
    \item 1. Propósito y Contexto General del Curso
    \begin{itemize}
        \item El proyecto  'Internet de las Cosas con Python y Raspberry Pi ' se centra en construir una  plataforma IoT basada en la nube  donde 
        múltiples usuarios pueden iniciar sesión de forma segura y controlar y monitorear sus dispositivos autorizados en tiempo real.
        \item La Sección 5 está dedicada al  desarrollo de terminologías de seguridad , incluyendo la obtención de un nombre de dominio personalizado 
        y su aseguramiento con la autoridad de certificación Let's Encrypt. También se implementará una funcionalidad de login de usuario seguro y el 
        almacenamiento de los detalles del usuario en una base de datos integrada.
        \item El objetivo final es asegurar que cada cliente conectado al servidor tenga una  comunicación cifrada de extremo a extremo .
    \end{itemize}

    \item 2. Uso de Let's Encrypt para Certificados SSL/TLS
    \begin{itemize}
        \item El primer paso para asegurar el sitio web es utilizar la  autoridad de certificación Let's Encrypt .
        \item Características de Let's Encrypt : Es una autoridad de certificación  gratuita, automatizada y de código abierto . Es ampliamente 
        utilizada por desarrolladores y empresas, y cuenta con el respaldo de importantes patrocinadores.
        \item Período de Validez y Renovación : Los certificados de Let's Encrypt son  válidos por aproximadamente tres meses  y requieren ser 
        renovados después de su vencimiento.
    \end{itemize}

    \item 3. Proceso de Aseguramiento del Dominio
    \item El proceso implica una serie de pasos técnicos detallados:
    \begin{itemize}
        \item Instalación de Certbot :
        \begin{itemize}
            \item El primer paso es  instalar un software de terceros  (Certbot) en el servidor.
        \item Esto se hace agregando el repositorio de Certbot (PPA) y luego actualizando la lista de paquetes. 
            \item Finalmente, se instala Certbot para Apache mediante el comando \verb|sudo apt-get install python-certbot-apache|.
        \end{itemize}
        \item Configuración del Certificado SSL para Apache :
        \begin{itemize}
            \item Una vez instalado Certbot, se ejecuta el comando \verb|sudo certbot --apache -D [nombre-de-dominio] -D www.[nombre-de-dominio]| 
            para configurar el certificado SSL.
            \item Se solicitará al usuario que  proporcione una dirección de correo electrónico  para la recuperación de la clave en caso de pérdida.
            \item Se ofrecerá la opción de  redirigir el tráfico HTTP a HTTPS , lo cual se recomienda aceptar.
        \item Los archivos de certificado generados se pueden encontrar en \verb|/etc/letsencrypt/live|.
        \end{itemize}
        \item Configuración de Reglas de Seguridad Inbound (AWS) :
        \begin{itemize}
            \item Es crucial asegurarse de que las  reglas de seguridad inbound para HTTPS (puerto 443)  estén asignadas en el servidor remoto, 
            especialmente si se utiliza AWS EC2. La omisión de este paso puede causar fallos en la conexión al servidor durante la verificación 
            del certificado.
            \item Para corregirlo, se deben editar las reglas de entrada y  agregar la regla HTTPS .        
        \end{itemize}
    \end{itemize}
    \item 4. Verificación y Resultado
    \begin{itemize}
        \item Verificación del Estado del Certificado : El estado del certificado SSL se puede verificar utilizando herramientas como SSL Labs (ssl.com/ssltest).
        \item Confirmación de Seguridad : Una vez completado el proceso, el certificado aparecerá como válido, firmado por Let's Encrypt para el servidor (ej.  'pact IOT server ').
        \item Redirección y Comunicación Segura : El sitio web se  redirigirá automáticamente de HTTP a HTTPS , mostrando un  'candado verde ' en el navegador, lo que indica una  comunicación segura . Esto garantiza una  comunicación cifrada de extremo a extremo  para todos los clientes conectados al servidor.    
    \end{itemize}
\end{itemize}

En resumen, la seguridad del dominio personalizado con SSL/TLS utilizando Let's Encrypt es un pilar fundamental en la creación de un  servidor IoT seguro , 
facilitando una comunicación cifrada y protegiendo el login de usuario y los datos intercambiados entre los dispositivos IoT y la plataforma en la nube.

\subsection{AUTORIDAD DE CERTIFICACIóN: Let's Encrypt}
% 5.1.1 AUTORIDAD DE CERTIFICACIóN: Let's Encrypt.
Se detalla el papel crucial de la Autoridad de Certificación Let's Encrypt en el contexto más amplio 
de asegurar un dominio personalizado con SSL/TLS.
Esto es este aspecto:
\begin{itemize}
    \item ¿Qué es Let's Encrypt?
        \begin{itemize}
            \item Let's Encrypt es una autoridad de certificación gratuita, automatizada y de código abierto.
            \item Es ampliamente utilizada por muchos desarrolladores y empresas, y está respaldada por importantes patrocinadores.
            \item Su objetivo principal es permitir la obtención de certificados SSL/TLS (Secure Sockets Layer/Transport Layer Security).
        \end{itemize}
    \item Proceso de Uso para Asegurar un Dominio Personalizado:
        \begin{itemize}
            \item 1. Instalación de certbot: El primer paso para usar Let's Encrypt es instalar un software de terceros llamado certbot en el servidor.
                \begin{itemize}
                    \item Esto implica añadir el repositorio de certbot (PPA para paquetes preparados por el equipo de Let's Encrypt de 
                    Debian y respaldados para Ubuntu).
                    \item Luego, se actualiza la lista de paquetes para incorporar el nuevo repositorio.
                    \item Finalmente, se instala certbot específicamente para Apache, utilizando el comando \verb|sudo apt-get install python-certbot-apache|.
                \end{itemize}
            \item 2. Configuración del Certificado SSL: Una vez instalado certbot, se utiliza para configurar el certificado SSL para Apache.
                \begin{itemize}
                    \item Esto se logra ejecutando \verb|sudo certbot --apache -d [nombre_de_dominio]| y también añadiendo el dominio con 'www' 
                    (\verb|-d www.nombre_de_dominio|).
                    \item Esto se logra ejecutando \verb|sudo certbot --apache -d [nombre_de_dominio]| y también añadiendo el dominio con 'www' 
                    (\verb|-d www.nombre_de_dominio|).
                    \item Durante este proceso, se solicita una dirección de correo electrónico para la recuperación de la clave en caso de pérdida.
                    \item También se pregunta si se desea redirigir el tráfico HTTP a HTTPS, una acción que se confirma para asegurar la comunicación.
                \end{itemize}
            \item 3. Generación y Ubicación del Certificado: Una vez finalizada la instalación, los archivos del certificado generado 
            se pueden encontrar en la ruta /etc/letsencrypt/live.
        \end{itemize}
    \item Verificación y Seguridad Resultante:
        \begin{itemize}
            \item Verificación del Estado: El estado del certificado SSL puede verificarse utilizando herramientas como SSL labs.com/ssltest.
            \item Reglas de Seguridad Inbound (AWS): Para que la evaluación sea exitosa, es fundamental asegurarse de que las reglas de 
            seguridad de entrada (inbound security rules) del servidor remoto (por ejemplo, en AWS EC2) permitan el tráfico HTTPS en el puerto 
            predeterminado 443. Inicialmente, si solo HTTP y SSH están permitidos, se debe añadir explícitamente la regla HTTPS.
            \item Resultados de la Certificación: Tras una configuración correcta, el certificado SSL será válido, firmado específicamente 
            para el servidor con la autoridad emisora Let's Encrypt, y tendrá una validez inicial de aproximadamente tres meses.
            \item Renovación: Dado que es una autoridad de certificación gratuita, los certificados de Let's Encrypt expiran cada tres meses 
            y deben renovarse.
            \item Redirección y Comunicación Segura: Una vez implementado, cualquier acceso al servidor se redirigirá automáticamente de 
            HTTP a HTTPS, mostrando un 'candado verde' en el navegador que indica que la conexión es segura. Esto garantiza que 'todo cliente 
            conectado con este servidor tendrá una comunicación encriptada de extremo a extremo'.
        \end{itemize}
\end{itemize}
En el contexto más amplio de asegurar un dominio personalizado con SSL/TLS para un servidor IoT, Let's Encrypt es presentado como 
la solución preferida y gratuita para obtener los certificados necesarios que permiten la comunicación segura y encriptada (HTTPS) 
entre el servidor y todos sus clientes, siendo un paso esencial en el desarrollo de la seguridad de la infraestructura IoT.

\subsubsection{Gratuito}
% 5.1.1.1 Gratuito
Se enfatiza claramente que Let's Encrypt es una autoridad de certificación gratuita en el contexto más amplio 
de asegurar un dominio personalizado con SSL/TLS.
Esto es este aspecto:
\begin{itemize}
    \item Naturaleza Gratuita, Automatizada y de Código Abierto:
        \begin{itemize}
            \item Let's Encrypt se describe como una 'autoridad de certificación gratuita, automatizada y de código abierto'.
            \item Es ampliamente utilizada por muchos desarrolladores y empresas, y cuenta con el respaldo de importantes patrocinadores.
            \item Su objetivo es permitir la obtención de certificados SSL/TLS.
        \end{itemize}
    \item Implicación de la Gratuidad: Renovación Frecuente:
        \begin{itemize}
            \item Una consecuencia directa de su naturaleza gratuita es que los certificados de Let's Encrypt 'expiran cada tres meses 
            y deben renovarse'. Esto se menciona explícitamente como una característica de esta 'autoridad de certificación gratuita'.
        \end{itemize}
\end{itemize}
En resumen, se  recalcan que Let's Encrypt ofrece una solución sin costo para la obtención de certificados SSL/TLS, lo que la 
convierte en una opción muy accesible para asegurar la comunicación en servidores, incluyendo aquellos utilizados en proyectos IoT. 
La contrapartida de ser gratuito es la necesidad de renovaciones periódicas cada tres meses.

\subsubsection{Automatizado}
% 5.1.1.2 Automatizado
Se destaca que Let's Encrypt es una autoridad de certificación automatizada en el contexto más amplio de 
asegurar un dominio personalizado con SSL/TLS.
Esto es este aspecto:
\begin{itemize}
    \item Naturaleza Automatizada de Let's Encrypt:
        \begin{itemize}
            \item Let's Encrypt se describe explícitamente como una 'autoridad de certificación gratuita, automatizada 
            y de código abierto'. Es ampliamente utilizada por muchos desarrolladores y empresas, y cuenta con el respaldo de 
            importantes patrocinadores.
            \item Su propósito es facilitar la obtención de certificados SSL/TLS.
        \end{itemize}
    \item Implementación a Través de certbot:
        \begin{itemize}
            \item La naturaleza automatizada de Let's Encrypt se materializa mediante el uso de software de terceros como certbot.
            \item El proceso para utilizarlo implica:
                \begin{itemize}
                    \item 1. Instalar certbot en el servidor, añadiendo su repositorio y luego instalando el paquete \verb|python-certbot-apache|.
                    \item 2. Configurar el certificado SSL ejecutando comandos específicos de certbot 
                    (ej. \verb|sudo certbot --apache -d [nombre_de_dominio]|). Este proceso automatizado guía al usuario a través de la 
                    configuración, solicitando un correo electrónico para recuperación de clave y preguntando si se desea redirigir el 
                    tráfico HTTP a HTTPS.
                \end{itemize}
        \end{itemize}
    \item Implicaciones de la Automatización:
        \begin{itemize}
            \item La automatización simplifica significativamente el proceso de obtención y configuración de certificados SSL/TLS, haciendo 
            que la seguridad HTTPS sea más accesible.
            \item A pesar de ser automatizado, los certificados de Let's Encrypt 'expiran cada tres meses y deben renovarse'. 
            Aunque no se detallan el comando exacto para la renovación automática, el hecho de ser una CA 'automatizada' 
            implica que este proceso puede ser gestionado de forma programada por herramientas como certbot, reduciendo la 
            intervención manual necesaria.
        \end{itemize}
\end{itemize}
En resumen, se  enfatizan que la capacidad de Let's Encrypt para ser una autoridad de certificación automatizada es una de sus 
características fundamentales, lo que simplifica en gran medida la tarea de obtener, configurar y gestionar certificados SSL/TLS para 
asegurar la comunicación en servidores, incluyendo aquellos en el ámbito del IoT, aunque estos certificados requieran renovaciones periódicas.

\subsubsection{Código Abierto}
% 5.1.1.3 Código Abierto
Se destaca que Let's Encrypt es una autoridad de certificación de código abierto en el contexto más 
amplio de asegurar un dominio personalizado con SSL/TLS.
Esto es este aspecto:
\begin{itemize}
    \item Naturaleza de Código Abierto, Gratuita y Automatizada:
        \begin{itemize}
            \item Let's Encrypt se describe explícitamente como una 'autoridad de certificación gratuita, automatizada y de código abierto'.
            \item Se menciona que es ampliamente utilizada por muchos desarrolladores y empresas, y que cuenta con el respaldo de importantes patrocinadores.
            \item Su objetivo es permitir la obtención de certificados SSL/TLS.
        \end{itemize}
\end{itemize}
En resumen, se  indican que la característica de ser de código abierto es fundamental para Let's Encrypt, junto con ser 
gratuita y automatizada. Esto contribuye a su adopción generalizada y al respaldo que recibe, facilitando así la obtención de 
certificados SSL/TLS para asegurar la comunicación en servidores, como los utilizados en proyectos IoT.

\subsubsection{Respaldado por Grandes Patrocinadores}
% 5.1.1.3 Respaldado por Grandes Patrocinadores
Aquí se indica claramente que Let's Encrypt cuenta con el respaldo de grandes patrocinadores en el contexto 
más amplio de su rol como Autoridad de Certificación.
Esto es este aspecto:
\begin{itemize}
    \item Respaldo de Patrocinadores Importantes:
        \begin{itemize}
            \item Let's Encrypt se describe como una 'autoridad de certificación gratuita, automatizada y de código abierto'.
            \item Se utiliza ampliamente por 'muchos desarrolladores y empresas' y, crucialmente, está 'respaldada por grandes 
            patrocinadores'. Aunque no se  lista explícitamente los nombres de los patrocinadores, menciona 'like all of these', 
            sugiriendo que son numerosos y significativos.
        \end{itemize}
\end{itemize}
En resumen, se  enfatizan que el hecho de que Let's Encrypt esté respaldada por grandes patrocinadores es una de 
sus características clave, junto con ser gratuita, automatizada y de código abierto. Este respaldo contribuye a su 
credibilidad y capacidad para ofrecer certificados SSL/TLS ampliamente adoptados para asegurar dominios personalizados, 
incluyendo aquellos en el ámbito del Internet de las Cosas (IoT).

\subsection{INSTALACIóN DE CERTBOT (SOFTWARE DE TERCEROS)}
% 5.1.2 INSTALACIóN DE CERTBOT (SOFTWARE DE TERCEROS).
Se proporciona información detallada sobre la instalación de software de terceros, específicamente Certbot, en el contexto 
de asegurar un dominio personalizado con certificados SSL/TLS de Let's Encrypt.

\begin{itemize}
    \item Contexto General: Asegurar un Dominio Personalizado con SSL/TLS.
    \item El objetivo principal de este proceso es asegurar un dominio personalizado utilizando certificados SSL/TLS de Let's Encrypt 
    para habilitar HTTPS. Esto es parte de la sección 5 del curso 'Internet of Things using Python and Raspberry Pi', que se enfoca en proteger 
    un servidor HTTP IoT y la autenticación de usuarios. Se menciona también que la seguridad en Internet y la criptografía son temas 
    clave en la sección 3, incluyendo protocolos SSL/TLS (HTTPs). Después de asegurar el dominio con Let's Encrypt, el servidor redirigirá el 
    tráfico de HTTP a HTTPS, mostrando un candado verde que indica que la conexión es segura, y toda comunicación entre el cliente y el servidor 
    estará encriptada de extremo a extremo.
    \item Instalación y Uso de Certbot:
    \begin{itemize}
        \item 1. Autoridad de Certificación: Para obtener un certificado SSL/TLS, se utiliza Let's Encrypt, una autoridad de certificación 
        gratuita, automatizada y de código abierto, respaldada por importantes patrocinadores y utilizada por muchos desarrolladores y empresas.
        \item 2. Software de Terceros (Certbot): El primer paso para usar Let's Encrypt es instalar software de terceros en el servidor. 
        Este software es Certbot, el cliente de Let's Encrypt.
        \item 3. Pasos de Instalación:
        \begin{itemize}
            \item Conexión al Servidor: Se debe conectar al servidor, por ejemplo, usando PuTTY.
            \item Añadir el Repositorio: Se añade el repositorio de Certbot utilizando el comando \verb|sudo add-apt-repository ppa:certbot/certbot|. 
            Este es un PPA (Package Archive) preparado por el equipo de Debian Let's Encrypt y de respaldo para Ubuntu.
            \item Actualizar Lista de Paquetes: Después de añadir el repositorio, se actualiza la lista de paquetes con \verb|sudo apt-get update| 
            para que el nuevo repositorio sea reconocido.
            \item Instalar Certbot: Finalmente, se instala Certbot desde el nuevo repositorio con el comando 
            \verb|sudo apt-get install python-certbot-apache|.
        \end{itemize}
        \item 4. Configuración del Certificado SSL con Certbot: Una vez instalado, Certbot se usa para configurar el certificado SSL para Apache:
        \begin{itemize}
            \item Se ejecuta el comando \verb|sudo certbot --apache -d [nombre_de_dominio] -d www.[nombre_de_dominio]|.
            \item Información Requerida: Se pedirá una dirección de correo electrónico para la recuperación de la clave en caso de pérdida.
            \item Redirección HTTPS: Se preguntará si se desea redirigir el tráfico HTTP a HTTPS, lo cual es altamente recomendable.
            \item Ubicación de los Certificados: Una vez finalizada la instalación, los archivos del certificado generado se pueden 
            encontrar en /etc/letsencrypt/live.
        \end{itemize}
        \item 5. Verificación del Certificado:
        \begin{itemize}
            \item Se puede verificar el estado del certificado SSL en SSL labs.com/ssltest proporcionando el nombre de dominio.
            \item Reglas de Seguridad Inbound: Si la evaluación falla ('unable to connect to the server'), puede ser debido a la 
            falta de reglas de seguridad inbound para HTTPS en el servidor remoto (por ejemplo, en AWS). Es necesario editar las 
            reglas de inbound para añadir HTTPS (puerto 443) y luego guardar.
            \item Una vez configurado correctamente, la verificación mostrará los detalles del certificado SSL, indicando que está 
            firmado para el servidor (ej. 'packt IOT server'), emitido por Let's Encrypt, y con una validez de aproximadamente tres 
            meses. Dado que es una autoridad de certificación gratuita, los certificados expiran cada tres meses y deben ser renovados.
        \end{itemize}
    \end{itemize}
\end{itemize}
En resumen, Certbot es una herramienta esencial y de código abierto que facilita la obtención y configuración de certificados 
SSL/TLS de Let's Encrypt para asegurar un dominio, lo que permite la comunicación encriptada (HTTPS) entre los clientes y el servidor.

\subsubsection{CONECTAR AL SERVIDOR (Putty)}
En el contexto de la instalación de software de terceros (Certbot) para asegurar un dominio personalizado con SSL/TLS, se  
indican que la conexión al servidor es el primer paso fundamental, y PuTTY es la herramienta específica mencionada para llevar 
a cabo esta conexión.
Aquí se detalla lo que se  dice sobre 'Conectar al Servidor (PuTTY)':
\begin{itemize}
    \item Paso Inicial para la Instalación de Certbot: Antes de proceder con la instalación de Certbot, es necesario establecer 
    una conexión con el servidor. Se especifica que el primer paso para utilizar Let's Encrypt y obtener un certificado SSL 
    es instalar software de terceros en el servidor, y para ello, primero se conecta al servidor usando PuTTY.
    \item Secuencia de Comandos: Una vez conectado al servidor mediante PuTTY, el siguiente paso inmediato es añadir el repositorio 
    de Certbot utilizando el comando \verb|sudo add-apt-repository ppa:certbot/certbot|. Esto establece claramente la posición de la conexión 
    al servidor como un requisito previo a cualquier comando de instalación.
    \item Contexto de Servidores Remotos: Aunque no se menciona directamente PuTTY en todos los contextos de conexión, la necesidad de 
    acceder a un servidor remoto (como una instancia EC2 de AWS o una Raspberry Pi) para realizar configuraciones es recurrente. Por ejemplo, 
    en el caso de una Raspberry Pi, se menciona que el 'escritorio remoto' y SSH pueden requerir que SSH esté habilitado por defecto si 
    está deshabilitado, y para ello se debe usar \verb|sudo raspi-config| en el terminal de la Raspberry Pi. PuTTY es una herramienta comúnmente 
    utilizada en Windows para establecer conexiones SSH a servidores remotos, lo que alinea su uso con la gestión de 
    estos entornos de servidor.
\end{itemize}
En resumen, PuTTY se presenta como la interfaz de conexión esencial para iniciar el proceso de instalación de Certbot 
en el servidor, permitiendo ejecutar los comandos necesarios para configurar los certificados SSL/TLS.

\subsubsection{AGREGAR REPOSITORIO CERTBOT}
En el contexto más amplio de la instalación de software de terceros (Certbot) para asegurar un dominio personalizado con 
certificados SSL/TLS, se  detallan que agregar el repositorio de Certbot (\verb|sudo add-apt-repository ppa:certbot/certbot|)
es el primer comando crucial que se ejecuta en el servidor, después de haber establecido la conexión.
A continuación, se detalla este aspecto:
\begin{itemize}
    \item 1. El Primer Paso para la Instalación de Certbot: Se establece que, para utilizar la autoridad de certificación 
    Let's Encrypt y obtener un certificado SSL, el primer paso es instalar software de terceros en el servidor. Inmediatamente después 
    de conectarse al servidor (por ejemplo, usando PuTTY, como se mencionó en nuestra conversación anterior), la primera acción es añadir 
    el repositorio de Certbot.
    \item 2. El Comando Específico: El comando exacto proporcionado es \verb|sudo add-apt-repository ppa:certbot/certbot|.
    \item 3. Propósito del Repositorio PPA:
        \begin{itemize}
            \item Se explica que este es un PPA (Package Archive).
            \item Este PPA ha sido preparado por el equipo de Debian Let's Encrypt y de respaldo para Ubuntu.
            \item La adición de este repositorio es fundamental porque permite que el sistema operativo del servidor sepa dónde 
            encontrar los paquetes de instalación más recientes y correctos para Certbot.
        \end{itemize}
    \item 4. Secuencia Post-Adición del Repositorio: Una vez que el repositorio ha sido añadido, el siguiente paso es actualizar 
    la lista de paquetes con \verb|sudo apt-get update|. Esta actualización es necesaria para que el sistema reconozca el nuevo 
    repositorio que se acaba de añadir y pueda ver los paquetes disponibles en él. Solo después de esta actualización se puede 
    proceder a instalar Certbot (\verb|sudo apt-get install python-certbot-apache|) desde el nuevo repositorio.
\end{itemize}
En resumen, el comando \verb|sudo add-apt-repository ppa:certbot/certbot| es una instrucción indispensable que prepara el entorno 
del servidor para la instalación de Certbot, asegurando que el sistema operativo tenga acceso a las versiones necesarias del software, 
lo que a su vez es un paso fundamental para obtener y configurar los certificados SSL/TLS de Let's Encrypt y, en última instancia, 
asegurar el servidor HTTP IoT y permitir la comunicación encriptada de extremo a extremo.

\subsubsection{ACTUALIZAR LISTA DE PAQUETES}
En el contexto más amplio de la instalación de software de terceros (Certbot) para asegurar un dominio personalizado con 
certificados SSL/TLS, se  indican que 'Actualizar Lista de Paquetes (\verb|sudo apt-get update|)' es un paso 
intermedio esencial después de añadir el repositorio de Certbot y antes de su instalación final.
A continuación, se detalla este aspecto:
\begin{itemize}
    \item 1. Propósito del Comando: Se explica explícitamente que después de añadir el nuevo repositorio (el PPA de Certbot), 
    es necesario actualizar la lista de paquetes para 'recoger' (pick up) el nuevo repositorio que se acaba de añadir. Esto significa que 
    el sistema operativo necesita ser informado sobre los nuevos orígenes de software disponibles.
    \item 2. El Comando Específico: El comando proporcionado para esta acción es \verb|sudo apt-get update|.
    \item 3. Lugar en la Secuencia de Instalación de Certbot:
        \begin{itemize}
            \item Este comando se ejecuta después de haber añadido el repositorio de Certbot con \verb|sudo add-apt-repository ppa:certbot/certbot|.
            \item Se ejecuta antes de instalar Certbot con \verb|sudo apt-get install python-certbot-apache|.
            \item La secuencia de pasos que se sigue es: conectarse al servidor (por ejemplo, con PuTTY), añadir el repositorio de Certbot, 
            actualizar la lista de paquetes e, inmediatamente después, instalar Certbot.
        \end{itemize}
\end{itemize}
En resumen, \verb|sudo apt-get update| es un paso crítico para asegurar que el sistema del servidor esté al tanto de los paquetes 
de software disponibles en el repositorio de Certbot recién agregado. Sin esta actualización, el sistema no podría encontrar ni 
instalar el cliente de Let's Encrypt (Certbot), lo que detendría el proceso de asegurar el dominio personalizado con certificados 
SSL/TLS y habilitar la comunicación HTTPS encriptada de extremo a extremo.

\subsubsection{INSTALAR CERTBOT PARA APACHE}
En el contexto más amplio de la instalación de software de terceros (Certbot) para asegurar un dominio personalizado con certificados 
SSL/TLS, se  indican que el comando \verb|sudo apt-get install python-certbot-apache| es el paso final para la instalación del 
cliente Certbot en el servidor, haciéndolo operativo y listo para configurar los certificados.
A continuación, se detalla este aspecto:
\begin{itemize}
    \item 1. Propósito del Comando: Este comando instala el cliente de Let's Encrypt, Certbot, junto con su plugin específico para 
    el servidor web Apache. Se afirma que después de ejecutarlo, 'el cliente de Let's Encrypt, Certbot, ya está listo para ser usado'.
    \item 2. Lugar en la Secuencia de Instalación de Certbot:
        \begin{itemize}
            \item Este comando se ejecuta después de haber añadido el repositorio de Certbot (\verb|sudo add-apt-repository ppa:certbot/certbot|).
            \item También se ejecuta después de haber actualizado la lista de paquetes (\verb|sudo apt-get update|) para que el sistema reconozca 
            el nuevo repositorio.
            \item Es el último paso en la secuencia de preparación del sistema para la configuración del certificado, siguiendo la conexión 
            al servidor (ej. PuTTY), la adición del repositorio y la actualización de la lista de paquetes.
        \end{itemize}
    \item 3. Habilitación de la Configuración SSL: Una vez que \verb|sudo apt-get install python-certbot-apache| ha sido ejecutado con éxito, 
    el cliente Certbot está preparado para configurar el certificado SSL para Apache. El siguiente paso directo es ejecutar 
    \verb|sudo certbot --apache -d [nombre_de_dominio] -d www.[nombre_de_dominio]| para iniciar la configuración del certificado en sí.
    \item 4. Contexto de Asegurar el Dominio: La instalación de este software de terceros (Certbot) es el primer paso crucial para utilizar 
    Let's Encrypt como autoridad de certificación. Al instalar python-certbot-apache, se obtiene la herramienta necesaria para automatizar 
    la obtención y configuración de los certificados SSL/TLS, que es fundamental para asegurar el servidor HTTP IoT y permitir la comunicación 
    encriptada de extremo a extremo (HTTPS).
\end{itemize}
En síntesis, \verb|sudo apt-get install python-certbot-apache| es la culminación de la fase de preparación del software Certbot 
en el servidor, permitiendo que el cliente de Let's Encrypt esté disponible para configurar y gestionar los certificados SSL/TLS 
que asegurarán el dominio personalizado y habilitarán HTTPS.

\subsection{CONFIGURACIóN DEL CERTIFICADO SSL PARA APACHE}
% 5.1.3 CONFIGURACIóN DEL CERTIFICADO SSL PARA APACHE.
Se detalla que 'Configurar Certificado SSL para Apache' es el paso crucial que sigue a la instalación del cliente Certbot 
y es el que finalmente permite asegurar un dominio personalizado con SSL/TLS de Let's Encrypt. Este proceso es fundamental para la 
sección 5 del curso 'Internet of Things using Python and Raspberry Pi', que se centra en proteger un servidor HTTP IoT y la 
autenticación de usuarios.
Aquí se desglosa lo que se  dice sobre este proceso:
\begin{itemize}
    \item 1. Propósito y Contexto Amplio de Asegurar el Dominio: El objetivo es asegurar un dominio personalizado con certificados 
    SSL/TLS para habilitar HTTPS. Una vez que el dominio está asegurado con Let's Encrypt, el servidor redirigirá automáticamente el 
    tráfico de HTTP a HTTPS, mostrando un candado verde en el navegador que indica que la conexión es segura, y toda la comunicación entre 
    el cliente y el servidor estará encriptada de extremo a extremo. Esto se basa en la autoridad de certificación gratuita, automatizada y 
    de código abierto Let's Encrypt.
    \item 2. Ejecución del Cliente Certbot para Configurar el Certificado SSL: Después de haber instalado Certbot para Apache 
    (mediante \verb|sudo apt-get install python-certbot-apache|), el cliente de Let's Encrypt ya está listo para ser usado. El siguiente 
    paso es configurar el certificado SSL para Apache.
        \begin{itemize}
            \item Comando Específico: Para ello, se ejecuta el comando \verb|sudo certbot --apache -d [nombre_de_dominio] -d www.[nombre_de_dominio]|. 
            Este comando instruye a Certbot para que genere y configure el certificado SSL para el dominio especificado, incluyendo su 
            versión con 'www'.        
        \end{itemize}
    \item 3. Interacciones y Opciones Durante la Configuración: Durante la ejecución del comando, Certbot solicitará cierta información y 
    ofrecerá opciones:
        \begin{itemize}
            \item Dirección de Correo Electrónico: Se pedirá una dirección de correo electrónico, la cual se utilizará para la recuperación de 
            la clave en caso de pérdida.
            \item Redirección HTTP a HTTPS: Se preguntará si se desea redirigir el tráfico HTTP a HTTPS. Se enfatiza que definitivamente 
            se desea hacer esto, indicando que se debe seleccionar la opción de redirigir.
        \end{itemize}
    \item 4. Resultado de la Configuración: Una vez finalizada la instalación y configuración, los archivos del certificado generado se pueden 
    encontrar en la ubicación /etc/letsencrypt/live.
    \item 5. Verificación del Estado del Certificado SSL: Es fundamental verificar que el certificado se haya configurado correctamente:
        \begin{itemize}
            \item Herramienta de Verificación: Se puede verificar el estado del certificado SSL y su configuración yendo a SSL labs.com/ssltest 
            y proporcionando el nombre del DNS.
            \item Solución de Problemas (Reglas de Seguridad Inbound): Si la evaluación inicial falla, indicando 'unable to connect to the server' 
            (incapaz de conectar al servidor), la razón común es la falta de reglas de seguridad inbound para HTTPS en el servidor remoto 
            (por ejemplo, en AWS). Es necesario editar las reglas de inbound para añadir HTTPS (el puerto predeterminado para HTTPS es 443) y 
            luego guardar los cambios.
            \item Confirmación Exitosa: Después de ajustar las reglas de seguridad, la verificación mostrará los detalles del certificado SSL, 
            confirmando que está firmado para el servidor (ej. 'packt IOT server'), emitido por Let's Encrypt, y con una validez de 
            aproximadamente tres meses. Es importante recordar que, al ser una autoridad de certificación gratuita, los certificados 
            expiran cada tres meses y deben renovarse.        
        \end{itemize}
\end{itemize}
En conclusión, la configuración del certificado SSL para Apache utilizando Certbot es el paso que materializa la seguridad del 
dominio. Implica la ejecución de un comando específico, la interacción con algunas preguntas para definir preferencias 
(como la redirección HTTPS), y la verificación posterior para asegurar que el certificado esté activo y que las reglas de seguridad 
del servidor permitan la comunicación segura, logrando así una comunicación encriptada de extremo a extremo.

\subsubsection{EJECUTAR CERBOT}
En el contexto más amplio de la configuración de un Certificado SSL para Apache, se  indican que el comando 
\verb|sudo certbot --apache -d [nombre_de_dominio] -d www.[nombre_de_dominio]| es la instrucción clave que se ejecuta para activar el 
cliente de Let's Encrypt y proceder con la obtención y configuración del certificado SSL/TLS para el servidor web Apache. Este es un paso 
fundamental para asegurar el dominio personalizado y habilitar la comunicación HTTPS.
A continuación, se detalla lo que se  dice sobre la ejecución de este comando:
\begin{itemize}
    \item 1. Propósito del Comando:
        \begin{itemize}
            \item Una vez que el cliente de Let's Encrypt (Certbot) ha sido instalado y está listo para usarse, este comando se 
            utiliza específicamente para configurar el certificado SSL para Apache.
            \item El comando especifica el servidor web (\verb|--apache|) y los dominios para los cuales se desea el certificado 
            (\verb|-d [nombre_de_dominio] -d www.[nombre_de_dominio]|), asegurando que tanto el dominio principal como su versión con 
            'www' estén cubiertos.
        \end{itemize}
    \item 2. Interacciones Durante la Ejecución:
        \begin{itemize}
            \item Cuando se ejecuta el comando, Certbot solicita una dirección de correo electrónico. Esta dirección se utilizará para la 
            recuperación de la clave en caso de que se pierda.
            \item También se le preguntará al usuario si desea redirigir el tráfico HTTP a HTTPS. Se enfatiza que 
            'definitivamente queremos eso' y se debe seleccionar la opción de redirigir, lo que significa que el servidor automáticamente 
            reenviará todas las solicitudes HTTP al puerto seguro HTTPS (puerto 443).
        \end{itemize}
    \item 3. Resultado de la Ejecución Exitosa:
        \begin{itemize}
            \item Una vez que la instalación y configuración han finalizado, los archivos del certificado generado se pueden encontrar 
            en la ubicación /etc/letsencrypt/live.
            \item El servidor se configurará para redirigir de HTTP a HTTPS, y se mostrará un candado verde en el navegador, indicando una 
            conexión segura y comunicación encriptada de extremo a extremo entre el cliente y el servidor.
        \end{itemize}
    \item 4. Verificación Post-Configuración:
        \begin{itemize}
            \item Se sugiere verificar el estado del certificado SSL y su configuración utilizando una herramienta como 
            SSL labs.com/ssltest, donde se debe proporcionar el nombre de dominio.
            \item Si la evaluación inicial falla con un mensaje como 'unable to connect to the server', se señala que la causa más probable 
            es la falta de reglas de seguridad inbound para HTTPS (puerto 443) en el servidor remoto (por ejemplo, en AWS). Es necesario editar 
            estas reglas para añadir el puerto HTTPS y luego guardar.
            \item Una vez corregidas las reglas, la verificación exitosa mostrará los detalles del certificado SSL, confirmando que está 
            firmado para el servidor (ej. 'packt IOT server'), emitido por Let's Encrypt, y válido por aproximadamente tres meses. 
            Se recuerda que, al ser una autoridad de certificación gratuita, los certificados deben renovarse cada tres meses.
        \end{itemize}
\end{itemize}
En resumen, el comando \verb|sudo certbot --apache -d [nombre_de_dominio] -d www.[nombre_de_dominio]| es el corazón del proceso de 
configuración SSL para Apache, ya que orquesta la obtención, instalación y ajuste de los certificados de Let's Encrypt. Es un paso crítico 
para transformar una conexión HTTP en una conexión HTTPS segura, protegiendo así la comunicación de un servidor IoT.

\subsubsection{PROPORCIONAR CORREO ELECTRóNICO (RECUPERACIóN DE CLAVE)}
En el contexto más amplio de la Configuración de Certificado SSL para Apache, se  indican que proporcionar una dirección de 
correo electrónico es un paso interactivo y obligatorio durante la ejecución del comando certbot, y su propósito es fundamental 
para la recuperación de la clave del certificado.
Aquí se detalla este aspecto:
\begin{itemize}
    \item 1. Momento de la Solicitud: Después de ejecutar el comando \verb|sudo certbot --apache -d [nombre_de_dominio] -d www.[nombre_de_dominio]| 
    para configurar el certificado SSL para Apache, Certbot solicitará información específica al usuario.
    \item 2. Información Solicitada: Una de las solicitudes es 'proporcionar un correo electrónico'.
    \item 3. Propósito: Recuperación de Clave: La razón explícita para solicitar este correo electrónico es que 
    'este correo electrónico se utilizará para la recuperación en caso de que perdamos la clave'. Esto asegura que si se pierde 
    el acceso a la clave privada del certificado SSL, hay un mecanismo para recuperarla, manteniendo así la continuidad 
    de la seguridad del dominio.
    \item 4. Contexto General de Seguridad: Esta interacción es parte integral del proceso de asegurar un dominio personalizado 
    con certificados SSL/TLS de Let's Encrypt. La capacidad de recuperar la clave es un aspecto importante de la gestión de certificados 
    y la seguridad a largo plazo de un servidor HTTP IoT, asegurando que el servidor pueda mantener su comunicación 
    encriptada de extremo a extremo.
\end{itemize}
En resumen, cuando se está configurando el certificado SSL para Apache utilizando Certbot, se requiere que el usuario proporcione 
una dirección de correo electrónico, cuya función principal es servir como método de recuperación en caso de que la clave del 
certificado SSL se pierda, garantizando así la capacidad de mantener el dominio seguro con HTTPS.

\subsubsection{REDIRECCIóN DE HTTP A HTTPS (TYPE 2 TO ENTER)}
En el contexto más amplio de la Configuración del Certificado SSL para Apache, se  destacan que la redirección de HTTP a HTTPS, 
específicamente al seleccionar la opción '2', es una elección crucial que se presenta durante la ejecución de Certbot y es altamente 
recomendada para garantizar la seguridad total del dominio.
Aquí se detalla este aspecto:
\begin{itemize}
    \item 1. Momento de la Solicitud: Después de ejecutar el comando \verb|sudo certbot --apache -d [nombre_de_dominio] -d www.[nombre_de_dominio]| 
    para configurar el certificado SSL para Apache, Certbot interactúa con el usuario solicitando información y preferencias. Una de las 
    preguntas que se formula es 'si queremos redirigir nuestro tráfico HTTP a HTTPS'.
    \item 2. La Opción Específica (Seleccionar 2): Se menciona que se debe 'escribir 2 para entrar' para seleccionar esta opción. 
    Esto indica que '2' es la opción numérica correspondiente a la acción de redirigir el tráfico.
    \item 3. Recomendación Enfatizada: La instrucción es clara: 'definitivamente queremos eso'. Esto subraya la importancia de esta 
    redirección, haciendo entender que es una práctica estándar y altamente deseable en la configuración de un servidor seguro.
    \item 4. Propósito y Beneficios de la Redirección:
        \begin{itemize}
            \item La redirección asegura que todo el tráfico que inicialmente llegue a través del protocolo HTTP inseguro sea automáticamente 
            reenviado al protocolo HTTPS seguro.
            \item Esto es fundamental para la seguridad de extremo a extremo (end-to-end encrypted communication) entre el cliente y el servidor.
            \item Al activar esta redirección, el servidor exhibirá un 'candado verde' en el navegador, lo que indica visualmente a los 
            usuarios que la conexión es segura.
            \item Este proceso es parte integral de la sección 5 del curso, que se enfoca en proteger un servidor HTTP IoT y la autenticación 
            de usuarios, donde la seguridad en Internet y la criptografía, incluidos los protocolos SSL/TLS (HTTPS), son temas clave.
        \end{itemize}
\end{itemize}
En resumen, seleccionar la opción '2' para redirigir el tráfico de HTTP a HTTPS durante la configuración del certificado SSL con Certbot 
es un paso interactivo que se  recomiendan enfáticamente. Es esencial para garantizar que todas las comunicaciones con el dominio 
personalizado sean encriptadas, proporcionando una conexión segura y una mejor experiencia para el usuario al mostrar 
un candado verde en el navegador.

\subsection{UBICACIóN DE CERTIFICADOS GENERADOS: /etc/letsencrypt/live/}
% 5.1.4 UBICACIóN DE CERTIFICADOS GENERADOS: /etc/letsencrypt/live/.
En el contexto más amplio de Asegurar un Dominio Personalizado con SSL/TLS, se  indican que la ubicación de los 
certificados generados, específicamente en /etc/letsencrypt/live, es el resultado final y tangible del proceso de 
configuración exitosa con Certbot.
Aquí se detalla este aspecto:
\begin{itemize}
    \item 1. Resultado de la Instalación y Configuración: Una vez que la instalación y configuración del certificado SSL para Apache 
    utilizando Certbot ha finalizado, el sistema almacenará los archivos correspondientes. Se afirma claramente que, 
    después de completar este proceso, 'deberíamos poder encontrar nuestros archivos de certificado generados en /etc/letsencrypt/live'.
    \item 2. Lugar en el Proceso General: Esta ubicación se menciona inmediatamente después de los pasos interactivos de Certbot, 
    como proporcionar una dirección de correo electrónico para la recuperación de la clave y seleccionar la opción de redirigir el 
    tráfico HTTP a HTTPS. Es decir, una vez que Certbot ha ejecutado su función y completado la configuración, estos archivos se 
    almacenan en la ruta especificada.
    \item 3. Importancia en la Seguridad del Dominio: La presencia de estos archivos en esta ubicación significa que el dominio 
    personalizado ha sido exitosamente asegurado con certificados SSL/TLS de Let's Encrypt. Estos certificados son los que permiten 
    que el servidor redirija el tráfico de HTTP a HTTPS, muestre un 'candado verde' en el navegador y establezca una comunicación 
    encriptada de extremo a extremo entre el cliente y el servidor.
\end{itemize}
En resumen, /etc/letsencrypt/live es el directorio estándar donde Certbot almacena los certificados SSL/TLS emitidos por Let's Encrypt 
para un dominio. Su existencia confirma que el proceso de asegurar el dominio personalizado ha sido completado con éxito, lo que es 
fundamental para la seguridad del servidor HTTP IoT y la autenticación de usuarios.

\subsection{VERIFICAR ESTADO DEL CERTIFICADO: SSL LABS (ssltest.com)}
% 5.1.5 VERIFICAR ESTADO DEL CERTIFICADO: SSL LABS (ssltest.com).
En el contexto más amplio de Asegurar un Dominio Personalizado con SSL/TLS, se  indican que la verificación del estado del 
certificado SSL utilizando SSL Labs (ssltest.com) es un paso esencial y final para confirmar que la configuración del certificado 
se ha realizado correctamente y que el dominio está efectivamente protegido.
A continuación, se detalla este aspecto:
\begin{itemize}
    \item 1. Propósito de la Verificación: Una vez que se ha instalado Certbot y se ha configurado el certificado SSL para Apache, es 
    fundamental verificar su estado. Se afirma que 'ahora también podemos verificar el estado de nuestro certificado SSL yendo 
    a SSL labs.com/ssltest'. Esta verificación asegura que el proceso fue exitoso.
    \item 2. Cómo Realizar la Verificación: Para utilizar la herramienta, se debe proporcionar el nombre de DNS (nombre de dominio) 
    en SSL Labs.
    \item 3. Posibles Resultados y Solución de Problemas:
        \begin{itemize}
            \item Fallo en la Evaluación: Si la evaluación inicial falla, mostrando un mensaje como 'unable to connect to the server' 
            (incapaz de conectar al servidor), se  identifican una razón común para esto: la falta de reglas de seguridad inbound 
            para HTTPS en el servidor remoto (ej. en AWS).
            \item Corrección Necesaria: Para solucionar este problema, es necesario editar las reglas de seguridad de entrada 
            (inbound security rules) del servidor y añadir HTTPS, cuyo puerto predeterminado es 443, y luego guardar los cambios.
        \end{itemize}
    \item 4. Confirmación Exitosa de la Verificación:
        \begin{itemize}
            \item Una vez que las reglas de seguridad se han configurado correctamente, la verificación en SSL Labs mostrará los detalles 
            del certificado SSL.
            \item Estos detalles incluyen que el certificado está específicamente firmado para el servidor (ej. 'packt IOT server'), 
            que la autoridad de emisión es Let's Encrypt, y que es válido desde la fecha actual hasta aproximadamente tres meses. Se hace una 
            observación importante sobre que, al ser una autoridad de certificación gratuita, los certificados de Let's Encrypt expiran cada t
            res meses y deben renovarse.
        \end{itemize}
    \item 5. Relevancia en el Contexto General: La verificación exitosa en SSL Labs es la confirmación de que el dominio está completamente 
    asegurado. Esto significa que el servidor redirigirá el tráfico de HTTP a HTTPS, mostrando el 'candado verde' en el navegador, y que 
    toda la comunicación entre el cliente y el servidor estará encriptada de extremo a extremo. Este logro es central para la seguridad del 
    servidor HTTP IoT y la autenticación de usuarios, temas clave en la sección 5 del curso.
\end{itemize}
En resumen, SSL Labs (ssltest.com) es una herramienta de diagnóstico externa fundamental para validar la correcta implementación 
del certificado SSL/TLS en un dominio personalizado. Permite al usuario confirmar que el certificado está activo y bien configurado, 
y, en caso de problemas, ayuda a identificar y corregir fallos como las reglas de seguridad del servidor.

\subsubsection{PROBLEMA INICIAL: FALTAN REGLAS DE SEGURIDAD DE ENTRADA HTTPS EN AWS EC2}
En el contexto más amplio de la Verificación del Estado del Certificado SSL mediante SSL Labs (ssltest.com), se  
identifican un 'Problema Inicial: Faltan Reglas de Seguridad de Entrada HTTPS en AWS EC2' como una causa común de falla en 
la evaluación del certificado, lo que impide que el dominio sea reconocido como seguro.
Aquí se detalla este aspecto:
\begin{itemize}
    \item 1. Momento de la Detección del Problema: Después de haber instalado Certbot y configurado el certificado SSL para Apache, 
    se  sugieren verificar el estado del certificado en SSL labs.com/ssltest proporcionando el nombre de DNS. Sin embargo, 
    la evaluación inicial 'ha fallado', mostrando el mensaje 'unable to connect to the server' (incapaz de conectar al servidor).
    \item 2. Identificación de la Causa Raíz (AWS EC2 y Reglas de Seguridad Inbound): Se identifica inmediatamente la razón de 
    este fallo: 'no hemos asignado reglas de seguridad de entrada (inbound security rules) para HTTPS en nuestro servidor remoto de AWS'. 
    Esto indica que el servidor (en este caso, una instancia EC2 de AWS) no tiene configuradas las reglas necesarias para permitir el 
    tráfico seguro a través de HTTPS.
    \item 3. Solución al Problema:
        \begin{itemize}
            \item Para corregirlo, se accede a la configuración de la instancia EC2 (por ejemplo, a través de 'ec2 instance launch wizard 
            two inbounds').
            \item Se observa que inicialmente solo se tienen reglas para HTTP y SSH.
            \item Se debe editar estas reglas de seguridad y 'añadir regla HTTPS'.
            \item El puerto predeterminado para HTTPS es 443, y se debe guardar esta configuración.
        \end{itemize}
    \item 4. Confirmación de la Solución y Verificación Exitosa:
        \begin{itemize}
            \item Una vez que las reglas de seguridad han sido modificadas para permitir el tráfico HTTPS, se reintenta la 
            verificación en SSL Labs.
            \item Tras un breve período, la evaluación se completa con éxito, mostrando los detalles del certificado SSL, 
            incluyendo que está firmado para el servidor (ej. 'packt IOT server'), emitido por Let's Encrypt, y válido por 
            aproximadamente tres meses.
            \item Esto confirma que el servidor ahora es accesible a través de HTTPS y que el certificado está correctamente configurado.
        \end{itemize}
\end{itemize}
En resumen, la falta de reglas de seguridad de entrada para HTTPS en un servidor remoto como AWS EC2 es un obstáculo significativo que 
puede hacer que la verificación del certificado SSL falle. Se resalta la importancia de configurar el puerto 443 para HTTPS 
en las reglas de seguridad de entrada como un paso crítico para asegurar que el certificado SSL pueda ser correctamente validado y 
que el dominio personalizado logre una comunicación encriptada de extremo a extremo.

\subsubsection{SOLUCIóN: AGREGAR REGLAS HTTPS (PUERTO 443) EN AWS INBOUND RULES}
En el contexto más amplio de la Verificación del Estado del Certificado SSL mediante SSL Labs (ssltest.com), se  
detallan que la 'Solución: Agregar Regla HTTPS (Puerto 443) en AWS Inbound Rules' es la acción correctiva directa para resolver 
el problema de conexión (unable to connect to the server) que se presenta cuando el servidor remoto (específicamente en AWS) 
no tiene configuradas las reglas de seguridad necesarias para permitir el tráfico HTTPS.
A continuación, se detalla este aspecto:
\begin{itemize}
    \item 1. Detección del Problema y Necesidad de la Solución: Después de intentar verificar el certificado SSL en SSL labs.com/ssltest, 
    el 'assessment ha fallado' con el mensaje 'unable to connect to the server'. La razón identificada para este fallo es que 'no hemos 
    asignado reglas de seguridad de entrada (inbound security rules) para HTTPS en nuestro servidor remoto de AWS'. Esto significa que, 
    aunque el certificado podría estar instalado, el firewall del servidor está bloqueando el acceso a través del puerto seguro.
    \item 2. Implementación de la Solución (Pasos en AWS EC2):
        \begin{itemize}
            \item Para corregir esto, se debe acceder a la configuración de la instancia EC2.
            \item Las reglas de seguridad existentes (inbound rules) típicamente solo incluyen HTTP y SSH.
            \item La solución consiste en 'editar esto [las reglas de seguridad] y añadir regla HTTPS'.
            \item Se especifica claramente que el 'puerto predeterminado para HTTP[S] es 443'. Por lo tanto, se debe configurar esta 
            regla para el puerto 443.
            \item Finalmente, es necesario guardar los cambios realizados en las reglas de seguridad.
        \end{itemize}
    \item 3. Resultado de la Aplicación de la Solución:
        \begin{itemize}
            \item Una vez que se añaden las reglas de seguridad para HTTPS y se guardan, se puede intentar 'refresh this' (actualizar) 
            la verificación en SSL Labs.
            \item Después de un corto período de tiempo ('takes a bit time to complete'), la evaluación se completará con éxito.
            \item La verificación exitosa mostrará los detalles del certificado SSL, confirmando que está firmado para el servidor 
            (ej. 'packt IOT server'), emitido por Let's Encrypt, y válido por aproximadamente tres meses.
            \item Esto es la confirmación de que el servidor ahora es accesible a través de HTTPS y que el dominio personalizado está 
            efectivamente asegurado, permitiendo una comunicación encriptada de extremo a extremo.
        \end{itemize}
\end{itemize}
En síntesis, la acción de agregar la regla HTTPS (puerto 443) a las reglas de seguridad de entrada en AWS EC2 es la solución directa y 
necesaria para permitir que SSL Labs (y, por extensión, cualquier cliente web) se conecte al servidor a través del protocolo seguro. 
Este paso es indispensable para validar la correcta implementación del certificado SSL/TLS y garantizar que el dominio esté verdaderamente 
protegido con HTTPS.

\subsection{DETALLES DEL CERTIFICADO: FIRMADO PARA  'pact IOT server', EMITIDO POR Let's Encrypt. Válido por 3 meses.}
% 5.1.6 DETALLES DEL CERTIFICADO: FIRMADO PARA  'pact IOT server ', EMITIDO POR Let's Encrypt. Válido por 3 meses.
En el contexto más amplio de Asegurar un Dominio Personalizado con SSL/TLS, se  detallan que los 'Detalles del Certificado: 
Firmado para 'pact IOT server', Emitido por Let's Encrypt, Válido por 3 Meses' son la confirmación final y tangible del éxito en la 
configuración del certificado, lo que indica que el dominio está debidamente protegido. Estos detalles se observan después de una 
verificación exitosa del estado del certificado, por ejemplo, utilizando SSL Labs (ssltest.com) y tras haber resuelto posibles 
problemas como la falta de reglas de seguridad de entrada HTTPS en el servidor remoto.
A continuación, se desglosa lo que se  dice sobre estos detalles:
\begin{itemize}
    \item 1. Firmado para 'pact IOT server':
        \begin{itemize}
            \item Se indica que el certificado SSL está 'específicamente firmado para nuestro propio servidor, pact IOT server'.
            \item Al verificar el certificado, se muestra que está 'emitido a pact IOT server'.
            \item Esto significa que el certificado ha sido generado y es válido para el dominio asociado con el 'pact IOT server', 
            confirmando que la seguridad se aplica correctamente a ese servidor específico.
        \end{itemize}
    \item 2. Emitido por Let's Encrypt:
        \begin{itemize}
            \item La 'autoridad emisora es Let's Encrypt'.
            \item Let's Encrypt se describe como una autoridad de certificación (CA) gratuita, automatizada y de código abierto, 
            utilizada por muchos desarrolladores y empresas, y respaldada por importantes patrocinadores.
            \item La decisión de utilizar Let's Encrypt como CA es el primer paso para obtener un certificado SSL y es parte de la 
            estrategia para asegurar el dominio personalizado. El curso se enfoca en asegurar el dominio utilizando esta autoridad 
            de certificación.
        \end{itemize}
    \item 3. Válido por 3 Meses:
        \begin{itemize}
            \item El certificado es 'válido desde hoy hasta casi tres meses'.
            \item Se explica que, dado que Let's Encrypt es una autoridad de certificación gratuita, los certificados expiran 
            cada tres meses.
            \item Esto implica que los usuarios 'tienen que renovarlo después de que expire'. La renovación es un aspecto importante 
            de la gestión continua de la seguridad del dominio.
        \end{itemize}
\end{itemize}
En el contexto más amplio de asegurar el dominio personalizado con SSL/TLS, la obtención de un certificado con estas características es 
la culminación de los pasos de instalación y configuración de Certbot. Una vez que se confirman estos detalles, significa que:
\begin{itemize}
    \item El servidor 'será redirigido de HTTP a HTTPS'.
    \item Se mostrará un 'candado verde' en el navegador, indicando que la conexión es segura.
    \item Toda la comunicación entre el cliente y el servidor tendrá una 'comunicación encriptada de extremo a extremo'.
\end{itemize}
Estos detalles son cruciales para el cumplimiento de los objetivos de la sección 5 del curso, que se centra en proteger un servidor HTTP IoT 
y la autenticación de usuarios, donde la seguridad en Internet y la criptografía, incluyendo los protocolos SSL/TLS (HTTPS), son temas clave.

\subsubsection{RESULTADO: REDIRECCIóN A HTTPS CON CANDADO VERDE (COMUNICACIóN ENCRIPTADA DE EXTREMO A EXTREMO)}
En el contexto más amplio de Asegurar un Dominio Personalizado con SSL/TLS, se  detallan que el 'Resultado: Redirección a 
HTTPS con Candado Verde (Comunicación Encriptada de Extremo a Extremo)' es la culminación exitosa de todo el proceso de configuración 
del certificado SSL/TLS con Certbot y Let's Encrypt. Este resultado indica que el dominio está completamente protegido y ofrece una 
experiencia segura al usuario.
A continuación, se desglosa lo que se  dice sobre este resultado:
\begin{itemize}
    \item 1. Redirección de HTTP a HTTPS:
        \begin{itemize}
            \item Una vez que el dominio está asegurado con Let's Encrypt, el servidor 'será redirigido de HTTP a HTTPS'. Esto ocurre 
            automáticamente después de la configuración exitosa del certificado.
            \item Durante el proceso de configuración con Certbot, se le pregunta al usuario si desea 'redirigir nuestro tráfico HTTP a HTTPS'. 
            Se enfatiza que 'definitivamente queremos eso' y se debe seleccionar la opción correspondiente (escribiendo '2'). 
            Esta elección es crucial para asegurar que todas las solicitudes sean manejadas por el protocolo seguro.
        \end{itemize}
    \item 2. Candado Verde en el Navegador:
        \begin{itemize}
            \item Como resultado de la redirección a HTTPS, el navegador mostrará un 'candado verde'.
            \item Este candado verde es una indicación visual para el usuario de que la conexión es 'segura'. Es un símbolo universalmente 
            reconocido de confianza y seguridad en la navegación web.
        \end{itemize}
    \item 3. Comunicación Encriptada de Extremo a Extremo:
        \begin{itemize}
            \item Se afirma que, una vez que el sitio web ha sido completamente asegurado, 'todo cliente conectado con 
            este servidor tendrá una comunicación encriptada de extremo a extremo'.
            \item Esto significa que todos los datos intercambiados entre el navegador del cliente y el servidor están protegidos 
            mediante criptografía, impidiendo que terceros intercepten o lean la información. La seguridad en Internet y la criptografía, 
            incluyendo los protocolos SSL/TLS (HTTPS), son temas clave del curso y esenciales para proteger un servidor HTTP IoT.
        \end{itemize}
\end{itemize}
En resumen, la redirección automática de HTTP a HTTPS, la aparición del candado verde en el navegador y la garantía de una comunicación 
encriptada de extremo a extremo son los indicadores más importantes del éxito en el proceso de asegurar un dominio personalizado 
con SSL/TLS. Estos resultados no solo validan la configuración técnica realizada, sino que también establece un canal de comunicación 
seguro y confiable, fundamental para la seguridad de un servidor IoT y la autenticación de usuarios.

\chapter{REGLAS PARA USUARIOS ADMINISTRADORES Y NO ADMINISTRADORES}
% 6.REGLAS PARA USUARIOS ADMINISTRADORES Y NO ADMINISTRADORES.
En el contexto del proyecto  'Internet de las Cosas con Python y Raspberry Pi ', la creación de  Reglas para Usuarios Administradores y No Administradores  
es una parte fundamental para desarrollar una plataforma IoT en la nube robusta, segura y multiusuario. Esta funcionalidad, abordada en la Sección 
5 y 6 del proyecto, permite que múltiples usuarios inicien sesión de forma segura y controlen o monitoreen sus dispositivos autorizados en tiempo real.

Se detalla cómo se implementan estas reglas y la gestión de acceso:
\begin{itemize}
    \item 1. Concepto y Objetivo de los Roles de Usuario
    \item El objetivo principal es establecer un sistema donde los  usuarios puedan tener roles de administrador o no administrador , con diferentes 
    niveles de acceso y control sobre la plataforma y los dispositivos.
    \begin{itemize}
        \item Plataforma Multi-Usuario : El proyecto se enfoca en construir una plataforma en la nube sin satélites donde múltiples usuarios pueden 
        interactuar de forma segura con sus dispositivos IoT.
        \item Control y Monitoreo Autorizado : Los usuarios pueden controlar y monitorear sus dispositivos, pero solo aquellos a los que están 
        autorizados.
        \item Gestión de Permisos en Tiempo Real : Los administradores pueden otorgar permisos de lectura y escritura en tiempo real a usuarios 
        no administradores y dispositivos.
    \end{itemize}

    \item 2. Panel de Control del Administrador
    \item Los usuarios con rol de administrador tienen acceso a un panel de control específico que les permite gestionar otros usuarios y sus permisos.
    \begin{itemize}
        \item Lista de Usuarios en Línea : El panel de control del administrador muestra una  lista de todos los usuarios que están actualmente en línea .
        \item Botones de Conmutación para Permisos : Junto al nombre de cada usuario en línea, el panel incluye  botones de conmutación (switch buttons) 
        para otorgar permisos de lectura y escritura .
        \item Botón  'Apply ' (Aplicar) : Hay un botón para aplicar los cambios realizados en los permisos de los usuarios.
    \end{itemize}

    \item 3. Implementación de la Lógica en el Servidor (Python/Flask)
    \item La gestión de usuarios y la preparación de los datos para el panel de control se manejan en el servidor Flask.
    \begin{itemize}
        \item Envío de Detalles Adicionales: Al retornar la página web principal al usuario, el servidor también envía detalles adicionales como el 
        userID de la sesión y una lista de `online-user-records`.
        \item `online-user-records` : Esta variable es un mapa (o lista de listas) que se rellena en una función como `get-all-logged-in-users`. 
        Cada entrada contiene:
        \begin{itemize}
            \item El  nombre del usuario  (índice 0).
            \item El  ID del usuario  (índice 1).
            \item El  estado de acceso de lectura  (1 para marcado, 0 para desmarcado) (índice 2).
            \item El  estado de acceso de escritura  (1 para marcado, 0 para desmarcado) (índice 3).
        \end{itemize}
        \item Conversión a 'checked' o  'unchecked ' : Los valores numéricos (1 o 0) para los permisos de lectura y escritura se convierten en las 
        cadenas 'checked' o  'unchecked ' respectivamente, ya que el código HTML las lee para establecer el estado de los botones de conmutación.
    \end{itemize}

    \item 4. Implementación en el Cliente (HTML/Jinja2) y Visibilidad Condicional
    \item La interfaz de usuario para la gestión de permisos se construye utilizando plantillas HTML y Jinja2, con una lógica para controlar quién puede verla.
    \begin{itemize}
        \item Bucle para Usuarios : La página 'index.html' utiliza un bucle `for` (plantilla Jinja2) para iterar sobre la lista `online-user-records` 
        y crear una fila de tabla (`<li>`) para cada usuario en línea.
        \item Display de Datos : Cada fila muestra el nombre del usuario y los botones de conmutación para lectura y escritura, cuyos estados 
        (checked/unchecked) se establecen dinámicamente según los datos recibidos del servidor.
        \item Panel de Control Solo para Administradores : El  panel de control completo para gestionar permisos solo es visible para los usuarios 
        administradores . Esto se logra mediante una declaración `if` en el código HTML que compara el `userID` del usuario actual con el `userID` 
        de un administrador (previamente hardcodeado o verificado). Un usuario no administrador, como  'Anam Chaudhary ' en el ejemplo, no verá este panel.
    \end{itemize}

    \item 5. Otorgamiento de Permisos en Tiempo Real
    \item El proceso para que un administrador otorgue o revoque permisos se realiza en tiempo real, involucrando el cliente y el servidor, y utilizando 
    un gestor de acceso.
    \begin{itemize}
        \item  Lado del Cliente (JavaScript en `main.js`) :
        \begin{itemize}
            \item Un método en `main.js` escucha los eventos de los botones de conmutación.
            \item Cuando se activa un botón, extrae el ID del usuario, el estado de lectura y el estado de escritura.
            \item Luego, envía una  solicitud POST  al servidor con el formato `grant-user ID-read state-write state`.
        \end{itemize}
        
        \item  Lado del Servidor (Aplicación Flask) :
        \begin{itemize}
            \item El servidor tiene un endpoint (`/grant`) para recibir estas solicitudes POST.
            \item Verificación de Administrador : Primero, el servidor  verifica si la solicitud proviene de un usuario administrador . Si no es así, 
            deniega el acceso.
            \item Almacenamiento en Base de Datos : Si la solicitud es válida, el servidor  almacena los permisos de lectura y escritura del usuario 
            en la base de datos.
            \item PubNub Access Manager : Finalmente, se realiza una llamada al  servidor PubNub para otorgar estos permisos al usuario específico .
        \end{itemize}

        \item  Generación de Clave de Autorización : Es importante destacar que el primer paso antes de conceder permisos de lectura y escritura 
        es  generar una clave de autorización para ese usuario específico y almacenarla en la base de datos .    
    \end{itemize}
\end{itemize}

En resumen, se  describen un sistema de seguridad y gestión de usuarios robusto que utiliza certificados SSL/TLS para asegurar la 
comunicación, y una arquitectura de roles (administrador/no administrador) para controlar el acceso y los permisos a dispositivos, 
implementada con Flask en el servidor, HTML/Jinja2 y JavaScript en el cliente, y la gestión de acceso en tiempo real a través de PubNub.

\section{MEJORAS DEL SERVIDOR IoT}
% 6.1 MEJORAS DEL SERVIDOR IoT
Se describe un conjunto de  mejoras significativas para el servidor IoT , centradas en la implementación de  reglas robustas para 
usuarios administradores y no administradores . Estas mejoras constituyen una parte esencial de la construcción de una plataforma IoT segura 
y escalable, permitiendo a los administradores gestionar los permisos de otros usuarios y dispositivos en tiempo real.

Aquí se dan los detalles de estas mejoras y la gestión de usuarios:
\begin{itemize}
    \item 1. Contexto General y Objetivos de las Mejoras
    \begin{itemize}
        \item La Sección 6 del proyecto se dedica a implementar una   'manera segura '  en la que usuarios y dispositivos IoT pueden conectarse 
        al servidor.
        \item El objetivo es  añadir funcionalidades mejoradas  al servidor IoT, comenzando por la creación de reglas para usuarios administradores 
        y no administradores.
        \item Se busca desarrollar un   'ecosistema IoT fuerte, seguro, en tiempo real y escalable ' .
    \end{itemize}

    \item 2. Dashboard de Administración y Gestión de Usuarios
    \begin{itemize}
        \item Se introduce una  nueva sección en el dashboard  (panel de control) del servidor donde se listan todos los  usuarios en línea .
        \item Al lado del nombre de cada usuario en línea, el dashboard presenta  botones para conceder permisos de lectura y escritura, junto con 
        un botón para  'aplicar los cambios '.
        \item El dashboard para usuarios administradores mostrará una lista de todos los usuarios en línea con estos botones de control para 
        conceder o denegar permisos.
    \end{itemize}

    \item 3. Implementación de Permisos de Lectura y Escritura
    \begin{itemize}
        \item Inicialmente, el código HTML puede  'hard-codear ' (codificar de forma rígida) la información del usuario, pero se implementa 
        una funcionalidad para  poblar dinámicamente esta lista  desde el servidor.
        \item El servidor envía detalles adicionales a la página web principal, como el 'user ID' y una  lista de usuarios en línea .
        \item Para cada usuario, el servidor devuelve un registro que incluye el nombre, el 'user ID', y el estado de los permisos de 
        lectura (`read`) y escritura (`write`).
        \item Estos estados de lectura y escritura se convierten a 'checked' o  'unchecked ' (marcado o desmarcado) para reflejarse 
        correctamente en los botones de conmutación (switch buttons) en la interfaz HTML.
        \item La página 'index.html' utiliza  plantillas Jinja  para iterar sobre la lista de usuarios en línea ('online users record') y 
        mostrar múltiples filas, cada una con el nombre del usuario, su ID, y los estados de los permisos de lectura y escritura.
    \end{itemize}

    \item 4. Restricción del Panel de Control para Usuarios Administradores
    \begin{itemize}
        \item Una mejora crucial es asegurar que el  panel de control completo (que permite cambiar permisos) sea visible  'únicamente para 
        los usuarios administradores' .
        \item Esto se logra añadiendo una  sentencia `if` en el código HTML  antes de que comience el panel de control, verificando si el 'user ID' 
        del usuario actual coincide con un 'user ID' de administrador codificado.
        \item Se demuestra que un usuario no administrador (por ejemplo,  'Anam Chaudhary ') no tiene acceso al panel de control, mientras que el 
        usuario administrador sí puede ver la lista completa de usuarios en línea y sus controles.
    \end{itemize}

    \item 5. Lógica del Servidor para la Concesión de Permisos
    \begin{itemize}
        \item Cuando el botón  'aplicar ' es presionado en el dashboard, una solicitud es enviada desde el código JavaScript del cliente a la 
        aplicación Flask en el servidor.
        \item El código JavaScript (`main.js`) detecta la pulsación de cualquier botón de conmutación cuyo ID comience con  'XS ' (acceso), 
        extrae el ID del usuario y el estado de los interruptores de lectura y escritura.
        \item Esta información se envía al servidor como una  solicitud `POST`  con el formato `grant-[user ID]-[read state]-[write state]`.
        \item En la aplicación Flask del servidor, se añade un  punto final (endpoint)  para recibir esta solicitud.
        \item Es  fundamental verificar que la solicitud provenga de un usuario administrador  antes de procesarla. Si no es así, el servidor 
        envía una respuesta de  'acceso denegado '.
        \item Si la solicitud es válida y proviene de un administrador, el servidor procede a:
        \begin{itemize}
            \item 1.   Almacenar los permisos  de lectura y escritura del usuario en la base de datos.
            \item 2.   Llamar al servidor PubNub  para otorgar acceso de lectura y escritura en tiempo real a ese usuario específico.        
        \end{itemize}
        \item Esta concesión de permisos es el segundo paso en un proceso que incluye primero la  generación de una clave de autorización  para 
        el usuario y su almacenamiento en la base de datos.
        \item El cliente recibe la respuesta del servidor, extrae el JSON y, si el  'acceso es concedido ', se vuelve a suscribir al canal, y se 
        modifica el método de suscripción para confirmar el éxito de la operación.
    \end{itemize}
\end{itemize}

En resumen, las mejoras del servidor IoT en el contexto de reglas para usuarios administradores y no administradores se centran en la creación de 
un  panel de control robusto y seguro . Este panel permite a los administradores  visualizar y gestionar los permisos de lectura y escritura  de 
otros usuarios y dispositivos en tiempo real, con una validación estricta en el lado del servidor para garantizar que solo los administradores 
puedan realizar cambios y que la comunicación sea segura de extremo a extremo.

\section{INTERFAZ DE USUARIO (index.html)}
% 6.2 INTERFAZ DE USUARIO (index.html)
La interfaz de usuario, específicamente el archivo  'index.html' , juega un papel fundamental en la implementación de las  mejoras del servidor IoT  
y en la aplicación de  reglas para usuarios administradores y no administradores . Este archivo es la base del  dashboard  que permite visualizar 
y gestionar los permisos de los usuarios en la plataforma IoT.

Aquí se dan los detalles de 'index.html' en este contexto:
\begin{itemize}
    \item 1. Diseño del Dashboard para la Gestión de Usuarios
    \begin{itemize}
        \item El 'index.html' se mejora para incluir una  nueva sección debajo de la existente  (que controla el movimiento y el zumbador).
        \item Esta nueva sección está dedicada a mostrar una  lista de todos los usuarios en línea .
        \item Para cada usuario en línea, el 'index.html' presenta:
        \begin{itemize}
            \item El  nombre del usuario .
            \item Dos botones tipo  'switch '  (conmutadores), uno para  permisos de lectura  y otro para  permisos de escritura . Estos botones tienen 
            IDs como `read user ID` y `write user ID`, donde 'user ID' se reemplazará por el ID real del usuario.
            \item Un  botón  'apply '  (aplicar) para guardar los cambios en los permisos. Este también tendrá un ID que incluye el 'user ID' del 
            usuario correspondiente, como `access user ID`.
        \end{itemize}
        \item Inicialmente, los botones de conmutación se establecen como 'checked' (activados) en el código HTML.
    \end{itemize}

    \item 2. Población Dinámica de la Lista de Usuarios
    \begin{itemize}
        \item Aunque inicialmente se puede  'hard-codear ' (codificar de forma rígida) un nombre de usuario de ejemplo en el HTML, el objetivo es  
        poblar dinámicamente esta lista  desde el servidor.
        \item El servidor envía detalles adicionales a la página web principal ('index.html'), incluyendo el 'user ID' del usuario actual y una  
        lista de registros de usuarios en línea  ('online users record').
        \item Cada registro de usuario incluye el nombre del usuario (índice 0), su 'user ID' (índice 1), y el estado de los permisos de lectura 
        (índice 2) y escritura (índice 3).
        \item El servidor convierte el estado numérico de los permisos de lectura y escritura (por ejemplo, 1 para permitido) a las cadenas 
        'checked' o  'unchecked ', que el HTML interpreta para configurar el estado de los botones de conmutación.
        \item El 'index.html' utiliza  plantillas Jinja  con un  bucle `for` (`for n in online users record`)  para iterar sobre la lista de 
        usuarios en línea y generar  múltiples filas , una para cada usuario. Cada fila muestra el nombre del usuario, su ID y los estados correctos 
        de los permisos de lectura y escritura.
    \end{itemize}

    \item 3. Restricción del Panel de Control para Usuarios Administradores
    \begin{itemize}
        \item Una mejora crucial es asegurar que  el panel de control completo (que contiene la lista de usuarios y los botones de permisos) 
        sea visible  'únicamente para los usuarios administradores ' .
        \item Esto se implementa añadiendo una  sentencia `if` directamente en el código HTML  del 'index.html' antes de que comience el panel 
        de control. Esta condición verifica si el 'user ID' del usuario actual coincide con un 'user ID' de administrador predefinido.
        \item De esta manera, un  usuario no administrador  (como  'Anam Chaudhary ') no tendrá acceso visual a este panel de control, mientras 
        que el administrador sí lo verá y podrá gestionar los permisos.
    \end{itemize}

    \item 4. Interacción del Usuario y Envío de Solicitudes
    \begin{itemize}
        \item Cuando un usuario administrador interactúa con los botones de conmutación o presiona el botón  'apply ' en el 'index.html', el 
        código  JavaScript (`main.js`)  del cliente detecta estos eventos.
        \item El JavaScript extrae el ID del usuario afectado y el estado actual (activado/desactivado) de los interruptores de lectura y 
        escritura.
        \item Esta información se envía al servidor como una  solicitud `POST`  en un formato específico: 
        `grant-[user ID]-[read state]-[write state]`.
        \item Esta interacción es fundamental para que los administradores puedan  otorgar acceso de lectura y escritura en tiempo real  a 
        usuarios específicos, lo que luego se procesa en el servidor y se comunica a servicios como PubNub.
    \end{itemize}
\end{itemize}

En síntesis, el archivo 'index.html' es el componente visual y de interacción clave del dashboard de administración, diseñado para ser dinámico, 
seguro y funcional. Permite a los administradores gestionar los permisos de otros usuarios de manera intuitiva, mientras que las reglas integradas 
en su estructura (mediante sentencias `if` y Jinja) garantizan que solo los usuarios autorizados puedan acceder y manipular estas funcionalidades 
críticas en el servidor IoT.

\subsection{SECCIóN PARA USUARIOS ONLINE}
% 6.2.1 SECCIóN PARA USUARIOS ONLINE.
En el contexto más amplio de la Interfaz de Usuario (index.html), se  detallan que la 'Sección para Usuarios Online' es una 
característica clave del panel de control del servidor IoT. Esta sección está diseñada para mostrar una lista de los usuarios actualmente 
conectados y permitir a los administradores gestionar sus permisos en tiempo real.
A continuación, se desglosa lo que se  dice sobre esta sección:
\begin{itemize}
    \item 1. Propósito y Ubicación en el Dashboard:
        \begin{itemize}
            \item Esta sección es una nueva adición al dashboard.
            \item Su objetivo es mostrar una lista de todos los usuarios online.
            \item Es parte de las características mejoradas que se están desarrollando para el servidor IoT, específicamente para crear reglas para 
            usuarios administradores y no administradores.
        \end{itemize}
    \item 2. Estructura y Contenido Visual (index.html):
        \begin{itemize}
            \item Se implementa en el archivo index.html, debajo de una sección existente (como la del detector de movimiento y el botón para 
            controlar el zumbador).
            \item Se utiliza un panel primario de Bootstrap con el encabezado 'online users' (usuarios online).
            \item Cada fila de la tabla de usuarios se define como un elemento de lista (<li>).
            \item A la izquierda de cada fila, se muestra el nombre del usuario.
            \item A la derecha, se incluyen dos botones tipo 'switch' y un botón de 'aplicar' (apply):
                \begin{itemize}
                    \item Un switch para otorgar permisos de lectura (read permissions).
                    \item Otro switch para otorgar permisos de escritura (write permissions).
                    \item Un botón 'apply' para aplicar los cambios.
                \end{itemize}
            \item Los switches tienen IDs como read user ID, write user ID, y el botón de aplicar tiene un ID como access user ID. Estos user 
            ID se reemplazan dinámicamente con los IDs de usuario reales proporcionados por el servidor.
            \item Inicialmente, ambos switches de lectura y escritura están configurados como 'checked' (activados).
        \end{itemize}
    \item 3. Población de Datos desde el Servidor:
        \begin{itemize}
            \item Se requiere una funcionalidad en el lado del servidor para proporcionar el nombre de usuario y el ID de usuario para 
            poblar correctamente esta lista.
            \item La función get all logged in users en el servidor devuelve un mapa o lista (online user records) que contiene detalles 
            de los usuarios online.
            \item Cada registro incluye el nombre de usuario (índice 0), el ID de usuario (índice 1), el estado de lectura (índice 2) y 
            el estado de escritura (índice 3).
            \item El servidor convierte el estado de acceso (1 o 0) a 'checked' o 'unchecked' para que el código HTML lo interprete y 
            establezca el estado de los botones switch.
            \item En index.html, se utiliza Jinja templates y un bucle for (for n in online users record) para iterar sobre la lista de 
            usuarios online y generar dinámicamente múltiples filas (<li>).
        \end{itemize}
    \item 4. Control de Visibilidad (Solo para Administradores):
        \begin{itemize}
            \item Se enfatiza que este 'panel de control' (incluyendo la sección de usuarios online) debe ser visible solo 
            para usuarios administradores.
            \item Esto se implementa añadiendo una sentencia if en el código HTML que verifica si el user ID del usuario actual 
            (obtenido de session user ID) coincide con el ID de usuario del administrador.
            \item Se realiza una prueba donde un usuario no administrador (Anam Chaudhary) no tiene acceso a este panel, mientras que el 
            administrador sí lo ve y puede observar a otros usuarios en la lista.
        \end{itemize}
\end{itemize}
En resumen, la Sección para Usuarios Online en index.html es una característica dinámica y controlada por el lado del servidor, 
diseñada para proporcionar a los administradores una vista en tiempo real de los usuarios conectados y la capacidad de gestionar 
sus permisos de lectura y escritura.

\subsection{LISTA DE USUARIOS ON-LINE}
% 6.2.2 LISTA DE USUARIOS ON-LINE.
En el contexto más amplio de la Interfaz de Usuario (index.html), se  describen la 'Lista de Usuarios Online' como una 
característica mejorada y crucial del panel de control del servidor IoT. Esta lista está diseñada para mostrar a los usuarios 
actualmente conectados y, específicamente para los administradores, permitirles gestionar los permisos de lectura y escritura de 
esos usuarios en tiempo real.
A continuación, se detalla lo que se  dice sobre la Lista de Usuarios Online:
\begin{itemize}
    \item 1. Propósito y Ubicación:
    \begin{itemize}
        \item Es una nueva sección añadida al dashboard del servidor IoT.
        \item Su objetivo es mostrar una lista de todos los usuarios online.
        \item Forma parte del desarrollo de características mejoradas para crear reglas para usuarios administradores y no administradores.
        \item Para los usuarios administradores, el dashboard mostrará una lista de todos los usuarios online con botones de interruptor para otorgar 
        permisos de lectura y escritura.
    \end{itemize}
    \item 2. Implementación en index.html:
    \begin{itemize}
        \item Se añade debajo de una sección existente (como la del detector de movimiento y el botón para controlar el zumbador) en el archivo index.html.
        \item Visualmente, utiliza un panel primario de Bootstrap con el encabezado 'online users'.
        \item Cada usuario online se representa como una fila en una tabla, definida como un elemento de lista (<li>) dentro de una clase de lista de Bootstrap.
        \item Contenido de cada fila:
            \begin{itemize}
                \item En el extremo izquierdo, se muestra el nombre del usuario.
                \item En el extremo derecho, hay dos botones tipo 'switch' y un botón de 'aplicar' (apply).
                \item Los switches son para otorgar permisos de lectura (read permissions) y permisos de escritura (write permissions).
                \item Los IDs de los switches (por ejemplo, read user ID, write user ID) y del botón 'apply' (access user ID) son dinámicamente 
                reemplazados por el ID de usuario real proporcionado por el servidor para ese usuario online particular.
                \item Inicialmente, ambos switches de lectura y escritura están configurados como 'checked' (activados).
            \end{itemize}
    \end{itemize}
    \item 3. Población Dinámica de Datos desde el Servidor:
    \begin{itemize}
        \item Para poblar correctamente esta lista, el servidor necesita proporcionar el nombre de usuario y el ID de usuario.
        \item Una función get all logged in users en el lado del servidor se encarga de esto, devolviendo un mapa o lista (online user records) 
        que contiene detalles de los usuarios online.
        \item Cada registro en esta lista incluye: el nombre de usuario (índice 0), el ID de usuario (índice 1), el estado de lectura (índice 2) 
        y el estado de escritura (índice 3).
        \item El servidor convierte el estado de acceso (1 o 0) en 'checked' o 'unchecked' para que el código HTML lo interprete y establezca 
        el estado inicial de los switches de lectura y escritura.
        \item En index.html, se utilizan Jinja templates y un bucle for (for n in online users record) para iterar sobre la lista de usuarios 
        online y generar dinámicamente las múltiples filas (<li>).
    \end{itemize}
    \item 4. Control de Visibilidad para Administradores:
    \begin{itemize}
        \item Es fundamental que este panel de control, incluyendo la lista de usuarios online, sea visible solo para usuarios administradores.
        \item Esto se implementa añadiendo una sentencia if en el código HTML antes de que comience el panel de control. Esta condición verifica si 
        el user ID del usuario actual (obtenido de session user ID) coincide con el ID de usuario del administrador (que puede ser codificado en el HTML).
        \item Se demuestra con una prueba: un usuario no administrador (Anam Chaudhary) no tiene acceso a este panel, mientras que un administrador 
        sí lo ve y puede observar a otros usuarios en la lista.
    \end{itemize}
    \item 5. Funcionalidad del Botón 'Aplicar':
    \begin{itemize}
        \item Se añade un endpoint para el botón 'apply' que enviará una solicitud desde el código JavaScript a la aplicación Flask.
        \item Un método en main.js escucha los cambios en los switches, extrae el ID de usuario, y lee el estado de los switches de lectura y 
        escritura. Luego, envía una solicitud de tipo POST al servidor con el formato grant - user ID - read state - write state.
        \item En la aplicación Flask, se añade un endpoint para recibir esta solicitud, se verifica si el remitente es un usuario administrador y, 
        si todo va bien, se almacena los permisos de lectura y escritura en la base de datos y se llama al servidor PubNub para conceder 
        acceso a ese usuario específico. La generación de la clave de autorización para el usuario específico y su almacenamiento en la 
        base de datos es un paso previo a la concesión de permisos.
    \end{itemize}
\end{itemize}
En conclusión, la 'Lista de Usuarios Online' en index.html es una característica interactiva y dinámica que permite a los administradores 
de un servidor IoT visualizar y modificar en tiempo real los permisos de lectura y escritura de los usuarios conectados, asegurando que 
la gestión de accesos sea eficiente y controlada.

\subsection{BOTONES TIPO \texttt{ 'SWITCH '} PARA PERMISOS DE LECTURA Y ESCRITURA}
% 6.2.3 BOTONES TIPO  'SWITCH ' PARA PERMISOS DE LECTURA Y ESCRITURA.
En el contexto más amplio de la Interfaz de Usuario (index.html), se  describen los 'Botones de Permisos de Lectura/Escritura 
por Usuario' como una característica interactiva y esencial dentro de la sección de usuarios online del panel de control del servidor IoT. 
Estos botones están diseñados para permitir a los usuarios administradores gestionar en tiempo real los niveles de acceso de otros 
usuarios conectados.
A continuación, se detalla lo que se  dice sobre estos botones:
\begin{itemize}
    \item 1. Propósito y Ubicación en la Interfaz:
    \begin{itemize}
        \item Estos botones son parte de una nueva sección en el dashboard del servidor IoT, específicamente en la 'Lista de Usuarios Online'.
        \item Su objetivo es permitir a los administradores otorgar permisos de lectura y escritura a cada usuario online.
        \item Para cada usuario listado, los botones se ubican a la derecha del nombre del usuario.
        \item Esta funcionalidad es una característica mejorada para crear reglas específicas para usuarios administradores y no administradores.
    \end{itemize}
    \item 2. Descripción y Estructura en index.html:
    \begin{itemize}
        \item Los botones son visualmente 'botones tipo switch' (interruptores).
        \item Se implementan en el archivo index.html dentro de un panel primario de Bootstrap.
        \item Para cada usuario online, se genera un elemento de lista (<li>) que contiene:
        \begin{itemize}
            \item Un switch para permisos de lectura (read permissions).
            \item Otro switch para permisos de escritura (write permissions).
            \item Un botón adicional de 'aplicar' (apply) para guardar los cambios.
        \end{itemize}
        \item Los switches y el botón de aplicar tienen IDs dinámicos (por ejemplo, read user ID, write user ID, access user ID), donde user ID 
        se reemplaza con el ID real del usuario proporcionado por el servidor.
        \item Inicialmente, ambos switches de lectura y escritura están configurados como 'checked' (activados).
    \end{itemize}
    \item 3. Población Dinámica y Estado Inicial:
    \begin{itemize}
        \item El servidor juega un papel crucial al proporcionar los datos para estos botones.
        \item Una función get all logged in users en el servidor devuelve una lista o mapa (online user records) que incluye el estado de 
        lectura (índice 2) y el estado de escritura (índice 3) para cada usuario.
        \item El servidor convierte internamente estos estados (1 o 0) en las cadenas 'checked' o 'unchecked' para que el código HTML 
        (usando Jinja templates) pueda interpretarlos y establecer el estado inicial de los switches correctamente.
        \item Mediante un bucle for (for n in online users record) en index.html, se generan dinámicamente las filas de usuarios con sus 
        respectivos botones y estados iniciales.
    \end{itemize}
    \item 4. Funcionalidad del 'Botón de Aplicar' y Lógica del Servidor:
    \begin{itemize}
        \item Existe un endpoint específico para el botón 'apply'.
        \item Cuando un switch es modificado y se presiona 'apply', una función en main.js escucha estos cambios, extrae el ID de usuario y 
        el estado de los switches de lectura y escritura.
        \item Luego, envía una solicitud POST al servidor con el formato grant - user ID - read state - write state.
        \item En la aplicación Flask, un endpoint recibe esta solicitud y realiza varias acciones:
            \item Verifica si la solicitud proviene de un usuario administrador.
            \item Almacena los permisos de lectura y escritura del usuario en la base de datos.
            \item Llama al servidor PubNub para conceder acceso a ese usuario específico.
            \item Antes de conceder los permisos, se debe generar una clave de autorización para el usuario específico y almacenarla en la base de datos.
    \end{itemize}
    \item 5. Control de Visibilidad (Solo para Administradores):
    \begin{itemize}
        \item Es fundamental que este panel de control y sus botones de permisos sean visibles solo para usuarios administradores.
        \item Esto se implementa con una sentencia if en index.html que verifica si el user ID del usuario actual (obtenido de session user ID) 
        coincide con el ID del administrador. Los usuarios no administradores no verán estos controles.
    \end{itemize}
\end{itemize}
En conclusión, los botones de permisos de lectura/escritura por usuario en index.html son una parte interactiva y programática del panel de 
control del servidor IoT. Su existencia permite a los administradores una gestión granular y en tiempo real de los accesos de los usuarios, lo 
cual es fundamental para la seguridad y el control de un ecosistema IoT escalable.

\subsection{BOTóN 'APPLY' PARA APLICAR CAMBIOS}
% 6.2.4 BOTóN  'APPLY ' PARA APLICAR CAMBIOS.
En el contexto de la Interfaz de Usuario, específicamente el archivo index.html, el botón 'Aplicar Cambios' es un elemento crucial 
dentro del panel de control para usuarios administradores.
Aquí se detalla lo que se dice sobre este botón:
\begin{itemize}
    \item Contexto en index.html y la Interfaz de Usuario:
    \begin{itemize}
        \item El panel de control, visible solo para usuarios administradores, incluye una sección para listar a los usuarios en línea.
        \item Para cada usuario en línea, se muestran dos botones de tipo interruptor: uno para otorgar permisos de lectura y otro para 
        permisos de escritura.
        \item Junto a estos interruptores, se encuentra un tercer botón, descrito como un 'botón adjunto' o 'botón táctil' con un ID como 
        access user ID, cuya función es aplicar los cambios realizados en los permisos de ese usuario específico.
        \item La estructura del index.html para esta sección utiliza plantillas Jinja para poblar dinámicamente la lista de usuarios, 
        mostrando el nombre del usuario (índice 0), el ID del usuario (índice 1), y los estados de lectura y escritura (índices 2 y 3, 
        respectivamente) que determinan si los interruptores están 'checked' o 'unchecked'.
    \end{itemize}
    \item Funcionalidad del Botón 'Aplicar Cambios':
    \begin{itemize}
        \item Cuando se presiona el botón, el código JavaScript en el lado del cliente envía una solicitud a la aplicación Flask en el servidor.
        \item Esta solicitud, que se realiza como una petición POST (similar a send event), incluye el ID del usuario, el estado de los permisos 
        de lectura y el estado de los permisos de escritura que se desean aplicar (por ejemplo, \verb|grant - user ID - read state - write stick|).
        \item En el lado del servidor, la aplicación Flask recibe esta solicitud a través de un endpoint específico. Es fundamental que la 
        aplicación verifique que la solicitud proviene de un usuario administrador antes de proceder.
        \item Si la verificación es exitosa, el servidor realiza dos acciones principales:
        \begin{itemize}
            \item 1. Almacena los nuevos permisos de lectura y escritura del usuario en la base de datos.
            \item 2. Llama al servidor PubNub para otorgar acceso de lectura y escritura a ese usuario específico en tiempo real.
        \end{itemize}
        \item Cabe mencionar que un paso previo a la concesión de estos permisos es la generación de una clave de autorización para el usuario 
        en cuestión, la cual se almacena en la base de datos.
        \item El panel permite a los administradores 'otorgar acceso de lectura y escritura en tiempo real a todos los usuarios que no son 
        administradores y a los dispositivos'. Al seleccionar el botón 'Aplicar', 'los permisos de acceso de lectura y escritura de ese 
        usuario específico se cambiarán en consecuencia en tiempo real'.
    \end{itemize}
\end{itemize}

\subsection{EXTRUCTURA DEL CóDIGO HTML (usando Bootstrap, divs)}
% 6.2.5 EXTRUCTURA DEL CóDIGO HTML (usando Bootstrap, divs).
En el contexto más amplio de la Interfaz de Usuario, el archivo index.html es fundamental para el panel de control del servidor IoT, 
especialmente para los usuarios administradores. Las fuentes detallan una estructura de código HTML que utiliza divs y componentes de 
Bootstrap para organizar y presentar la información y las funcionalidades.
Aquí se desglosa la estructura del código HTML, con énfasis en el uso de divs y Bootstrap:
\begin{itemize}
    \item 1. Estructura General y Diseño:
        \begin{itemize}
            \item El dashboard busca tener diferentes secciones, una de las cuales es para la lista de usuarios en línea.
            \item Se menciona la replicación de una 'sección de caja negra' existente, lo que sugiere un diseño modular donde diferentes 
            funcionalidades están encapsuladas en bloques.
            \item Se utilizan divs para organizar el contenido, incluyendo un div completo para una sección, un 'inner div' para definir 
            una fila, y otro div para centrar los elementos ('define everything as a central block').
        \end{itemize}
    \item 2. Uso de Bootstrap para el Panel de Usuarios en Línea:
        \begin{itemize}
            \item Para la sección que lista a los usuarios en línea, se crea un 'Bootstrap primary panel'. Esto indica que se están 
            utilizando las clases y estilos predefinidos de Bootstrap para dar una apariencia coherente y profesional al panel.
            \item Dentro de este panel, se añade un encabezado para identificar la sección como 'online users'.
        \end{itemize}
    \item 3. Visualización de Usuarios y Permisos:
        \begin{itemize}
            \item La lista de usuarios se presenta mediante una tabla cuya clase es list group, y cada fila de la tabla se define como 
            un elemento li (lista). Esto sugiere el uso de componentes de lista de Bootstrap, que estilizan elementos ul y li para crear 
            listas de elementos conectados.
            \item Cada fila (li) está diseñada para mostrar:
                \begin{itemize}
                    \item El nombre del usuario en el extremo izquierdo.
                    \item Dos botones de tipo interruptor ('switch buttons') en el extremo derecho para otorgar permisos de lectura (read user ID) 
                    y escritura (write user ID). Estos botones se inicializan como 'checked' (activos).
                    \item Un tercer botón adyacente (conocido como 'touch button' o 'botón táctil') con un ID como access user ID, cuya función 
                    es aplicar los cambios de permisos para ese usuario específico.
                \end{itemize}
        \end{itemize}
    \item 4. Población Dinámica con Plantillas Jinja:
        \begin{itemize}
            \item El index.html utiliza plantillas Jinja para poblar dinámicamente la lista de usuarios en línea.
            \item Esto se logra mediante un bucle for (for n in online users record) que itera sobre una lista de registros de usuarios 
            recibida del servidor.
            \item Dentro de cada iteración del bucle, se accede a los datos de cada usuario:
                \begin{itemize}
                    \item El nombre del usuario se encuentra en el índice 0 de la lista.
                    \item El ID del usuario se encuentra en el índice 1.
                    \item Los estados de lectura y escritura (que determinan si los interruptores están 'checked' o 'unchecked') 
                    se encuentran en los índices 2 y 3, respectivamente.
                \end{itemize}
        \end{itemize}
    \item 5. Visibilidad Condicional para Administradores:
        \begin{itemize}
            \item Para asegurar que el panel de control completo (que incluye la lista de usuarios en línea y la funcionalidad del 
            botón 'Aplicar Cambios') sea visible solo para los usuarios administradores, se implementa una sentencia if en el código HTML. 
            Esta condición verifica si el user ID del usuario actual coincide con el ID de un usuario administrador.
        \end{itemize}
\end{itemize}
En resumen, el index.html se estructura como un dashboard interactivo y dinámico para administradores, utilizando una combinación de 
divs para la organización modular y Bootstrap para el estilado de paneles y listas. La integración de Jinja permite la presentación de datos 
en tiempo real de los usuarios en línea y sus permisos, mientras que la lógica condicional en el HTML controla la visibilidad de estos elementos 
solo para usuarios autorizados.

\subsection{ELEMENTOS DE LA FILA DE USUARIO}
% 6.2.6 ELEMENTOS DE LA FILA DE USUARIO.
En el contexto de la Interfaz de Usuario (UI), específicamente dentro del archivo index.html y como parte del panel de control 
para administradores, los elementos de la fila de usuario son componentes diseñados para mostrar la información de los usuarios en 
línea y permitir a los administradores gestionar sus permisos en tiempo real.
Aquí se detallan los elementos que conforman cada fila de usuario:
\begin{itemize}
    \item 1. Estructura General de la Fila:
        \begin{itemize}
            \item Las filas de usuario forman parte de un 'Bootstrap primary panel' con el encabezado 'online users'.
            \item Cada fila en la tabla se define como un elemento li (lista), utilizando la clase list group de Bootstrap.
            \item La población de estas filas es dinámica, utilizando plantillas Jinja con un bucle for n in online users record, 
            donde n representa cada registro de usuario en línea.
        \end{itemize}
    \item 2. Nombre del Usuario:
        \begin{itemize}
            \item En el extremo izquierdo de la fila, se muestra el nombre del usuario.
            \item Este dato se obtiene del índice 0 de la lista n que representa al usuario actual en el bucle Jinja.
        \end{itemize}
    \item 3. Botones de Tipo Interruptor para Permisos:
        \begin{itemize}
            \item En el extremo derecho de la fila, se encuentran dos botones de tipo interruptor ('switch buttons').
            \item Uno es para otorgar permisos de lectura, con un ID como read user ID.
            \item El otro es para otorgar permisos de escritura, con un ID como write user ID.
            \item Ambos interruptores se configuran inicialmente como 'checked' (activos).
            \item El estado ('checked' o 'unchecked') de estos interruptores se determina a partir del índice 2 para lectura y el índice 3 
            para escritura del registro de usuario (n).
        \end{itemize}
    \item 4. Botón 'Aplicar Cambios':
        \begin{itemize}
            \item Junto a los interruptores, hay un tercer botón, descrito como un 'touch button' (botón táctil) con un ID como access user ID.
            \item La función principal de este botón es aplicar los cambios realizados en los permisos de lectura y escritura para ese usuario específico.
            \item Cuando se presiona, el código JavaScript envía una solicitud POST a la aplicación Flask en el servidor. Esta solicitud incluye 
            el ID del usuario, el estado de los permisos de lectura y el estado de los permisos de escritura.
            \item El ID del usuario para estos botones se extrae del índice 1 del registro de usuario (n) y reemplaza el placeholder user ID en los 
            IDs de los botones.
            \item Una vez que el servidor procesa la solicitud, almacena los nuevos permisos en la base de datos y llama al servidor PubNub para 
            otorgar o revocar el acceso en tiempo real. Este proceso también implica la generación de una clave de autorización para el usuario.
        \end{itemize}
    \item 5. Visibilidad Condicional:
        \begin{itemize}
            \item Es importante destacar que todo el panel de control, que incluye estas filas de usuario y la funcionalidad asociada, es 
            visible únicamente para los usuarios administradores. Esta restricción se implementa mediante una declaración if en el código HTML 
            que verifica si el ID del usuario actual coincide con el ID de un administrador.
        \end{itemize}
\end{itemize}
En resumen, cada fila de usuario en el index.html del panel de control representa una unidad interactiva que permite a los administradores 
visualizar la información de los usuarios en línea y gestionar sus permisos de lectura y escritura a través de interruptores y un botón de 
aplicación de cambios, todo ello con una estructura dinámica impulsada por Jinja y estilizada con Bootstrap.

\subsubsection{NOMBRE DE USUARIO}
% 6.2.6.1 NOMBRE DE USUARIO
En el contexto más amplio de los Elementos de la Fila de Usuario dentro de la Interfaz de Usuario (UI) en el index.html, el nombre del 
usuario es un componente clave para identificar a cada usuario en línea en el panel de control para administradores.
Aquí se detalla lo que las fuentes dicen sobre el nombre de usuario:
\begin{itemize}
    \item 1. Visibilidad y Contexto:
        \begin{itemize}
            \item El nombre del usuario es parte de la lista de usuarios en línea que se muestra en un 'Bootstrap primary panel' dentro 
            del panel de control.
            \item Este panel completo es visible exclusivamente para usuarios administradores. Una sentencia if en el código HTML asegura 
            que solo los administradores (aquellos cuyo ID de usuario coincide con el de un administrador) puedan ver esta sección.
        \end{itemize}
    \item 2. Posición en la Fila de Usuario:
        \begin{itemize}
            \item Para cada fila que representa un usuario en línea, el nombre del usuario se muestra en el extremo izquierdo.
            \item Las filas están estructuradas como elementos li de una tabla con la clase list group.        
        \end{itemize}
    \item 3. Población Dinámica con Jinja:
        \begin{itemize}
            \item Inicialmente, el nombre del usuario se podría 'hardcodear' (codificar directamente) en el HTML, por ejemplo, con el nombre 
            del desarrollador.
            \item Sin embargo, la implementación final utiliza plantillas Jinja para poblar dinámicamente esta lista.
            \item El servidor Flask devuelve una lista de online user records (registros de usuarios en línea) a la página index.html.
            \item Dentro de un bucle for n in online users record de Jinja, cada n representa un usuario. El nombre del usuario se encuentra 
            en el índice 0 de esta lista n.        
        \end{itemize}
    \item 4. Generación de Datos en el Servidor:
        \begin{itemize}
            \item En el lado del servidor, la función get all logged in users (obtener todos los usuarios conectados) es responsable de crear 
            la lista online user records.
            \item Esta lista es un mapa donde el nombre del usuario se agrega en el índice 0. También incluye el ID del usuario (índice 1), 
            el estado de lectura (índice 2) y el estado de escritura (índice 3).
        \end{itemize}
\end{itemize}
En resumen, el nombre de usuario es el identificador visual principal para cada entrada en la lista de usuarios en línea del panel de 
control de administrador. Su presentación se realiza de manera dinámica a través de plantillas Jinja, asegurando que la información 
sea actualizada y que solo los usuarios con los permisos adecuados puedan acceder a ella.

\subsubsection{BOTóN DE LECTURA}
% 6.2.6.2 BOTóN DE LECTURA (ID:read_user ID)
En el contexto más amplio de los Elementos de la Fila de Usuario dentro de la Interfaz de Usuario (UI) del index.html, el Botón de 
Lectura (\verb|ID: read user ID|) es un componente esencial que permite a los usuarios administradores controlar los permisos de acceso de 
los usuarios en línea.
Aquí se detalla lo que se dice sobre este botón:
\begin{itemize}
    \item 1. Ubicación y Apariencia en la Interfaz de Usuario:
        \begin{itemize}
            \item El botón de lectura es uno de los dos botones de tipo interruptor ('switch buttons') que se encuentran en el extremo 
            derecho de cada fila de usuario en la lista de usuarios en línea.
            \item El otro botón interruptor es para los permisos de escritura.
            \item Inicialmente, ambos interruptores (lectura y escritura) se configuran como 'checked' (activos).
            \item Estas filas de usuario, que contienen el botón de lectura, son parte de un panel de control más amplio visible exclusivamente 
            para usuarios administradores. Una sentencia if en el código HTML asegura que solo los administradores puedan ver esta sección.        
        \end{itemize}
    \item 2. Identificación Dinámica (ID: read user ID):
        \begin{itemize}
            \item El ID del botón de lectura es read user ID.
            \item La parte user ID en este ID se reemplaza dinámicamente por el ID real del usuario en línea al que corresponde la fila. Este ID 
            de usuario es proporcionado por el servidor.        
        \end{itemize}
    \item 3. Estado y Población Dinámica:
        \begin{itemize}
            \item El estado inicial del interruptor (si está 'checked' o 'unchecked') se determina en el lado del servidor y se envía a la plantilla 
            Jinja en index.html.
            \item En el servidor, si el read access es 1, la variable de lectura se establece como 'checked'; de lo contrario, se establece como 
            'unchecked'.
            \item En el lado del cliente, las plantillas Jinja leen este estado para configurar el atributo 'checked' del botón interruptor.
            \item Los datos de los usuarios en línea se proporcionan a la página index.html en una lista, y el estado de lectura de cada usuario 
            se encuentra en el índice 2 de su respectivo registro.        
        \end{itemize}
    \item 4. Funcionalidad para Control de Permisos:
        \begin{itemize}
            \item La función principal del botón de lectura es permitir a los administradores otorgar o revocar permisos de lectura a usuarios 
            o dispositivos específicos en tiempo real.
            \item Cuando se presiona el botón 'Aplicar Cambios' (adyacente a los interruptores) para un usuario específico, el código JavaScript 
            lee el estado actual del botón de lectura (y escritura).
            \item Esta información se envía como parte de una solicitud POST a la aplicación Flask en el servidor, incluyendo el estado de 
            lectura (y escritura) junto con el ID del usuario.
            \item En el servidor, antes de aplicar los cambios, se verifica que la solicitud provenga de un usuario administrador.
            \item Si la verificación es exitosa, la aplicación Flask realiza dos acciones clave:
                \begin{itemize}
                    \item 1. Almacena los nuevos permisos de lectura (y escritura) en la base de datos.
                    \item 2. Llama al servidor PubNub para otorgar o revocar el acceso de lectura (y escritura) a ese usuario específico en 
                    tiempo real.                
                \end{itemize}
            \item Cabe destacar que la concesión de estos permisos es un segundo paso, ya que el primer paso es generar y almacenar una 
            clave de autorización para el usuario en la base de datos.        
        \end{itemize}
\end{itemize}
En síntesis, el Botón de Lectura es un elemento interactivo crucial en el panel de control de administradores, que forma parte de una fila 
de usuario dinámicamente generada. Permite a los administradores visualizar y modificar los permisos de lectura de los usuarios en línea de 
manera controlada y en tiempo real, utilizando una combinación de plantillas Jinja, JavaScript y lógica de servidor para 
gestionar y aplicar estos accesos.

\subsubsection{BOTóN DE ESCRITURA}
% 6.2.6.3 BOTóN DE ESCRITURA
En el contexto más amplio de los Elementos de la Fila de Usuario dentro de la Interfaz de Usuario (UI) en el index.html, el Botón 
de Escritura (\verb|ID: write user ID|) es un componente interactivo clave que permite a los usuarios administradores controlar los 
permisos de escritura de los usuarios en línea.
Aquí se detalla lo que se dice sobre este botón:
\begin{itemize}
    \item 1. Ubicación y Apariencia en la Interfaz de Usuario:
        \begin{itemize}
            \item El botón de escritura es uno de los dos botones de tipo interruptor ('switch buttons') que se encuentran en el extremo 
            derecho de cada fila de usuario en la lista de usuarios en línea.
            \item El otro interruptor es para los permisos de lectura.
            \item Inicialmente, ambos interruptores (lectura y escritura) se configuran como 'checked' (activos).
            \item Las filas de usuario, que incluyen este botón, forman parte de un panel de control más amplio que es visible exclusivamente 
            para usuarios administradores. Una sentencia if en el código HTML asegura esta visibilidad restringida.        
        \end{itemize}
    \item 2. Identificación Dinámica (ID: write user ID):
        \begin{itemize}
            \item El ID de este botón es write user ID.
            \item La parte user ID en este ID se reemplaza dinámicamente con el ID real del usuario en línea al que corresponde la fila, ya 
            que este ID es proporcionado por el servidor.        
        \end{itemize}
    \item 3. Estado y Población Dinámica:
        \begin{itemize}
            \item Aunque inicialmente se configuran como 'checked', el estado real del interruptor (si está 'checked' o 'unchecked') se 
            determina en el lado del servidor y se envía a la plantilla Jinja en index.html.
            \item En el servidor, si el write access es 1, la variable de escritura se establece como 'checked'; de lo contrario, se establece 
            como 'unchecked'.
            \item En el lado del cliente, las plantillas Jinja leen este estado para configurar el atributo 'checked' del botón interruptor.
            \item Los datos de los usuarios en línea se proporcionan a la página index.html en una lista de online user records, y el estado 
            de escritura de cada usuario se encuentra en el índice 3 de su respectivo registro (n).        
        \end{itemize}
    \item 4. Funcionalidad para Control de Permisos:
        \begin{itemize}
            \item La función principal del botón de escritura es permitir a los administradores otorgar o revocar permisos de escritura a 
            usuarios o dispositivos específicos en tiempo real.
            \item Cuando se presiona el botón 'Aplicar Cambios' (adyacente a los interruptores) para un usuario específico, el código 
            JavaScript lee el estado actual de los botones de lectura y escritura.
            \item Esta información se envía como parte de una solicitud POST a la aplicación Flask en el servidor, incluyendo el ID del 
            usuario, el estado de lectura y el estado de escritura (referido como right stick).
            \item En el servidor, antes de aplicar los cambios, la aplicación Flask verifica que la solicitud provenga de un usuario administrador.
            \item Si la verificación es exitosa, la aplicación Flask realiza dos acciones clave:
                \begin{itemize}
                    \item 1. Almacena los nuevos permisos de escritura (y lectura) en la base de datos.
                    \item 2. Llama al servidor PubNub para otorgar o revocar el acceso de escritura (y lectura) a ese usuario específico en 
                    tiempo real.
                \end{itemize}
            \item Es importante mencionar que la concesión de estos permisos de lectura y escritura es un segundo paso, ya que el primer 
            paso es generar la clave de autorización para el usuario específico y almacenarla en la base de datos.        
        \end{itemize}
\end{itemize}
En síntesis, el Botón de Escritura es un elemento interactivo fundamental en el panel de control de administradores, formando parte de 
una fila de usuario generada dinámicamente. Permite a los administradores visualizar y modificar los permisos de escritura de los usuarios 
en línea de manera controlada y en tiempo real, integrando plantillas Jinja, JavaScript y lógica de servidor para gestionar y aplicar estos 
accesos de forma segura.

\subsubsection{BOTóN APLICAR}
% 6.2.6.4 BOTóN APLICAR
En el contexto más amplio de los Elementos de la Fila de Usuario dentro de la Interfaz de Usuario (UI) en el index.html, el botón 'Aplicar' 
(\verb|ID: access user ID|) es un componente crucial que permite a los usuarios administradores guardar y aplicar los cambios de permisos 
para usuarios específicos en tiempo real.
Aquí se detalla lo que las fuentes dicen sobre este botón:
\begin{itemize}
    \item 1. Ubicación y Apariencia en la Interfaz de Usuario:
        \begin{itemize}
            \item Este botón es descrito como un 'botón táctil' ('touch button').
            \item Se encuentra en el extremo derecho de cada fila de usuario, adyacente a los dos botones de tipo interruptor para 
            permisos de lectura y escritura.
            \item Las filas de usuario, que incluyen este botón, son parte de un panel de control más amplio que es visible exclusivamente 
            para usuarios administradores. La visibilidad se controla mediante una sentencia if en el código HTML que compara el ID del 
            usuario actual con el ID de un administrador.        
        \end{itemize}
    \item 2. Identificación Dinámica (ID: access user ID):
        \begin{itemize}
            \item El ID del botón es access user ID.
            \item La parte user ID en este ID se reemplaza dinámicamente con el ID real del usuario en línea al que corresponde la fila. 
            Este ID es proporcionado por el servidor a la plantilla index.html.        
        \end{itemize}
    \item 3. Funcionalidad principal: Aplicar Cambios de Permisos:
        \begin{itemize}
            \item La función principal de este botón es aplicar los cambios realizados en los permisos de lectura y escritura para ese 
            usuario específico.
            \item Cuando se presiona el botón 'Aplicar' (o 'Apply'), los permisos de acceso de lectura y escritura de ese usuario en 
            particular se modifican en consecuencia en tiempo real.        
        \end{itemize}
    \item 4. Flujo de Interacción (Cliente-Servidor):
        \begin{itemize}
            \item En el lado del cliente (JavaScript):
                \begin{itemize}
                    \item Un método en main.js 'escucha' cualquier botón que comience con access en su ID (lo que implica que se refiere 
                    a este botón 'Aplicar').
                    \item Cuando se presiona, el JavaScript divide el ID del botón para extraer el ID del usuario.
                    \item Luego, lee el estado actual de los interruptores de lectura y escritura asociados a ese usuario.
                    \item Finalmente, envía una solicitud POST a la aplicación Flask en el servidor. Esta solicitud incluye el formato 
                    \verb|grant - user ID - read state - right stick| (donde right stick se refiere al estado de escritura).                
                \end{itemize}
            \item En el lado del servidor (Aplicación Flask):
                \begin{itemize}
                    \item La aplicación Flask tiene un endpoint configurado para recibir esta solicitud de grant user ID read and write.
                    \item Es crucial que el servidor verifique que la solicitud proviene de un usuario administrador antes de procesarla. 
                    Si no es así, se envía una respuesta de 'acceso denegado'.
                    \item Si la verificación es exitosa, se realizan dos acciones clave:
                        \begin{itemize}
                            \item 1. Almacenar los nuevos permisos de lectura y escritura del usuario en la base de datos.
                            \item 2. Llamar al servidor PubNub para otorgar o revocar el acceso de lectura y escritura a ese usuario 
                            específico en tiempo real.
                        \end{itemize}
                    \item Un paso previo a la concesión de estos permisos es la generación de una clave de autorización para el usuario 
                    específico y su almacenamiento en la base de datos.                        
                \end{itemize}
        \end{itemize}
\end{itemize}
En resumen, el botón 'Aplicar' en cada fila de usuario es el punto de interacción clave para los administradores en el index.html del panel 
de control. Su función es traducir la selección de permisos en la interfaz de usuario en acciones concretas en el servidor y en la plataforma 
de comunicación en tiempo real (PubNub), asegurando una gestión dinámica y segura de los accesos para los usuarios en línea.

\section{FUNCIONALIDAD DEL LADO DEL SERVIDOR (Python Flask)}
% 6.3 FUNCIONALIDAD DEL LADO DEL SERVIDOR (Python Flask).
La funcionalidad del lado del servidor, implementada con Python Flask, es el núcleo de la gestión de las reglas para usuarios administradores y no 
administradores  en la plataforma IoT. El servidor Flask no solo procesa las solicitudes y proporciona la interfaz, sino que también es responsable de la 
seguridad, la persistencia de los datos y la comunicación en tiempo real para aplicar estas reglas.

A continuación, se dan los detalles de la funcionalidad del lado del servidor con Python Flask en este contexto:
\begin{itemize}
    \item 1. Base de Seguridad del Servidor
    \begin{itemize}
        \item Antes de implementar las reglas de usuario, el servidor IoT se asegura utilizando \textbf{certificados SSL/TLS de Let's Encrypt} para su 
        dominio personalizado. Esto garantiza que toda la comunicación con el servidor sea \textbf{cifrada de extremo a extremo}, lo que es fundamental para 
        una gestión segura de permisos de usuario. El servidor redirige el tráfico HTTP a HTTPS.
        \item La instalación del software 'certbot' y la configuración de las reglas de seguridad de entrada (como el puerto HTTPS 443 en AWS EC2) son pasos 
        previos cruciales en el lado del servidor para establecer un entorno seguro.
    \end{itemize}

    \item 2. Suministro Dinámico de Datos a la Interfaz de Usuario ('index.html')
    \begin{itemize}
        \item El servidor Flask es responsable de \textbf{enviar detalles adicionales a la página web principal ('index.html')}, lo que permite poblar 
        dinámicamente el dashboard con la información de los usuarios en línea.
        \item Esto incluye el 'user ID' del usuario actual (obtenido de la sesión) y una lista de 'registros de usuarios en línea' ('online users record').
        \item Para cada usuario en línea, el servidor prepara un registro que contiene su nombre, 'user ID', y el \textbf{estado de sus permisos de 
        lectura y escritura}.
        \item Es crucial que el servidor convierta los estados numéricos de los permisos (por ejemplo, 1 para permitido) a las cadenas 'checked' o 
         'unchecked ', que el código HTML ('index.html') interpreta para configurar correctamente el estado de los botones de conmutación (switch buttons) 
        en la interfaz. Esta lógica de conversión se realiza en el servidor antes de renderizar la plantilla Jinja.
    \end{itemize}

    \item 3. Identificación del Usuario Administrador
    \begin{itemize}
        \item El servidor Flask gestiona la sesión del usuario y proporciona el 'user ID' del usuario actualmente conectado al cliente.
        \item Aunque la verificación de la visibilidad del panel de control se realiza en el HTML con una sentencia `if` que compara el 'user ID' 
        actual con un 'user ID' de administrador codificado, la información del 'user ID' proviene del servidor.
    \end{itemize}

    \item 4. Manejo de Solicitudes de Concesión de Permisos
    \begin{itemize}
        \item El servidor Flask implementa un  punto final (endpoint) específico  para recibir las solicitudes del cliente (enviadas por JavaScript) 
        cuando un administrador intenta cambiar los permisos de otro usuario. Estas solicitudes tienen un formato como 
        `grant-[user ID]-[read state]-[write state]` y se envían como peticiones `POST`.
        \item Una de las funcionalidades más críticas del lado del servidor es la  validación de que la solicitud para conceder o denegar permisos 
        provenga de un usuario administrador . Si la solicitud no es de un administrador, el servidor envía una respuesta de  'acceso denegado '.
        \item Si la solicitud es válida y proviene de un administrador, el servidor realiza dos acciones principales:
        \begin{itemize}
            \item 1.   Almacena los permisos  de lectura y escritura del usuario afectado  en la base de datos .
            \item 2.   Llama al servidor PubNub  para  otorgar acceso de lectura y escritura en tiempo real  a ese usuario específico. Esto se logra 
            suscribiendo al usuario al canal correspondiente con los permisos adecuados.
        \end{itemize}
        \item La concesión de permisos es el  segundo paso  en un proceso que requiere, como primer paso, la  generación de una clave de autorización  
        para el usuario y su almacenamiento en la base de datos.
    \end{itemize}

    \item 5. Comunicación y Actualización en Tiempo Real
    \begin{itemize}
        \item El servidor utiliza  PubNub  como protocolo de comunicación principal para la funcionalidad de acceso de los usuarios administradores, 
        lo que permite el otorgamiento de permisos de lectura y escritura en tiempo real a usuarios no administradores y dispositivos.
        \item Después de que el servidor procesa una solicitud de concesión de acceso y notifica a PubNub, el cliente (navegador del usuario) 
        se vuelve a suscribir al canal para reflejar los nuevos permisos.
    \end{itemize}
\end{itemize}

En resumen, la funcionalidad del lado del servidor con Python Flask es el cerebro detrás de la gestión de usuarios. Se encarga de la seguridad 
fundamental, la entrega de datos dinámicos a la interfaz de usuario, la validación de roles de administrador para acciones críticas y la 
ejecución de cambios de permisos que se persisten en la base de datos y se aplican en tiempo real a través de servicios como PubNub.

\subsection{Población de la Lista de Usuarios Online}
% 6.3.1 Pobación de la Lista de Usuarios Online
En el contexto más amplio de la funcionalidad del lado del servidor utilizando Python Flask, la \textbf{población de la lista de usuarios en línea} es 
una característica clave para permitir a los administradores gestionar los permisos de los usuarios en tiempo real en una plataforma IoT.

Aquí se detalla cómo se gestiona esta funcionalidad desde el lado del servidor:
\begin{itemize}
    \item Propósito de la Lista de Usuarios en Línea
        \begin{itemize}
            \item El objetivo es desarrollar un servidor IoT que permita a los usuarios administradores ver una lista de todos los usuarios en 
            línea y concederles permisos de lectura y escritura a usuarios no administradores y dispositivos en tiempo real.
            \item Esta lista forma parte de un panel de control visible solo para usuarios administradores.
        \end{itemize}

    \item Implementación en el Lado del Servidor (Python Flask)
        \begin{itemize}
            \item Recopilación de Datos: Cuando la aplicación Flask devuelve la página web principal (\texttt{'index.html'}) al usuario, debe incluir 
            detalles adicionales como el ID de sesión del usuario (\texttt{`session-user-ID`}) y la lista de usuarios en línea.
            \item Función \texttt{`get-all-logged-in-users`}: Esta función (presumiblemente definida en un módulo \texttt{`my-DB`}) es la encargada de 
            preparar los datos de los usuarios.
                \begin{itemize}
                    \item Crea una variable \texttt{`online-user-records`}, que es un mapa inicialmente con la clave \texttt{`user-record`} y un valor de 
                    lista vacía.
                    \item Dentro de un bucle \texttt{`for`}, esta variable se \textbf{rellena} añadiendo los detalles de cada usuario. Por cada usuario, se 
                    adjunta una lista que contiene:
                        \begin{itemize}
                            \item El nombre del usuario (en el índice 0).
                            \item El ID del usuario (en el índice 1).
                            \item El estado de acceso de lectura (en el índice 2).
                            \item El estado de acceso de escritura (en el índice 3).                        
                        \end{itemize}
                    \item Los estados de acceso de lectura y escritura (que son 1 o 0) se convierten a las cadenas \texttt{'checked'} o \texttt{'unchecked'} 
                    respectivamente. Esto es crucial porque el código HTML interpretará estas cadenas para establecer el estado de los botones de alternancia 
                    (\texttt{`switch buttons`}).
                    \item Finalmente, la función \texttt{`get-all-logged-in-users`} devuelve este mapa, utilizando la clave \texttt{`online-users`}.
                \end{itemize} 
        \end{itemize}

    \item Visualización en el Lado del Cliente (HTML con Jinja Templates)
        \begin{itemize}
            \item La página \texttt{'index.html'} utiliza \textbf{Jinja templates} para procesar los datos recibidos del servidor.
            \item Se implementa un bucle \texttt{`for`} \texttt{(`for n in online-users-record`}) para iterar a través de la lista de \texttt{`online-users`}, 
            creando una fila (\texttt{'<li>'}) para cada usuario en línea.
            \item Dentro de cada fila, se muestra el nombre del usuario (\texttt{`n`}), y se utilizan \texttt{`n`} para el ID de usuario en los botones, y 
            \texttt{`n`} y \texttt{`n`} para establecer el estado de los botones de alternancia de permisos de lectura y escritura.
            \item Para asegurar que solo los administradores vean el panel de control de usuarios, se añade una declaración \texttt{`if`} en el código 
            HTML. Esta declaración compara el \texttt{`user-ID`} del usuario actual (obtenido del \texttt{`session-user-ID`} enviado por el servidor) con un 
            \texttt{`user-ID`} de administrador codificado. Un usuario no administrador no tendrá acceso al panel de control de acceso, mientras que un 
            administrador verá a otros usuarios en la lista en línea.
        \end{itemize}
\end{itemize}

Este proceso permite que la aplicación Flask gestione y presente dinámicamente una lista interactiva de usuarios en línea, que es esencial para la 
administración de permisos en tiempo real dentro de la plataforma IoT.

\subsection{Retorna \texttt{user-id} y \texttt{online-users-record} a la Página web}
% 6.3.2 Retorna  'user-id ' y  'online-users-record ' a la Página web.
En el contexto de la funcionalidad del lado del servidor con Python Flask para una plataforma IoT, la acción de \textbf{retornar el \texttt{`user-ID`} 
y la lista \texttt{`online-user-records`} a la página web} es fundamental para habilitar el panel de control de administración y la gestión de permisos 
en tiempo real.

Aquí se detalla cómo se gestiona esta transferencia de datos desde el lado del servidor a la página web:
\begin{itemize}
    \item Retorno del \texttt{`user-ID`} (identificador del usuario):
        \begin{itemize}
            \item Cuando la aplicación Flask devuelve la página web principal (\texttt{'index.html'}) al usuario, incluye detalles adicionales como el 
            \texttt{`user-ID`} del usuario actual.
            \item Este \texttt{`user-ID`} se envía como \texttt{`session-user-ID`}.
            \item En el lado del cliente (HTML), este \texttt{`session-user-ID`} se utiliza en una declaración \texttt{`if`} para determinar si el usuario 
            actual es un administrador. Si el \texttt{`user-ID`} coincide con un \texttt{`user-ID`} de administrador codificado, se le concede acceso al 
            panel de control de acceso. Los usuarios no administradores no verán este panel.
        \end{itemize}

    \item Retorno de \texttt{`online-user-records`} (registros de usuarios en línea):
        \begin{itemize}
            \item Para proporcionar la lista de usuarios en línea, se invoca una función (presumiblemente `get-all-logged-in-users` en un módulo `my-DB`) 
            en el lado del servidor.
            \item Esta función crea una variable `online-user-records`, que es un mapa inicial con la clave `user-record` y una lista vacía como valor.
            \item Luego, dentro de un bucle `for`, esta variable se  rellena  con los detalles de cada usuario en línea. Por cada usuario, se adjunta una 
            lista que contiene:
                \begin{itemize}
                    \item El nombre del usuario (en el índice 0).
                    \item El ID del usuario (en el índice 1).
                    \item El estado de acceso de lectura (en el índice 2).
                    \item El estado de acceso de escritura (en el índice 3).                
                \end{itemize}
            \item Es crucial que los estados de acceso de lectura y escritura (que son 1 o 0) se conviertan a las cadenas `\texttt{'checked'}` o 
            `\texttt{'unchecked'}` respectivamente. Esto se debe a que el código HTML interpretará estas cadenas para establecer el estado de los botones 
            de alternancia (\texttt{`switch buttons`}) en la interfaz de usuario.
            \item Finalmente, la función \texttt{`get-all-logged-in-users`} devuelve este mapa, utilizando la clave \texttt{`online-users`}.
        \end{itemize}

    \item Uso en la Página Web (Jinja Templates) :
        \begin{itemize}
            \item La página \texttt{'index.html'} utiliza \textbf{Jinja templates} para procesar la variable \texttt{`online-users`} recibida del servidor.
            \item Un bucle \texttt{`for`} (\texttt{`for n in online-users-record`}) itera a través de la lista de usuarios en línea, creando una fila 
            (\texttt{`<li>`}) para cada usuario.
            \item Dentro de cada fila, se muestra el nombre del usuario (\texttt{`n`}) y se utilizan \texttt{`n`} para el ID del usuario en los botones, y
            \texttt{`n`} y \texttt{`n`} para establecer dinámicamente el estado de los botones de alternancia de permisos de lectura y escritura.
        \end{itemize}
\end{itemize}

Este proceso permite que la aplicación Flask no solo identifique al usuario actual y su rol (admin/no-admin), sino también que \textbf{genere y envíe una 
lista detallada de todos los usuarios en línea con sus permisos actuales}, permitiendo a los administradores visualizar y modificar estos permisos en tiempo 
real desde el panel de control.

\subsection{Función \texttt{get-all-logged-in-users}}
% 6.3.3 Función  'get-all-logged-in-users '.
La función \texttt{`get-all-logged-in-users`} es un componente central de la funcionalidad del lado del servidor en una aplicación Python Flask, 
diseñada para \textbf{gestionar y presentar la lista de usuarios en línea} a los administradores de una plataforma IoT.

En el contexto más amplio de la funcionalidad del lado del servidor (Python Flask), esta función cumple los siguientes propósitos y se implementa de la 
siguiente manera:
\begin{itemize}
    \item Propósito y Contexto
        \begin{itemize}
            \item La función es invocada por la aplicación Flask en el momento en que se devuelve la página web principal (\texttt{'index.html'}) al usuario.
            \item Su objetivo principal es preparar y proporcionar la lista de todos los usuarios actualmente conectados para que un usuario administrador 
            pueda verla en un panel de control. Este panel permite a los administradores conceder permisos de lectura y escritura a usuarios no administradores 
            y dispositivos en tiempo real.
        \end{itemize}

    \item Ubicación y Estructura
        \begin{itemize}
            \item Aunque no se especifica explícitamente, se sugiere que esta función reside en un módulo llamado \texttt{`my-DB`}, indicando su rol 
            en la interacción con la base de datos o el sistema que mantiene el estado de los usuarios.
            \item La función crea una variable llamada \texttt{`online-user-records`}. Inicialmente, esta variable es un \textbf{mapa} (diccionario) con 
            una clave \texttt{`user-record`} cuyo valor es una lista vacía.
        \end{itemize}

    \item Población de Datos
        \begin{itemize}
            \item Dentro de un bucle \texttt{`for`}, la variable \texttt{`online-user-records`} se \textbf{rellena} con los detalles de cada usuario 
            en línea. Por cada usuario, se adjunta una lista que contiene la siguiente información:
                \begin{itemize}
                    \item El nombre del usuario  (en el índice 0).
                    \item El ID del usuario  (en el índice 1).
                    \item El estado de acceso de lectura (en el índice 2).
                    \item El estado de acceso de escritura (en el índice 3).
                \end{itemize} 
        \end{itemize}

    \item Conversión de Estados de Permiso para HTML
        \begin{itemize}
            \item Un paso crucial realizado por esta función es la conversión de los estados de acceso de lectura y escritura. Estos estados, que originalmente 
            son valores binarios (1 o 0), se transforman en las cadenas \texttt{`'checked'`} o \texttt{`'unchecked'`}.
            \item Esta conversión es fundamental porque el código HTML en la página web principal (\texttt{'index.html'}) interpretará directamente estas 
            cadenas para establecer el estado de los \textbf{botones de alternancia} (\texttt{`switch buttons`}) que controlan los permisos en la interfaz 
            de usuario.
        \end{itemize}

    \item Retorno de Datos al Servidor Flask
        \begin{itemize}
            \item Finalmente, la función \texttt{`get-all-logged-in-user`get-all-logged-in-users`s`} devuelve el mapa \texttt{`online-user-records`}, 
            pero lo hace utilizando la clave \texttt{`online-users`}.
            \item Esta variable \texttt{`online-users`} es entonces accesible en la página \texttt{'index.html'} a través de los \textbf{ Jinja templates }, 
            permitiendo que la interfaz de usuario genere dinámicamente las filas de la tabla con los usuarios y sus respectivos botones de permiso.
        \end{itemize}
\end{itemize}

En resumen, la función \texttt{`get-all-logged-in-users`} es esencial en la arquitectura del lado del servidor de Flask para la \textbf{visualización y 
administración en tiempo real de los permisos de los usuarios en línea}, transformando los datos brutos de la base de datos en un formato fácilmente 
consumible por el cliente web (HTML/Jinja) para una interacción dinámica.

\begin{itemize}
    \item Variable \texttt{online-users-records} (Mapa: \texttt{user-record} -> Lista Vacia).
    \item En el contexto más amplio de la función \texttt{'get-all-logged-in-users'} y la funcionalidad del lado del servidor con Python Flask, 
    la variable \texttt{'online-user-records'} (Mapa: \texttt{'user-record'} -> Lista Vacía) es fundamental para la construcción y gestión de la 
    lista de usuarios en línea que se presenta a los administradores.

    \item Aquí se dan los detalles de esta variable:
    \begin{itemize}
        \item 1. Inicialización de la Variable online-user-records.
            \begin{itemize}
                \item Dentro de la función get-all-logged-in-users (que se encuentra en el módulo my-DB), el primer paso es crear una variable llamada 
                online-user-records.
                \item Esta variable se inicializa como un mapa (diccionario) con la clave user-record y un valor que es una lista vacía. Esta estructura 
                permite que la función comience a recopilar datos de usuarios de manera organizada.
            \end{itemize}
        \item 2. Propósito y Población dentro de la Función get-all-logged-in-users.
            \begin{itemize}
                \item La función get-all-logged-in-users es invocada desde el lado del servidor Flask para preparar la lista de usuarios en línea y sus 
                permisos.
                \item El mapa online-user-records se rellena dentro de un bucle for. Por cada usuario en línea, se le adjunta una lista con sus 
                detalles:
                    \begin{itemize}
                        \item Nombre del usuario (en el índice 0).
                        \item ID del usuario (en el índice 1).
                        \item Estado de acceso de lectura (en el índice 2).
                        \item Estado de acceso de escritura (en el índice 3
                    \end{itemize}
                \item Es crucial destacar que los estados de acceso de lectura y escritura (que son 1 o 0) se convierten a las cadenas 'checked' o 'unchecked' 
                respectivamente. Esta conversión es necesaria porque el código HTML en la página index.html (utilizando Jinja templates) leerá estas cadenas 
                para establecer el estado de los botones de alternancia (switch buttons) que controlan los permisos
            \end{itemize}
        \item 3. Retorno de online-user-records para la Página Web.
            \begin{itemize}
                \item Finalmente, la función get-all-logged-in-users devuelve este mapa online-user-records, pero lo hace con la clave online-users. 
                Esta variable online-users es la que luego se utiliza en la página index.html mediante Jinja templates para construir dinámicamente el 
                panel de control. Un bucle for itera sobre online-users-record para crear una fila (<li>) por cada usuario, mostrando su nombre y los 
                botones de permisos con sus estados correctos (checked/unchecked).
            \end{itemize}
        \item En síntesis, la inicialización de online-user-records como un mapa con una lista vacía es el punto de partida para que la función 
        get-all-logged-in-users pueda estructurar y procesar los datos de los usuarios en línea, incluyendo sus permisos de lectura y escritura, y 
        convertirlos a un formato legible por HTML, facilitando así la gestión de permisos en tiempo real desde el panel de control de administración.
    \end{itemize}

    \item Poblar con (Nombre, User ID, Lectura, Escritura)
    \item En el contexto más amplio de la función get-all-logged-in-users y la funcionalidad del lado del servidor con Python Flask, el proceso de poblar 
    la variable online-user-records con el nombre, el ID de usuario, y los estados de acceso de lectura y escritura es fundamental para construir el panel 
    de control de administración en tiempo real.
    \item Se detalla este proceso de la siguiente manera:
        \begin{itemize}
            \item 1. Contexto de la Función get-all-logged-in-users.
                \begin{itemize}
                    \item Esta función, ubicada presumiblemente en el módulo my-DB, es la responsable de recopilar y estructurar los datos de los usuarios 
                    actualmente en línea.
                    \item Se invoca cuando la aplicación Flask devuelve la página web principal (index.html) al usuario, con el fin de proporcionar detalles 
                    adicionales para la interfaz.
                \end{itemize}
            \item 2. Inicialización de online-user-records.
                \begin{itemize}
                    \item Dentro de esta función, se crea una variable llamada online-user-records. Se inicializa como un mapa (diccionario) con la clave 
                    user-record y una lista vacía como valor. Esta estructura sirve como contenedor para los datos de los usuarios.            
                \end{itemize}
            \item 3. Proceso de Población (Nombre, User ID, Lectura, Escritura)
                \begin{itemize}
                    \item La variable online-user-records se rellena dentro de un bucle for. Por cada usuario en línea, la función adjunta una lista 
                    que contiene sus detalles específicos en índices predefinidos:
                    \begin{itemize}
                        \item El nombre del usuario se añade en el índice 0.
                        \item El ID del usuario se añade en el índice 1.
                        \item El estado de acceso de lectura se añade en el índice 2.
                        \item El estado de acceso de escritura se añade en el índice 3.                        
                    \end{itemize}
                \end{itemize}
            \item 4. Transformación de los Estados de Permiso (Lectura y Escritura).
            \begin{itemize}
                \item Un paso crítico durante la población es la conversión de los estados de acceso de lectura y escritura. Inicialmente, estos estados son 
                valores numéricos (1 para permitido, 0 para denegado).
                \item Sin embargo, la función los transforma a las cadenas 'checked' o 'unchecked'.
                \item Esta conversión es esencial porque el código HTML en la página index.html (que utiliza Jinja templates) leerá directamente estas cadenas 
                para establecer el estado visual de los botones de alternancia (switch buttons) que controlan los permisos en el panel de administración.        
            \end{itemize}
            
            \item 5. Retorno de los Datos Poblados.
            \begin{itemize}
                \item Una vez que online-user-records ha sido poblada con los datos transformados de todos los usuarios en línea, la función 
                get-all-logged-in-users devuelve este mapa. Lo hace utilizando la clave online-users para que sea accesible en la plantilla Jinja de 
                index.html        
            \end{itemize}
    
            Este meticuloso proceso de población garantiza que la función get-all-logged-in-users no solo recopile la información necesaria de los usuarios 
            en línea, sino que también la formatee adecuadamente para su presentación en la interfaz web, permitiendo a los administradores una gestión 
            intuitiva y en tiempo real de los permisos en la plataforma IoT.
        \end{itemize}

    \item Convertir Permisos de Lectura/Escritura a \texttt{cheked} o \texttt{unchecked} para HTML.
    \item En el contexto más amplio de la función get-all-logged-in-users y la funcionalidad del lado del servidor con Python Flask, la conversión de los 
    permisos de lectura y escritura a las cadenas 'checked' o 'unchecked' para HTML es un paso crucial para la visualización y gestión efectiva de permisos 
    en el panel de control del administrador.
    \item Aquí se dan los detalles de este proceso:
    \begin{itemize}
        \item 1. Propósito de la Función get-all-logged-in-users:
            \begin{itemize}
                \item Esta función, presumiblemente ubicada en el módulo my-DB, es responsable de recopilar los detalles de todos los usuarios en línea.
                \item Su objetivo es preparar estos datos para ser presentados en la página web principal (index.html) de la aplicación Flask, 
                específicamente en el panel de control visible para los usuarios administradores. Este panel permite a los administradores conceder 
                permisos de lectura y escritura en tiempo real a otros usuarios y dispositivos.
            \end{itemize}
        \item 2. La Necesidad de la Conversión:
            \begin{itemize}
                \item Originalmente, los estados de acceso de lectura y escritura se almacenan como valores numéricos, típicamente 1 para un permiso concedido 
                y 0 para un permiso denegado.
                \item Sin embargo, el código HTML en la interfaz de usuario utiliza botones de alternancia (switch buttons) para representar estos permisos. 
                Para que estos botones se muestren correctamente como  'activados ' o  'desactivados ', el código HTML no interpreta directamente 1 o 0, 
                sino que necesita cadenas específicas como 'checked' (para activado) o 'unchecked' (para desactivado).
            \end{itemize}
        \item 3. Proceso de Conversión:
            \begin{itemize}
                \item La función get-all-logged-in-users realiza esta conversión explícitamente.
                \item Durante la población de la variable online-user-records (un mapa que contiene listas con los detalles de cada usuario), los valores 
                numéricos de lectura y escritura se transforman.
                \item Se utiliza una lógica condicional (sentencias if/else) para realizar esta transformación:
                    \begin{itemize}
                        \item Si el acceso de lectura (o escritura) es 1, la variable correspondiente se establece en la cadena 'checked'.
                        \item De lo contrario (si es 0), la variable se establece en la cadena 'unchecked'.
                    \end{itemize}
            \end{itemize}
        \item 4. Integración con HTML y Jinja Templates:
            \begin{itemize}
                \item Una vez que los estados de lectura y escritura se han convertido a 'checked' o 'unchecked', estos valores se incluyen en la lista de 
                detalles de cada usuario que se adjunta a online-user-records.
                \item La función devuelve este mapa bajo la clave online-users, que luego es utilizada por los Jinja templates en la página index.html.
                \item En el HTML, cuando se itera a través de online-users-record para crear cada fila de usuario, estas cadenas 'checked' o 'unchecked' 
                se insertan directamente en los atributos de los botones de alternancia (switch buttons), lo que permite que el navegador muestre su estado 
                correcto de forma dinámica.
            \end{itemize}
        \item En resumen, la conversión de los permisos de lectura y escritura a 'checked' o 'unchecked' es un paso de mapeo de datos crítico en el lado del 
        servidor que facilita la comunicación entre la lógica de Python Flask (donde se gestionan los permisos) y la presentación en el lado del cliente 
        (HTML con Jinja), asegurando que los administradores puedan visualizar y manipular los permisos de los usuarios en línea de manera intuitiva y en 
        tiempo real.
    \end{itemize}

    \item Retorna Mapa con Clave 'on-line-users' y valor una lista.
    \item En el contexto más amplio de la función get-all-logged-in-users y la funcionalidad del lado del servidor con Python Flask, el hecho de que 
    retorne un mapa con la clave 'online-users' y un valor que es una lista es un aspecto fundamental para la comunicación efectiva entre el servidor y 
    la interfaz de usuario.
    \item Se detalla este proceso de la siguiente manera:
        \begin{itemize}
            \item 1. Propósito General de la Función get-all-logged-in-users:
                \begin{itemize}
                    \item Esta función, presumiblemente definida en un módulo my-DB, tiene la responsabilidad de recopilar y preparar los datos de 
                    todos los usuarios que están actualmente en línea.
                    \item Su objetivo es hacer que esta información esté disponible para la página web principal (index.html), especialmente para 
                    el panel de control del administrador, donde se visualizan y gestionan los permisos de lectura y escritura en tiempo real.            
                \end{itemize}
            \item 2. Preparación de los Datos (online-user-records):
                \begin{itemize}
                    \item Dentro de la función, se inicializa una variable llamada online-user-records. Esta variable comienza como un mapa (diccionario) con 
                    la clave user-record y un valor que es una lista vacía.
                    \item A continuación, en un bucle for, esta variable online-user-records se rellena añadiendo los detalles de cada usuario en línea. Cada 
                    entrada es una lista que contiene el nombre del usuario (índice 0), el ID del usuario (índice 1), el estado de acceso de lectura (índice 2) 
                    y el estado de acceso de escritura (índice 3).
                    \item Un paso crucial antes del retorno es la conversión de los estados de acceso de lectura y escritura (que son 1 o 0) a las cadenas 
                    'checked' o 'unchecked'. Esto es esencial porque el código HTML interpretará estas cadenas para establecer el estado de los botones de 
                    alternancia (switch buttons) en la interfaz de usuario.            
                \end{itemize}
            \item 3. El Retorno del Mapa con la Clave 'online-users':
                \begin{itemize}
                    \item Esta variable online-users (el mapa retornado) es entonces accesible en la página index.html del lado del cliente a través de los 
                    Jinja templates.
                    \item La plantilla HTML utiliza un bucle for (for n in online-users-record) para iterar a través de la lista de usuarios. Este bucle 
                    permite crear dinámicamente una fila (<li>) para cada usuario en línea.
                    \item Dentro de cada fila, se extraen y muestran el nombre del usuario (usando n), el ID del usuario (usando n), y los estados de 
                    lectura y escritura (n y n), que al ser 'checked' o 'unchecked', configuran directamente los botones de alternancia en la interfaz.
                \end{itemize}
            \item 4. Uso en el Lado del Cliente (Jinja Templates):
                \begin{itemize}
                     \item Esta variable online-users (el mapa retornado) es entonces accesible en la página index.html del lado del cliente a través de los 
                     Jinja templates.
                    \item La plantilla HTML utiliza un bucle for (for n in online-users-record) para iterar a través de la lista de usuarios. Este bucle 
                    permite crear dinámicamente una fila (<li>) para cada usuario en línea.
                    \item Dentro de cada fila, se extraen y muestran el nombre del usuario (usando n), el ID del usuario (usando n), y los estados de lectura 
                    y escritura (n y n), que al ser 'checked' o 'unchecked', configuran directamente los botones de alternancia en la interfaz.
                \end{itemize}
            \item En conclusión, el patrón de retornar un mapa con la clave 'online-users' y una lista de registros desde la función get-all-logged-in-users 
            es una estrategia deliberada en el lado del servidor para empaquetar y entregar de manera estructurada los datos de los usuarios en línea a la 
            plantilla HTML. Esto facilita la generación dinámica del panel de control de administración, permitiendo a los administradores visualizar y 
            gestionar los permisos de los usuarios en tiempo real en la plataforma IoT.
        \end{itemize}
\end{itemize}

\section{RENDERIZADO DE LA INTERFAZ (Jinja Templates)}
% 6.4 RENDERIZADO DE LA INTERFAZ (Jinja Templates)
Las  Plantillas Jinja (Jinja Templates)  son un componente esencial del  renderizado de la interfaz de usuario ('index.html')  que permite implementar 
de manera dinámica y segura las  reglas para usuarios administradores y no administradores  en la plataforma IoT. Permiten que el servidor Flask inyecte 
datos y lógica condicional directamente en el HTML antes de que se envíe al navegador del cliente.

Aquí se detalla cómo las Plantillas Jinja se utilizan en este contexto:
\begin{itemize}
    \item 1. Población Dinámica de la Lista de Usuarios en Línea
    \begin{itemize}
        \item El servidor Flask prepara una lista de registros de usuarios en línea ('online users record') que incluye el nombre del usuario, 
              su ID, y el estado de sus permisos de lectura y escritura.
        \item Las plantillas Jinja en 'index.html' utilizan un bucle `for` (`for n in online users record`) para iterar sobre esta lista.
        \item Cada iteración del bucle genera una fila (`<li>`) en la interfaz, mostrando la información de un usuario en línea.
        \item Dentro de cada fila, se accede a los elementos del registro del usuario utilizando índices:
        \begin{itemize}
            \item `n` para el  nombre del usuario.
            \item `n` para el  ID del usuario.
        \end{itemize}
    \end{itemize}

    \item 2. Configuración Dinámica de Permisos de Lectura y Escritura
    \begin{itemize}
        \item El servidor es responsable de convertir los estados numéricos de los permisos de lectura y escritura (por ejemplo, 1 para permitido) a 
              las cadenas checked (activado) o unchecked (desactivado).
        \item Estas cadenas son luego inyectadas por Jinja directamente en el atributo `checked` de los botones tipo  'switch ' en el HTML.
        \item De esta forma, los botones de conmutación de permisos de lectura y escritura (`read user ID` y `write user ID`) para cada usuario 
              aparecen con el estado correcto reflejando sus permisos actuales.
        \item Los estados de lectura y escritura se obtienen de `n` y `n` del registro del usuario, respectivamente.
    \end{itemize}

    \item 3. Restricción del Panel de Control para Usuarios Administradores
    \begin{itemize}
        \item Una de las funcionalidades más críticas implementadas con Jinja es asegurar que  el panel de control completo, que contiene la lista de 
        usuarios y los botones para gestionar sus permisos, sea visible  'únicamente para los usuarios administradores ' .
        \item Esto se logra añadiendo una  sentencia `if` directamente en el código HTML  del 'index.html'.
        \item Esta condición verifica si el 'user ID' del usuario actualmente conectado (proporcionado por el servidor a través de la sesión) coincide 
        con un 'user ID' de administrador predefinido.
        \item Si e    l 'user ID' actual no es el del administrador (por ejemplo, para un usuario como  'Anam Chaudhary ' que no es administrador), el
         panel de control  no se renderiza en absoluto  en la página web que recibe el cliente.
        \item Esto garantiza que los usuarios no administradores no tengan acceso visual ni funcional a las herramientas de gestión de permisos, 
        reforzando la seguridad de la plataforma.
    \end{itemize}
\end{itemize}

En resumen, las Plantillas Jinja son fundamentales para crear una interfaz de usuario dinámica que se adapta al rol del usuario. Permiten al servidor 
renderizar contenido HTML personalizado, mostrando listas de usuarios con sus permisos actuales y, lo que es más importante, controlando la visibilidad 
de funciones críticas de administración, asegurando que solo los usuarios con los privilegios adecuados puedan ver y modificar las reglas de acceso en 
la plataforma IoT.

\subsection{Bucle \texttt{'for n in online-users-record'}}
% 6.4.1 Bucle \texttt{'for n in online-users-record'}.
El bucle \texttt{`for n in online-users-record`} es una parte fundamental del renderizado de la interfaz en el lado del cliente utilizando 
\textbf{plantillas Jinja} dentro del contexto más amplio de la Visibilidad del Panel de Control (Admin Solamente). Permite mostrar dinámicamente 
la lista de usuarios en línea en el panel de administración.

A continuación, se detalla este aspecto:
\begin{itemize}
    \item Origen de \texttt{`online-users-record`}:
    \begin{itemize}
        \item Para poblar la lista de usuarios en línea, el servidor (específicamente la función \texttt{`get all logged in users`} en \texttt{`my DB`}) 
        es responsable de \textbf{devolver un mapa} que contiene los detalles de todos los usuarios que han iniciado sesión.
        \item Este mapa se nombra \texttt{`online user records`} (o \texttt{`online-users-record`}) y se construye con el \texttt{`name`} del
              usuario en el índice 0, el \texttt{`user-id`} en el índice 1, el estado de \texttt{`read`} (lectura) en el índice 2 y el estado de 
              \texttt{`write`} (escritura) en el índice 3.
        \item Los estados de \texttt{`read`} y \texttt{`write`} se convierten a \texttt{checked} o \texttt{unchecked} para que el código HTML pueda 
              establecer correctamente el estado de los botones de interruptor.
        \item Este \texttt{`online-users-record`} (que es una lista, a pesar de ser referido como un mapa con un valor de lista) se envía al cliente 
              junto con el \texttt{`user-id`} de la sesión cuando se devuelve la página principal al usuario.
    \end{itemize}

    \item Uso del Bucle \texttt{`for`} en \texttt{'index.html'} (Jinja):
    \begin{itemize}
        \item Una vez que la variable \texttt{`online-users-record`} llega a la página \texttt{'index.html'}, las plantillas Jinja se utilizan para 
        iterar sobre ella y renderizar la información.
        \item El bucle se escribe como \texttt{`for n in online users record`}.
        \item Este bucle \texttt{`for`} se ejecuta para cada elemento en la lista \texttt{`online-users-record`}, creando \textbf{múltiples filas} 
        en la tabla del panel de control, una por cada usuario en línea.
        \item Cada \texttt{`n`} dentro del bucle \textbf{representa una lista individual} que contiene los detalles de un usuario en particular dentro 
        de \texttt{`user records`}.    
    \end{itemize}

    \item Acceso a los Datos del Usuario dentro del Bucle:
    \begin{itemize}
        \item Dentro de cada iteración del bucle (\texttt{`n`}), se accede a los datos específicos del usuario utilizando los índices definidos:
        \begin{itemize}
            \item El \textbf{nombre del usuario} se muestra tomando el \texttt{`0th index`} de \texttt{`n`}.
            \item El \texttt{`user-id`} del usuario se obtiene del \texttt{`1st index`} de \texttt{`n`}. Este \texttt{`user-id`} se utiliza para 
            reemplazar placeholders en los ID de los botones de interruptor y el botón \texttt{APLICAR} (acceso).
            \item El estado de lectura \texttt{(`read`)} se encuentra en el \texttt{`2nd index`} de \texttt{`n`}.
            \item El estado de escritura \texttt{(`write`)} se encuentra en el \texttt{`3rd index`} de \texttt{`n`}.
        \end{itemize}
        \item Estos estados de lectura y escritura \texttt{(`n` y `n`)} se utilizan para establecer las condiciones de \texttt{checked} o 
        \texttt{ 'unchecked '} para los botones de interruptor de permisos en la interfaz.    
    \end{itemize}

    \item Propósito en el Renderizado de la Interfaz:
    \begin{itemize}
        \item El bucle \texttt{`for n in online-users-record`} es esencial para construir la sección del panel de control de administradores donde se lista 
        a \textbf{todos los usuarios en línea}.
        \item Esta lista permite al administrador ver a otros usuarios (por ejemplo, \texttt{Anam Chaudhary} en el escenario de prueba) y, lo más 
        importante, \textbf{interactuar con botones de interruptor para conceder permisos de lectura y escritura en tiempo real}.    
    \end{itemize}
\end{itemize}

En resumen, el bucle \texttt{`for n in online-users-record`} combinado con las plantillas Jinja es el mecanismo del lado del cliente que permite 
al panel de control del administrador renderizar de forma dinámica y detallada la lista de usuarios en línea y sus respectivos controles de permisos, 
basándose en los datos proporcionados por el servidor.

\subsection{Acceso a Datos por índice}
% 6.4.2 Acceso a Datos por índice, \texttt{(0: User Nombre, 1: User ID, 2: Lectura, 3:Escritura)}.
Se explican en detalle cómo se accede a los datos de los usuarios en línea mediante índices (0: Nombre, 1: User ID, 2: Lectura, 3: Escritura) 
en el contexto del renderizado de la interfaz con \textbf{Plantillas Jinja}. Este método es clave para mostrar dinámicamente la información en el panel 
de control del administrador.

A continuación, se describe cómo se maneja este acceso a los datos:
\begin{itemize}
    \item Estructura de \texttt{`online-users-record`}:
    \begin{itemize}
        \item La información de los usuarios en línea se organiza en una variable llamada \texttt{`online-users-record`}. Aunque se la describe como 
        un \textbf{mapa con la clave como registro de usuario y el valor es una list}, se  especifican que es una \textbf{lista} en sí misma, 
        donde cada elemento dentro de ella es otra lista que representa los detalles de un usuario individual.
        \item Esta estructura es devuelta por el servidor (desde la función \texttt{`get all logged in users`} en \texttt{`my DB`}) al cliente, 
        junto con el \texttt{`user-id`} de la sesión.
    \end{itemize}

    \item Organización de Datos dentro de Cada Registro de Usuario:
    \begin{itemize}
        \item Dentro de cada lista o \textbf{registro} que representa a un usuario, la información se almacena en posiciones específicas o \textbf{índices}:
            \begin{itemize}
                \item Índice 0: Contiene el \textbf{nombre del usuario}.
                \item Índice 1: Almacena el \textbf{ID del usuario (`user-id`)}.
                \item Indice 2 : Indica el estado de los permisos de \textbf{lectura (`read`)} del usuario. Este valor se convierte a \textbf{'checked'} o 
                \textbf{ 'unchecked '} para que el código HTML pueda interpretarlo directamente.
                \item Índice 3: Indica el estado de los permisos de \textbf{escritura (`write`)} del usuario. Similar al de lectura, también se convierte
                a \textbf{checke} o \textbf{unchecke}.        
        \end{itemize}    
    \end{itemize}

    \item Renderizado en \texttt{'index.html'} con Jinja Templates:
    \begin{itemize}
        \item Para mostrar esta lista de usuarios en la interfaz, la página \texttt{'index.html'} utiliza un \textbf{bucle `for` de Jinja}: 
        \texttt{`for n in online users record`}.
        \item Este bucle itera sobre la lista \texttt{`online-users-record`}, y en cada iteración, la variable \texttt{`n`} representa la lista de 
        datos de un usuario individual.
        \item Dentro del bucle, se accede a los datos específicos de cada usuario utilizando sus índices:
        \begin{itemize}
            \item El \textbf{nombre del usuario} se muestra al acceder a \texttt{`n`} en el índice 0 (\texttt{`n`}).
            \item El \texttt{`user-id`} se obtiene de \texttt{`n`} en el índice 1 (\texttt{`n`}) y se utiliza para construir dinámicamente los ID de 
            los botones de interruptor (\texttt{`read user ID`}, \texttt{`write user ID`}) y el botón \texttt{APLICAR} (\texttt{`access user ID`}).
            \item Los estados de \textbf{lectura y escritura} se acceden a través de \texttt{`n`} en los índices 2 y 3 (\texttt{`n`} y \texttt{`n`}), 
            respectivamente, para establecer el estado inicial (\texttt{`checked`} o \texttt{`unchecked`}) de los botones de interruptor correspondientes 
            en la interfaz.
        \end{itemize}
    \end{itemize}

    \item Contexto Amplio de Renderizado de la Interfaz:
    \begin{itemize}
        \item Este método de acceso a datos por índice es crucial para \textbf{renderizar dinámicamente la sección del panel de control de administradores} 
        donde se lista a todos los usuarios en línea.
        \item La posibilidad de ver a otros usuarios (como \texttt{Anam Chaudhary} en el ejemplo) y sus permisos se logra gracias a esta estructura de 
        datos y su renderizado por medio de Jinja.
        \item Permite a los administradores \textbf{gestionar en tiempo real los permisos de lectura y escritura} de usuarios y dispositivos no 
        administradores mediante los botones de interruptor y el botón \textbf{APLICAR} que se generan en cada fila de la tabla.    
    \end{itemize}
\end{itemize}

En síntesis, el acceso a datos por índice dentro del bucle \texttt{`for n in online-users-record`} y las plantillas Jinja es el mecanismo fundamental 
para que el panel de control del administrador pueda construir una \textbf{vista interactiva y actualizada de todos los usuarios en línea}, 
facilitando la gestión de permisos en el ecosistema IoT.

\section{VISIBILIDAD DEL PANEL DE CONTROL (Admin Solamente)}
% 6.5 VISIBILIDAD DEL PANEL DE CONTROL (Admin Solamente).
La  visibilidad del panel de control  es un aspecto fundamental de las  reglas para usuarios administradores y no administradores , y se  
detallan cómo se asegura que  solo los usuarios administradores  puedan acceder a las funcionalidades de gestión de permisos. Esta restricción se 
implementa principalmente a través del  renderizado de la interfaz del lado del servidor utilizando plantillas Jinja .

Aquí se detalla este aspecto:
\begin{itemize}
    \item 1. Definición y Contenido del Panel de Control
    \begin{itemize}
        \item El panel de control al que se hace referencia es una sección específica del dashboard de la plataforma IoT.
        \item Esta sección está diseñada para mostrar una  lista de todos los usuarios en línea .
        \item Junto al nombre de cada usuario en línea, el panel incluye  botones de conmutación (switch buttons)  para  otorgar permisos de lectura y 
        escritura . También hay un botón para aplicar los cambios.
        \item La interfaz permite a los administradores gestionar los permisos de otros usuarios en tiempo real.
    \end{itemize}

    \item 2. Restricción de Visibilidad: Solo para Administradores
    \begin{itemize}
        \item Se enfatiza que la última funcionalidad a implementar es asegurarse de que   'este panel de control es únicamente visible para los 
        usuarios administradores ' .
        \item Esto significa que los usuarios que no tienen el rol de administrador no deben ver esta sección en absoluto.
        \item El dashboard para usuarios administradores mostrará la lista de usuarios en línea con los botones para conceder permisos de lectura y escritura.    
    \end{itemize}

    \item 3. Implementación a Través de Plantillas Jinja y Lógica Condicional
    \begin{itemize}
        \item La restricción de visibilidad se logra añadiendo una  sentencia `if` directamente en el código HTML de 'index.html' .
        \item Esta sentencia `if` compara el 'user ID' del usuario actualmente conectado (que es proporcionado por el servidor a la plantilla Jinja) 
        con un  'user ID' de administrador codificado (hard-coded) .
        \item El 'user ID' del usuario actual se obtiene de la sesión y se envía al cliente. Para identificar el 'user ID' del administrador, se 
        sugiere imprimir el 'user ID' de la sesión en la consola y luego codificarlo en el HTML.
        \item La lógica es clara:   'si el user ID es igual a mi user ID '  (refiriéndose al ID del administrador), entonces el panel de control se 
        renderizará. Si no coincide, el panel no se mostrará.
        \item Es importante recordar   'cerrar esta sentencia `if` '  al final de la sección del panel de control.    
    \end{itemize}

    \item 4. Experiencia del Usuario Administrador vs. No Administrador
    \begin{itemize}
        \item Cuando un  usuario administrador  inicia sesión, verá el panel de control con la lista de usuarios y las opciones para modificar permisos.
        \item Sin embargo, un  usuario no administrador , como el ejemplo de  'Anam Chaudhary ', al iniciar sesión en el servidor IoT de Packt, 
        'no tendrá acceso al panel de control de acceso ' . Esto demuestra que la implementación condicional de Jinja funciona eficazmente para 
        ocultar la sección de administración a usuarios no autorizados.    
    \end{itemize}
\end{itemize}

En resumen, la visibilidad del panel de control para gestionar permisos de usuarios es una característica de seguridad crítica. Se implementa en el lado 
del servidor mediante la inyección condicional de HTML a través de plantillas Jinja, asegurando que solo los usuarios administradores, identificados por 
su 'user ID', puedan ver y utilizar esta funcionalidad. Esto garantiza que las reglas para usuarios administradores y no administradores se apliquen 
estrictamente, protegiendo la integridad de la gestión de acceso en la plataforma IoT.

\subsection{Sentencia \texttt{'if user-id == my-admin-user-id'} en HTML (hardcodeado)}

La sentencia \texttt{'if user-id == my-admin-user-id'} en el código HTML (hardcodeado) se utiliza para controlar la visibilidad del panel de control, 
asegurando que solo los usuarios administradores puedan verlo.

A continuación, se detalla su contexto y funcionamiento:
\begin{itemize}
    \item Propósito de la Sentencia \texttt{'if'}.
    \begin{itemize}
        \item La funcionalidad principal de esta sentencia es \textbf{hacer que el panel de control sea visible únicamente para los usuarios administradores}.
        \item Se añade una sentencia \texttt{`if`} antes de que comience la sección del panel de control en el código HTML.
        \item El \texttt{`user-id`} que se envía al cliente se utiliza en esta sentencia para determinar si el usuario actual es el administrador.
    \end{itemize}

    \item Implementación del \texttt{`my-admin-user-id`} (Hardcodeado).
    \begin{itemize}
        \item Para implementar esta verificación, el \texttt{`my-admin-user-id`} se obtiene imprimiendo el \texttt{`session user-id`} en la consola y luego
              hardcodeándolo directamente en el código HTML.
        \item Después de la sentencia \texttt{`if`}, se debe incluir una sentencia \texttt{`end if`} para cerrar el bloque condicional.
    \end{itemize}

    \item Comportamiento para Usuarios No Administradores.
    \begin{itemize}
        \item Cuando un usuario que no es administrador (por ejemplo,  'Anam Chaudhary ' en el ejemplo proporcionado) inicia sesión en el servidor IOT, 
              \textbf{no tendrá acceso al panel de control de acceso}.
        \item Sin embargo, desde el panel de control del administrador, el usuario no administrador sí aparecerá en la lista de usuarios en línea.
    \end{itemize}

    \item Contexto Amplio de Visibilidad del Panel de Control (Admin Solamente).
    \begin{itemize}
        \item Esta implementación forma parte de un enfoque más amplio para \textbf{permitir a los usuarios administradores gestionar permisos}.
        \item El panel de control del administrador está diseñado para mostrar una lista de todos los usuarios en línea con botones para conceder permisos 
              de lectura y escritura en tiempo real a los usuarios y dispositivos no administradores.
        \item Al seleccionar el botón \texttt{APLICAR}, los permisos de lectura y escritura de un usuario específico se cambiarán en tiempo real.
        \item Antes de conceder permisos de lectura y escritura, es necesario generar una clave de autorización para el usuario específico y 
              almacenarla en la base de datos.
    \end{itemize}
\end{itemize}

En resumen, la sentencia \texttt{`if user-id == my-admin-user-id`} es un mecanismo del lado del cliente, hardcodeado en el HTML, para 
\textbf{restringir la interfaz de usuario del panel de control a los administradores}, lo que les permite gestionar las autorizaciones de otros usuarios 
y dispositivos dentro del ecosistema IOT.

\subsection{Verificar \texttt{'session.user-id'} en consola}
Para controlar la visibilidad del panel de control de manera que solo los administradores puedan acceder a él, se  indican que es necesario
\textbf{verificar el \texttt{`session.user-id`} en la consola}.

Aquí se detalla el proceso y su relevancia en el contexto de la visibilidad del panel de control (solo para administradores):
\begin{itemize}
    \item Objetivo: Asegurar que el panel de control sea visible \textbf{únicamente para los usuarios administradores}.
    \item Mecanismo de Verificación: Se añade una \textbf{sentencia \texttt{`if`} en el código HTML} antes de que comience la sección del panel de control. 
          Esta sentencia comparará el \texttt{`user-id`} actual del usuario con el \texttt{`ID`} de un usuario administrador predefinido.
    \item Determinación del `ID` del Administrador \texttt{(`my-admin-user-id`)}: Para obtener el \texttt{`ID`} del usuario administrador que se utilizará 
          en la sentencia \texttt{`if`}, se debe \textbf{imprimir el \texttt{`session user-id`} en la consola}. Una vez obtenido, este \texttt{`ID`} se
          \textbf{hardcodea (se incrusta directamente) en el código HTML}. Es importante recordar cerrar la sentencia \texttt{`if`} con un \texttt{`end if`}.
    \item Comportamiento para Usuarios No Administradores: Como resultado de esta implementación, si un usuario que no es el administrador (por ejemplo, 
          \texttt{'Anam Chaudhary'}) inicia sesión, \textbf{no tendrá acceso al panel de control de acceso}. Sin embargo, desde el panel de control del 
          administrador, el usuario no administrador sí aparecerá en la lista de usuarios en línea.
    \item Contexto más Amplio (Visibilidad del Panel de Control Admin Solamente): Esta funcionalidad es crucial para el diseño del panel de control de 
          administradores, que permite a los administradores \textbf{gestionar permisos en tiempo real}. El panel muestra una lista de todos los usuarios 
          en línea con botones para conceder permisos de lectura y escritura a usuarios y dispositivos no administradores. Al seleccionar el botón 
          \texttt{APLICAR}, los permisos se cambian en tiempo real. Antes de conceder permisos, se requiere generar una clave de autorización para el 
          usuario específico y almacenarla en la base de datos.
\end{itemize}

En síntesis, la verificación de \texttt{`session.user-id`} en la consola es un paso intermedio para \textbf{identificar y hardcodear el \texttt{`ID`} 
del administrador directamente en el código HTML}, lo que permite una comprobación del lado del cliente para \textbf{restringir la visibilidad de la 
interfaz del panel de control exclusivamente a los usuarios administradores}.

\subsection{Prueba con Usuario No Administrador (no tiene acceso al panel)}
Se describe claramente la prueba con un usuario no administrador en el contexto de la visibilidad del panel de control (Admin Solamente), 
confirmando que \textbf{dichos usuarios no tienen acceso al panel}.

A continuación, se detalla este aspecto:
\begin{itemize}
    \item Objetivo de la Prueba:
    \begin{itemize}
        \item El propósito de esta prueba es \textbf{asegurarse de que el panel de control sea visible únicamente para los usuarios 
              administradores}. Esto se logra mediante la implementación de una sentencia \texttt{`if`} en el código HTML que compara el \texttt{`user-id`}
              del usuario actual con el \texttt{`ID`} del administrador, el cual ha sido previamente obtenido de la \texttt{`session user-id`} en la consola
              y hardcodeado.
    \end{itemize}

    \item Escenario de Prueba con Usuario No Administrador:
    \begin{itemize}
        \item Para verificar la funcionalidad, se describe un escenario donde se reinicia el servidor y se conecta \textbf{otro usuario que no es el 
              administrador}.
    \end{itemize}

    \item Comportamiento Esperado y Observado:
    \begin{itemize}
        \item Como se esperaba, si el usuario no es el administrador, \textbf{no obtendrá acceso al panel de control}.
        \item Se utiliza el ejemplo de una cuenta de Facebook con el nombre \texttt{Anam Chaudhary}, configurada como usuario no administrador. 
              Al iniciar sesión desde esa cuenta en el servidor IOT, \textbf{Anam Chaudhary no obtiene acceso al panel de control de acceso}.
    \end{itemize}

    \item Impacto en el Panel del Administrador:
    \begin{itemize}
        \item A pesar de que el usuario no administrador no ve el panel de control, \textbf{desde el panel de control del administrador, si se actualiza 
              la página, se verá al usuario \texttt{Anam Chaudhary} en la lista de usuarios en línea}.    
    \end{itemize}
    
    \item Contexto Amplio de Visibilidad (Admin Solamente):
    \begin{itemize}
        \item Esta prueba confirma la efectividad de la medida de seguridad implementada para el panel de control. El panel de control del administrador 
              está diseñado para mostrar una \textbf{lista de todos los usuarios en línea}, permitiendo a los administradores \textbf{conceder permisos 
              de lectura y escritura en tiempo real} a usuarios y dispositivos no administradores mediante botones específicos. Al seleccionar 
              \texttt{APLICAR}, los permisos de un usuario específico se cambian en tiempo real.
        \item Antes de otorgar estos permisos, es necesario \textbf{generar una clave de autorización} para el usuario y almacenarla en la base de datos.
        \item Por lo tanto, la restricción de visibilidad a los administradores es fundamental para garantizar que solo ellos puedan realizar estas 
              acciones críticas de gestión de permisos.
    \end{itemize}
\end{itemize}

En síntesis, se  demuestran que, tras implementar la lógica de verificación de \texttt{`user-id`}, \textbf{los usuarios no administradores son 
efectivamente impedidos de acceder al panel de control}, lo que valida la funcionalidad de seguridad y la exclusividad del panel para la gestión de 
permisos por parte de los administradores.

\subsection{Prueba con Administrador (puede ver otros usuarios en lista online)}
Se indica claramente que la \textbf{prueba con un administrador demuestra que este puede ver a otros usuarios en la lista de usuarios en línea}, 
lo cual es fundamental en el contexto más amplio de la Visibilidad del Panel de Control (Admin Solamente).

A continuación, se detalla este aspecto:
\begin{itemize}
    \item Diseño del Panel de Control para Administradores:
    \begin{itemize}
        \item El panel de control está diseñado para incluir una sección específica que lista a \textbf{todos los usuarios en línea}.
        \item Junto al nombre de cada usuario en línea, se prevén botones para \textbf{otorgar permisos de lectura y escritura}, así como un botón 
              para \texttt{APLICAR} los cambios.
        \item La meta de esta característica es que los administradores puedan \textbf{gestionar permisos de lectura y escritura en tiempo real} para 
              los usuarios y dispositivos no administradores.
    \end{itemize}
    
    \item Lógica de Servidor para la Lista de Usuarios en Línea:
    \begin{itemize}
        \item Para poblar esta lista, se implementa una funcionalidad en el servidor para \textbf{devolver el \texttt{`user-id`} y una lista de todos los 
              usuarios que han iniciado sesión}.
        \item La función \texttt{`get all logged in users`} (obtener todos los usuarios que han iniciado sesión) crea un mapa (\texttt{`online user records`}) 
              que se llena con el nombre del usuario (índice 0), el \texttt{`user-id`} (índice 1) y los estados de lectura y escritura (índices 2 y 3 
              respectivamente).
        \item Estos datos se devuelven al cliente para ser utilizados en la página \texttt{'index.html'} mediante plantillas Jinja, utilizando un 
              bucle \texttt{`for`} para crear una fila por cada usuario en línea.
    \end{itemize}
    
    \item Confirmación de la Visibilidad del Administrador en una Prueba:
    \begin{itemize}
        \item Se describe un escenario de prueba en el que un usuario no administrador (por ejemplo, \texttt{Anam Chaudhary}) inicia sesión 
              y, como se espera, no tiene acceso al panel de control.
        \item Sin embargo, desde el \textbf{panel de control del administrador, al refrescar la página, se puede ver al usuario \texttt{Anam Chaudhary} 
              en la lista de usuarios en línea}. Esta observación directa confirma que el administrador tiene la capacidad de ver a otros usuarios.
    \end{itemize}

    \item Contexto Amplio de Visibilidad del Panel de Control (Admin Solamente):
    \begin{itemize}
        \item La capacidad del administrador de ver una lista de todos los usuarios en línea es una \textbf{característica clave del diseño del sistema para 
              la gestión de la seguridad y los permisos}.
        \item Esta visibilidad permite al administrador interactuar con cada usuario en la lista para \textbf{concederles o denegarles permisos específicos}.
        \item Antes de conceder permisos de lectura y escritura, es necesario \textbf{generar una clave de autorización para el usuario y almacenarla en la 
              base de datos}.
    \end{itemize}
\end{itemize}

En resumen, se  detallan cómo el panel de control del administrador está diseñado para mostrar activamente una lista de todos los usuarios en 
línea, una capacidad que se verifica mediante pruebas directas. Esta funcionalidad es esencial para que los administradores puedan llevar a cabo su rol 
de \textbf{gestión y control de permisos en tiempo real} dentro del ecosistema de IoT.

\section{ENDPOINT PARA EL BOTóN \texttt{APLICAR}}
% 6.6 ENDPOINT PARA EL BOTóN  'APLICAR '.
El Endpoint para el Botón \texttt{APLICAR} es una funcionalidad central en la gestión de permisos de usuarios dentro de la plataforma IoT, y su 
implementación está directamente ligada a las \textbf{Reglas para Usuarios Administradores y No Administradores}, asegurando que solo los usuarios autorizados 
puedan realizar cambios.

Aquí se detalla lo que se propone al respecto en este proyecto:
\begin{itemize}
    \item 1. Propósito y Ubicación del Botón 'Aplicar'
    \begin{itemize}
        \item El botón \texttt{APLICAR} forma parte del  panel de control de acceso , que es visible  'únicamente para los usuarios administradores '.
        \item Este botón se encuentra junto a las opciones de permisos de lectura y escritura para cada usuario en línea listado en el dashboard.
        \item Su función es \texttt{'aplicar los cambios'} realizados en los permisos de lectura y escritura de un usuario específico. Al seleccionarlo, 
                  los \texttt{ 'permisos de acceso de lectura y escritura de ese usuario específico cambiarán en tiempo real '}.    
    \end{itemize}

    \item 2. Interacción del Lado del Cliente (JavaScript)
    \begin{itemize}
        \item Se añade un  endpoint para este botón 'Aplicar'  que enviará una solicitud desde el código JavaScript a la aplicación Flask.
        \item En `main.js`, se implementa un método que  'escucha cualquier botón de conmutación (switch button) que se extrae de nuestro dashboard '.
        \item Este método verifica si la ID del botón de conmutación  'comienza con 'XS' ' (posiblemente refiriéndose a 'access') y luego  divide esta ID 
        para extraer la ID del usuario .
        \item Además, lee el  estado de los interruptores de lectura y escritura  asociados a ese usuario.
        \item Finalmente, envía la solicitud al servidor Flask en un formato específico:  `grant - user ID - read state - write state` .    
    \end{itemize}

    \item 3. Recepción y Procesamiento en el Servidor (Flask)
    \begin{itemize}
        \item En la aplicación Flask, se añade un  endpoint para recibir esta solicitud `grant user ID read and write` .
        \item Verificación de permisos de administrador:  Es crucial que el servidor  'verifique una vez más si esta solicitud de concesión proviene de 
        un usuario administrador '. Esto refuerza las reglas de acceso, asegurando que solo un administrador pueda iniciar cambios en los permisos.
        \item Respuesta condicional: 
        \begin{itemize}
            \item Si la solicitud no proviene de un administrador, el servidor enviará una respuesta de   'acceso denegado ' (`access denied`) .
            \item Si todo está correcto y la solicitud es de un administrador, se enviará una respuesta de   'acceso concedido ' (`access granted`) .
        \end{itemize}
        \item Acciones posteriores:  Si el acceso es concedido, el servidor realiza dos acciones principales:
        \begin{itemize}
            \item 1.   Almacenar los permisos de lectura y escritura del usuario en la base de datos .
            \item 2.   Llamar al servidor PubNub para otorgar acceso de lectura y escritura a este usuario específico .    
        \end{itemize}
    \end{itemize}

    \item 4. Manejo de la Respuesta en el Cliente
    \begin{itemize}
        \item En `main.js`, se maneja la respuesta del servidor, extrayendo el JSON.
        \item Se verifica si el JSON tiene la clave  'access ' y si el  'access ' es  'granted '.
        \item Si el acceso es concedido, se  'restablece la suscripción en el canal nuevamente '.    
    \end{itemize}

    \item 5. Requisito Previos: Clave de Autorización
    \begin{itemize}
        \item Se aclara que la  'concesión de permisos de lectura y escritura ' es un segundo paso. El primer paso que lo precede es 
        'generar la clave de autorización para ese usuario específico y almacenarla en la base de datos ' . Esto asegura que haya una base de 
        seguridad y autenticación antes de que se puedan modificar los permisos.    
    \end{itemize}
\end{itemize}
En resumen, el endpoint para el botón 'Aplicar' no solo es una interfaz para modificar permisos, sino que está profundamente integrado con la 
arquitectura de seguridad de la plataforma. Garantiza que cualquier cambio en los derechos de acceso sea iniciado por un usuario administrador 
validado y que estos cambios se reflejen de manera persistente en la base de datos y en tiempo real a través de PubNub, consolidando así las 
reglas para usuarios administradores y no administradores.

\subsection{Método JavaScript en \textbf{main.js}}
% 6.6.1 Método JavaScript en  'main.js'
En el contexto más amplio del Endpoint para el Botón 'Aplicar', el método JavaScript en main.js juega un papel fundamental para habilitar 
la funcionalidad de gestión de permisos de usuario en el panel de control del administrador en el index.html.
Aquí se detalla lo que se dice sobre este método:
\begin{itemize}
    \item 1. Propósito y Visibilidad del Botón 'Aplicar' (\verb|ID: access_user ID|):
        \begin{itemize}
            \item El botón 'Aplicar' es un 'botón táctil' que se encuentra en cada fila de usuario del panel de control, adyacente a los interruptores de 
            permisos de lectura y escritura.
            \item Su función es cambiar los permisos de acceso de lectura y escritura de un usuario específico en tiempo real cuando se selecciona.
            \item Todo el panel, incluyendo este botón, es visible únicamente para usuarios administradores.        
        \end{itemize}
    \item 2. El Método JavaScript en main.js (Lado del Cliente):
        \begin{itemize}
            \item El archivo main.js contiene un método diseñado para escuchar cualquier botón en el dashboard.
            \item Este método verifica si el ID de un botón de interruptor comienza con 'access', lo que indica que es el botón 'Aplicar' o 'Access' para un usuario.
            \item Cuando se detecta una pulsación en uno de estos botones:
                \begin{itemize}
                    \item Extrae el ID del usuario: El método divide el ID del botón (por ejemplo, access-user ID) y toma la segunda parte (el índice 1) como el ID del usuario.
                    \item Lee el estado de los interruptores: Luego, el método lee el estado actual de los interruptores de permisos de lectura y escritura asociados a ese usuario específico, identificándolos por sus IDs (por ejemplo, read user ID y write user ID).
                    \item Envía la Solicitud POST: Finalmente, el método JavaScript envía una solicitud POST a la aplicación Flask en el servidor. La solicitud se formatea como 'grant - user ID - read state - right stick', donde 'right stick' se refiere al estado del interruptor de escritura. Esta acción se realiza utilizando una función similar a send event, que se usa para enviar solicitudes POST al servidor.
                \end{itemize}
        \end{itemize}
    \item 3. El Endpoint en la Aplicación Flask (Lado del Servidor):
        \begin{itemize}
            \item La aplicación Flask debe tener un endpoint configurado para recibir esta solicitud POST de 'grant user ID read and write'.
            \item Verificación de Administrador: Es crucial que la aplicación Flask verifique que la solicitud proviene de un usuario administrador antes de procesarla. Si no es un administrador, el servidor enviará una respuesta de 'acceso denegado'.
            \item Acciones del Servidor (si se concede el acceso): Si la verificación de administrador es exitosa, el servidor realiza dos acciones principales:
                \begin{itemize}
                    \item 1. Almacena los nuevos permisos: Guarda los permisos de lectura y escritura del usuario en la base de datos.
                    \item 2. Llama al servidor PubNub: Contacta al servidor PubNub para otorgar o revocar el acceso de lectura y escritura en tiempo real a ese usuario específico.                
                \end{itemize}
            \item Paso Previo: Es importante señalar que la concesión de estos permisos de lectura y escritura es un segundo paso. El primer paso es generar la clave de autorización para ese usuario específico y almacenarla en la base de datos.        
        \end{itemize}
\end{itemize}
En resumen, el método JavaScript en main.js es el motor que traduce la interacción del administrador con el botón 'Aplicar' en una 
solicitud estructurada enviada al servidor Flask. Esta interacción bidireccional cliente-servidor, con su correspondiente validación 
y procesamiento en el backend, asegura que los cambios de permisos de los usuarios se apliquen de manera segura y en tiempo real.

\subsubsection{ESCUCHA CLICKS EN BOTONES CON ID QUE EMPIEZAN POR 'XS'}
% 6.6.1.1 ESCUCHA CLICKS EN BOTONES CON ID QUE EMPIEZAN POR 'XS'
En el contexto más amplio del Método JavaScript en main.js, el aspecto de escuchar clics en botones con ID que empieza por 'XS' 
es fundamental para la interacción de los administradores con el panel de control, específicamente para la gestión de permisos de usuario.
Aquí se detalla lo que se dice al respecto:
\begin{itemize}
    \item 1. Propósito del Método JavaScript en main.js:
        \begin{itemize}
            \item El archivo main.js contiene un método diseñado para escuchar (listen on) cualquier botón de interruptor ('switch button') 
            que se encuentre en el dashboard.
            \item Este método es esencial para capturar las interacciones del usuario con los controles del panel.        
        \end{itemize}
    \item 2. Identificación de Botones Relevantes (ID que empieza por 'XS'):
        \begin{itemize}
            \item Para identificar el botón 'Aplicar Cambios' de entre todos los botones del dashboard, el método JavaScript verifica si el 
            ID de un botón de interruptor 'starts with XS'.
            \item A pesar de la fonética 'XS', este identificador se refiere al botón 'Aplicar Cambios', cuyo ID es access user ID. La 
            intención es detectar el botón cuyo ID comienza con 'access'.            
        \end{itemize}
    \item 3. Funcionalidad al Detectar un Clic:
        \begin{itemize}
            \item Una vez que el método JavaScript detecta que se ha presionado un botón cuyo ID comienza con 'XS' (es decir, el botón 
            'Aplicar Cambios' para un usuario específico):
                \begin{itemize}
                    \item Extrae el ID del usuario: El método divide el ID del botón (por ejemplo, access-user ID) y toma la segunda parte 
                    (el índice 1) como el ID del usuario.
                    \item Lee el estado de los interruptores: Procede a leer el estado actual de los botones de tipo interruptor de lectura 
                    (read user ID) y escritura (write user ID) asociados a ese mismo usuario.
                    \item Envía la solicitud POST: Finalmente, utiliza una función similar a send event para enviar una solicitud POST a la 
                    aplicación Flask en el servidor. Esta solicitud incluye el ID del usuario, el estado de lectura (read state) y el estado 
                    de escritura (right stick), siguiendo el formato \verb|'grant - user ID - read state - right stick'|.                            
                \end{itemize}
        \end{itemize}
\end{itemize}
En resumen, la escucha de clics en botones con IDs que empiezan por 'XS' es el mecanismo que el método JavaScript en main.js utiliza 
para identificar cuándo un administrador desea aplicar cambios de permisos a un usuario. Esta detección es el primer paso en un flujo 
cliente-servidor que culminará en la actualización de los permisos en la base de datos y la plataforma PubNub en tiempo real, tras la 
validación de que la solicitud proviene de un usuario administrador.

\subsubsection{EXTRAE USER ID DEL ID DEL BOTóN}
%6.6.1.2 EXTRAE USER ID DEL ID DEL BOTóN
En el contexto más amplio del Método JavaScript en main.js, la extracción del ID del usuario del ID del botón es un paso 
fundamental para identificar al usuario específico cuyos permisos se van a modificar en el panel de control del administrador.
Aquí se detalla lo que se dice sobre este proceso:
\begin{itemize}
    \item 1. Propósito del Método JavaScript en main.js:
        \begin{itemize}
            \item El archivo main.js contiene un método diseñado para escuchar los clics en cualquier botón de tipo interruptor ('switch button') que se encuentre en el dashboard.
            \item Este método es esencial para capturar las interacciones del administrador cuando desea modificar los permisos de los usuarios en línea.        
        \end{itemize}
    \item 2. Identificación del Botón 'Aplicar Cambios':
        \begin{itemize}
            \item Para saber qué botón ha sido pulsado, el método JavaScript verifica si el ID de un botón de interruptor 'starts with XS'. Aunque se menciona 'XS', en el contexto de la interfaz, se refiere al botón 'Aplicar Cambios' cuyo ID es access user ID.        
        \end{itemize}
    \item 3. Proceso de Extracción del ID del Usuario:
        \begin{itemize}
            \item Una vez que el método JavaScript detecta un clic en un botón cuyo ID comienza con access (o XS), procede a dividir este ID en dos partes usando el guion - como separador.
            \item Luego, toma la segunda parte (el índice 1) de este ID dividido, y esta parte es el ID del usuario.
            \item Por ejemplo, si el ID del botón es access-user ID, el método lo dividiría en ['access', 'user ID'] y tomaría 'user ID' como el identificador del usuario.
        \end{itemize}
    \item 4. Uso del ID del Usuario Extraído:
        \begin{itemize}
            \item Una vez que el ID del usuario ha sido extraído, el método JavaScript lo utiliza junto con el estado actual de los interruptores de permisos de lectura y escritura (read state y right stick) de ese usuario.
            \item Toda esta información se envía en una solicitud POST a la aplicación Flask en el servidor, con un formato como 'grant - user ID - read state - right stick'. Esta solicitud activa el endpoint en el servidor para procesar los cambios de permisos.
        \end{itemize}
\end{itemize}
En resumen, la extracción del ID del usuario del ID del botón por parte del método JavaScript en main.js es un mecanismo inteligente y dinámico. 
Permite al sistema identificar con precisión a qué usuario corresponden los cambios de permisos deseados por el administrador, lo cual es vital 
para una gestión efectiva y segura de los accesos en tiempo real.

\subsubsection{LEE ESTADO DE BOTONES DE LECTURA ESCRITURA}
% 6.6.1.3 LEE ESTADO DE BOTONES DE LECTURA ESCRITURA
En el contexto más amplio del Método JavaScript en main.js, la acción de leer el estado de los botones de lectura y escritura es una 
función crítica que permite al sistema capturar las preferencias de permisos establecidas por el administrador para un usuario en línea específico.
Aquí se detalla lo que se dice sobre este proceso:
\begin{itemize}
    \item 1. Propósito del Método JavaScript en main.js:
        \begin{itemize}
            \item El archivo main.js contiene un método diseñado para escuchar los clics en cualquier botón de tipo interruptor 
            ('switch button') que se encuentre en el dashboard.
            \item Este método es esencial para manejar las interacciones de los administradores con el panel de control, especialmente 
            cuando desean modificar los permisos de los usuarios en línea.
        \end{itemize}
    \item 2. Activación de la Lectura de Estados:
        \begin{itemize}
            \item La lectura de los estados de los botones de lectura y escritura se activa cuando el método JavaScript detecta un clic 
            en el botón 'Aplicar Cambios'.
            \item Este botón 'Aplicar Cambios' se identifica porque su ID comienza con 'XS' (que en la práctica se refiere a IDs como 
            access user ID).
        \end{itemize}
    \item 3. Proceso de Lectura de los Estados:
        \begin{itemize}
            \item Después de identificar el botón 'Aplicar Cambios' y extraer el ID del usuario asociado, el método JavaScript procede a leer 
            el estado del interruptor de lectura y del interruptor de escritura.
            \item Para hacer esto, se especifica el ID de cada uno de estos interruptores (por ejemplo, read user ID y write user ID), lo que 
            permite al JavaScript obtener su estado actual (si están 'checked' o 'unchecked').
            \item Los botones de lectura y escritura son interruptores que se configuran inicialmente como 'checked' (activos), pero su estado 
            puede ser modificado por el administrador. El estado real ('checked' o 'unchecked') se determina a partir de los datos del servidor 
            (índice 2 para lectura y 3 para escritura del registro del usuario) cuando se carga la página.
        \end{itemize}
    \item 4. Uso de los Estados Leídos:
        \begin{itemize}
            \item Una vez que se han leído los estados de los botones de lectura (read state) y escritura (right stick), esta información, junto 
            con el ID del usuario extraído, se empaca en una solicitud POST.
            \item Esta solicitud se envía a la aplicación Flask en el servidor con un formato específico: 
            \verb|grant - user ID - read state - right stick|.
            \item El servidor utiliza esta información para almacenar los nuevos permisos de lectura y escritura del usuario en la base de 
            datos y para llamar al servidor PubNub con el fin de otorgar o revocar el acceso en tiempo real a ese usuario específico.
        \end{itemize}
\end{itemize}
En resumen, la capacidad del método JavaScript en main.js para leer el estado de los botones de lectura y escritura es fundamental. 
Es el mecanismo mediante el cual la interfaz de usuario traduce la intención del administrador en datos procesables que luego se envían al 
servidor para la gestión y aplicación en tiempo real de los permisos de acceso de los usuarios.

\subsubsection{ENVIA SOLISITUD POST AL SERVIDOR}
% 6.6.1.4 ENVIA SOLISITUD POST AL SERVIDOR
En el contexto más amplio del Método JavaScript en main.js, la acción de enviar una solicitud POST al servidor con el formato 
\verb|grant-user ID-read state-write state| es el paso culminante que traduce la interacción del administrador en la interfaz de usuario a una 
acción concreta en el backend para gestionar los permisos.
Aquí se detalla lo que se dice sobre este envío de solicitud POST:
\begin{itemize}
    \item 1. Activación por el Botón 'Aplicar Cambios':
        \begin{itemize}
            \item El método JavaScript en main.js está diseñado para escuchar cualquier botón en el dashboard.
            \item Específicamente, verifica si el ID de un botón de tipo interruptor 'starts with XS', lo que, como hemos visto, se 
            refiere al botón 'Aplicar Cambios' (ID: access user ID) presente en cada fila de usuario para aplicar las modificaciones de permisos.
        \end{itemize}
    \item 2. Preparación de la Solicitud por JavaScript:
        \begin{itemize}
            \item Una vez que el método detecta un clic en el botón 'Aplicar Cambios':
            \begin{itemize}
                \item Extracción del ID del usuario: Divide el ID del botón pulsado (por ejemplo, access-user ID) utilizando el 
                guion - como separador y toma la segunda parte (el índice 1) como el ID del usuario.
                \item Lectura de los estados de permisos: Luego, lee el estado actual del interruptor de lectura (read state) y el 
                estado del interruptor de escritura (right stick) para ese usuario específico, identificándolos por sus respectivos 
                IDs (por ejemplo, read user ID y write user ID).
            \end{itemize}
            \item Estos datos (ID de usuario, estado de lectura y estado de escritura) son cruciales para formar la solicitud.
        \end{itemize}
    \item 3. Formato y Envío de la Solicitud POST:
        \begin{itemize}
            \item El método JavaScript envía la solicitud al servidor con el formato específico: \verb|grant - user ID - read state - right stick|.
            \item Esta acción se realiza mediante una función similar a send event, que se utilizó en partes anteriores del proyecto 
            para enviar solicitudes POST al servidor.
        \end{itemize}
    \item 4. Recepción y Procesamiento en el Servidor (Aplicación Flask):
        \begin{itemize}
            \item La aplicación Flask en el servidor está equipada con un endpoint específico para recibir esta solicitud POST, que se 
            describe como grant user ID read and write.
            \item Verificación de Administrador: Un paso fundamental en el servidor es verificar que la solicitud grant provenga de un 
            usuario administrador. Si la solicitud no es de un administrador, el servidor enviará una respuesta de 'acceso denegado'.
            \item Acciones del Servidor (si la verificación es exitosa): Si la solicitud es válida y proviene de un administrador, el 
            servidor realiza dos acciones clave:
                \begin{itemize}
                    \item 1. Almacena los nuevos permisos de lectura y escritura del usuario en la base de datos.
                    \item 2. Llama al servidor PubNub para otorgar o revocar el acceso de lectura y escritura en tiempo real a ese usuario específico.
                \end{itemize}
            \item Paso Previo: Es importante señalar que la concesión de estos permisos es un segundo paso. El primer paso es generar y 
            almacenar una clave de autorización para el usuario específico en la base de datos.
        \end{itemize}
\end{itemize}

\subsection{Endpoint en FLASK para Recibir Solisitud \textbf{grant user-id read and write}}
% 6.6.2 Endpoint en Flask para Recibir Solicitud  'grant user-id read and write '.
En el contexto más amplio del Endpoint para el Botón 'Aplicar', el Endpoint Flask para Recibir Solicitud es el componente en el lado 
del servidor responsable de procesar los cambios de permisos de usuario iniciados por un administrador a través de la interfaz de 
usuario en index.html.
Esto es lo que se dice sobre este endpoint:
\begin{itemize}
    \item 1. Propósito y Origen de la Solicitud:
        \begin{itemize}
            \item Cuando un administrador interactúa con el botón 'Aplicar Cambios' (cuyo ID comienza con access user ID o XS) en el panel 
            de control, el método JavaScript en main.js envía una solicitud al servidor.
            \item Este botón permite cambiar los permisos de lectura y escritura de un usuario específico en tiempo real.
            \item La solicitud es de tipo POST, y su formato es grant - user ID - read state - right stick (donde right stick representa el 
            estado de escritura). Se describe también como una solicitud grant user ID read and write.
        \end{itemize}
    \item 2. Configuración del Endpoint en la Aplicación Flask:
        \begin{itemize}
            \item La aplicación Flask debe tener un endpoint configurado para recibir esta solicitud POST específica.
            \item Este endpoint está diseñado para manejar la lógica de otorgar permisos.
        \end{itemize}
    \item 3. Verificación de Seguridad (Administrador):
        \begin{itemize}
            \item Un paso crucial e indispensable en el lado del servidor es verificar que la solicitud grant provenga de un usuario 
            administrador. Esta es una medida de seguridad fundamental para evitar que usuarios no autorizados modifiquen permisos.
            \item Si la solicitud no es de un administrador, el servidor enviará una respuesta de 'acceso denegado'.
            \item Si la verificación es exitosa y la solicitud proviene de un administrador, el servidor procede con las acciones, 
            enviando implícitamente una respuesta de 'acceso concedido' (access granted).
        \end{itemize}
    \item 4. Acciones del Servidor tras la Verificación Exitosa:
        \begin{itemize}
            \item Una vez que se confirma que la solicitud es válida y proviene de un administrador, el servidor Flask realiza dos acciones principales:
                \begin{itemize}
                    \item 1. Almacenar los nuevos permisos: Guarda los permisos de lectura y escritura (read y write access) del usuario 
                    específico en la base de datos.
                    \item 2. Llamar al servidor PubNub: Contacta al servidor PubNub para otorgar o revocar el acceso de lectura y escritura 
                    en tiempo real a ese usuario específico. PubNub es una de las principales tecnologías utilizadas en el proyecto para 
                    protocolos de comunicación en tiempo real.
                \end{itemize}
        \end{itemize}
    \item 5. Paso Previo a la Concesión de Permisos:
        \begin{itemize}
            \item Es importante destacar que la concesión de estos permisos de lectura y escritura es un segundo paso. El primer paso que 
            lo precede es generar la clave de autorización para ese usuario específico y almacenarla en la base de datos.
        \end{itemize}
\end{itemize}
En resumen, el Endpoint Flask para Recibir Solicitud actúa como el punto de entrada seguro en el servidor para la gestión de permisos. 
Recibe solicitudes POST del cliente, realiza una estricta verificación de administrador y, si es válida, actualiza los permisos del usuario 
en la base de datos y en la plataforma de comunicación en tiempo real (PubNub), asegurando una gestión robusta y 
segura de los accesos en el entorno IoT.

\subsubsection{Verificar si la Solisitud Procede de un Usuario Administrador}
% 6.6.2.1 Verificar si la Solisitud Procede de un Usuario Administrador.
En el contexto más amplio del Endpoint Flask para Recibir Solicitud, la acción de verificar si la solicitud procede de un usuario 
administrador es una medida de seguridad crítica e indispensable para asegurar la integridad y el control de los permisos de los usuarios 
dentro del panel de control del servidor IoT.
Aquí se detalla lo que se dice sobre este proceso:
\begin{itemize}
    \item 1. Rol del Endpoint Flask:
        \begin{itemize}
            \item La aplicación Flask en el servidor tiene un endpoint específicamente configurado para recibir solicitudes POST que 
            provienen del lado del cliente (main.js).
            \item Estas solicitudes se activan cuando un administrador hace clic en el botón 'Aplicar Cambios' (ID: access user ID) en la 
            interfaz de usuario de index.html para modificar los permisos de un usuario en línea.
            \item El formato de la solicitud es \verb|grant - user ID - read state - right stick|, indicando el deseo de otorgar (o revocar) 
            permisos de lectura y escritura a un usuario específico.
        \end{itemize}
    \item 2. La Verificación de Seguridad del Administrador:
        \begin{itemize}
            \item Una vez que la solicitud grant user ID read and write es recibida por la aplicación Flask, un paso crucial 
            es 'verificar una vez más si esta solicitud grant proviene de un usuario administrador'.
            \item Esta verificación es fundamental porque el panel de control y sus funcionalidades de gestión de permisos están diseñados 
            para ser visibles y operables únicamente por usuarios administradores. Existe una sentencia if en el HTML que controla la 
            visibilidad de este panel.
        \end{itemize}
    \item 3. Resultados de la Verificación:
        \begin{itemize}
            \item Acceso Denegado: Si el servidor determina que la solicitud no proviene de un usuario administrador, enviará una 
            respuesta al solicitante con el mensaje 'acceso denegado' (access denied). Esto previene que usuarios no autorizados realicen 
            cambios en los permisos.
            \item Acceso Concedido: Si la verificación es exitosa y se confirma que la solicitud es válida y ha sido iniciada por un usuario 
            administrador, el servidor procede a ejecutar las acciones solicitadas y envía implícitamente una respuesta de 'acceso concedido' 
            (access granted).
        \end{itemize}
    \item 4. Acciones Posteriores a la Verificación Exitosa:
    \begin{itemize}
            \item Una vez que se ha verificado que la solicitud procede de un administrador, el servidor Flask realiza dos acciones principales:
                \begin{itemize}
                    \item 1. Almacenar los nuevos permisos: Guarda los permisos de lectura y escritura del usuario específico en la base de datos.
                    \item 2. Llamar al servidor PubNub: Contacta al servidor PubNub para otorgar o revocar el acceso de lectura y escritura en 
                    tiempo real a ese usuario o dispositivo específico.
                \end{itemize}
            \item Es importante recordar que la concesión de estos permisos es un segundo paso; el primer paso es generar la clave de autorización 
            para el usuario y almacenarla en la base de datos.
    \end{itemize}
\end{itemize}
En síntesis, la verificación de que la solicitud procede de un usuario administrador dentro del Endpoint Flask es una piedra angular de la 
seguridad del sistema. Actúa como un guardián, permitiendo que solo los usuarios autorizados realicen cambios sensibles en los permisos de 
otros usuarios, asegurando así un control robusto y seguro en el entorno del Internet de las Cosas.

\subsubsection{Responder 'Acceso Denegado' o 'Acceso Concedido'}
% 6.6.2.2 Responder 'Acceso Denegado' o 'Acceso Concedido'
En el contexto más amplio del Endpoint Flask para Recibir Solicitud, la acción de responder 'Acceso Denegado' o 'Acceso Concedido' 
es el resultado directo y crucial de la verificación de seguridad que el servidor realiza al procesar las solicitudes de cambio de permisos.
Aquí se detalla lo que se dice sobre estas respuestas:
\begin{itemize}
    \item 1. Contexto del Endpoint Flask:
        \begin{itemize}
            \item La aplicación Flask en el servidor está configurada con un endpoint para recibir solicitudes POST provenientes del código 
            JavaScript en main.js.
            \item Estas solicitudes se activan cuando un administrador interactúa con el botón 'Aplicar Cambios' 
            (cuyo ID comienza con access user ID) en el panel de control del index.html.
            \item La solicitud tiene el formato grant - user ID - read state - right stick o grant user ID read and write, y su propósito es 
            modificar los permisos de lectura y escritura de un usuario en línea.
        \end{itemize}
    \item 2. La Verificación de Seguridad:
        \begin{itemize}
            \item Un paso crucial e indispensable en el lado del servidor es 'verificar una vez más si esta solicitud grant proviene de un 
            usuario administrador'. Esta verificación es fundamental para la seguridad del sistema, ya que el panel de control para la 
            gestión de permisos está diseñado para ser accesible y operable solo por administradores.
        \end{itemize}
    \item 3. Respuestas Basadas en la Verificación:
        \begin{itemize}
            \item 'Acceso Denegado' (access denied): Si el servidor determina que la solicitud no procede de un usuario administrador, 
            enviará una respuesta al solicitante (es decir, al cliente que envió la solicitud) con el mensaje 'acceso denegado'. Esto sirve 
            para prevenir que usuarios no autorizados puedan realizar cambios sensibles en los permisos.
            \item 'Acceso Concedido' (access granted): Si la verificación es exitosa, lo que significa que la solicitud es válida y proviene 
            de un usuario administrador, el servidor procede a ejecutar las acciones solicitadas. Se menciona que si 'todo sale bien' 
            (if everything goes well), se enviará la respuesta de 'acceso concedido' (access granted). Esta respuesta implícita o 
            explícita señala que la operación ha sido autorizada y se procederá con la gestión de permisos.
        \end{itemize}
    \item 4. Manejo de la Respuesta en el Lado del Cliente (JavaScript):
        \begin{itemize}
            \item El main.js en el lado del cliente está preparado para recibir esta respuesta del servidor.
            \item Una vez recibida, el JavaScript extrae el JSON de la respuesta y verifica si contiene la clave 'access'. Si el valor 
            asociado a 'access' es granted (es decir, 'acceso concedido'), se restablece la suscripción al canal nuevamente. Esto sugiere 
            que, tras la confirmación de los permisos, podría ser necesario actualizar las suscripciones en tiempo real para reflejar los 
            cambios. También se recomienda modificar el método de suscripción para que indique si la suscripción fue exitosa.
        \end{itemize}
    \item 5. Acciones Posteriores a un 'Acceso Concedido':
        \begin{itemize}
            \item Si la verificación es exitosa y se concede el acceso, el servidor Flask realiza dos acciones principales:
                \begin{itemize}
                    \item 1. Almacenar los nuevos permisos de lectura y escritura del usuario en la base de datos.
                    \item 2. Llamar al servidor PubNub para otorgar o revocar el acceso de lectura y escritura a ese usuario específico en tiempo real.
                \end{itemize}
            \item Cabe recordar que la concesión de estos permisos es un segundo paso, ya que el primer paso es generar la clave de 
            autorización para el usuario y almacenarla en la base de datos.
        \end{itemize}
\end{itemize}
En conclusión, el Endpoint Flask utiliza la respuesta de 'Acceso Denegado' o 'Acceso Concedido' como un mecanismo de retroalimentación 
esencial para comunicar el resultado de su verificación de seguridad al cliente. Esta funcionalidad no solo protege el sistema de accesos 
no autorizados, sino que también permite al lado del cliente reaccionar adecuadamente a la confirmación de los cambios de permisos, 
manteniendo la coherencia y la seguridad en la gestión en tiempo real de los dispositivos IoT.

\subsubsection{Almacenar Permisos de Lectura/Escritura en la Base de Datos}
% 6.6.2.3 Almacenar Permisos de Lectura/Escritura en la Base de Datos.
En el contexto más amplio del Endpoint Flask para Recibir Solicitud, la acción de almacenar los permisos de lectura y escritura 
en la base de datos es una función principal y esencial que el servidor realiza después de verificar la validez y la autorización 
de una solicitud de cambio de permisos.
Aquí se detalla lo que se dice sobre este proceso:
\begin{itemize}
    \item 1. Activación de la Solicitud y Rol del Endpoint Flask:
        \begin{itemize}
            \item El método JavaScript en main.js envía una solicitud POST a la aplicación Flask en el servidor cuando un administrador 
            hace clic en el botón 'Aplicar Cambios' (cuyo ID comienza con access user ID) en el panel de control del index.html.
            \item Esta solicitud, con un formato como grant - user ID - read state - right stick o grant user ID read and write, tiene como 
            objetivo modificar los permisos de lectura y escritura de un usuario en línea.
            \item El endpoint Flask está configurado específicamente para recibir y procesar estas solicitudes de otorgamiento de permisos.
        \end{itemize}
    \item 2. Verificación de Seguridad Previa al Almacenamiento:
        \begin{itemize}
            \item Antes de almacenar cualquier cambio de permisos, el servidor realiza una verificación crucial: 'comprobar una vez más si 
            esta solicitud grant proviene de un usuario administrador'.
            \item Esta medida de seguridad es indispensable, ya que el panel de control de permisos es visible y operable solo para 
            usuarios administradores.
            \item Si la solicitud no es de un administrador, el servidor responderá con 'acceso denegado'. Solo si la verificación es 
            exitosa ('todo sale bien' y se obtiene 'acceso concedido') se procede con el almacenamiento.
        \end{itemize}
    \item 3. Almacenamiento de Permisos en la Base de Datos:
        \begin{itemize}
            \item Una vez que la solicitud ha sido validada y confirmada como proveniente de un usuario administrador, una de las dos 
            acciones principales que el servidor Flask debe realizar es 'almacenar los permisos de lectura y escritura del usuario 
            en la base de datos'.
            \item Esto implica registrar el estado actual (otorgado o revocado) de los permisos de lectura y escritura para el user ID 
            especificado en la solicitud.
        \end{itemize}
    \item 4. Contexto de los Pasos de Almacenamiento:
        \begin{itemize}
            \item Las fuentes aclaran que el 'otorgar permisos de lectura y escritura es un segundo paso'.
            \item El primer paso que precede a esto es 'generar la clave de autorización para ese usuario específico y almacenarla en la 
            base de datos'. Esto sugiere que la base de datos no solo guarda los estados de lectura y escritura, sino también una clave 
            fundamental para la autorización del usuario.
        \end{itemize}
    \item 5. Integración con PubNub:
        \begin{itemize}
            \item Además de almacenar los permisos en la base de datos, el servidor también realiza la segunda acción principal: 
            'llamar al servidor PubNub para otorgar acceso de lectura y escritura a este usuario específico' en tiempo real. 
            Esto asegura que los cambios no solo se persistan, sino que también se apliquen inmediatamente en la plataforma de comunicación.
        \end{itemize}
\end{itemize}
En resumen, la capacidad de almacenar los permisos de lectura y escritura en la base de datos es una funcionalidad central del Endpoint 
Flask para Recibir Solicitud. Este proceso se lleva a cabo de forma segura, precedido por una estricta verificación de administrador y 
como parte de una secuencia de dos pasos que incluye la generación de claves de autorización, asegurando así una gestión robusta y persistente 
de los accesos de usuario en el sistema IoT.

\subsubsection{Llamar a Pub/Nub para Conceder Acceso al Usuario}
% 6.6.2.4 Llamar a Pub/Nub para Conceder Acceso al Usuario.
En el contexto más amplio del Endpoint Flask para Recibir Solicitud, la acción de llamar a PubNub para conceder acceso al usuario es una de 
las dos funciones principales y críticas que el servidor realiza después de validar y autorizar una solicitud de cambio de permisos. 
Esta acción asegura que los cambios se apliquen en tiempo real a la plataforma de comunicación.
Aquí se detalla lo que se dice sobre este proceso:
\begin{itemize}
    \item 1. Activación de la Solicitud y Rol del Endpoint Flask:
        \begin{itemize}
            \item El método JavaScript en main.js envía una solicitud POST a la aplicación Flask en el servidor cuando un administrador 
            hace clic en el botón 'Aplicar Cambios' (cuyo ID comienza con access user ID) en el panel de control del index.html.
            \item Esta solicitud, con un formato como grant - user ID - read state - right stick o grant user ID read and write, tiene como 
            objetivo modificar los permisos de lectura y escritura de un usuario en línea.
            \item El endpoint Flask está configurado específicamente para recibir y procesar estas solicitudes.
        \end{itemize}
    \item 2. Verificación de Seguridad Previa a la Llamada a PubNub:
        \begin{itemize}
            \item Antes de realizar cualquier acción, el servidor realiza una verificación crucial: 'comprobar una vez más si esta solicitud 
            grant proviene de un usuario administrador'.
            \item Esta medida de seguridad es indispensable, ya que el panel de control de permisos es visible y operable solo para usuarios 
            administradores.
            \item Si la solicitud no es de un administrador, el servidor responderá con 'acceso denegado'. Solo si la verificación es exitosa 
            ('todo sale bien' y se obtiene 'acceso concedido') se procede con la llamada a PubNub.
        \end{itemize}
    \item 3. Llamada a PubNub para Conceder Acceso:
        \begin{itemize}
            \item Una vez que la solicitud ha sido validada y confirmada como proveniente de un usuario administrador, una de las dos acciones 
            principales que el servidor Flask debe realizar es 'llamar al servidor PubNub para conceder acceso de lectura y escritura a este 
            usuario específico'.
            \item Esto garantiza que los cambios de permisos no solo se persistan en la base de datos, sino que también se apliquen inmediatamente 
            en la plataforma de comunicación en tiempo real.
        \end{itemize}
    \item 4. Propósito y Beneficios de Usar PubNub:
        \begin{itemize}
            \item PubNub es una de las principales tecnologías utilizadas en el proyecto para los protocolos de comunicación en tiempo real.
            \item Se utiliza la funcionalidad de PubNub Access Manager para que los usuarios administradores puedan 'otorgar acceso de lectura 
            y escritura en tiempo real a todos los usuarios que no son administradores y a los dispositivos'.
            \item Esto permite una 'comunicación en tiempo real, fuerte, segura y escalable' para el ecosistema IoT. PubNub es una solución para 
            protocolos de comunicación en tiempo real y ligeros.
        \end{itemize}
    \item 5. Contexto de los Pasos:
        \begin{itemize}
            \item Las fuentes aclaran que el 'otorgar permisos de lectura y escritura (a través de PubNub) es un segundo paso'.
            \item El primer paso que precede a esto es 'generar la clave de autorización para ese usuario específico y almacenarla en la base de datos'. 
            Esto indica que la autorización en PubNub se basa en una clave que ya debe existir.
        \end{itemize}
\end{itemize}
En resumen, la llamada a PubNub para conceder acceso al usuario es una función central del Endpoint Flask, que se ejecuta de forma 
segura después de una verificación de administrador. Esta acción es vital para aplicar los cambios de permisos de lectura y escritura 
en tiempo real a través de la plataforma PubNub, facilitando así una gestión dinámica y segura de los accesos de usuarios y dispositivos 
en el sistema IoT.

\subsection{Manejar Respuesta en \textbf{main.js}}
% 6.6.3 Manejar Respuesta en  'main.js'.
En el contexto más amplio del Endpoint Flask para el Botón 'Aplicar', el manejo de la respuesta en main.js es el paso final 
en el lado del cliente, donde se procesa la confirmación del servidor sobre los cambios de permisos solicitados, permitiendo 
que la interfaz de usuario reaccione adecuadamente.
Aquí se detalla lo que se dice sobre cómo main.js maneja esta respuesta:
\begin{itemize}
    \item 1. Contexto de la Solicitud y el Endpoint:
        \begin{itemize}
            \item Recordemos que el método JavaScript en main.js es el encargado de enviar una solicitud POST a la aplicación 
            Flask en el servidor. Esto ocurre cuando un administrador presiona el botón 'Aplicar Cambios' (cuyo ID comienza con 
            access user ID) para modificar los permisos de lectura y escritura de un usuario en línea.
            \item El Endpoint Flask recibe esta solicitud (grant user ID read and write), realiza una verificación crucial para asegurar 
            que proviene de un usuario administrador y, si todo es válido, procede a almacenar los permisos en la base de datos y a llamar 
            a PubNub para conceder el acceso en tiempo real.
        \end{itemize}
    \item 2. Recepción de la Respuesta en main.js:
        \begin{itemize}
            \item El archivo main.js está preparado para recibir la respuesta que el servidor Flask envía después de procesar la solicitud.
            \item El servidor envía una respuesta que puede ser 'acceso denegado' si la solicitud no proviene de un administrador, o 
            'acceso concedido' si la verificación es exitosa y se procede con los cambios.
        \end{itemize}
    \item 3. Procesamiento de la Respuesta JSON:
        \begin{itemize}
            \item Una vez recibida la respuesta, el código JavaScript en main.js debe extraer el JSON de esta respuesta.
            \item Después de extraer el JSON, se verifica si contiene la clave 'access'. Esta clave es la que indica el estado de la autorización.
        \end{itemize}
    \item 4. Acción al Recibir 'Acceso Concedido':
        \begin{itemize}
            \item Si el valor asociado a la clave 'access' es 'granted' (es decir, 'acceso concedido'), entonces main.js procede a 
            'restablecer la suscripción al canal nuevamente' (reset scribe on the channel again). Esto sugiere que, una vez confirmados 
            los cambios de permisos, es posible que sea necesario reconfigurar las suscripciones en tiempo real para reflejar las nuevas 
            autorizaciones del usuario en la plataforma de comunicación (probablemente PubNub).
        \end{itemize}
    \item 5. Recomendación para el Método SUBSCRIBE:
        \begin{itemize}
            \item Las fuentes también mencionan que 'es bueno modificar el método SUBSCRIBE para que nos diga si se ha suscrito con éxito o no' 
            (it is also good to modify the SUBSCRIBE method to let us know whether it has subscribed successfully or not). Esta es una mejora 
            sugerida para proporcionar una mejor retroalimentación sobre el éxito de la operación de suscripción después de que se conceden los permisos.
        \end{itemize}
\end{itemize}
En resumen, el manejo de la respuesta en main.js es un componente reactivo y vital en el flujo de gestión de permisos. Se encarga 
de recibir, interpretar y actuar sobre la confirmación del servidor (especialmente el 'acceso concedido'), asegurando que la 
interfaz de usuario y las suscripciones en tiempo real se sincronicen con los cambios de permisos aplicados por el administrador.

\subsubsection{Extraer JSON de la Respuesta}
% 6.6.3.1 Extraer JSON de la Respuesta.
En el contexto más amplio del Manejo de Respuesta en main.js, la acción de extraer el JSON de la respuesta es un paso fundamental 
que permite al lado del cliente interpretar la decisión del servidor sobre la solicitud de cambio de permisos y actuar en consecuencia.
Aquí se detalla lo que se dice sobre este proceso:
\begin{itemize}
    \item 1. Ciclo de Comunicación Cliente-Servidor:
        \begin{itemize}
            \item Previamente, el método JavaScript en main.js envía una solicitud POST al endpoint de la aplicación Flask en el servidor. 
            Esta solicitud se activa cuando un administrador presiona el botón 'Aplicar Cambios' (ID: access user ID) en el index.html para 
            modificar los permisos de lectura y escritura de un usuario.
            \item El servidor Flask procesa esta solicitud, que incluye una verificación crucial para determinar si proviene de un usuario 
            administrador. Tras esta verificación y la aplicación de los cambios (almacenamiento en la base de datos y llamada a PubNub), el 
            servidor envía una respuesta de vuelta al cliente.
        \end{itemize}
    \item 2. Recepción y Extracción del JSON en main.js:
        \begin{itemize}
            \item El código JavaScript en main.js está diseñado para recibir esta respuesta del servidor.
            \item El siguiente paso es 'extraer el JSON' (extract the JSON) de la respuesta recibida. Esto es esencial porque la respuesta del 
            servidor probablemente está formateada en JSON para facilitar el intercambio estructurado de datos.
        \end{itemize}
    \item 3. Interpretación de la Respuesta para 'Acceso Concedido' o 'Denegado':
        \begin{itemize}
            \item Una vez extraído el JSON, main.js procede a 'verificar si tiene la clave 'access'' (check whether it has the key as access).
            \item Esta clave 'access' es la que indica el estado de la autorización otorgada por el servidor. Si el valor asociado a 'access' es 
            'granted' (es decir, 'acceso concedido'), significa que la solicitud de cambio de permisos fue validada y ejecutada con éxito por el servidor.
            \item Si la verificación de administrador en el servidor fallara, la respuesta contendría un 'acceso denegado' (access denied). 
            Aunque la fuente no detalla cómo main.js reacciona específicamente a 'acceso denegado', la estructura implica que solo se procede 
            con acciones si el acceso es 'concedido'.
        \end{itemize}
    \item 4. Acciones Posteriores a la Extracción Exitosa del JSON:
        \begin{itemize}
            \item Si el JSON indica que el 'access' fue 'granted', main.js realiza la acción de 'restablecer la suscripción al canal nuevamente' 
            (reset scribe on the channel again). Esto sugiere que, una vez que los permisos se han modificado y confirmado, las suscripciones en 
            tiempo real (posiblemente a través de PubNub) deben actualizarse para reflejar las nuevas capacidades del usuario.
            \item Además, se recomienda modificar el método SUBSCRIBE para que indique si la suscripción fue exitosa o no, lo que mejoraría la 
            retroalimentación sobre el proceso.
        \end{itemize}
\end{itemize}
En resumen, la extracción del JSON de la respuesta en main.js es un paso crucial en el ciclo de gestión de permisos. Permite al 
cliente interpretar de manera estructurada el resultado de la operación del servidor (si el acceso fue concedido o denegado) y, 
en caso de éxito, realizar las acciones necesarias en la interfaz de usuario y en la comunicación en tiempo real para mantener la 
coherencia y la funcionalidad del sistema.

\subsubsection{Si 'access' es 'granted', re-suscribirse al canal}
% 6.6.3.2 Si 'access' es 'granted', re-suscribirse al canal.
En el contexto más amplio del Manejo de Respuesta en main.js, la acción de re-suscribirse al canal si 'access' es 'granted' es un paso crucial y 
consecuente en el lado del cliente, que asegura que el sistema se adapte dinámicamente a los cambios de permisos del usuario confirmados por el servidor.
Aquí se detalla lo que se dice sobre este proceso:
\begin{itemize}
    \item 1. Contexto de la Interacción y el Rol de main.js:
        \begin{itemize}
            \item El método JavaScript en main.js es el encargado de iniciar una solicitud POST al servidor Flask. Esto sucede cuando un 
            administrador interactúa con el botón 'Aplicar Cambios' (cuyo ID empieza con access user ID) en el panel de control del index.html 
            para modificar los permisos de lectura y escritura de un usuario en línea.
            \item El servidor Flask procesa esta solicitud, realizando una verificación de seguridad crucial para asegurar que proviene de un 
            usuario administrador. Tras la verificación y la aplicación de los cambios (almacenamiento en la base de datos y llamada a PubNub), 
            el servidor envía una respuesta al cliente.
        \end{itemize}
    \item 2. Recepción y Procesamiento de la Respuesta en main.js:
        \begin{itemize}
            \item main.js está diseñado para recibir esta respuesta del servidor.
            \item El primer paso es extraer el JSON de la respuesta. Este formato estructurado es esencial para que el cliente pueda interpretar 
            los datos enviados por el servidor.
            \item Luego, main.js verifica si el JSON contiene la clave 'access'. Esta clave es la que indica el estado de la autorización o el 
            resultado de la operación en el servidor.
        \end{itemize}
    \item 3. Condición para Re-suscripción ('access' es 'granted'):
        \begin{itemize}
            \item Si el valor asociado a la clave 'access' es 'granted' (es decir, 'acceso concedido'), esto significa que la solicitud de cambio 
            de permisos fue validada, ejecutada con éxito por el servidor, y que la verificación de administrador fue positiva.
            \item En este caso de éxito, el código JavaScript en main.js procederá a 'restablecer la suscripción al canal nuevamente' 
            (reset scribe on the channel again).
        \end{itemize}
    \item 4. Implicaciones y Propósito de la Re-suscripción:
        \begin{itemize}
            \item Esta acción sugiere que, una vez que los permisos de un usuario han sido modificados y confirmados por el servidor, las 
            suscripciones en tiempo real (probablemente a través de PubNub, que se usa para comunicación en tiempo real y protocolos ligeros) 
            deben actualizarse.
            \item Al re-suscribirse, el sistema asegura que el usuario (o el dispositivo) reciba los datos de acuerdo con sus nuevos permisos 
            de lectura y escritura, manteniendo la coherencia entre el backend y el frontend en un ecosistema IoT en tiempo real. PubNub es clave 
            para otorgar acceso de lectura y escritura en tiempo real a usuarios y dispositivos.
        \end{itemize}
    \item 5. Mejora Sugerida para el Método SUBSCRIBE:
        \begin{itemize}
            \item Las fuentes también mencionan que 'es bueno modificar el método SUBSCRIBE para que nos diga si se ha suscrito con éxito o no'. 
            Esto indica la importancia de tener una retroalimentación clara sobre el resultado de la operación de suscripción, lo cual es vital 
            para el diagnóstico y la estabilidad del sistema.
        \end{itemize}
\end{itemize}
En resumen, la lógica en main.js para re-suscribirse al canal si 'access' es 'granted' es un componente fundamental para la reactividad y 
la coherencia del sistema de gestión de permisos en tiempo real. Permite que la interfaz de usuario y los canales de comunicación se ajusten 
inmediatamente después de que el servidor haya confirmado y aplicado los cambios de acceso, garantizando que el ecosistema IoT funcione de 
manera segura y actualizada.

\subsubsection{Modificar Método 'SUBSCRIBE para Confirmación de éxito}
% 6.6.3.3 Modificar Método 'SUBSCRIBE para Confirmación de éxito.
En el contexto más amplio del Manejo de Respuesta en main.js, la recomendación de modificar el método SUBSCRIBE para confirmación de éxito es 
una mejora sugerida para aumentar la robustez y la retroalimentación del sistema después de que se han aplicado cambios en los permisos de los usuarios.
Aquí se detalla lo que se dice sobre esta modificación:
\begin{itemize}
    \item 1. Contexto de la Interacción Cliente-Servidor en main.js:
        \begin{itemize}
            \item El método JavaScript en main.js es el punto de inicio para enviar una solicitud POST al servidor Flask. Esta solicitud se 
            dispara cuando un administrador presiona el botón 'Aplicar Cambios' (cuyo ID comienza con access user ID) en el index.html para modificar 
            los permisos de lectura y escritura de un usuario en línea.
            \item El servidor Flask procesa esta solicitud, realiza una verificación de seguridad crucial para asegurarse de que proviene de un usuario 
            administrador y, si todo es válido, procede a almacenar los permisos en la base de datos y a llamar a PubNub para conceder el acceso en 
            tiempo real.
            \item Posteriormente, el servidor envía una respuesta al cliente (main.js).
        \end{itemize}
    \item 2. Manejo de la Respuesta en main.js y la Re-suscripción:
        \begin{itemize}
            \item En main.js, una vez que se recibe la respuesta del servidor, el código JavaScript extrae el JSON de esta respuesta y verifica si 
            contiene la clave 'access'.
            \item Si el valor asociado a 'access' es 'granted' (es decir, 'acceso concedido'), esto indica que la solicitud de cambio de permisos 
            fue validada y ejecutada con éxito por el servidor.
            \item En este escenario de éxito, main.js realiza la acción de 'restablecer la suscripción al canal nuevamente' (reset scribe on the 
            channel again). Esto es crucial para que el usuario (o dispositivo) reciba actualizaciones de datos de acuerdo con sus nuevos permisos, 
            especialmente en un entorno de comunicación en tiempo real como el proporcionado por PubNub.
        \end{itemize}
    \item 3. La Recomendación de Modificar el Método SUBSCRIBE:
        \begin{itemize}
            \item Directamente relacionado con la acción de re-suscripción, las fuentes sugieren que 'es bueno modificar el método SUBSCRIBE para 
            que nos diga si se ha suscrito con éxito o no' (it is also good to modify the SUBSCRIBE method to let us know whether it has subscribed 
            successfully or not).
        \end{itemize}
    \item 4. Propósito y Beneficios de la Modificación:
        \begin{itemize}
            \item Esta modificación tiene como objetivo proporcionar una retroalimentación explícita sobre el resultado de la operación de suscripción.
            \item En un sistema IoT que depende de la comunicación en tiempo real para el control y monitoreo, es vital saber si una suscripción 
            (o re-suscripción) fue exitosa. Si la re-suscripción falla después de un cambio de permisos, el cliente podría no recibir las actualizaciones 
            esperadas, lo que afectaría la funcionalidad del panel de control o del dispositivo.
            \item Una confirmación de éxito en el método SUBSCRIBE permitiría al main.js manejar posibles fallos de re-suscripción, mejorando la 
            fiabilidad y el diagnóstico del sistema.
        \end{itemize}
\end{itemize}
En resumen, la sugerencia de modificar el método SUBSCRIBE para confirmar el éxito en el contexto del manejo de respuesta en main.js es 
una mejora de diseño propuesta para el lado del cliente. Su objetivo es asegurar que la re-suscripción a los canales de comunicación, que 
ocurre después de que el servidor ha concedido (o modificado) los permisos, se verifique adecuadamente, proporcionando una mayor certeza 
sobre la funcionalidad en tiempo real y la integridad del sistema IoT.

\subsection{Requisito Previo: Generar Clave de Autorización y Almacenar en Base de Datos}
% 6.6.4 Requisito Previo: Generar Clave de Autorización y Almacenar en Base de Datos.
En el contexto más amplio del Endpoint Flask para el Botón 'Aplicar', la generación de la clave de autorización para un usuario y 
su almacenamiento en la base de datos se presenta como un prerrequisito fundamental y el 'primer paso' en el proceso de gestión de permisos. 
Esto es lo que las fuentes dicen al respecto:
\begin{itemize}
    \item 1. Rol del Botón 'Aplicar' y la Solicitud al Endpoint Flask:
        \begin{itemize}
            \item Cuando un usuario administrador interactúa con el botón 'Aplicar Cambios' (ID: access user ID) en el panel de control (index.html), 
            el código JavaScript en main.js envía una solicitud POST al endpoint de la aplicación Flask en el servidor.
            \item Esta solicitud (grant - user ID - read state - right stick o grant user ID read and write) tiene como objetivo modificar los permisos 
            de lectura y escritura de un usuario específico en tiempo real.
        \end{itemize}
    \item 2. La Generación y Almacenamiento de la Clave de Autorización como Primer Paso:
        \begin{itemize}
            \item Las fuentes aclaran que 'otorgar permisos de lectura y escritura es un segundo paso'.
            \item El 'primer paso que viene antes de esto es generar la clave de autorización para ese usuario específico y almacenarla en la base de datos'.
            \item Esto indica que, antes de que el endpoint de Flask procese una solicitud para cambiar los permisos de lectura y escritura de un usuario 
            a través del botón 'Aplicar', ya debe existir una clave de autorización generada para ese usuario y debe estar almacenada en la base de datos. 
            Esta clave es esencial para el esquema de seguridad y acceso del sistema IoT.
        \end{itemize}
    \item 3. Acciones del Endpoint Flask (Segundo Paso):
        \begin{itemize}
            \item Una vez que el endpoint Flask recibe la solicitud POST del botón 'Aplicar' y después de haber verificado que la solicitud proviene de 
            un usuario administrador, el servidor realiza dos acciones principales:
                    \item 1. Almacenar los nuevos permisos de lectura y escritura del usuario en la base de datos.
                    \item 2. Llamar al servidor PubNub para otorgar o revocar el acceso de lectura y escritura a ese usuario específico en tiempo real.
            \item Estas dos acciones constituyen el 'segundo paso' del proceso, que se basa en la existencia previa de la clave de autorización.
        \end{itemize}
\end{itemize}
En resumen, la generación de la clave de autorización para un usuario y su almacenamiento en la base de datos es un prerrequisito 
fundamental que debe cumplirse antes de que el endpoint Flask, activado por el botón 'Aplicar', pueda procesar y aplicar los cambios 
en los permisos de lectura y escritura de ese usuario. Este proceso de dos pasos garantiza un control de acceso robusto y seguro en el 
ecosistema IoT, donde la clave de autorización inicial sienta las bases para la gestión dinámica de permisos.

\newpage
\bibliographystyle{ieeetr}
\bibliography{bibliografia}
\end{document}
